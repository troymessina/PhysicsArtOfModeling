\subsection{Appendix C - Guidelines for lab related activities}

\subsubsection{Overview}\label{app:labs}

This chapter introduces the skills that are necessary for thinking about how to design an experiment and to report on its results.

\begin{framed}
\textbf{Learning Objectives}\\
\begin{itemize}
\item Develop skills in general scientific writing.
\item Learn to write scientific proposals and experimental reports.
\item Learn to review others' scientific proposals and experimental reports.
\end{itemize}
\end{framed}

\subsubsection{The process of science and the need for scientific writing}

Conducting experiments that test a scientific theory is integral to the advancement of science and to the refining of scientific theories. In practice, scientists do not have a lab full of equipment ready to go and to be used for testing whichever theory suits their fancy. Instead, they need to write a ``proposal'' for conducting a particular experiment to a funding source (e.g. a funding agency). That funding source will then select a panel of experts in the field to review whether the proposal is feasible and useful in advancing science, to decide whether it should be funded. If the scientist is awarded with funds, they are then expected to carry out their experiment and report on the results in a peer-reviewed scientific journal. Again, before the results are published, the scientific journal will ask a panel of experts to review the results to ensure that they are scientifically valid and interesting.

In order for a proposal to be funded, it must thus propose an experiment that is well-thought out and feasible. For example, the reviewers will want to make sure that the proposed experiment is designed in the best possible way to test a theory. Often, this means that thought has been put into designing an experiment that minimizes the uncertainty on the result, so that the test of the theory is as stringent as possible.

A proposal needs to be well-written and precise. We generally call this type of writing ``scientific writing'', and it is a style of writing that takes some practice. Similarly, when reporting on the results of an experiment, the report will need to be clear and precise as well. For example, in scientific writing, one avoids giving opinions or using sentences that do not add necessary information or that are not factual.

This chapter provides some guidelines for scientific writing, writing proposals, and writing reports. In addition to this, guidelines for reviewing others' proposals and reports are also presented. Not only is it important to develop the ability to critically evaluate others' work, but it is also helpful in learning to reflect and improve on one's own work.

\subsubsection{Scientific writing}

Scientific writing is important in communicating with other scientists. Think of scientific writing as a style of writing where \textbf{every word counts}. It makes for rather ``dry'' reading, but it is important for clearly and precisely communicating factual information. The main guidelines for scientific writing are \textbf{be concise, precise, factual, and clear}. Below are some tips to help with scientific writing:

\begin{itemize}
\item Avoid subjective/imprecise terms: avoid using subjective and imprecise terms, stick to factual statements and avoid opinions.  Instead of saying ``our calculated value of g was much greater than the expected value'', say ``our calculated value of g was greater than the expected value''. Your opinion that it was ``much greater'' does not communicate anything and is imprecise (much greater in relation to what?).
\item Definitive statements: avoid attributing definitive causes to your experimental outcomes. You can never prove a theory to be correct, so at most, your results will be consistent with a theory. For example, instead of saying ``as the data exhibit, we have detected the Purple Particle'', you should state that ``the data are consistent with the detection of the Purple Particle''.
\item Data is the plural of datum. ``This data shows'' is incorrect, rather, ``these data show'', or ``this set of data shows''.
\item Active vs. passive voice: when writing scientific papers, it is recommended to use the third person, passive voice. For example, this would mean saying ``the drop time for balls at various heights was measured'' rather than ``we measured the drop time for balls at various heights''. However, both passive and active voices are acceptable in scientific writing, as long as it is consistent throughout the text.
\item Tense. Generally, for a proposal, you would use the future tense, and you would use the past tense for reporting on your results.
\end{itemize}

\begin{framed}
\textbf{Emma's Thoughts}\\
\textbf{Writing and editing - how can I be more concise?}
We've all felt that our writing was lacking at some point or another. Here are some general tips to avoid overall ``wordiness'' and to increase ease of reading when writing scientifically:

\begin{itemize}
\item What would you want to read? Let's say that you wanted to know the strength of Earth's magnetic field, and how it was found, so you decide to do a literature search. Would you choose a brief, succinct article, or a wordy Magnetic Field Manifesto?
\item The kindergarten test: If you had to explain your concept to a six year old cousin, how would you break it down in a way that they could understand it? If you can't break it down enough to explain to a six year old, perhaps you need to revisit your own understanding of the concept before writing about it scientifically.
\item Avoid unnecessary adjectives: while this might be ok in a creative writing class, in scientific writing, the goal is to get your point across as succinctly as possible. Using ``big'' words might be ok (as long as they properly describe what you are trying to say), but it is important to communicate your message in the simplest manner.
\item Think about it: every time you use a comma, dash or even an ``and'', you should reconsider the brevity of your statement. In scientific writing, commas are carefully placed, and semicolons are rare.
\item Cut it in half: For every word you read, think of another that you can cut. For every sentence that you read, think of three sentences that communicate the same idea. Pick the sentence that is the shortest and most concise.
\item Proofread - the more, the better.
\end{itemize}
\end{framed}

The following sections provide basic outlines for writing a proposal and a lab report, as well as rubrics for evaluating/reviewing proposals and reports. Additionally, samples of a proposal, proposal review, report, and report review for the experiment ``Measuring g using a pendulum'' are provided. In the sample proposal and lab report, errors are purposefully included and addressed in the reviews. It is important to entirely read the rest of this section to capture the common proposal/lab mistakes and their corresponding corrections. That is, do not take the sample proposal as a ``perfect proposal'', but rather, consider it in the light of the corresponding review.

\subsubsection{Guide for writing a proposal}

\textbf{Summary and Goal}

Write a few short sentences briefly summarizing the aim of your experiment, how it will be conducted, and how precise of a result you expect to obtain.

\textbf{Method and equipment}

Clearly describe, in as much detail as required, the method/procedure that you will use to carry out your experiment, and how you will analyse the results. Justify the choices that you made (no need to say you chose to use a ruler because you will need to measure a distance, but perhaps say why you need to measure a given distance, or that you chose to measure something in a particular way as it would reduce the corresponding uncertainty). Provide a list of the equipment that you will need. Also, propose a method of assessing whether or not your project was successful.

Consider the following questions:

\begin{itemize}
\item What theory are you testing and through what model?
\item How precisely do you estimate that you will be able to make your measurement? Estimate the uncertainty that you will obtain with the proposed experiment. Use this in guiding the design of your experiment.
\item What materials, equipment and/or tools are necessary in making your measurements?
\item What are the cost of these materials? Can they be easily obtained?
\item Where should this experiment be conducted?
\item Are there any safety concerns?
\item How will you make your measurements? How many times will you make them?
\item How will you record your measurements?
\item How will you maximize the precision of your experiments?
\item How will you determine uncertainties?
\item How will you analyse the data?
\item What issues could arise in your experiment? How do you plan to resolve these issues?
\end{itemize}

\textbf{Timeline and Team}

Provide the names of team members, and assign relevant duties to each member. Give a rough outline of the timeline to conduct the experiment, to analyse the data, and to report on the results.

\subsubsection{Guide for reviewing a proposal}

\textbf{Summary}

Summarize your overall evaluation of the proposal in 2-3 sentences. Focus on the experiment's methods and goals. For example, ``The authors wish to drop balls from different heights to determine the value of g''. You don't need to go into the specific details, just give a high level summary of the proposal and your opinion on whether this is a strong proposal. If the proposal is unclear, specify this.

\textbf{Review}

This is where you give your detailed review of the proposal. Consider the following questions:

\begin{itemize}
\item Is the proposed experiment well thought-out and feasible?
\item Is the experimental procedure clear and concise? Could you could carry out the experiment without asking the authors for additional information? Do the authors specify what instruments to use to measure different quantities and how to determine the associated uncertainties?
\item Does the experimental design minimize uncertainties?
\item Is it possible to complete the experiment in a reasonable period of time?
\item Is it possible to obtain the equipment/materials to conduct the experiment?
\item Do the authors describe how to analyse the data (correctly)?
\item Does the plan incorporate a mechanism to assess success?
\item Is a troubleshooting plan in place, in case of unexpected difficulties?
\end{itemize}

\textbf{Overall Rating of the Experiment}

Give the proposal an overall score, based on the criteria described above. Use one of the following to rate the proposal and include a sentence to justify your choice.

\begin{itemize}
\item Excellent
\item Good
\item Satisfactory
\item Needs work
\item Incomplete
\end{itemize}

\subsubsection{Guide for writing a lab report}

\textbf{Abstract}

Write a few short sentences briefly summarizing what you did, how you did it, what you found and whether anything went wrong in your experiment.

\textbf{Procedure}

Describe relevant theories that relate to your experiment here, and the steps to carry out your procedure.

Consider the following questions:

\begin{itemize}
\item What are the relevant theories/principles that you used?
\item What equations did you use? Show how you modelled your experiment.
\item What materials, equipment and/or tools were necessary in making your measurements?
\item Where was this experiment conducted?
\item How did you make your measurements? How many times did you make them?
\item How did you record your measurements?
\item How did you determine and minimize the uncertainties in your measurements? Why did you choose to measure a specific quantity in a certain way?
\end{itemize}

\textbf{Prediction}
It can be useful to predict the value (and uncertainty) that you expect to measure before conducting the measurement. You should report on this initial prediction in order to help you better understand the data from your experiment.

Consider the following questions:

\begin{itemize}
\item Predict your measured values and uncertainties. How precise do you expect your measurements to be?
\item What assumptions did you have to make to predict your results?
\item Have these predictions influenced how you should approach your procedure? Make relevant adjustments to the procedure based on your predictions.
\end{itemize}

\textbf{Data and Analysis}

Present your data. Include relevant tables/graphs. Describe in detail how you analysed the data, including how you propagated uncertainties. If the data do not agree with your model prediction (or the prediction from your proposal), examine whether you can improve your model.

Consider the following questions:

\begin{itemize}
\item How did you obtain the ``final'' measurement/value from your collected data?
\item How did you propagate uncertainties? Why did you do it that way?
\item What is the relative uncertainty on your value(s)?
\end{itemize}

\textbf{Discussion and Conclusion}

Summarize your findings, and address whether or not your model described the data. Discuss possible reasons why your measured value is not consisted with your model expectation (is it the model? is it the data?).

Consider the following questions:

\begin{itemize}
\item Were there any systematic errors that you didn't consider?
\item Did you learn anything that you didn't previously know? (eg. about the subject of your experiment, about the scientific method in general)
\item If you could redo this experiment, what would you change (if anything)?
\end{itemize}

\paragraph{Guide for reviewing a lab report}

\textbf{Summary}

Summarize your overall evaluation of the report in 2-3 sentences. Focus on the experiment's method and its result. For example, ``The authors dropped balls from different heights to determine the value of g''. You don't need to go into the specific details, just give a high level summary of the report. If the report is unclear, specify this.

\textbf{Review}

Consider the following questions:

\begin{itemize}
\item Is the the procedure well thought-out, clearly and concisely described?
\item Do you have sufficient information that you could repeat this experiment?
\item Does the report clearly describe how different quantities were measured and how the uncertainties were determined?
\item Does the report motivate why the specific procedure was chosen? (e.g. to minimize uncertainties).
\item Does the experiment clearly state how uncertainties were propagated and how the data were analysed?
\item Do you believe their result to be scientifically valid?
\end{itemize}

\textbf{Overall Rating of the Experiment}

Give the report an overall score, based on the criteria described above. Use one of the following to rate the proposal and include a sentence to justify your choice.

\begin{itemize}
\item Excellent
\item Good
\item Satisfactory
\item Needs work
\item Incomplete
\end{itemize}

\subsubsection{Sample proposal (Measuring g using a pendulum)}

\textbf{Summary and Goal}

One can measure the gravitational constant, $g$, by measuring the period of a pendulum of a known length, requiring only a string, mass, ruler and timer. Because the experimental design can be easily adjusted and the experiment is simple, the experiment has a high chance of success.

\textbf{Method and equipment}

The period of a pendulum of length $L$ is easily shown to be given by:

\begin{equation}
T=2\pi \sqrt {\frac{L}{g}}
\end{equation}

Thus, by measuring the period, $T$, of a pendulum as well as its length, one can determine the value of $g$:

\begin{equation}
g=\frac{4\pi^{2}L}{T^{2}}
\end{equation}

One can carry out the experiment using the following materials:

\begin{itemize}
\item a mass
\item inextensible string
\item a metre stick
\item stand to attach string
\item cell-phone with timer and slow-motion camera
\end{itemize}

The materials listed above are all inexpensive and can be easily obtained.  It is recommended that the experiment be completed indoors at room temperature, in order to minimize any environmental effects.

One should tie the string to the mass at one end and the stand at the other, and measure the length, $L$, of the string from the point on the stand to the centre of mass of the mass.

The period of the pendulum is measured by timing how long it takes the pendulum to complete 20 oscillations and dividing that time by 20. This will be more precise than trying to time the period of a single oscillation.

The pendulum should be released from $90\degree$. When releasing the pendulum, the string should be pulled taught, and the team member's eye that is measuring the angle should be situated parallel to the measuring device.

A slow-motion video will be taken of the pendulum to track the time of the oscillation in order to minimize error due to reaction time. The team member in charge of taking the video will start the video shortly before the pendulum is released. After releasing the pendulum, the team should record 20 oscillations before stopping the pendulum and the video. Data from the video should be entered into a Jupyter Notebook. It is recommended that this measurement be repeated at least 5 times.

The uncertainty in the time should be taken as half of the smallest division of the cell-phone timer, and the uncertainty in the length of the pendulum as half the smallest division of the metre stick used to measure the length of the pendulum.

Foreseeable issues in this experiment may arise when trying to find a string that is optimally inextensible, as any extensibility will cause error in the results. Additionally, being able to measure exactly $90\degree$ as the drop-angle for the pendulum could be difficult. In order to correct for this, the team member who is dropping the pendulum must stand directly parallel to the measuring device, minimizing parallax error.

The measure of success will be determined by the uncertainty and precision of the measured value of $g$. If the measured value of $g$ has a relative uncertainty that is less than 10 \%, and is consistent with the accepted value, then one can consider the experiment to have been carried out successfully.

\textbf{Team and timeline}

One should be able to complete the experiment and analysis in approximately 1 hour and 30 minutes with the data being collected in the first 30 minutes. The remainder of the time should be spent processing the data and writing the experimental report.\newline
Following the strengths of the members of the team, the following people should be responsible for leading the following tasks, while everyone participates:

\begin{itemize}
\item Alice: building the pendulum
\item Brice: taking the measurements
\item Chlo"e: analysing the data
\item Dennis: writing and formatting
\end{itemize}

\subsubsection{Sample proposal review (Measuring g using a pendulum)}

\textbf{Summary and Goal}

The authors propose to measure the value of $g$ to within 10\% by measuring the period of a simple pendulum, using the SHM equations and theory. The proposal is reasonably clear, but lacks some details in how to measure the initial angle of the pendulum. The authors propose to use a an amplitude of $90\degree$ for the pendulum, but at such a large angle, the motion is not expected to be SHM, since it is only so at small angles. By using a smaller angle, the experiment has a good chance of being successful in the proposed timeline.

\textbf{Review}

The experimental methods are described clearly and succinctly, with most information clearly stated. For the materials list, it is stated that ``a mass'' must be used. Here, it should be stated that a small, solid, non-deformable mass should be used to minimize drag and to act as a point mass. The authors refer to a ``measuring device'' when determining the amplitude of the pendulum, but this is not described. Anyhow, the amplitude of the oscillations in irrelevant for a pendulum in SHM, as long as the amplitude is small.

Most equations are described in the theory section, but it is incorrectly assumed that the period of a pendulum is independent of the drop angle for all angles. The small angle approximation is not expected to apply with an oscillation amplitude of $90\degree$.

No justification is provided for the use of 20 oscillations prior to measuring the period - it may be necessary to iterate on the reason why 20 oscillations was chosen.

The equipment can be easily obtained and is fairly inexpensive. Adequate resources are available to the group to perform this experiment. A clear troubleshooting plan is described and a method for evaluating success is included.

\textbf{Timeline and team}

This experiment is fairly simple and the equipment/setup is not difficult to handle. The proposed team should be qualified to perform this experiment in the proposed amount of time, although I worry a little bit about Dennis, as he seems to be a bit of a menace.

\textbf{Overall Rating of the Proposal}

Good - this proposal was clearly explained and is scientifically sound, apart from the use of a large angle for the oscillations. It was succinctly written, and most components of the experiment were clearly described. A little more detail in the justification for using 20 oscillations is necessary.

\subsubsection{Sample lab report (Measuring g using a pendulum)}

\textbf{Abstract}

In this experiment, we measured $g$ by measuring the period of a pendulum of a known length. We measured $g = 7.65\pm 0.378 {\rm m/s^2}$. This correspond to a relative difference of 22\% with the accepted value ($9.8 {\rm m/s^2}$), and our result is not consistent with the accepted value.

\textbf{Theory}

A pendulum exhibits simple harmonic motion (SHM), which allowed us to measure the gravitational constant by measuring the period of the pendulum. The period, $T$, of a pendulum of length $L$ undergoing simple harmonic motion is given by:
\begin{equation}
T=2\pi \sqrt {\frac{L}{g}}
\end{equation}

Thus, by measuring the period of a pendulum as well as its length, we can determine the value of $g$:
\begin{equation}
g=\frac{4\pi^{2}L}{T^{2}}
\end{equation}
We assumed that the frequency and period of the pendulum depend on the length of the pendulum string, rather than the angle from which it was dropped.

\textbf{Predictions}

We built the pendulum with a length $L=1.0000\pm 0.0005 {\rm m}$ that was measured with a ruler with $1 {\rm mm}$ graduations (thus a negligible uncertainty in $L$). We plan to measure the period of one oscillation by measuring the time to it takes the pendulum to go through 20 oscillations and dividing that by 20. The period for one oscillation, based on our value of $L$ and the accepted value for $g$, is expected to be $T=2.0 {\rm s}$. We expect that we can measure the time for 20 oscillations with an uncertainty of $0.5 {\rm s}$. We thus expect to measure one oscillation with an uncertainty of $0.025 {\rm s}$ (about 1\% relative uncertainty on the period). We thus expect that we should be able to measure $g$ with a relative uncertainty of the order of 1\%

\textbf{Procedure}

The experiment was conducted in a laboratory indoors.

\begin{enumerate}
\item Construction of the pendulum
\end{enumerate}

We constructed the pendulum by attaching a inextensible string to a stand on one end and to a mass on the other end. The mass, string and stand were attached together with knots. We adjusted the knots so that the length of the pendulum was $1.0000\pm0.0005 {\rm m}$. The uncertainty is given by half of the smallest division of the ruler that we used.

\begin{enumerate}[resume]
\item Measurement of the period
\end{enumerate}

The pendulum was released from $90\degree$ and its period was measured by filming the pendulum with a cell-phone camera and using the phone's built-in time. In order to minimize the uncertainty in the period, we measured the time for the pendulum to make 20 oscillations, and divided that time by 20. We repeated this measurement five times. We transcribed the measurements from the cell-phone into a Jupyter Notebook.

\textbf{Data and Analysis}

Using a $100 {\rm g}$ mass and $1.0 {\rm m}$ ruler stick, the period of 20 oscillations was measured over 5 trials. The corresponding value of $g$ for each of these trials was calculated. The following data for each trial and corresponding value of $g$ are shown in Table~\ref{tab:labs:exreport}.

\begin{table}
\centering
\caption[]{Results of five trials measuring a pendulum period.}
\label{tab:labs:exreport}
\begin{tabular}{p{\dimexpr 0.250\linewidth-2\tabcolsep}p{\dimexpr 0.250\linewidth-2\tabcolsep}p{\dimexpr 0.250\linewidth-2\tabcolsep}p{\dimexpr 0.250\linewidth-2\tabcolsep}}
\toprule
Trial & Angle (Degrees) & Measured Period (s) & Value of g (${\rm m/s^2}$) \\
\hline
1 & 90 & 2.24 & 7.87 \\
2 & 90 & 2.37 & 7.03 \\
3 & 90 & 2.28 & 7.59 \\
4 & 90 & 2.26 & 7.73 \\
5 & 90 & 2.22 & 8.01 \\
\bottomrule
\end{tabular}
\end{table}

Our final measured value of $g$ is $7.65\pm 0.378 {\rm m/s^2}$. This was calculated using the mean of the values of g from the last column and the corresponding standard deviation. The relative uncertainty on our measured value of $g$ is 4.9\% and the relative difference with the accepted value of $9.8 {\rm m/s^2}$ is 22\%, well above our relative uncertainty.

\textbf{Discussion and Conclusion}

In this experiment, we measured $g=7.65\pm 0.378 {\rm m/s^2}$. This has a relative difference of 22\% with the accepted value and our measured value is not consistent with the accepted value. All of our measured values were systematically lower than expected, as our measured periods were all systematically higher than the $2.0 {\rm s}$ that we expected from our prediction. We also found that our measurement of $g$ had a much larger uncertainty (as determined from the spread in values that we obtained), compared to the 1\% relative uncertainty that we predicted.

We suspect that by using 20 oscillations, the pendulum slowed down due to friction, and this resulted in a deviation from simple harmonic motion. This is consistent with the fact that our measured periods are systematically higher. We also worry that we were not able to accurately measure the angle from which the pendulum was released, as we did not use a protractor.

If this experiment could be redone, measuring 10 oscillations of the pendulum, rather than 20 oscillations, could provide a more precise value of $g$. Additionally, a protractor could be taped to the top of the pendulum stand, with the ruler taped to the protractor. This way, the pendulum could be dropped from a near-perfect $90 {\rm \degree}$ rather than a rough estimate.

\subsubsection{Sample lab report review (Measuring g using a pendulum)}

\textbf{Summary}

The authors measured the period of a pendulum to determine $g$. They measured $g$ to be $7.65\pm0.378 {\rm m/s^2}$ which is inconsistent with the accepted value. The authors were incorrect in assuming that the pendulum would undergo simple harmonic motion in the conditions that they used.

\textbf{Review}

The experimental procedure was clearly written and one could mostly reproduce this experiment with the given description.

The authors thought about minimizing uncertainties by measuring the period over several oscillations, although it appears that 20 was perhaps too large, as friction was likely to have an effect. The authors should have taken more care in determining the number of oscillations to use so that the uncertainty in the time is minimized while also keeping the effects of friction negligible. Ultimately, the authors did not specify the uncertainty in the time that they measured.

The authors also claim to have measured the length of the pendulum with a precision of $0.5 {\rm mm}$, but did not specify the length of the ruler that they used. I would not expect the measurement to be that precise unless they used a very precise ruler that is longer than $1 {\rm m}$. However, the authors made the length of the pendulum as long as possible so as to minimize the uncertainty in the length.

The authors did not describe the mass that was attached at the end of the pendulum, and whether its size would be expected to cause significant air drag.

The authors made a mistake in assuming that a pendulum would undergo simple harmonic motion with an amplitude of $90 {\rm \degree}$, as the small angle approximation used to determine the period does not apply in this case.

The experimental procedure was scientifically sound, other than the choices for the number of oscillations and their amplitude.

\textbf{Overall rating of the Experiment}

Satisfactory - The experiment was well described, but the authors should have paid more attention to their choice of 20 oscillations, and they made a mistake in assuming that their pendulum would exhibit simple harmonic oscillation with a large amplitude.