% Created with jtex v.1.0.18
%%%%%%%%%%%%%%%%%%%%%%%%%%%%%%%%%%%%%%%%%%%%%%%%%%%%%%%%%%%%
%%% LaPreprint: PREPRINT TEMPLATE
%%%%%%%%%%%%%%%%%%%%%%%%%%%%%%%%%%%%%%%%%%%%%%%%%%%%%%%%%%%%

% Here I could talk about what one should do in this document.
% Instead I'll refer you to the explore on your own and check the Github Repo. :-)
% Line spacing is 1.2 by default (can't be smaller).

%%%%%%%%%%%%%%%%%%%%%%%%%%%%%%%%%%%%%%%%%%%%%%%%%%%%%%%%%%%%
%%% PREAMBLE
%%%%%%%%%%%%%%%%%%%%%%%%%%%%%%%%%%%%%%%%%%%%%%%%%%%%%%%%%%%%

% Declare document class
\documentclass[9pt,arxiv,red]{lapreprint}
% Choose between "biorxiv", "medrxiv", "arxiv" and "chemrxiv". Otherwise defaults "Preprint".
% Choose between "blue" and "red" colour scheme. Defaults to "blue".
% Use the "onehalfspacing" option for 1.5 line spacing.
% Use the "doublespacing" option for 2.0 line spacing.
% Use the "lineno" option for line numbers.
% Use the "endfloat" option to place floats after the bibliography.
% Use the "secnum" option to have include numbers.

% Import packages
% \usepackage{lipsum}     % Required to insert dummy text
\usepackage[version=4]{mhchem} % For chemical notation
\usepackage{siunitx}    % For SI units
\usepackage{pdflscape}  % For putting pages in landscape mode
\usepackage{rotating}   % For rotating specific elements
\usepackage{textgreek}  % Greek symbols
\usepackage{gensymb}    % Symbols
\usepackage[misc]{ifsym} % For the \Letter symbol
\usepackage{orcidlink}  % For the \orcidlink
\usepackage{listings}   % For inserting code chunks
\usepackage{colortbl}   % For Knitr table colouring
\usepackage{tabularx}   % For making Knitr tables compatible
\usepackage{longtable}  % For multi-page tables
\usepackage{subcaption}
\usepackage{multirow}
\usepackage{snotez}     % For sidenote environments. enotez for endnotes
\usepackage{csquotes}   % For language-based quote rules (helps BiBLaTeX)

%%%%%%%%%%%%%%%%%%%%%%%%%%%%%%%%%%%%%%%%%%%%%%%%%%
%%%%%%%%%%%%%%%%%%%%  imports  %%%%%%%%%%%%%%%%%%%
\usepackage{framed}
\usepackage{url}
%%%%%%%%%%%%%%%%%%%%%%%%%%%%%%%%%%%%%%%%%%%%%%%%%%


% Make declarations
\DeclareSIUnit\Molar{M}

% Please note that these options may affect formatting.

%%%%%%%%%%%%%%%%%%%%%%%%%%%%%%%%%%%%%%%%%%%%%%%%%%%%%%%%%%%%
%%% BIBLIOGRAPHY
%%%%%%%%%%%%%%%%%%%%%%%%%%%%%%%%%%%%%%%%%%%%%%%%%%%%%%%%%%%%
\usepackage[			% use biblatex for bibliography
	backend=biber,      % use biber or bibtex backend
    style=authoryear,   % choose style
	natbib=true,		% allow natbib commands
	hyperref=true,	    % activate hyperref support
	alldates=year,      % only show year (not month)
]{biblatex}

% Just avoiding some rogue fields that cause issues with certain styles
\AtEveryBibitem{
    \clearfield{urlyear}
    \clearfield{urlmonth}
    \clearlist{language}
}



%%%%%%%%%%%%%%%%%%%%%%%%%%%%%%%%%%%%%%%%%%%%%%%%%%%%%%%%%%%%
%%% ARTICLE SETUP
%%%%%%%%%%%%%%%%%%%%%%%%%%%%%%%%%%%%%%%%%%%%%%%%%%%%%%%%%%%%

% Paper title
\title{Modeling with Physics}

% Authors - you can use \orcidlink{} and \authfn{} - see contribution note

% Affiliations

% Other metadata. Feel free to add your own
\metadata[]{Keywords}{Introductory Physics}


% Surname of the lead author(s) for the running footer
\leadauthor{}
\shorttitle{Modeling with Physics}

%%%%%%%%%%%%%%%%%%%%%%%%%%%%%%%%%%%%%%%%%%%%%%%%%%%%%%%%%%%%
%%% ARTICLE START
%%%%%%%%%%%%%%%%%%%%%%%%%%%%%%%%%%%%%%%%%%%%%%%%%%%%%%%%%%%%

\begin{document}
\maketitle


\section{Part 1 - Mechanics}

\include{ModelingWithPhysics-introduction}

\include{ModelingWithPhysics-modelandexperiment}

\include{ModelingWithPhysics-vectors}

\subsection{Chapter 4 NEW - Linear momentum and the centre of mass}

\subsubsection{Overview}\label{chap:momentumandcm}

In this chapter, we introduce the concepts of linear momentum and of centre of mass. Momentum is a quantity that, like energy, can be defined from Newton's Second Law, to facilitate building models. Since momentum is often a conserved quantity within a system, it can make calculations much easier than using forces. The concepts of momentum and of centre of mass will also allow us to apply Newton's Second Law to systems comprised of multiple particles including solid objects.

\begin{framed}
\textbf{LearningObjectives}\\
\begin{itemize}
\item Understand how to calculate linear momentum.
\item Understand how to calculate impulse and that it corresponds to a change in momentum.
\item Understand when and how to apply conservation of linear momentum to model situations.
\item Understand the difference between elastic and inelastic collisions, and when mechanical energy is conserved.
\item Understand how to calculate the centre of mass of an object.
\end{itemize}
\end{framed}

\begin{framed}
\textbf{Think About It}\\
You hit a pool ball square on with the cue ball. If both balls have the same mass, and you can neglect any ``english'' on the cue ball, what happens to the cue ball?

\begin{enumerate}
\item It stops.
\item It continues, with half of its original speed.
\item It continues, with its original speed.
\item It rebounds, with its original speed.
\end{enumerate}

\begin{framed}
\textbf{Answer}\\
\begin{enumerate}
\item
\end{enumerate}
\end{framed}
\end{framed}

\subsubsection{Momentum and Newton's First Law}

Momentum is a quantity that describes an object's motion. Imagine an object that has a mass of 1 kg and a velocity of 1 m/s. Now, imagine doubling the mass. How would you say the object's motion has changed. It may help to think of the quantity of motion as ``oomph''. Does a more massive object have more or less oomph than a less massive object when they both move at the same speed? What if the 1 kg object doubles its speed? Does it have more oomph?

In 1687, Sir Isaac Newton published his Philosophiae Naturalis Principia Mathematica, where, among other things, he detailed his three laws of motion. The first law is summarized as

\begin{quote}
An object will remain in its state of motion, be it at rest or moving with constant velocity, unless a net external force is exerted on the object.
\end{quote}

In perhaps simpler terms, this says that an object's oomph or \textbf{momentum} will remain a constant quantity if nothing pushes or pulls on it.

\paragraph{Momentum of a point particle}

We can define the momentum, $\vec p$, of a particle of mass $m$ and velocity $\vec v$ as the vector quantity:
\begin{equation}
\boxed{\vec p = m\vec v}
\end{equation}
Since this is a vector equation, it corresponds to three equations, one for each component of the momentum vector. It should be noted that the numerical value for the momentum of a particle is arbitrary, as it depends in which frame of reference the velocity of the particle is defined. For example, your velocity with respect to the surface of the Earth is zero, so your momentum relative to the surface of the Earth is zero. However, relative to the surface of the Sun, your velocity, and momentum, are not zero. As we will see, forces are related to a \textit{changes} in momentum, just as they are related to a change in velocity (acceleration).

Consider a point particle moving at constant velocity such as a rock sliding across a frozen pond. If we capture an image of the rock at the same time we start a stopwatch ($t_o = 0 {\rm s}$), it might be located at a position 0.5 meters from the shore as shown in Figure~\ref{fig:momentumandcm:constvelmotiona}. We will designate the direction the rock slides as being along the $x$-axis.

\begin{figure}[!htbp]
\centering
\begin{figure}[!htbp]
\centering
\includegraphics[width=0.7\linewidth]{files/ConstVelDotsa-bb1e54c58a33c4382d81aa1e636f3fb7.png}
\caption[]{The initial position x\_o=0.5 \{{\textbackslash}rm m\} of a rock sliding across a frozen pond.}
\label{fig:momentumandcm:constvelmotiona}
\end{figure}

\begin{figure}[!htbp]
\centering
\includegraphics[width=0.7\linewidth]{files/ConstVelDotsa-bb1e54c58a33c4382d81aa1e636f3fb7.png}
\caption[]{The position of a rock sliding across a frozen pond as time progresses.}
\label{fig:momentumandcm:constvelmotionb}
\end{figure}
\caption[]{The motion diagram of a rock sliding across a frozen pond.}
\label{fig:momentumandcm:constvelmotionb}
\end{figure}

As time progresses, the rock will be at greater and greater distances from the shore. As shown in Figure~\ref{fig:momentumandcm:constvelmotionb}, the rock slides 0.5 meters every second. Suppose we recorded its $x$-position every second in a table and obtained the values in Table~\ref{tab:MomentumAndCM:1dmotion} (we will ignore measurement uncertainties discussed in Section~\ref{sec:ModelAndExperiment:uncertainties} and pretend that the values are exact).

\begin{table}
\centering
\caption[]{Time and $x$-position of a rock sliding across a frozen pond recorded every second.}
\label{tab:MomentumAndCM:1dmotion}
\begin{tabular}{p{\dimexpr 0.500\linewidth-2\tabcolsep}p{\dimexpr 0.500\linewidth-2\tabcolsep}}
\toprule
Time $t$ (s) & $x$-position (m) \\
\hline
0.0 & 0.5 \\
1.0 & 1.0 \\
2.0 & 1.5 \\
3.0 & 2.0 \\
4.0 & 2.5 \\
5.0 & 3.0 \\
6.0 & 3.5 \\
7.0 & 4.0 \\
8.0 & 4.5 \\
9.0 & 5.0 \\
\bottomrule
\end{tabular}
\end{table}

The easiest way to visualize the values in the table is to plot them on a graph, as in Figure~\ref{fig:MomentumAndCM:1dxvst}. Plotting position as a function of time is one of the most common graphs to make in physics, since it is often a complete description of the motion of an object.

\begin{figure}[!htbp]
\centering
\includegraphics[width=0.7\linewidth]{files/1dxvst-d6543db2884dccdda7dabba7850de246.png}
\caption[]{Plot of position as a function of time using the values from Table~\ref{tab:MomentumAndCM:1dmotion}.}
\label{fig:MomentumAndCM:1dxvst}
\end{figure}

The data plotted in Figure~\ref{fig:MomentumAndCM:1dxvst} show that the $x$ position of the ball increases linearly with time (i.e. it is a straight line and the position increases at a constant rate). This means that in equal time increments, the rock will cover equal distances. Note that we also had the liberty to choose when we define $t=0$; in this case, we chose that time is zero when the rock is at $x=0.5 {\rm m}$.

\begin{framed}
\textbf{Checkpoint}\\
Using the data from Table~\ref{tab:MomentumAndCM:1dmotion}, at what position along the $x$-axis will the ball be when time is $t=9.5 {\rm s}$, if it continues its motion undisturbed?

\begin{enumerate}
\item $5.0 {\rm m}$
\item $5.25 {\rm m}$
\item $5.75 {\rm m}$
\item $6.0 {\rm m}$
\end{enumerate}

\begin{framed}
\textbf{Answer}\\
\begin{enumerate}[resume]
\item
\end{enumerate}
\end{framed}
\end{framed}

Since the position as a function of time for the ball plotted in Figure~\ref{fig:MomentumAndCM:1dxvst} is linear, we can summarize our description of the motion using a function, $x(t)$, instead of having to tabulate the values as we did in Table Table~\ref{tab:MomentumAndCM:1dmotion}. The function will have the functional form:
\begin{equation}
\label{eqn:MomentumAndCM:1dxvst_noa}
\boxed{x(t) = x_0 + v_x t}
\end{equation}
The constant $x_0$ is the ``offset'' of the function; the value that the function has at $t= 0 {\rm s}$. We call $x_0$ the ``initial position'' of the object (its position at $t=0$). The constant $v_x$ is the ``slope'' of the function and gives the rate of change of the position as a function of time. We call $v_x$ the ``velocity'' of the object.

The initial position is simply the value of the position at $t=0$, and is given from the table as:
\begin{equation}
x_0 = 0.5 {\rm m}
\end{equation}

The velocity, $v_x$, is simply the difference in position, $\Delta x$, between any two points divided by the amount of time, $\Delta t$, that it took the object to move between those two points (``rise over run'' for the graph of $x(t)$):

\begin{equation}
\label{eq:momentumandcm:constveleq}
\vec v &= \frac{\Delta x}{\Delta t} \hat x\rightarrow \frac{dx}{dt} \hat x\\
\vec v &= \frac{0.5 {\rm m}}{1 {\rm s}} \hat x \\
\vec v &= \left(0.5, 0, 0\right) {\rm m/s}
\end{equation}
Therefore, the rock has a velocity of 0.5 m/s along the $x$-direction. In the first line of equation (\ref{eq:momentumandcm:constveleq}), the derivative after the arrow corresponds to when $\Delta x$ and $\Delta t$ become infinitesimal changes. Rearranging the first line of equation (\ref{eq:momentumandcm:constveleq}) we see that the rock moves a small increment $\Delta x$ each increment of time $\Delta t$
\begin{equation}
\Delta x = v\Delta t
\end{equation}
Therefore, we can think of the position equation (\ref{eqn:MomentumAndCM:1dxvst_noa}) as
\begin{equation}
x(t) = x_0 + \Delta x = x_0 + vt
\end{equation}
As long as the velocity is constant, we can use equation (\ref{eqn:MomentumAndCM:1dxvst_noa}) to determine the position of an object between any two points in time.

\subparagraph{Simulating the rock}

\begin{framed}
\textbf{Review}\\
Before proceeding, you may wish to review:

\begin{itemize}
\item Section~\ref{app:visualpython} for an introduction to programming with Visual Python using \href{http://trinket.io}{trinket.io}.
\end{itemize}
\end{framed}

Under constant velocity motion we can describe a future position of an object using the current position, the velocity, and an increment of time to progress into the future. With equation (\ref{eqn:MomentumAndCM:1dxvst_noa}), we can model this motion on computer. We will use the Python language in this textbook, and in particular, we will use Visual Python so that we can apply physics to objects that visualize physical motion in 3D. To get started, we first need to define an object that will move through space. We will make a sphere call it \texttt{rock}. The \texttt{rock} object can beplaced at a position using a vector \texttt{pos=vec(x,y,z)}.

\begin{verbatim}
rock = sphere(pos=vec(0,0,0), color=color.green, radius=0.1)
\end{verbatim}

We can define the velocity to be a vector along the x-direction also using the built-in vector function \texttt{velocity=vec(vx, vy, vz)}. To set the velocity to the appropriate vector, it would be coded

\begin{verbatim}
velocity = vec(0.5, 0, 0)
\end{verbatim}

Alternatively, it is possible to make the velocity one of the attributes of the rock object. We should be careful to only use this method when the attribute is a property of the object, e.g.,

\begin{verbatim}
rock.vel = vec(0.5, 0, 0)
\end{verbatim}

We can visualize the velocity with a vector arrow that remains attached to the rock's position and has a length that is the magnitude of the velocity. To do this we write the following code.

\begin{verbatim}
arrow(pos=rock.pos, axis=rock.vel, color=color.white)
\end{verbatim}

Try, putting these three lines of code in the trinket below and see what happens when you run the program.

\begin{figure}[!htbp]
\centering
\caption[]{A blank trinket to simulate a rock sliding on a frozen pond.}
\label{chap:momentumandcm:blanktrinket}
\end{figure}

\paragraph{Non-constant Momentum}

If the particle has a constant mass, then the time derivative of its momentum is given by:
\begin{equation}
\frac{d}{dt}\vec p = \frac{d}{dt}m\vec v = m\frac{d}{dt}\vec v=m\vec a
\end{equation}
and we can write this as Newton's Second Law, since $m\vec a$ must be equal to the vector sum of the forces on the particle of mass $m$:
\begin{equation}
\boxed{\frac{d}{dt}\vec p = \sum \vec F = \vec F^{net}}
\end{equation}
The equation above is the original form in which Newton first developed his theory. It says that the net force on an object is equal to the rate of change of its momentum. \textbf{If the net force on the object is zero, then its momentum is constant} (as is its velocity). In terms of components, Newton's Second Law written for the rate of change of momentum is given by:
\begin{equation}
\frac{dp_x}{dt} =& \sum F_x\\
\frac{dp_y}{dt} =& \sum F_y\\
\frac{dp_z}{dt} =& \sum F_z
\end{equation}

\begin{framed}
\textbf{Example 4.1}\\
A particle of mass $m$ is released from rest and allowed to fall freely under the influence of gravity near the Earth's surface (assume that drag is negligible). Is the mechanical energy of the particle conserved? Is the momentum of the particle conserved? If momentum is not conserved, how does momentum change with time? Do your answers change if the force of drag cannot be ignored?\}

\begin{framed}
\textbf{Solution}\\
First, we model the falling particle assuming that there is no force of drag. The only force exerted on the particle is thus its weight.

The mechanical energy of the particle will be conserved only if there are no non-conservative forces doing work on the particle. Since the force of gravity is the only force acting on the particle, its mechanical energy is conserved.

The total momentum of the particle is not conserved, because the sum of the forces on the particle is not zero. Choosing the $z$ axis to be vertical and positive upwards, Newton's Second Law in the $z$ direction is given by:
\begin{equation}
\sum F_z = -mg=\frac{dp_z}{dt}
\end{equation}
Note that the $x$ and $y$ components of momentum are conserved, since there are no forces with components in that direction. We can find how the $z$ component of the momentum changes with time by taking the anti-derivative of the force with respect to time (from $t=0$ to $t=T$):
\begin{equation}
\frac{dp_z}{dt} &= -mg\\
\int dp_z &= \int_0^T (-mg) dt\\
p_z(T) - p_z(0) &= -mgT\\
\therefore p_z(T) &= p_z(0) - mgT
\end{equation}
where the $z$ component of momentum, $p_z(T)$ at some time $T$, is given by its value at time $t=0$ plus $-mgT$. If the object started at rest ($\vec v=0$), then the magnitude of the momentum, as a function of time, is given by:
\begin{equation}
p(t) = p_z(t) = -mgt
\end{equation}
and indeed changes with time.

If the force of drag were not negligible, there would be a non-conservative force acting on the particle, so its mechanical energy would no longer be conserved. The particle will accelerate until it reaches terminal velocity. During that phase of acceleration, the net force on the particle is not zero (it is accelerating), so its momentum is not conserved. Once the particle reaches terminal velocity, the net force on the particle is zero, and its momentum is conserved from then on.

\textbf{Discussion:} This simple example highlights the fact that mechanical energy and momentum are conserved under different conditions. Just because one is conserved does not mean that the other is conserved. It also shows that Newton's Second Law is a statement about change in momentum, not momentum itself (just like it is a statement about acceleration, change in velocity, not velocity).
\end{framed}
\end{framed}

\paragraph{Impulse}

When we introduced the concept of energy, we started by calculating the ``work'', $W$, done by a force exerted on an object over a specific path between two points:
\begin{equation}
W = \int_A^B \vec F \cdot d\vec l
\end{equation}
We then introduced kinetic energy, $K$, to be that quantity whose change is equal to the net work done on the particle
\begin{equation}
W^{net} = \int_A^B \vec F^{net}\cdot d\vec l = \Delta K
\end{equation}
where the net force, $\vec F^{net}$, is the vector sum of the forces on the particle.

We can do the same thing, but instead of integrating the force over distance, we can integrate it over time. We thus introduce the concept of ``impulse'', $\vec J$, of a force, as that force integrated from an initial time, $t_A$, to a final time, $t_B$:
\begin{equation}
\vec J = \int_{t_A}^{t_B}\vec F dt
\end{equation}
where it should be clear that impulse is a vector quantity (and the above vector equation thus corresponds to one integral per component). Impulse is, in general, defined as an integral because the force, $\vec F$, could change with time. If the force is constant in time (magnitude and direction), then we can define the impulse without using an integral:
\begin{equation}
\vec J = \vec F \Delta t
\end{equation}
where $\Delta t$ is the amount of time over which the force was exerted. Although the force might never be constant, we can sometimes use the above formula to calculate impulse using an average value of the force.

\begin{framed}
\textbf{Checkpoint}\\
What is the SI unit for impulse?

\begin{enumerate}
\item ${\rm kg\cdot m/s^2}$
\item ${\rm kg \cdot s^2}$
\item ${\rm kg \cdot m/s}$
\item ${\rm kg \cdot m/s^3}$
\end{enumerate}

\begin{framed}
\textbf{Answer}\\
\begin{enumerate}[resume]
\item
\end{enumerate}
\end{framed}
\end{framed}

\begin{framed}
\textbf{Example 4.2}\\
Estimate the impulse that is given to someone's hand when they swat a fly on the surface of a table.

\begin{framed}
\textbf{Solution}\\
When we swat a fly with our hand, our hand exerts a force on the table surface during the period of time, $\Delta t$, over which our hand is in contact with the table surface. During that period of time, the force on the hand goes from being 0, to some unpleasantly high value, and then back to zero, so the force cannot be considered constant.

Let us estimate the average magnitude of the swatting force by considering the deceleration of our swatting hand and modelling the motion as one-dimensional. Let us assume that our swatting hand has a mass $m=1 {\rm kg}$ and that it is has a speed of $2 {\rm m/s}$ just before it makes contact. Furthermore, let us assume that it is in contact with the table for a period of time $\Delta t$. This allows us to find the average acceleration of our hand and thus the average force exerted by the table on our hand to stop it:
\begin{equation}
a &= \frac{\Delta v}{\Delta t}\\
\therefore F &= ma = m  \frac{\Delta v}{\Delta t}
\end{equation}
By Newton's Third Law, the force decelerating our hand has the same magnitude as the force that our hand exerts on the table, allowing us to calculate the impulse given to the person's hand:
\begin{equation}
J &= F\Delta t =  \left(m  \frac{\Delta v}{\Delta t}\right) \Delta t = m\Delta v\\
&=(1 {\rm kg})(2 {\rm m/s})=2 {\rm kg\cdot m/s}
\end{equation}
\textbf{Discussion:} Note that the impulse given to the table corresponds exactly to the change in momentum of the hand ($\Delta p=m\Delta v$).
\end{framed}
\end{framed}

So far, we calculated the impulse that is given by a single force. We can also consider the net impulse given to an object by the net force exerted on the object:
\begin{equation}
\vec J^{net} = \int_{t_A}^{t_B}\vec F^{net} dt
\end{equation}
Compare this to Newton's Second Law written out using momentum:
\begin{equation}
\frac{d}{dt}\vec p &= \vec F^{net}\\
\int_{\vec p_A}^{\vec p_B} d\vec p &=  \int_{t_A}^{t_B}\vec F^{net} dt\\
\vec p_B - \vec p_A &=  \int_{t_A}^{t_B}\vec F^{net}dt\\
\therefore \Delta \vec p &= \int_{t_A}^{t_B}\vec F^{net}	 dt
\end{equation}
and we find that the net impulse received by a particle is precisely equal to its change in momentum:
\begin{equation}
\boxed{\Delta \vec p = \vec J^{net}}
\end{equation}
This is similar to the statement that the net work done on an object corresponds to its change in kinetic energy, although one should keep in mind that momentum is a vector quantity, unlike kinetic energy.

\begin{framed}
\textbf{Example 4.3}\\
A car moving with a speed of $100 {\rm km/h}$ collides with a building and comes to a complete stop. The driver and passenger each have a mass of $80 {\rm kg}$. The driver wore a seat belt that extended during the collision, so that the force exerted by the seatbelt on the driver acted for about $2.5 {\rm s}$. The passenger did not wear a seat belt and instead was slowed down by the force exerted by the dashboard, over a much smaller amount of time, $0.2 {\rm s}$. Compare the average decelerating force experienced by the driver and the passenger.

\begin{framed}
\textbf{Solution}\\
We can calculate the change in momentum of both people, which will be equal to the impulse they received as they collided with the seatbelt or with the dashboard. Since we know the duration in time that the forces were exerted, we can calculate the average force involved in order to give the required impulse. We can assume that this all happens in one dimension, so we use scalar quantities instead of vectors.

The change in momentum along the direction of motion for either the driver or passenger is given by:
\begin{equation}
\Delta p = p_B - p_A = (0)-p_A=-mv_A
\end{equation}
where $v_A$ is the initial speed of the car, and the final momentum of either person is zero.

The change in momentum is equal to the impulse received by either person during a period of time $\Delta t$, which is related to the force that was exerted on them:
\begin{equation}
J=F\Delta t &= \Delta p = -mv_A\\
F&=-m \frac{v_A}{\Delta t}
\end{equation}
For the driver, this corresponds:
\begin{equation}
F=(80 {\rm kg})\frac{(27.8 {\rm m/s})}{(2.5 {\rm s})}=890 {\rm N}
\end{equation}
and for the passenger:
\begin{equation}
F=(80 {\rm kg})\frac{(27.8 {\rm m/s})}{(0.2 {\rm s})}=11120 {\rm N}
\end{equation}
The force on the driver is thus comparable to their weight, whereas the passenger experiences an average force that is more than 10 times their weight.

\textbf{Discussion:} Any mechanism that results in a longer collision time will help to reduce the forces that are involved. This is why cars are designed to crumple in head-on collisions. We can understand this in terms of the crumpling of the car absorbing some of the kinetic energy of the car, as well as lengthening the time of the collision so that the forces involved are smaller. You may also hear people that look at modern cars that are all crumpled up after a crash and say something along the lines of ``They sure don't make cars the way they used to''. But of course, that is by design; it is safer if the car crumples up (and cars are designed to crumple up in specific areas, not the passenger cabin).

Note that we did not need to use impulse to calculate the average force, since we could have just used kinematics to determine the acceleration and Newton's Second Law to calculate the corresponding force. Using impulse is equivalent by construction, but sometimes, it is easier mathematically.
\end{framed}
\end{framed}

\paragraph{Systems of particles: internal and external forces}

So far, we have only used Newton's Second Law to describe the motion of a single point mass particle or to describe the motion of an object whose orientation we did not need to describe (e.g. a block sliding down a hill). In this section, we consider what happens when there are multiple point particles that form a ``system''.

In physics, we loosely define a system as the ensemble of objects/particles that we wish to describe. So far, we have only described systems made of one particle, so describing the motion of the system was equivalent to describing the motion of that single particle. A  system of two particles could be, for example, two billiard balls on a pool table. To describe that system, we would need to provide functions that describe the positions, velocities, and forces exerted on both balls. We can also define functions/quantities that describe the system as a whole, rather than the details. For example, we can define the total kinetic energy of the system, $K$, corresponding to the sum of kinetic energies of the two balls. We can also define the total momentum of the system, $\vec P$, given by the vector sum of the momenta of the two balls.

When considering a system of multiple particles, we distinguish between \textbf{internal} and \textbf{external} forces. Internal forces are those forces that the particles in the system exert on each other. For example, if the two billiard balls in the system collide with each other, they will each exert a force on the other during the collision; those forces are internal. External forces are all other forces exerted on the particles of the system. For example, the force of gravity and the normal force from the pool table are both external forces exerted on the balls in the system (exerted by the Earth, or by the pool table, neither of which we considered to be part of the system). The force exerted by a person hitting one of the balls with a pool queue is similarly an external force. What we consider to be a system is arbitrary; we could consider the pool table and the Earth to be part of the system along with the two balls; in that case, the normal force and the weight of the balls would become internal forces. The classification of whether a force is internal or external to a system of course depends on what is considered part of the system.

\begin{framed}
\textbf{Checkpoint}\\
Two pool balls crash against each other. Is this force of gravity exerted by one ball on the other an internal or external force?

\begin{enumerate}
\item Internal.
\item External.
\end{enumerate}

\begin{framed}
\textbf{Answer}\\
\begin{enumerate}
\item
\end{enumerate}
\end{framed}
\end{framed}

The key property of internal forces is that \textbf{the vector sum of the internal forces in a system is zero}. Indeed, Newton's Third Law states that for every force exerted by object A on object B, there is a force that is equal in magnitude and opposite in direction exerted by object B on object A. If we consider both objects to be in the same system, then the sum of the internal forces between objects A and B must sum to zero. It is important to note that this is quite different than what we have discussed so far about summing forces. The forces that sum to zero are exerted on \textit{different} objects. Thus far, we had only ever considered summing forces that are exerted on the same object in order to apply Newton's Second Law. We have never encountered a situation where ``action'' and ``reaction'' forces are summed together, because they act on different objects.

\begin{framed}
\textbf{Emma's Thoughts}\\
\textbf{Internal vs. External forces - what is the ``system'' and what forces should we consider?}

As discussed above, internal and external forces can only be considered in the context of a specific system. So, how do we define this ``system''? How far do we go when defining the system?

For example, let's say that you kick a soccer ball, and it hits a nearby lawn chair, knocking it down. You want to determine what will happen to the soccer ball after it hits the lawn chair. What is defined to be the system here, and how should the forces be classified? Is the force exerted by the soccer ball on the lawn chair an external force? Should we consider the friction between the first foot particle that touches the first soccer ball particle?

The best way to approach ``defining the system'' is to pin down exactly what you're trying to model. Here, specifically, you are trying to determine the velocity of the ball after it hits the lawn chair. In this situation, thinking about the friction between individual foot and soccer ball particles wouldn't help us to figure out the final velocity of the soccer ball. Rather, thinking of the soccer ball and lawn chair as two giant, continuous particles, colliding and exchanging energy would be helpful. In this situation, it would be useful to consider the ``system'' to be the soccer ball and lawn chair only.

The force exerted by the soccer ball on the lawn chair would be an internal force, as this gives us information as to the final velocity of the soccer ball and is a force exchanged between the particles within the system. The force that gravity exerts on the lawn chair, normal force on the person's foot and the force exerted by the foot on the soccer ball are all forces that we would consider ``external''.

Remember - ``internal'' and ``external'' are not magical properties of a specific type of force. These definitions are made by us in the quest of building useful models.
\end{framed}

\paragraph{Conservation of momentum}

Consider a system of two particles with momenta $\vec p_1$ and $\vec p_2$.  Newton's Second Law must hold for each particle:
\begin{equation}
\frac{d\vec p_1}{dt}&=\sum_k \vec F_{1k}\\
\frac{d\vec p_2}{dt}&=\sum_k \vec F_{2k}
\end{equation}
where $F_{ik}$ is the $k$-th force that is acting on particle $i$.  We can sum these two equations together:
\begin{equation}
\frac{d\vec p_1}{dt}+\frac{d\vec p_2}{dt} &= \sum_k \vec F_{1k} + \sum_k \vec F_{2k}
\end{equation}
The quantity on the right is the sum of the forces exerted on particle 1 plus the sum of the forces exerted on particle 2. In other words, it is the sum of all of the forces exerted on all of the particles in the system, which we can write as a single sum. On the left hand side, we have the sum of the two time derivatives of the momenta, which is equal to the time-derivative of the sum of the momenta. We can thus re-write the equation as:
\begin{equation}
\frac{d}{dt}(\vec p_1 + \vec p_2) = \sum \vec F
\end{equation}
where, again, the sum on the right is the sum over all of the forces exerted on the system. Some of those forces are external (e.g. gravity exerted by Earth on the particles), whereas some of the forces are internal (e.g. a contact force between the two particles). We can separate the sum into a sum over all external forces ($\vec F^{ext}$) and a sum over internal forces ($\vec F^{int}$):
\begin{equation}
\sum \vec F = \sum \vec F^{ext} + \sum \vec F^{int}
\end{equation}
The sum of the internal forces is zero:
\begin{equation}
\sum \vec F^{int} = 0
\end{equation}
because for every force that particle 1 exerts on particle 2, there will be an equal and opposite force exerted by particle 2 on particle 1. We thus have:
\begin{equation}
\frac{d}{dt}(\vec p_1 + \vec p_2) = \sum \vec F^{ext}
\end{equation}
Furthermore, if we introduce the ``total momentum of the system'', $\vec P=\vec p_1 + \vec p_2$, as the sum of the momenta of the individual particles, we find:
\begin{equation}
\frac{d\vec P}{dt} &= \sum \vec F^{ext}
\end{equation}
which is the equivalent of Newton's Second Law for a system where, $\vec P$, is the total momentum of the system, and the sum of the forces is only over external forces to the system.

Note that the derivation above easily extends to any number, $N$, of particles, even though we only did it with $N=2$. In general, for the ``ith particle'', with momentum $\vec p_i$, we can write Newton's Second Law:
\begin{equation}
\frac{d\vec p_i}{dt}=\sum_k \vec F_{ik}
\end{equation}
where the sum is over only those forces exerted on particle $i$. Summing the above equation for all $N$ particles in the system:
\begin{equation}
\frac{d}{dt}\sum_i \vec p_i=\sum \vec F^{ext} + \sum \vec F^{int}
\end{equation}
where the sum over internal forces will vanish for the same reason as above. Introducing the total momentum of the system, $\vec P$:
\begin{equation}
\vec P = \sum_i \vec p_i\\
\end{equation}
We can write an equation for the time-derivative of the total momentum of the system:
\begin{equation}
\boxed{\frac{d\vec P}{dt} = \sum \vec F^{ext}}
\end{equation}
where the sum of the forces is the sum over all forces external to the system. Thus, \textbf{if there are no external forces on a system, then the total momentum of that system is conserved} (if the time-derivative of a quantity is zero then that quantity is constant).

We already argued in the previous section that we can make all forces internal if we choose our system to be large enough. If we make the system be the Universe, then there are no forces external to the Universe, and the total momentum of the Universe must be constant:
\begin{equation}
\frac{d\vec P^{Universe}}{dt} &= \sum_{Universe} \vec F^{ext} = 0 \\
\therefore \vec P^{Universe}&=\text{constant}
\end{equation}

In summary, we saw that:

\begin{itemize}
\item If no forces are exerted on a single particle, then the momentum of that particle is constant (conserved).
\item In a system of particles, the total momentum of the system is conserved if there are no external forces on the system.
\item If there are no non-conservative forces exerted on a particle, then that particle's mechanical energy is constant (conserved).
\item In a system of multiple particles, the total mechanical energy of the system will be conserved if there are no non-conservative forces exerted on the system.
\end{itemize}

When we refer to a force being ``exerted on a system'', we mean exerted on one or more of the particles in the system. In particular, the sum of the work done by internal forces is not necessarily zero, so \textbf{energy and momentum are thus conserved under different conditions}.

\begin{framed}
\textbf{Example 4.4}\\
\begin{figure}[!htbp]
\centering
\includegraphics[width=0.8\linewidth]{files/train-d2c8c3f4754121be98bc464317511478.png}
\caption[]{A train with $N$ cars of mass $m$ about to collide with a car of mass $m$ that is at rest on the track.}
\label{fig:momentumandcm:train}
\end{figure}

Consider a train made of $N$ cars of equal mass $m$ that is travelling at constant speed $v$ along a straight piece of track where friction and drag are negligible, as depicted in Figure~\ref{fig:momentumandcm:train}. An empty car of mass $m$ was left at rest on the track in front of the train. The train collides with the empty car which stays attached to the front of the train. What is the speed of the train after the collision? Is the total mechanical energy of the system conserved?

\begin{framed}
\textbf{Solution}\\
When the train collides with the car, it will exert a ``collision'' force on the car, and the car will exert an opposite force on the train. If we consider both of the train and the car as being part of the same system, then those collision forces will be internal, and the momentum of the system (train + car) will be conserved. The train and car both experience external forces from Earth's gravity and the normal force from the train tracks. However, those two sets of forces cancel each other out, since neither the train nor the car have any acceleration in the vertical direction (the sum of the forces on each object has no net vertical component). Thus, there are no net external forces on the car+train system, and the total momentum of the system is conserved through the collision.

We can model this system in one dimension (along the track), defining our $x$ axis. We choose the ground as a frame of reference, the positive direction parallel to the initial velocity of the train, and the origin to be located where the car initially starts. Before the collision, the $x$ component of the momenta of the train (mass $Nm$) and car (mass $m$) are:
\begin{equation}
p_{train}&=Nmv\\
p_{car}&=0
\end{equation}
After the collision, the car is attached to the train (and thus has the same speed, $v'$), so the momenta of the train and car after the collision are:
\begin{equation}
p'_{train}&=Nmv'\\
p'_{car}&=mv'
\end{equation}
where the primes $'$ denote quantities after the collision. Applying conservation of momentum to the system, the total momentum before and after the collision must be equal:
\begin{equation}
p_{train}+p_{car}&=p'_{train}+p'_{car}\\
\therefore Nmv &= Nmv' +mv'\\
\therefore v' &=\frac{N}{N+1}v
\end{equation}
and the speed of the train with the additional car attached is reduced by a factor $N/(N+1)$ compared to what it was before the collision.

We can check to see if the mechanical energy of the system is conserved, since we know the speeds of the train and car before and after the collision. Since all of the motion is horizontal, gravity and the normal force do no work on either the train or car, so their mechanical energy can be taken as their kinetic energy (their gravitational potential energy does not change after the collision). The total mechanical energy of the system, $E$, before the collision is the kinetic energy of the train:
\begin{equation}
E= \frac{1}{2}Nmv^2
\end{equation}
The total mechanical energy of the system, $E'$, after the collision is:
\begin{equation}
E' &= \frac{1}{2}Nmv'^2 + \frac{1}{2}mv'^2 = \frac{1}{2}(N+1)mv'^2 \\
&=\frac{1}{2}(N+1)m \left( \frac{N}{N+1}v \right)^2\\
&=\frac{1}{2}m\frac{N^2}{N+1}v^2
\end{equation}
and we see that $E'<E$, and thus that the total mechanical energy of the system is not conserved (it is reduced after the collision).

**Discussion: **We could have solved this problem by carefully modelling the force exerted by the car on the train during the collision, which would have allowed us to find the speed of the train after the collision using its acceleration. This would have required a detailed model for that force, which we do not have. However, by realizing that the train and car could be considered as a system with no net external forces exert on it, we were able to easily find the speed of the train after the collision using conservation of momentum.

We also found that mechanical energy was not conserved. This makes physical sense because, for the car to remain attached to the train, there presumably had to be some significant forces in play that ``crushed'' the car into the train. Some of the initial kinetic energy of the train was used to deform the train and the car during the collision. We can also think of deforming a material as giving it energy. Sometimes that energy is recoverable (e.g. compressing a spring), sometimes, it is not (e.g. crushing a car).

If the car and train were equipped with large springs to absorb the energy of the impact, the collision could have conserved mechanical energy, as the springs compress and then expand back. The speed of the car and train would then be different after the collision in this case (see Example~4.7). It is a feature of collisions where the two bodies remain attached to each other that mechanical energy is not conserved.
\end{framed}
\end{framed}

\subsubsection{Collisions}

In this section we go through a few examples of applying conservation of momentum to model collisions. Collisions can loosely be defined as events where the momenta of individual particles in a system are different before and after the event.

We distinguish between two types of collisions: \textbf{elastic} and \textbf{inelastic} collisions. Elastic collisions are those for which the total mechanical energy of the system is conserved during the collision (i.e. it is the same before and after the collision). Inelastic collisions are those for which the total mechanical energy of the system is not conserved. In either case, to model the system, one chooses to define the system such that there are no external forces on the system so that total momentum is conserved.

\paragraph{Inelastic collisions}

In this section, we give a few examples of modelling inelastic collisions. Inelastic collisions are usually easier to handle mathematically, because one only needs to consider conservation of momentum and does not use conservation of energy (which usually involves equations that are quadratic in the speeds because of the kinetic energy term).

\begin{framed}
\textbf{Example 4.5}\\
\begin{figure}[!htbp]
\centering
\includegraphics[width=0.4\linewidth]{files/skaters-3e4f05ca0612b57a349cdcab710b732c.png}
\caption[]{One skater pushing another on a frictionless horizontal surface.}
\label{fig:momentumandcm:skaters}
\end{figure}

You (mass $m_s$) and your friend (mass $m_f$) face each other on ice skates on an ice surface that is slippery enough that friction can be considered negligible, as shown in Figure~\ref{fig:momentumandcm:skaters}. You shove your friend away from you so that he moves with velocity $\vec v_f$ away from you (the velocity is measured relative to the ice). Is the collision elastic? What is your speed relative to the ice after you shoved your friend?

\begin{framed}
\textbf{Solution}\\
We can consider the system as being comprised of you and your friend. There are no net external forces on the system (gravity and normal forces cancel each other), so the momentum of the system will be conserved.

The mechanical energy will not be conserved. You had to use chemical potential energy stored in your muscles to shove your friend. Thus, external energy (i.e. not mechanical energy from you or your friend) was injected into the system, and we should expect the total mechanical energy to be larger after the collision.

Before the collision, both you and your friend have zero speed, and thus zero kinetic energy and zero momentum. After the collision, your friend has a velocity $\vec v_f$. We can use conservation of total momentum, $\vec P$, to determine your velocity, $\vec v_s$, after the collision.
\begin{equation}
\vec P &=\vec P'\\
0 &= m_s\vec v_s + m_f\vec v_f\\
\therefore \vec v_s &= -\frac{m_f}{m_s}\vec v_f
\end{equation}
where primes ($'$) denote a quantity after the collision. We find that your velocity is in the opposite direction from that of your friend. Before the collision, the mechanical energy, $E$, of the system is zero (we can ignore gravitational potential energy, since everything is in the horizontal plane). After the collision, the mechanical energy, $E'$, is:
\begin{equation}
E' = \frac{1}{2}m_sv_s^2+\frac{1}{2}m_fv_f^2
\end{equation}
which is clearly bigger than the mechanical energy before the collision (i.e. 0), as we suspected it would be.

\textbf{Discussion:} We find that you recoil in the opposite direction, which makes sense. If you push your friend in one direction, Newton's Third Law says that your friend pushes you in the opposite direction. Your speed furthermore depends on the ratio of your friend's mass to yours. This also makes sense, because if you both feel the same force, the person with the smallest mass will have the highest speed; if your mass is higher than your friend's, then your speed after the collision will be smaller than your friend's.

We also saw that mechanical energy was not conserved. In terms of energy, we can explain this by saying that you burned up chemical potential energy stored in your muscles in order to shove your friend. Because we included both you and your friend in the system, the shove was an internal force and momentum is conserved. Of course, if we had considered only you as the system, then your momentum would not have been conserved during the collision.

The type of collision that we described here is also sometimes called an ``explosion''. You can imagine all of the parts that make up a bomb as small particles. When the bomb explodes, chemical potential energy is converted into the kinetic energy of the bomb fragments. If you consider all of the particles/fragments of the bomb as a system, then the total momentum of all of the bomb fragments is conserved (and equal to zero if the bomb was initially at rest). Again, mechanical energy would not be conserved (and would increase) as the chemical potential energy is converted into mechanical energy.
\end{framed}
\end{framed}

\begin{framed}
\textbf{Example 4.6}\\
A proton of mass $m_p$ and initial velocity $\vec v_p$ collides inelastically with a nucleus of mass $m_N$ at rest, as shown in Figure~\ref{fig:momentumandcm:protonnucleus}. A coordinate system is set up as shown, such that the initial velocity of the proton is in the $x$ direction. After the collision, the proton's speed is measured to be $v'_p$ and its velocity vector is found to make an angle $\theta$ with the $x$ axis as shown. What is the velocity vector of the nucleus after the collision? Assume that the collision takes place in vacuum.

\begin{figure}[!htbp]
\centering
\includegraphics[width=0.7\linewidth]{files/protonnucleus-ae0194616d7519d54c8de46b5351c92c.png}
\caption[]{A proton of mass $m_p$ colliding inelastically with a nucleus of mass $m_N$.}
\label{fig:momentumandcm:protonnucleus}
\end{figure}

\begin{framed}
\textbf{Solution}\\
As a system, we consider the proton and the nucleus together, so that the total momentum of the system is conserved during the collision, as no other external forces are exerted on the two particles (since they are in vacuum). Because momentum is a vector, each component of the total momentum, $\vec P$, is conserved during the collision:
\begin{equation}
\vec P &= \vec P'\\
\therefore P_x &= P'_x\\
\therefore P_y &= P'_y
\end{equation}
where, as usual, primes ($'$) denote quantities after the collision. After the collision, both particles will have velocity vectors that have $x$ and $y$ components. Let the velocity vector of the nucleus after the collision be $\vec v'_N$ and let $\phi$ be the angle that it makes with the $x$ axis, as shown in Figure~\ref{fig:momentumandcm:protonnucleus}.

We can start by considering the conservation of the $x$ component of the total momentum. The initial and final momenta in the $x$ direction are given by:
\begin{equation}
P_x &= m_p v_p\\
P'_x &= m_p v'_p\cos\theta + m_N v'_N\cos\phi\\
\therefore m_p v_p &= m_p v'_p\cos\theta + m_N v'_N\cos\phi
\end{equation}
which gives us a first equation to determine the final velocity of the nucleus.

The $y$ component of the total momentum before the collision is zero since we chose the coordinate system such that the initial velocity of the proton is in the $x$ direction. The initial and final momenta in the $y$ direction are given by:
\begin{equation}
P_y &= 0\\
P'_y &= m_p v'_p\sin\theta - m_N v'_N\sin\phi\\
\therefore m_p v'_p\sin\theta &= m_N v'_N\sin\phi
\end{equation}
which gives us a second equation to solve for the velocity of the nucleus. With the two equations from momentum conservation, we can solve for the magnitude and direction of the velocity of the nucleus. From the $y$ component of momentum conservation, we can find an expression for the speed of the nucleus:
\begin{equation}
m_p v'_p\sin\theta &= m_N v'_N\sin\phi\\
\therefore v'_N &= \frac{m_p}{m_N}v'_p\sin\theta \frac{1}{\sin\phi}
\end{equation}
which we can substitute into the $x$ equation for momentum conservation to solve for the angle $\phi$:
\begin{equation}
m_p v_p &= m_p v'_p\cos\theta + m_N v'_N\cos\phi\\
m_p v_p &= m_p v'_p\cos\theta + m_N\frac{m_p}{m_N}v'_p\sin\theta \frac{\cos\phi}{\sin\phi} \\
v_p &= v'_p\cos\theta + v'_p\sin\theta \frac{1}{\tan\phi}\\
\therefore \tan\phi &=  \frac{v'_p\sin\theta}{v_p-v'_p\cos\theta}
\end{equation}
If we were given numbers for the initial and final speed of the proton, as well as the angle $\theta$, we would be able to find a value for the angle $\phi$, which we could then use to determine the final speed of the nucleus:
\begin{equation}
 v'_N &= \frac{m_p}{m_N}v'_p\sin\theta \frac{1}{\sin\phi}
\end{equation}
\textbf{Discussion:} By using the conservation of momentum equation and writing out the $x$ and $y$ components, we were able to find two equations to determine the magnitude and direction of the nucleus' velocity after the collision. In the limit where $m_N >> m_p$, the final speed of the nucleus would be very small (close to zero).
\end{framed}
\end{framed}

\paragraph{Elastic collisions}

In this section, we give a few examples of modelling elastic collisions. Even though it is mechanical energy that is conserved in an elastic collision, one can almost always simplify this to only kinetic energy being conserved. If a collision takes place in a well localized position in space (i.e. before and after the collision are the same point in space), then the potential energies of the objects involved will not change, thus any change in their mechanical energy is due to a change in kinetic energy.

\begin{framed}
\textbf{Example 4.7}\\
\begin{figure}[!htbp]
\centering
\includegraphics[width=0.6\linewidth]{files/1delastic-53943ca2dce994dcc3828939b01f3217.png}
\caption[]{Two blocks about to collide elastically.}
\label{fig:momentumandcm:1delastic}
\end{figure}

A block of mass $M$ moves with velocity $\vec v_M$ in the $x$ direction, as shown in Figure~\ref{fig:momentumandcm:1delastic}. A block of mass $m$ is moving with velocity $\vec v_m$ also in the $x$ direction and collides elastically with block $M$. Both blocks slide with no friction on the horizontal surface. What are the velocities of the two blocks after the collision?

\begin{framed}
\textbf{Solution}\\
Because this is an elastic collision, both the total momentum and total mechanical energy are conserved. Equating the total momentum before and after the collision, and considering only the $x$ component gives the following equation:
\begin{equation}
\vec P &=\vec P'\\
Mv_M+mv_m&=Mv'_M+mv'_m
\end{equation}
where the primes ($'$) correspond to the quantities after the collision. Note that, in principle, the $x$ components of the velocities ($v_M$, $v'_M$, $v_m$, $v'_m$) could be negative numbers if the corresponding block is moving in the negative $x$ direction.

For the mechanical energy of the two blocks, we only need to consider their kinetic energy since their gravitational potential energies are the same before and after the collision on the horizontal surface. The total mechanical energy of the system, before and after the collision is given by:
\begin{equation}
E &=E'\\
\frac{1}{2}Mv_M^2+\frac{1}{2}mv_m^2&=\frac{1}{2}Mv'^2_M+\frac{1}{2}mv'^2_m\\
\therefore Mv_M^2+mv_m^2&=Mv'^2_M+mv'^2_m
\end{equation}
where we cancelled the factor of one half in the last line. This gives two equations (conservation of energy and momentum) and two unknowns (the two speeds after the collision). This is not a linear system of equations, because the equation from conservation of energy is quadratic in the speeds.

The following method allows many models for elastic collisions between two particles to be solved easily by converting the quadratic equation from energy conservation into an equation that is linear in the speeds. First, write both equations so that the quantities related to each particle are on opposite sides of the equation. For momentum, this gives:
\begin{equation}
\label{eq:momentumandcm:exptemp}
Mv_M+mv_m&=Mv'_M+mv'_m\nonumber\\
\therefore M(v_M-v'_M) &= m(v'm-v_m)
\end{equation}
For conservation of energy, this gives:
\begin{equation}
\label{eq:momentumandcm:exptemp2}
Mv_M^2+mv_m^2&=Mv'^2_M+mv'^2_m\nonumber\\
\therefore  M(v_M^2-v'^2_M)&= M(v'^2_m-v^2_m)
\end{equation}
which we can re-write as:
\begin{equation}
M(v_M^2-v'^2_M)&= M(v'^2_m-v^2_m)\\
M(v_M-v'_M)(v_M+v'_M)&= M(v'_m-v_m)(v'_m+v_m)
\end{equation}
We can then divide Equation (\ref{eq:momentumandcm:exptemp2}) by Equation (\ref{eq:momentumandcm:exptemp}):
\begin{equation}
\frac{M(v_M-v'_M)(v_M+v'_M)}{M(v_M-v'_M)}&= \frac{M(v'_m-v_m)(v'_m+v_m)}{m(v'm-v_m)}\\
\therefore v_M+v'_M&=v'_m+v_m
\end{equation}
which gives us an equation that is much easier to work with, since it is linear in the speeds. If we re-arrange this last equation back so that quantities before and after the collision are on different sides of the equality:
\begin{equation}
\boxed{v_M-v_m = - (v'_M-v'_m)}
\end{equation}
we can see that the relative speed between $M$ and $m$ is the same before and after the collision. That is, if block $M$ ``saw'' block $m$ approaching with a speed of $3 {\rm m/s}$ before the collision, it would ``see'' block $m$ moving \textit{away} with speed $3 {\rm m/s}$ after the collision, regardless of the actual directions and velocities of the block, if the collision was elastic.

By using this equation with the original conservation of momentum equation, we now have two equations and two unknowns that are easy to solve:
\begin{equation}
v_M-v_m &= - (v'_M-v'_m)\\
Mv_M+mv_m&=Mv'_M+mv'_m
\end{equation}
Solving for $v'_m$ in both equations gives:
\begin{equation}
v_M-v_m &= - (v'_M-v'_m)\\
\therefore v'_m &= v_M+v'_M-v_m\\
Mv_M+mv_m&=Mv'_M+mv'_m\\
\therefore v'_m&=\frac{1}{m}(Mv_M+mv_m-Mv'_M)
\end{equation}
Equating the two expressions for $v'_m$ allows us to solve for $v'_M$:
\begin{equation}
\frac{1}{m}(Mv_M+mv_m-Mv'_M)&=v_M+v'_M-v_m\\
Mv_M+mv_m-Mv'_M&=mv_M+mv'_M-mv_m\\
(M-m)v_M+2mv_m&=(M+m)v'_M\\
\therefore v'_M&=\frac{M-m}{M+m}v_M+\frac{2m}{M+m}v_m
\end{equation}
One can easily solve for the other speed, $v'_m$:
\begin{equation}
\therefore v'_m &= \frac{m-M}{M+m}v_m+\frac{2M}{M+m}v_M
\end{equation}
And writing these together:
\begin{equation}
v'_M&=\frac{M-m}{M+m}v_M+\frac{2m}{M+m}v_m\\
v'_m &= \frac{m-M}{M+m}v_m+\frac{2M}{M+m}v_M
\end{equation}
\textbf{Discussion:} The formulas that we obtained above are valid for any one dimensional elastic collision.
\end{framed}
\end{framed}

\begin{framed}
\textbf{Checkpoint}\\
Two trains of equal masses collide elastically on a track. If train A had a speed $v$ and train B was at rest, what are the speeds of the trains after the collision?

\begin{enumerate}
\item Both trains A and B travel away from each other with speeds $\frac{1}{2}v$.
\item Train A will be at rest and train B will move away with a speed $v$.
\item Both trains A and B will stick together and move at a speed of $v$.
\item Train B will be at rest and train A will move away at a speed of $v$.
\end{enumerate}

\begin{framed}
\textbf{Answer}\\
\begin{enumerate}[resume]
\item
\end{enumerate}
\end{framed}
\end{framed}

\begin{framed}
\textbf{Example 4.8}\\
\begin{figure}[!htbp]
\centering
\includegraphics[width=0.7\linewidth]{files/protonproton-99e03bb3c8bd038f99f69d8e1d9c380b.png}
\caption[]{A proton elastically collides with a proton at rest.}
\label{fig:momentumandcm:protonproton}
\end{figure}

A proton of mass $m$ and initial velocity $\vec v_1$ collides elastically with a second proton that is at rest. After the collision, the two protons have velocities $\vec v'_1$ and $\vec v'_2$, as shown in Figure~\ref{fig:momentumandcm:protonproton}. Show that the velocity vectors of the two protons are perpendicular after the collision.

\begin{framed}
\textbf{Solution}\\
This example highlights a particular feature of elastic collisions when the two objects have the same mass and one of the objects is initially at rest. The conservation of momentum for the system comprised of the two protons can be written as:
\begin{equation}
m\vec v_1 &= m\vec v'_1 + m\vec v'_2\\
\vec v_1 &= \vec v'_1 + \vec v'_2
\end{equation}
where the left hand side corresponds to the initial total momentum and the right hand side to the total momentum after the collision. In the second line, we cancelled out the mass, and obtained a vector relation between the velocity vectors. We can graphically illustrate the vector relation as in Figure~\ref{fig:momentumandcm:vsum} which shows the triangle that is formed by adding the two outgoing velocity vectors to obtain the initial velocity vector.

\begin{figure}[!htbp]
\centering
\includegraphics[width=0.4\linewidth]{files/vsum-cc029478b75de1cb8eee0040eb74c0c5.png}
\caption[]{Graphical illustration of the relation between the initial and final velocity vectors as a vector sum.}
\label{fig:momentumandcm:vsum}
\end{figure}

Conservation of kinetic energy for the collision can be written as:
\begin{equation}
\frac{1}{2}mv_1^2 &= \frac{1}{2}mv'^2_1+\frac{1}{2}mv'^2_2\\
v_1^2 &= v'^2_1+ v'^2_2
\end{equation}
where the left hand side corresponds to the initial kinetic energy and the right hand side to the final kinetic energy. We cancelled the mass and factor of one half in the second line. This last equation gives a relation between the magnitudes of the velocity vectors. By comparing the equation above to Pythagoras' theorem, and by inspecting the triangle in Figure~\ref{fig:momentumandcm:vsum}, it is clear that the triangle must be a right angle triangle, and thus that $\vec v'_1$ and $\vec v'_2$ must be perpendicular.
\end{framed}
\end{framed}

\paragraph{Frames of reference}

\begin{framed}
\textbf{Review}\\
Before proceeding, you may wish to review Sections \href{\#sec:describingmotionin1D:relativemotion}{} and  \href{\#sec:desribingmotioninnd:relativemotion}{} on expressing velocities in different frames of reference.
\end{framed}

Because the momentum of a particle is defined using the velocity of the particle, its value depends on the reference frame in which we chose to measure that velocity. In some cases, it is useful to apply momentum conservation in a frame of reference where the total momentum of the system is zero. For example, consider two particles of mass $m_1$ and $m_2$, moving towards each other with velocities $\vec v_1$ and $\vec v_2$, respectively, as measured in a frame of reference $S$, as illustrated in Figure~\ref{fig:momentumandcm:2particles}.

\begin{figure}[!htbp]
\centering
\includegraphics[width=0.3\linewidth]{files/2particles-60b4d45d714329dfe64e722015685ec8.png}
\caption[]{Two particles moving towards each other.}
\label{fig:momentumandcm:2particles}
\end{figure}

In the frame of reference $S$, the total momentum, $\vec P$, of the two particles can be written:
\begin{equation}
\vec P = m_1\vec v_1 + m_2\vec v_2
\end{equation}
Consider a frame of reference, $S'$, that is moving with velocity, $\vec v_{CM}$, relative to the frame of reference $S$. In that frame of reference, the velocities of the two particles are different and given by:
\begin{equation}
\vec v'_1&=\vec v_1- \vec v_{CM}\\
\vec v'_2&=\vec v_2- \vec v_{CM}
\end{equation}
The total momentum, $\vec P'$, in the frame of reference $S'$ is then given by\footnote{Note that we are using primes ($'$) to denote quantities in a different reference frame, not after a collision.}:
\begin{equation}
\vec P' &= m_1\vec v'_1 + m_2 \vec v'_2\\
&=m_1(\vec v_1- \vec v_{CM})+m_2(\vec v_2- \vec v_{CM})\\
&= m_1\vec v_1 + m_2\vec v_2 - (m_1+m_2) \vec v_{CM}
\end{equation}
We can choose the velocity of the frame $S'$, $\vec v_{CM}$, such that the total momentum in that frame of reference is zero:
\begin{equation}
\vec P' &= 0\\
m_1\vec v_1 + m_2\vec v_2 - (m_1+m_2) \vec v_{CM} &=0\\
\therefore \vec v_{CM} &= \frac{m_1\vec v_1 + m_2\vec v_2 }{m_1+m_2}
\end{equation}
This ``special'' frame of reference, in which the total momentum of the system is zero, is called the ``centre of mass frame of reference''. The velocity of centre of mass frame of reference can easily be obtained if there are $N$ particles involved instead of two:
\begin{equation}
\boxed{\therefore \vec v_{CM} = \frac{m_1\vec v_1 + m_2\vec v_2 + m_3 \vec v_3 + \dots }{m_1+m_2+m_3+\dots}=\frac{\sum m_i\vec v_i}{\sum m_i}}
\end{equation}
Again, you should note that because the above equation is a vector equation, it represents one equation per component of the vectors. For example, the $x$ component of the velocity of the centre of mass frame of reference is given by:
\begin{equation}
\therefore  v_{CMx} = \frac{m_1 v_{1x} + m_2v_{2x} + m_3 v_{3x} + \dots }{m_1+m_2+m_3+\dots}=\frac{\sum m_iv_{ix}}{\sum m_i}
\end{equation}

\begin{framed}
\textbf{Example 4.9}\\
\begin{figure}[!htbp]
\centering
\includegraphics[width=0.4\linewidth]{files/labframe-25598d4a463ebbbaba2d721c6fc7d0bd.png}
\caption[]{One block approaching another identical block at rest, as seen in the lab frame of reference.}
\label{fig:momentumandcm:labframe}
\end{figure}

In the frame of reference of a lab, a block of mass $m$ has a velocity $\vec v_1$ directed along the positive $x$ axis and is approaching a second block of mass $m$ that is at rest ($\vec v_2=0$), as shown in Figure~\ref{fig:momentumandcm:labframe}. What is the velocity of the centre of mass frame? What is the velocity of each block in the centre of mass frame? Verify that the total momentum is zero in the centre of mass frame.

\begin{framed}
\textbf{Solution}\\
Since this is a one dimensional situation, we only need to evaluate the $x$ component of the velocity of the centre of mass:
\begin{equation}
\vec v_{CM} &= \frac{m_1\vec v_1 + m_2\vec v_2 }{m_1+m_2}\\
\therefore v_{CMx} &= \frac{m_1 v_{1x} + m_2 v_{2x}}{m_1+m_2}\\
&=\frac{mv_1 + m(0) }{m+m}\\
&=\frac{1}{2}v_1
\end{equation}
The centre of mass frame of reference is thus also moving along the positive direction of the $x$ axis, but with a speed that is half of that of the moving block. In the centre of mass frame of reference, it appears that the block on the left is slower than in the lab frame and that the block on the right is moving in the negative $x$ direction. The velocities of the two blocks in the centre of mass frame of reference are given by:
\begin{equation}
v'_1&=v_1-v_{CMx}=\frac{1}{2}v_1\\
v'_2&=(0)-v_{CMx}=-\frac{1}{2}v_1
\end{equation}
Thus, in the reference frame of the centre of mass, the two block are approaching each other with the same speed ($v_1/2$), which is only the case because the two blocks have the same mass. The blocks, as viewed in the centre of mass frame of reference, are shown in Figure~\ref{fig:momentumandcm:cmframe}.

\begin{figure}[!htbp]
\centering
\includegraphics[width=0.4\linewidth]{files/cmframe-e97482a4cf5f187af2d422a8562db512.png}
\caption[]{In the centre of mass frame of reference, the block approach each other with the same speed, because they have the same mass.}
\label{fig:momentumandcm:cmframe}
\end{figure}

Clearly, the total momentum is zero in the centre of mass frame of reference:
\begin{equation}
\vec P' = m\vec v'_1+ m\vec v'_2 = m \left(\frac{1}{2}\vec v_1 - \frac{1}{2}\vec v_1\right) = 0
\end{equation}

\textbf{Discussion:} As we have seen, in the centre of mass frame of reference the total momentum is zero. If there are only two particles, and they have the same mass, then, in the centre of mass frame of reference, they both have the same speed and move either towards or away from each other.
\end{framed}
\end{framed}

\subsubsection{The centre of mass}

In this section, we show how to generalize Newton's Second Law so that it may describe the motion of an object that is not a point particle. Any object can be described as being made up of point particles; for example, those particles could be the atoms that make up regular matter. We can thus use the same terminology as in the previous sections to describe a complicated object as a ``system'' comprised of many point particles, themselves described by Newton's Second Law. A system could be a rigid object where the point particles cannot move relative to each other, such as atoms in a solid\footnote{In reality, even atoms in a solid can move relative to each other, but they do not move by large amounts compared to the object.}. Or, the system could be a gas, made of many atoms moving around, or it could be a combination of many solid objects moving around.

In the previous section, we saw how the total momentum and the total mechanical energy of the system could be used to describe the system as a whole. In this section, we will define the centre of mass which will allow us to describe the position of the system as a whole.

Consider a system comprised of $N$ point particles. Each point particle $i$, of mass $m_i$, can be described by a position vector, $\vec r_i$, a velocity vector, $\vec v_i$, and an acceleration vector, $\vec a_i$, relative to some coordinate system in an inertial frame of reference. Newton's Second Law can be applied to any one of the particles in the system:
\begin{equation}
\sum_k \vec F_{ik} = m_i \vec a_i
\end{equation}
where $\vec F_{ik}$ is the k-th force exerted on particle $i$. We can write Newton's Second Law once for each of the $N$ particles, and we can sum those $N$ equations together:
\begin{equation}
\sum_k \vec F_{1k} + \sum_k \vec F_{2k} + \sum_k \vec F_{3k} +\dots &= m_1\vec a_1 + m_2 \vec a_2 + m_3 \vec a_3 + \dots\\
\sum \vec F = \sum_i m_i \vec a_i
\end{equation}
where the sum on the left is the sum of all of the forces exerted on all of the particles in the system\footnote{Again, we are summing together forces that are acting on \textbf{different} particles.} and the sum over $i$ on the right is over all of the $N$ particles in the system. As we have already seen, the sum of all of the forces exerted on the system can be divided into separate sums over external and internal forces:
\begin{equation}
\sum \vec F = \sum \vec F^{ext} + \sum \vec F^{int}
\end{equation}
and the sum over the internal forces is zero\footnote{Recall, the internal forces are those forces that particles in the system are exerting on one another. Because of Newton's Third Law, these will sum to zero.}. We can thus write that the sum of the external forces exerted on the system is given by:
\begin{equation}
\label{eqn:momentumandcm:cmtemp1}
\sum \vec F^{ext}&= \sum_i m_i \vec a_i
\end{equation}
We would like this equation to resemble Newton's Second Law, but for the system as a whole. Suppose that the system has a total mass, $M$:
\begin{equation}
M = m_1 + m_2 + m_3 +\dots = \sum_i m_i
\end{equation}
we would like to have an equation of the form:
\begin{equation}
\label{eqn:momentumandcm:cmtemp2}
\sum \vec F^{ext}&=M\vec a_{CM}
\end{equation}
to describe the system as a whole. However, it is not (yet) clear what is accelerating with acceleration, $\vec a_{CM}$, since the particles in the system could all be moving in different directions. Suppose that there is a point in the system, whose position is given by the vector, $\vec r_{CM}$, in such a way that the acceleration above is the second time-derivative of that position vector:
\begin{equation}
\vec a_{CM} = \frac{d^2 }{dt^2}\vec r_{CM}
\end{equation}
We can compare Equations (\ref{eqn:momentumandcm:cmtemp1}) and (\ref{eqn:momentumandcm:cmtemp2}) to determine what the position vector $\vec r_{CM}$ corresponds to:
\begin{equation}
\sum \vec F^{ext}&= \sum_i m_i \vec a_i = \sum_i m_i \frac{d^2 }{dt^2}\vec r_i \\
\sum \vec F^{ext}&=M\vec a_{CM} = M \frac{d^2 }{dt^2}\vec r_{CM}\\
\therefore M \frac{d^2 }{dt^2}\vec r_{CM}&= \sum_i m_i \frac{d^2 }{dt^2}\vec r_i
\end{equation}
Re-arranging, and noting that the masses are constant in time, and so they can be factored into the derivatives:
\begin{equation}
\frac{d^2 }{dt^2}\vec r_{CM} &= \frac{1}{M}\sum_i m_i \frac{d^2 }{dt^2}\vec r_i\\
\frac{d^2 }{dt^2}\vec r_{CM} &= \frac{d^2 }{dt^2}\left(\frac{1}{M}\sum_i m_i\vec r_i \right)\\
\therefore \vec r_{CM} &=\frac{1}{M}\sum_i m_i\vec r_i
\end{equation}
where in the last line we set the quantities that have the same time derivative equal to each other\footnote{Technically, the terms in the derivatives are only equal to within two constants of integration, $\vec r_{CM} =\frac{1}{M}\sum_i m_i\vec r_i + at + b$, which we can set to zero.}. $\vec r_{CM}$ is the vector that describes the position of the ``centre of mass'' (CM). The position of the centre of mass is described by Newton's Second Law applied to the system as a whole:
\begin{equation}
\boxed{\sum \vec F^{ext}=M\vec a_{CM}}
\end{equation}
where $M$ is the total mass of the system, and the sum of the forces is the sum over only external forces on the system.

Although we have formally derived Newton's Second Law for a system of particles, we really have been using this result throughout the text. For example, when we modelled a block sliding down an incline, we never worried that the block was made of many atoms all interacting with each other and the surroundings. Instead, we only considered the external forces on the block, namely, the normal force from the incline, any frictional forces, and the total weight of the object (the force exerted by gravity). Technically, the force of gravity is not exerted on the block as a whole, but on each of the atoms. However, when we sum the force of gravity exerted on each atom:
\begin{equation}
m_1\vec g+ m_2 \vec g + m_3\vec g + \dots = (m_1+m_2+m_3+\dots)\vec g = M\vec g
\end{equation}
we find that it can be modelled by considering the block as a single particle of mass $M$ upon which gravity is exerted. The centre of mass is sometimes described as the ``centre of gravity'', because it \textbf{corresponds to the location where we can model the total force of gravity, $M\vec g$, as being exerted}. When we applied Newton's Second Law to the block, we then described the motion of the block as a whole (and not the motion of the individual atoms). Specifically, we modelled the motion of the centre of mass of the block.

The position of the centre of mass is a vector equation that is true for each coordinate:
\begin{equation}
\vec r_{CM} &=\frac{1}{M}\sum_i m_i\vec r_i\nonumber\\
\therefore x_{CM} &= \frac{1}{M}\sum_i m_i x_i\nonumber\\
\therefore y_{CM} &= \frac{1}{M}\sum_i m_i y_i\nonumber\\
\therefore z_{CM} &= \frac{1}{M}\sum_i m_i z_i
\end{equation}
The centre of mass is that \textbf{position in a system that is described by Newton's Second Law when it is applied to the system as a whole}. The centre of mass can be thought of as an average position for the system (it is the average of the positions of the particles in the system, weighted by their mass). By describing the position of the centre of mass, we are not worried about the detailed positions of the all of the particles in the system, but rather only the average position of the system as a whole. In other words, this is equivalent to viewing the whole system as a single particle of mass $M$ located at the position of the centre of mass.

Consider, for example, a person throwing a dumbbell that is made from two spherical masses connected by a rod, as illustrated in Figure~\ref{fig:momentumandcm:cmparabola}. The dumbbell will rotate in a complex manner as it moves through the air. However, the centre of mass of the dumbbell will travel along a parabolic trajectory (projectile motion), because the only external force exerted on the dumbbell during its trajectory is gravity.

\begin{figure}[!htbp]
\centering
\includegraphics[width=0.6\linewidth]{files/cmparabola-51543e3284ba3fe18b173b830063c9ef.png}
\caption[]{The motion of the centre of mass of a dumbbell is described by Newton's Second Law, even if the motion of the rotating dumbbell is more complex.}
\label{fig:momentumandcm:cmparabola}
\end{figure}

If we take the derivative with respect to time of the centre of mass position, we obtain the velocity of the centre of mass, and its components, which allow us to describe how the system is moving as a whole:
\begin{equation}
\vec v_{CM} &= \frac{d}{dt}\vec r_{CM} = \frac{1}{M}\sum_i m_i\frac{d}{dt}\vec r_i=  \frac{1}{M}\sum_i m_i\vec v_i\nonumber\\
\therefore v_{CMx} &= \frac{1}{M}\sum_i m_i v_{ix}\nonumber\\
\therefore v_{CMy} &= \frac{1}{M}\sum_i m_i v_{iy}\nonumber\\
\therefore v_{CMz} &= \frac{1}{M}\sum_i m_i v_{iz}
\end{equation}
Note that this is the same velocity that we found earlier for the velocity of the centre of mass frame of reference. In the centre of mass frame of reference, the total momentum of the system is zero. This makes sense, because the centre of mass represents the average position of the system; if we move ``with the system'', then the system appears to have zero momentum.

We can also define the total momentum of the system, $\vec P$, in terms of the total mass, $M$, of the system and the velocity of the centre of mass:
\begin{equation}
\vec P &= \sum m_i \vec v_i = \frac{M}{M}\sum m_i \vec v_i\\
&=M\vec v_{CM}
\end{equation}
which we can also use in Newton's Second Law:
\begin{equation}
\frac{d}{dt}\vec P = \sum \vec F^{ext}
\end{equation}
and again, we see that the total momentum of the system is conserved if the net external force on the system is zero. In other words, the centre of mass of the system will move with constant velocity when momentum is conserved.

Finally, we can also define the acceleration of the centre of mass by taking the time derivative of the velocity:
\begin{equation}
\vec a_{CM} &= \frac{d}{dt}\vec v_{CM} = \frac{1}{M}\sum_i m_i\frac{d}{dt}\vec v_i=  \frac{1}{M}\sum_i m_i\vec a_i\nonumber\\
\therefore a_{CMx} &= \frac{1}{M}\sum_i m_i a_{ix}\nonumber\\
\therefore a_{CMy} &= \frac{1}{M}\sum_i m_i a_{iy}\nonumber\\
\therefore a_{CMz} &= \frac{1}{M}\sum_i m_i a_{iz}
\end{equation}

\begin{framed}
\textbf{Example 4.10}\\
\begin{figure}[!htbp]
\centering
\includegraphics[width=0.7\linewidth]{files/sunearthmars-db16062d454b433685e974e808bf801b.png}
\caption[]{A syzygy between the Sun, Earth, and Mars.}
\label{fig:momentumandcm:sunearthmars}
\end{figure}

In astronomy, a syzygy is defined as the event in which three bodies are all lined up along a straight line. For example, a syzygy occurs when the Sun (mass $M_S=2.00e30 {\rm kg}$), Earth (mass $M_E=5.97e24 {\rm kg}$), and Mars (mass $M_M=6.39e23 {\rm kg}$) are all lined up, as in Figure~\ref{fig:momentumandcm:sunearthmars}. How far from the centre of the Sun is the centre of mass of the Sun, Earth, Mars system during a syzygy?

\begin{framed}
\textbf{Solution}\\
Since this is a one-dimensional problem, we can define an $x$ axis that is co-linear with the three bodies, and find only the $x$ coordinate of the position of the centre of mass. We are free to choose the origin of the coordinate system, so we choose the origin to be located at the centre of the Sun. This way, the position of the centre of mass along the $x$ axis will directly correspond to its distance from the centre of the Sun.

The Sun, Earth, and Mars are not point particles. However, because they are spherically symmetric, their centres of mass correspond to their geometric centres. We can thus model them as point particles with the mass of the body located at the corresponding geometric centre. If $r_E=1.50e11 {\rm m}$ ($r_M=2.28e11 {\rm m}$) is the distance from the centre of the Earth (Mars) to the centre of the Sun, then the position of the centre of mass is given by:
\begin{equation}
x_{CM} &= \frac{1}{M}\sum_i m_i x_i\\
&=\frac{M_S(0)+M_Er_E+M_Mr_M}{M_S+M_E+M_M}\\
&=\frac{(2.00e30 {\rm kg})(0)+(5.97e24 {\rm kg})(1.50e11 {\rm m})+(6.39e23 {\rm kg})(2.28e11 {\rm m})}{(2.00e30 {\rm kg})+(5.97e24 {\rm kg})+(6.39e23 {\rm kg})}\\
&=5.21e5 {\rm m}
\end{equation}
The centre of mass of the Sun-Earth-Mars system during a syzygy is located approximately $500 {\rm km}$ from the centre of the Sun.

\textbf{Discussion:} The radius of the Sun is approximately $700000 {\rm km}$, so the centre of mass of the system is well inside of the Sun. The Sun is so much more massive than either of the Earth or Mars, that the two planets hardly contribute to shifting the centre of mass away from the centre of the Sun. We would generally consider the masses of the two planets to be negligible if one wanted to model how the solar system itself moves around the Milky Way galaxy.
\end{framed}
\end{framed}

\begin{framed}
\textbf{Example 4.11}\\
\begin{figure}[!htbp]
\centering
\includegraphics[width=0.7\linewidth]{files/cmraft-2ce0a582e824c4cb4f6807c04093e5f7.png}
\caption[]{Three people on rafts on a lake.}
\label{fig:momentumandcm:cmraft}
\end{figure}

Alice (mass $m_A$), Brice (mass $m_B$), and Chloe (mass $m_C$) are stranded on individual rafts of negligible mass on a lake, off of the coast of Nyon. The rafts are located at the corners of a right-angle triangle, as illustrated in Figure~\ref{fig:momentumandcm:cmraft}, and are connected by ropes. The distance between Alice and Brice is $r_{AB}$ and the distance between Alice and Chloe is $r_{AC}$, as illustrated. Alice decides to pull on the rope that connects her to Chloe, while Brice decide to pull on the rope that connects him to Alice. Where will the three rafts meet?

\begin{framed}
\textbf{Solution}\\
We consider the system comprised of the three people and their rafts and model each person and their raft as a point particle with the mass concentrated at the centre of the raft. The forces exerted by pulling on the ropes are internal forces (one particle on the other), and will thus have no impact on the motion of the centre of mass of the system. There are no net external forces exerted on the system (the forces of gravity are balanced out by the forces of buoyancy from the rafts). The centre of mass of the system does not move when the people are pulling on the ropes, so they must ultimately meet at the centre of mass.

We can define a coordinate system such that the origin is located where Alice is initially located, the $x$ axis is in the direction from Alice to Brice, and the $y$ axis is in the direction from Alice to Chloe. The initial positions of Alice, Brice, and Chloe are thus:
\begin{equation}
\vec r_A &= 0\hat x + 0\hat y\\
\vec r_B &= r_{AB}\hat x + 0\hat y\\
\vec r_C &= 0\hat x + r_{AC}\hat y
\end{equation}
respectively. The $x$ and $y$ coordinates of the centre of mass are thus:
\begin{equation}
x_{CM} &= \frac{1}{M}\sum_i m_i x_i = \frac{m_A(0) + m_Br_{AB} + m_C(0)}{m_A + m_B + m_C}=\left(\frac{m_B}{m_A + m_B + m_C}\right)r_{AB}\\
y_{CM} &= \frac{1}{M}\sum_i m_i y_i = \frac{m_A(0) + m_B(0) + m_Cr_{AC}}{m_A + m_B + m_C}=\left(\frac{m_C}{m_A + m_B + m_C}\right)r_{AC}\\
\end{equation}
which corresponds to the position where the three rafts will meet, relative to the initial position of Alice.

\textbf{Discussion:} By using the centre of mass, we easily found where the three rafts would meet. If we had used Newton's Second Law on the three rafts individually, the model would have been complicated by the fact that the forces exerted by Alice and Brice on the ropes change direction as the rafts begin to move, which would have required the use of integrals to determine the motion of each person.
\end{framed}
\end{framed}

\paragraph{The centre of mass for a continuous object}

So far, we have considered the centre of mass for a system made of point particles. In this section, we show how one can determine the centre of mass for a ``continuous object''\footnote{In reality, there are of course no continuous objects since, at the atomic level, everything is made of particles.}. We previously argued that if an object is uniform and symmetric, its centre of mass will be located at the centre of the object. Let us show this explicitly for a uniform rod of total mass $M$ and length $L$, as depicted in Figure~\ref{fig:momentumandcm:rod}.

\begin{figure}[!htbp]
\centering
\includegraphics[width=0.5\linewidth]{files/rod-c7b5d09b2e1a12f0968930c303a464e6.png}
\caption[]{A rod of length $L$ and mass $M$.}
\label{fig:momentumandcm:rod}
\end{figure}

In order to determine the centre of mass of the rod, we first model the rod as being made of $N$ small ``mass elements'' each of equal mass, $\Delta m$, and of length $\Delta x$, as shown in Figure~\ref{fig:momentumandcm:rod}. If we choose those mass elements to be small enough, we can model them as point particles, and use the same formulas as above to determine the centre of mass of the rod.

We define the $x$ axis to be co-linear with the rod, such that the origin is located at one end of the rod. We can define the ``linear mass density'' of the rod, $\lambda$, as the mass per unit length of the rod:
\begin{equation}
\lambda = \frac{M}{L}.
\end{equation}

A small mass element of length $\Delta x$, will thus have a mass, $\Delta m$, given by:
\begin{equation}
\Delta m = \lambda \Delta x
\end{equation}

If there are $N$ mass elements that make up the rod, the $x$ position of the centre of mass of the rod is given by:
\begin{equation}
x_{CM} &= \frac{1}{M}\sum_i^N m_i x_i = \frac{1}{M}\sum_i^N \Delta m x_i \\
&=\frac{1}{M}\sum_i^N \lambda \Delta x x_i\\
\end{equation}
where $x_i$ is the $x$ coordinate of the $i$-th mass element. Of course, we can take the limit over which the length, $\Delta x$, of each mass element goes to zero to obtain an integral:
\begin{equation}
x_{CM} = \lim_{\Delta x \to 0} \frac{1}{M}\sum_i^N \lambda \Delta x x_i = \frac{1}{M} \int_0^L \lambda x dx
\end{equation}
where the discrete variable $x_i$ became the continuous variable $x$, and $\Delta x$ was replaced by $dx$ (which is the same, but indicates that we are taking the limit of $\Delta x \to 0$). The integral is easily found:
\begin{equation}
x_{CM} &= \frac{1}{M} \int_0^L \lambda x dx = \frac{1}{M}\lambda \left[ \frac{1}{2} x^2\right]_0^L\\
&=\frac{1}{M}\lambda \frac{1}{2} L^2 = \frac{1}{M}\left( \frac{M}{L}\right) \frac{1}{2} L^2\\
&=\frac{1}{2}L
\end{equation}
where we substituted the definition of $\lambda$ back in to find, as expected, that the centre of mass of the rod is half its length away from one of the ends.

Suppose that the rod was instead not uniform and that its linear density depended on the position $x$ along the rod:
\begin{equation}
\lambda(x) = 2a + 3bx
\end{equation}

We can still find the centre of mass by considering an infinitesimally small mass element of mass $dm$, and length $dx$. In terms of the linear mass density and length of the mass element, $dx$, the mass $dm$ is given by:
\begin{equation}
dm = \lambda(x) dx
\end{equation}
The $x$ position of the centre of mass is thus found the same way as before, except that the linear mass density is now a function of $x$:
\begin{equation}
x_{CM} &= \frac{1}{M} \int_0^L \lambda(x) x dx =\frac{1}{M} \int_0^L (2a + 3bx) x dx=\frac{1}{M} \int_0^L (2ax + 3bx^2) dx\\
&=\frac{1}{M}  \left[  ax^2 + bx^3  \right]_0^L\\
&=\frac{1}{M} (aL^2 + bL^3 )
\end{equation}

In general, for a continuous object, the position of the centre of mass is given by:
\begin{equation}
\vec r_{CM} &=\frac{1}{M}\int \vec r dm\nonumber\\
\therefore x_{CM} &= \frac{1}{M}\int x dm\nonumber\\
\therefore y_{CM} &=  \frac{1}{M}\int y dm\nonumber\\
\therefore z_{CM} &=  \frac{1}{M}\int z dm\\
\end{equation}
where in general, one will need to write $dm$ in terms of something that depends on position (or a constant) so that the integrals can be evaluated over the spatial coordinates ($x$,$y$,$z$) over the range that describe the object. In the above, we wrote $dm = \lambda dx$ to express the mass element in terms of spatial coordinates.

\begin{framed}
\textbf{Example 4.12}\\
\begin{figure}[!htbp]
\centering
\includegraphics[width=0.5\linewidth]{files/cmbowl-1d1b369d890b85728f882628aa960aeb.png}
\caption[]{A symmetric bowl with parabolic sides is completely filled with water. The bowl has a height $h$.}
\label{fig:momentumandcm:cmbowl}
\end{figure}

A bowl of height $h$ has parabolic sides and a circular cross-section, as illustrated in Figure~\ref{fig:momentumandcm:cmbowl}. The bowl is filled with water. The bowl itself has a negligible mass and thickness, so that the mass of the full bowl is dominated by the mass of the water. Where is the centre of mass of the full bowl?

\begin{framed}
\textbf{Solution}\\
We can define a coordinate system such that the origin is located at the bottom of the bowl and the $z$ axis corresponds to the axis of symmetry of the bowl. Because the bowl is full of water, and the bowl itself has negligible mass, we can model the full bowl as a uniform body of water with the same shape as the bowl and (volume) mass density $\rho$ equal to the density of water. Furthermore, by symmetry, the centre of mass of the bowl will be on the $z$ axis.

Because the bowl has a circular cross-section, we can divide it up into disk-shaped mass elements, $dm$, that have an infinitesimally small height $dz$, and a radius $r(z)$, that depends on their $z$ coordinate (Figure~\ref{fig:momentumandcm:cmbowl}).

\begin{figure}[!htbp]
\centering
\includegraphics[width=0.5\linewidth]{files/cmbowlsoln-46d95246606d9afbfcb46d896208050a.png}
\caption[]{The parabolic bowl divided up into disk-shaped mass elements, $dm$, that have an infinitesimally small height $dz$, and a radius $r(z)$, that depends on their $z$ coordinate.}
\label{fig:momentumandcm:cmbowlsoln}
\end{figure}

The centre of mass of each disk-shaped mass element will be located where the corresponding disk intersects the $z$ axis. The mass of one disk element is given by:
\begin{equation}
dm = \rho dV = \rho \pi r^2(z) dz
\end{equation}
where $dV = \pi r(z)^2 dz$ is the volume of the disk with radius $r(z)$ and thickness $dz$. The radius of the infinitesimal disk depends on its $z$ position, since the radii of the different disks must describe a parabola:
\begin{equation}
z(r) &= \frac{1}{a^2}r^2\\
r(z) &= a\sqrt z\\
\therefore dm &= \rho \pi r^2(z) dz= \rho \pi a^2  z dz
\end{equation}
where we introduced the constant $a$ so that the dimensions are correct. The constant $a$ determines how ``steep'' the parabolic sides are. The $z$ coordinate of the centre of mass is thus given by:
\begin{equation}
z_{CM} &=  \frac{1}{M}\int z dm =\frac{1}{M}\int_0^h z  (\rho \pi a^2 z dz)=\frac{\rho \pi a^2}{M}\int_0^h z^2dz \\
&=\frac{\rho \pi a^2}{M}\left[ \frac{1}{3}z^3 \right]_0^h\\
&=\frac{\rho \pi a^2}{3M}h^3
\end{equation}
However, we are not quite done, since we do not know the total mass, $M$, of the water. To find the total mass of water, $M$, we proceed in an analogous way, and determine the value of the sum (integral) of all of the mass elements:
\begin{equation}
M = \int dm = \int_0^h \rho \pi a^2 z dz = \rho \pi a^2 \left[ \frac{1}{2}z^2 \right]_0^h= \frac{1}{2}\rho \pi a^2 h^2
\end{equation}
Substituting this value for $M$, we can determine the $z$ coordinate of the centre of mass of the full bowl:
\begin{equation}
z_{CM} &=\frac{\rho \pi a^2}{3M}h^3 = \frac{2\rho \pi a^2}{3\rho \pi a^2 h^2}h^3=\frac{2}{3}h
\end{equation}
Regardless of the actual shape of the parabola (the parameter $a$), the centre of mass will always be two thirds of the way up from the bottom of the bowl.

\textbf{Discussion:} In determining the centre of mass of a three dimensional object, we used symmetry to argue that the $x$ and $y$ coordinates would be zero. We then found the $z$ position of the centre of mass by dividing up the bowl into infinitesimally small mass elements (disks) along the direction in which we needed to find the centre of mass coordinate.
\end{framed}
\end{framed}

\begin{framed}
\textbf{Checkpoint}\\
True or False: The centre of mass of a continuous object is always located within the object.

\begin{enumerate}
\item True
\item False
\end{enumerate}

\begin{framed}
\textbf{Answer}\\
\begin{enumerate}[resume]
\item
\end{enumerate}
\end{framed}
\end{framed}

\subsubsection{Circular Motion}\label{sec:momentumandcm:circularmotion}

\subsubsection{Summary}

The momentum vector, $\vec p$, of a point particle of mass, $m$, with velocity, $\vec v$, is defined as:
\begin{equation}
\vec p = m\vec v
\end{equation}
We can write Newton's Second Law for a point particle in term of its momentum:
\begin{equation}
\frac{d}{dt}\vec p = \sum \vec F = \vec F^{net}
\end{equation}
where the net force on the particle determines the rate of change of its momentum. In particular, if there is no net force on a particle, its momentum will not change.

The net impulse vector, $\vec J^{net}$, is defined as the net force exerted on a particle integrated from a time $t_A$ to a time $t_B$:
\begin{equation}
\vec J^{net} = \int_{t_A}^{tB} \vec F^{net} dt
\end{equation}
The net impulse vector is also equal to the change in momentum of the particle in that same period of time:
\begin{equation}
\vec J^{net} = \Delta \vec p = \vec p_B - \vec p_A
\end{equation}

When we define a system of particles, we can distinguish between internal and external forces. Internal forces are those forces exerted by the particles in the system on each other. External forces are those forces on the particles in the system that are not exerted by the particles on each other. The sum over all of the forces on all of the particles in the system will be equal to the sum over the external forces, because the sum over all internal forces on a system is always zero (Newton's Third Law).

The total momentum of a system, $\vec P$, is the sum of the momenta, $\vec p_i$, of all of the particles in the system:
\begin{equation}
\vec P = \sum \vec p_i
\end{equation}

The rate of change of the momentum of a system is equal to the sum of the external forces exerted on the system:
\begin{equation}
\frac{d}{dt}\vec P = \sum \vec F^{ext}
\end{equation}
which can be thought of as an equivalent description as Newton's Second Law, but for the system as a whole. If the net (external) force on a system is zero, then the total momentum of the system is conserved.

Collisions are those events when the particles in a system interact (e.g. by colliding) and change their momenta. When modelling collisions, it is usually beneficial to first define a system for which the total momentum is conserved before and after the collision.

Collisions can be elastic or inelastic. Elastic collisions are those where, in addition to the total momentum, the total mechanical energy of the system is conserved. The total mechanical energy can usually be taken as the sum of the kinetic energies of the particles in the system.

Inelastic collisions are those in which the total mechanical energy of the system is not conserved. One can usually identify if mechanical energy was introduced or removed from the system and determine if the collision is elastic. It is important to identify when momentum and mechanical energy are conserved. Momentum is conserved if no net force is exerted on the system, whereas mechanical energy is conserved if no net work was done on the system by non-conservative forces (internal or external) or by external conservative forces.

We can always choose in which frame of reference to model a collision. In some cases, it is convenient to use the frame of reference of the centre of mass of the system, because in that frame of reference, the total momentum of the system is zero.

If a system has a total mass $M$, then one can use Newton's Second Law to describe its motion:
\begin{equation}
\sum \vec F^{ext} &= M \vec a_{CM}\\
\sum \vec F^{ext} &=\frac{d}{dt} \vec P
\end{equation}
where the sum of the forces is over all of the external forces on the system. The acceleration vector, $\vec a_{CM}$, describes the motion of the ``centre of mass'' of the system. $\vec P=M\vec v_{CM}$ is the total momentum of the system.

The centre of mass of a system is a mass-weighted average of the positions of all of the particles of mass $m_i$ and position $\vec r_i$ that comprise the system:
\begin{equation}
\vec r_{CM} &=\frac{1}{M}\sum_i m_i\vec r_i
\end{equation}
The vector equation can be broken into components to find the $x$, $y$, and $z$ component of the position of the centre of mass. Similarly, one can also define the velocity of the centre of mass of the system, in terms of the individual velocities, $\vec v_i$, of the particles in the system:
\begin{equation}
\vec v_{CM} &= \frac{1}{M}\sum_i m_i\vec v_i
\end{equation}
Finally, one can define the acceleration of the centre of mass of the system, in terms of the individual accelerations, $\vec a_i$, of the particles in the system:
\begin{equation}
\vec a_{CM} &=  \frac{1}{M}\sum_i m_i\vec a_i
\end{equation}

If the system is a continuous object, we can find its centre of mass using a sum (integral) of infinitesimally small mass elements, $dm$, weighted by their position:
\begin{equation}
\vec r_{CM} &=\frac{1}{M}\int \vec r dm\\
\therefore x_{CM} &= \frac{1}{M}\int x dm\\
\therefore y_{CM} &=  \frac{1}{M}\int y dm\\
\therefore z_{CM} &=  \frac{1}{M}\int z dm
\end{equation}
The strategy to set up the integrals above is usually to express the mass element, $dm$, in terms of the position and density of the material of which the object is made. One can then integrate over position in the range defined by the dimensions of the object.

\begin{framed}
\textbf{Important Equations}\\
\textbf{Momentum of a point particle:}
\begin{equation}
\vec p = m\vec v \\
\frac{d}{dt}\vec p = \sum \vec F = \vec F^{net}
\end{equation}
\textbf{Impulse:}
\begin{equation}
\vec J^{net} = \int_{t_A}^{tB} \vec F^{net} dt \\
\vec J^{net} = \Delta \vec p = \vec p_B - \vec p_A
\end{equation}
\textbf{Momentum of a system:}
\begin{equation}
\vec P = \sum \vec p_i \\
\frac{d}{dt}\vec P = \sum \vec F^{ext}
\end{equation}
\textbf{Newton's Second Law for a {\textbackslash}~system:}
\begin{equation}
\sum \vec F^{ext} &= M \vec a_{CM}\\
\sum \vec F^{ext} &=\frac{d}{dt} \vec P
\end{equation}
\textbf{Position of the Centre of Mass {\textbackslash}~of a system:}
\begin{equation}
\vec r_{CM} &=\frac{1}{M}\sum_i m_i\vec r_i
\end{equation}
\textbf{Velocity of the Centre of Mass {\textbackslash}~of a system:}
\begin{equation}
\vec v_{CM} &= \frac{1}{M}\sum_i m_i\vec v_i \\
\end{equation}
\textbf{Acceleration of the Centre of Mass {\textbackslash}~of a system:}
\begin{equation}
\vec a_{CM} &=  \frac{1}{M}\sum_i m_i\vec a_i \\
\end{equation}
\textbf{Position of the Centre of Mass for a {\textbackslash}~continuous object:}
\begin{equation}
\vec r_{CM} &=\frac{1}{M}\int \vec r dm\\
\therefore x_{CM} &= \frac{1}{M}\int x dm\\
\therefore y_{CM} &=  \frac{1}{M}\int y dm\\
\therefore z_{CM} &=  \frac{1}{M}\int z dm
\end{equation}
\end{framed}

\begin{framed}
\textbf{Important Definitions}\\
\begin{itemize}
\item \textbf{Momentum:} The product of velocity and mass. SI units: [${\rm kgms^{ -1}}$]. Common variable(s): $\vec p$.
\item \textbf{Impulse:} A property of matter which describes an object's resistance to rotational motion. SI units: [${\rm Ns}$]. Common variable(s): $\vec J$.
\end{itemize}
\end{framed}

\subsubsection{Thinking about the material}

\begin{framed}
\textbf{Reflect and research}\\
\begin{itemize}
\item Explain how Newton's Cradle illustrates the conservation of momentum. Are the collisions in Newton's Cradle elastic? Explain!
\item Gymnasts have specially engineered ``crash mats'' for landing after doing spins and flips in the air. Why do these crash mats have to be specially engineered, and why can't the gymnast just use a big pile of blankets?
\item Give 2 examples where the centre of mass of an object is not located inside of the object.
\item The Volvo XC60 is supposedly the safest car in the world that money can buy. Why is this?
\item In the boxing world, boxers try to ``ride the punch''. Research and explain how this method helps boxers to reduce injuries.
\end{itemize}
\end{framed}

\begin{framed}
\textbf{To try at home}\\
\begin{itemize}
\item Grab two or three of your friends and ask them to hold a bed sheet. Throw an egg at full speed onto the bed sheet. What happens to the egg, and why?
\item Verify that in a 1 one-dimensional elastic collision between two objects of the same mass, if one object starts at rest, the other object will end at rest after the collision (look up Newton's Cradle to get an idea).
\end{itemize}
\end{framed}

\begin{framed}
\textbf{To try in the lab}\\
\begin{itemize}
\item Propose an experiment to test whether a collision is elastic.
\item Propose an experiment to test whether momentum is conserved in a two dimensional collision.
\item Design a technique which measures the centre of mass of an arbitrary 3D object.
\end{itemize}
\end{framed}

\subsubsection{Sample problems and solutions}

\paragraph{Problems}

\begin{framed}
\textbf{Problem 4.1}\\
\begin{figure}[!htbp]
\centering
\includegraphics[width=0.7\linewidth]{files/ballistic-469d6db369fd5a1d3f5f71a368e7d9d4.png}
\caption[]{A bullet of mass $m$ strikes and embeds itself into a ballistic pendulum of mass $M$.}
\label{fig:momentumandcm:ballistic}
\end{figure}

A ballistic pendulum is a device that can be built to measure the speed of a projectile. The pendulum is constructed such that the projectile is fired at the bob of the pendulum (typically a block of wood) which then swings as illustrated in Figure~\ref{fig:momentumandcm:ballistic}, with the projectile embedded within. By measuring the height that is reached by the pendulum's bob, one can determine the speed of the projectile before it collided with the pendulum. If a ballistic pendulum with a mass $M$ suspended at the end of strings of length $L$ is observed to rise by a height $h$ after being struck by a bullet of mass $m$, how fast was the bullet moving?
\end{framed}

\begin{framed}
\textbf{Problem 4.2}\\
\begin{figure}[!htbp]
\centering
\includegraphics[width=0.6\linewidth]{files/springcollision-e92881ea9bc239e7be1906e3102e5e0c.png}
\caption[]{One block attached to a spring about to collide with another block.}
\label{fig:momentumandcm:springcollision}
\end{figure}

A block of mass $M$ with a spring of spring constant $k$ attached to it is sliding on a frictionless surface with velocity $\vec v_M$ in the $x$ direction. A second block of mass $m$ has velocity $\vec v_m$ also in the $x$ direction (shown above in the negative $x$ direction, but let us assume that we do not necessarily know the direction, only  that the two blocks will collide). During the collision between the blocks, what is the maximum amount by which the spring is compressed?
\end{framed}

\begin{framed}
\textbf{Problem 4.3}\\
A uniform wire is bent into a semi-circle of radius $R$. Where is the centre of mass of the wire?
\end{framed}

\paragraph{Solutions}

\begin{framed}
\textbf{Solution 4.1}\\
We can model this situation by dividing it into three phases:

\begin{enumerate}
\item Before the bullet collides with the pendulum, only the bullet has momentum in the $x$ direction.
\item Immediately after the \textbf{inelastic} collision, the bullet and pendulum form a combined object of mass $M+m$ that has the same momentum as the bullet, in the $x$ direction, before the pendulum starts to swing upwards.
\item The pendulum with the embedded bullet swings upwards until its kinetic energy is zero.
\end{enumerate}

The collision between the bullet and pendulum is inelastic, because some of the kinetic energy of the bullet is used to deform the bullet and the pendulum. In general, any collision where two objects end up ``stuck together'' is inelastic.

In order to model the pendulum's motion we first apply conservation of momentum to determine the speed, $v'$, of the pendulum and embedded bullet just after the collision. Applying conservation of momentum in the $x$ direction to the system formed by the pendulum and the bullet, just before and after the collision, we have:
\begin{equation}
P &= mv\\
P' &= (M+m)v'\\
\therefore mv &= (M+m)v'\\
\therefore v' &= \frac{m}{m+M}v
\end{equation}
where $P$ and $P'$ are the initial and final momenta of the system, respectively. The pendulum with the bullet embedded in it will thus have a speed of $v'$ at the bottom of the pendulum's motion, before it swings upwards.

We can now use conservation of energy to model the swinging motion since, at that point, only tension and gravity act on the pendulum, and there are no non-conservative forces. If we choose the origin to be the location of the pendulum at the bottom of its trajectory, its initial gravitational potential energy is zero and its initial mechanical energy, $E$, is given by:
\begin{equation}
E = \frac{1}{2}(m+M) v'^2
\end{equation}
At the top of the trajectory, the pendulum with the embedded bullet will stop and have no kinetic energy. The mechanical energy at the top of the trajectory, $E'$, is thus equal to the gravitational potential energy of the pendulum at a height $h$ above the origin:
\begin{equation}
E' = (m+M)gh
\end{equation}
Applying conservation of mechanical energy allows us to find the initial speed of the bullet:
\begin{equation}
E &= E'\\
\frac{1}{2}(m+M) v'^2 &= (m+M)gh\\
v'^2 &= 2gh\\
\left( \frac{m}{m+M}v\right)^2&= 2gh\\
\therefore v &= \frac{m+M}{m} \sqrt{2gh}
\end{equation}
where is the second last line we used the expression for $v'$ that we obtained from conservation of momentum.

**Discussion: **This example showed a situation in which momentum and energy were both conserved, but not at the same time. This example also highlighted how, by using conservation laws, one can derive models that are much easier to solve mathematically than if one had to model all of the forces involved.
\end{framed}

\begin{framed}
\textbf{Solution 4.2}\\
The collision is elastic because the energy used to compress the spring is ``given back'' when the spring extends again, since the spring force is conservative.

They key to modelling the compression of the spring is to identify the condition under which the spring is maximally compressed. This will occur at the point during the collision where the two masses will have exactly the same velocity, momentarily moving in unison as the spring is maximally compressed. Because, instantaneously, the masses have the same velocity, there is a frame of reference in which the two masses are at rest, and the momentum is zero. Of course, that frame of reference is the centre of mass frame of reference.

Because the collision is one-dimensional, we can calculate the velocity of the centre of mass as:
\begin{equation}
v_{CM} = \frac{Mv_M+mv_m}{m+M}
\end{equation}
where we note that $v_m$ is a negative number, since the block of mass $m$ is moving in the negative $x$ direction. The total momentum, $\vec P^{CM}$, in the centre of mass frame of reference must be zero. Writing this out for the $x$ component and transforming the velocities of the two blocks into the centre of mass frame of reference:
\begin{equation}
P^{CM}_x = M(v_M-v_{CM})+m(v_m-v_{CM})&=0\\
\therefore (v_m-v_{CM}) &= -\frac{M}{m}(v_M-v_{CM})
\end{equation}
Also note that we can write the velocity difference $v_M -v_{CM}$ without using the centre of mass velocity:
\begin{equation}
v_M-v_{CM} &= v_M-\frac{Mv_M+mv_m}{m+M}=\frac{1}{m+M}(v_M(m+M)-Mv_M-mv_m)\\
&=\frac{m}{m+M}(v_M-v_m)
\end{equation}
We can then use conservation of energy in the centre of mass frame to determine the maximal compression of the spring. Before the collision, the total mechanical energy in the system, $E$, is the sum of the kinetic energies of the two blocks (as the spring is not compressed):
\begin{equation}
E&=\frac{1}{2}m(v_m-v_{CM})^2+\frac{1}{2}M(v_M-v_{CM})^2\\
&=\frac{1}{2}\frac{M^2}{m}(v_M-v_{CM})^2+\frac{1}{2}M(v_M-v_{CM})^2\\
&=\frac{1}{2}M \left( 1 + \frac{M}{m}\right) (v_M-v_{CM})^2\\
&=\frac{1}{2}M \left(\frac{m+M}{m} \right)(v_M-v_{CM})^2\\
&=\frac{1}{2}M \left(\frac{m+M}{m} \right)\left(\frac{m}{m+M}(v_M-v_m)\right)^2\\
&=\frac{1}{2} \left(\frac{mM}{m+M}\right)(v_M-v_m)^2
\end{equation}
where we used our expressions above to simplify the expression. When the spring is maximally compressed, the two blocks are at rest and the mechanical energy of the system, $E'$, is all ``stored'' as spring potential energy:
\begin{equation}
E'&=\frac{1}{2}kx^2
\end{equation}
where $x$ is the distance by which the spring is compressed. Equating the two allows us to determine the maximal compression of the spring:
\begin{equation}
E &= E' \\
\frac{1}{2} \left(\frac{mM}{m+M}\right)(v_M-v_m)^2 &= \frac{1}{2}kx^2\\
\therefore x &= \sqrt{\frac{1}{k} \left(\frac{mM}{m+M}\right)}(v_M-v_m)
\end{equation}
\textbf{Discussion:} By modelling the collision in the centre of mass frame of reference, we were easily able to determine the maximal compression of the spring. This would have been more difficult in the lab frame of reference, because the two blocks would still be moving when the spring is maximally compressed, so we would have needed to determine their speeds to determine the total mechanical energy when the spring is compressed.

When we calculated the initial kinetic energy, we found that it was given by:
\begin{equation}
E=\frac{1}{2} \left(\frac{mM}{m+M}\right)(v_M-v_m)^2 &=\frac{1}{2}M_{red}(v_M-v_m)^2
\end{equation}
The combination of masses in parentheses is called the ``reduced mass'' of the system, and is a sort of effective mass that can be used to model the system as a whole.
\end{framed}

\begin{framed}
\textbf{Solution 4.3}\\
The curved wire is illustrated in Figure~\ref{fig:momentumandcm:curvedwire}, along with a small mass element, $dm$, on the wire, and our choice of coordinate system (centred at the centre of the semi-circle). By symmetry, the position of the centre of mass will be located at $x=0$, so we only need to determine the $y$ position.

\begin{figure}[!htbp]
\centering
\includegraphics[width=0.4\linewidth]{files/curvedwire-99d938f2b3157166a4be5707dd9ea5b3.png}
\caption[]{A uniform wire bent into a semi circle of radius $R$, and a small mass element, $dm$, on the wire.}
\label{fig:momentumandcm:curvedwire}
\end{figure}

The $y$ position of the centre of mass is given by:
\begin{equation}
y_{CM} = \frac{1}{M}\int y dm
\end{equation}
where $M$ is the total mass of the wire. We can define the mass per unit length, $\lambda$, for the wire as:
\begin{equation}
\lambda =\frac{M}{\pi R}
\end{equation}
We will choose to integrate the equation for the $y$ position of the centre of mass over $\theta$ (from 0 to $\pi$), instead of over $y$, as it will make the integral easier (it is easier to express $dm$ in terms of $d\theta$ than $dy$ because the wire is curved). $\theta$ is the angle at which the mass element is located. The mass element forms an arc on the wire of length $ds$ that subtends an angle $d\theta$. The two are related by:
\begin{equation}
ds = Rd\theta
\end{equation}
The mass element, $dm$, can then be expressed in terms of the mass per unit length of the wire and the length, $Rd\theta$, of the mass element:
\begin{equation}
dm = \lambda ds = \lambda Rd\theta
\end{equation}
We also need to express the $y$ position of the mass element using $\theta$:
\begin{equation}
y = R\sin\theta
\end{equation}
Now that we have expressed $dm$ and $y$ in terms of $\theta$, we can determine the $y$ position of the centre of mass:
\begin{equation}
y_{CM}  &= \frac{1}{M}\int y dm =  \frac{1}{M}\int_0^\pi R\sin\theta \lambda Rd\theta\\
&= \frac{R^2\lambda}{M}\int_0^\pi \sin\theta d\theta = \frac{R^2\lambda}{M} \bigl[-\cos\theta\bigr]_0^\pi\\
&=\frac{2R^2\lambda}{M}=\frac{2R}{\pi}
\end{equation}
where in the last equality, we used the expression for the mass per unit length, $\lambda$, obtained above.
\end{framed}

\include{ModelingWithPhysics-newtonslaws}

\include{ModelingWithPhysics-applyingnewtonslaws}

\subsection{Chapter 7 - Work and energy}

\subsubsection{Overview}\label{chap:workenergy}

In this chapter, we introduce a new way to build models derived from Newton's theory of classical physics. We will introduce the concepts of work and energy, which will allow us to model situations using scalar quantities, such as energy, instead of vector quantities, such as forces. It is important to remember that even when we are using energy and work, these tools are derived from Newton's Laws; that is, we may not be using Newton's Second Law explicitly, but the models that we develop are still based on the same theory of classical physics.

\begin{framed}
\textbf{Learning Objectives}\\
\begin{itemize}
\item Understand the concept of work and how to calculate the work done by a force.
\item Understand the concept of the net work done on an object and how that relates to a change in speed of the object.
\item Understand the concept of kinetic energy and where it comes from.
\item Understand the concept of power.
\end{itemize}
\end{framed}

\begin{framed}
\textbf{Think About It}\\
You are holding a heavy book with your arm extended horizontally. The book does not move as you struggle to keep it from falling to the ground. Does your arm do work on the book? If you start walking to class while holding the book, does your arm do work on the book?

\begin{framed}
\textbf{Answer}\\
Your arms do no work on the book. There is no displacement (the book does not move up or down), so you do no work, even if its tiring! If you are walking, the displacement is perpendicular to the force applied by your arms, your arms do no work.
\end{framed}
\end{framed}

\subsubsection{Work}

\begin{framed}
\textbf{Review}\\
\begin{itemize}
\item Section~\ref{sec:Vectors:scalarproduct} on the scalar product.
\item Section~\ref{sec:calculus:integrals} on integrals.
\end{itemize}
\end{framed}

We introduce the concept of work as the starting point for building models using energy instead of forces. Work is a scalar quantity that is meant to represent how a force exerted on an object over a given distance results in a change in speed of that object. We will first introduce the concept of work done by a force on an object, and then look at how work can change the kinematics of the object. This is analogous to how we first defined the concept of force, and then looked at how force affects motion (by using Newton's Second Law, which connected the concept of force to the acceleration of the object).

The work done by a force, $\vec F$, on an object over a displacement, $\vec d$, is defined to be:
\begin{equation}
\boxed{W = \vec F \cdot \vec d = Fd\cos\theta = F_xd_x+F_yd_y+F_zd_z}
\end{equation}
where $\theta$ is the angle between the vectors when they are placed tail to tail, as in Figure~\ref{fig:workenergy:fddotproduct}. The dimension of work, force times displacement, is also called ``energy''. The S.I. unit for energy is the Joule (abbreviated $\text{J}$) which is equivalent to $\text{Nm}$ or ${\rm kg m^2/s^2}$ in base units.

\begin{figure}[!htbp]
\centering
\includegraphics[width=0.3\linewidth]{files/fddotproduct-49904090da9ae6fe24ebcc94657f47b1.png}
\caption[]{When determining the scalar product $\vec F\cdot \vec d = Fd\cos\theta$, $\theta$ is the angle between the vectors when they are placed tail to tail.}
\label{fig:workenergy:fddotproduct}
\end{figure}

The work ``done'' by the force is the scalar product of the force vector and the displacement vector of the object. We say that the force ``does work'' if it is exerted while the object moves (has a displacement vector) and in such a way that the scalar product of the force and displacement vectors is non-zero. A force that is perpendicular to the displacement vector of an object does no work (since the scalar product of two perpendicular vectors is zero).  A force exerted in the same direction as the displacement will do positive work ($\cos\theta$ positive), and a force in the opposite direction of the displacement will do negative work ($\cos\theta$ negative). As we will see, positive work corresponds to increasing the speed of the object, whereas negative work corresponds to decreasing its speed. No work corresponds to no change in speed (but could corresponds to a change in velocity).

\begin{framed}
\textbf{Checkpoint}\\
A pendulum of length $R$ consists of a mass connected to a string (Figure~\ref{fig:workenergy:pendulumtension}). The string exerts a force of tension $\vec F_T$ on the mass. What is the work done by tension when the pendulum swings through an angle $\theta$?

\begin{figure}[!htbp]
\centering
\includegraphics[width=0.2\linewidth]{files/pendulumworktension-a90b632f802628285c5f3f9cc394dcae.png}
\caption[]{A pendulum swings through an angle $\theta$.}
\label{fig:workenergy:pendulumtension}
\end{figure}

\begin{enumerate}
\item $W=F_TR\theta$
\item $W=F_TR(1 -\cos\theta)$
\item Tension does no work on the mass.
\end{enumerate}

\begin{framed}
\textbf{Answer}\\
\begin{enumerate}[resume]
\item
\end{enumerate}
\end{framed}
\end{framed}

You may be tempted to ask, ``Why work? Why not something else? Why that scalar product in particular? How could we possibly have thought of that?''. In general, it seems arbitrary that we introduce the quantity ``work'' and then find that it leads to a convenient way of building models. However, we did not just pull this quantity out of thin air! Many theorists, over many years, tried all sorts of quantities and ways to rephrase Newton's Theory that were not helpful. The quantities that make it into textbooks are the ones that turned out to be useful. You should also keep in mind that, just like force, work is a ``made-up'' mathematical tool that is helpful in describing the world around us. There is no such thing as work or energy; they are just useful mathematical tools.

\paragraph{Work in one dimension.}

Work involves vectors, so we can first examine the concept in one dimension, before extending this to two and three dimensions. We can choose $x$ as the coordinate in one dimension, so that all vectors only have an $x$ component. We can write a force vector as $\vec F=F\hat x$, where $F$ is the $x$ component of the force (which could be positive or negative). A displacement vector can be written as $\vec d = d \hat x$, where again, $d$ is the $x$ component of the displacement, and can be positive or negative. In one dimension, work is thus:
\begin{equation}
W = \vec F \cdot \vec d = (F\hat x) \cdot ( d\hat x ) = Fd (\hat x\cdot\hat x)=Fd
\end{equation}
where $\hat x \cdot \hat x = 1$. Consider, for example, the work done by a force, $\vec F$, on a box, as the box moves along the $x$ axis from position $x=x_0$ to position $x=x_1$, as shown in Figure~\ref{fig:workenergy:work1d}.

\begin{figure}[!htbp]
\centering
\includegraphics[width=0.4\linewidth]{files/work1d-0daf97b5d5589ef1f3195ed30fc5eb42.png}
\caption[]{A force, $\vec F$, exerted on an object as it moves from position $x=x_0$ to position $x=x_1$.}
\label{fig:workenergy:work1d}
\end{figure}

We can write the length of the displacement vector as $||\vec d|| =d= \Delta x = x_1 -x_0$. The work done by the force is given by:
\begin{equation}
W = \vec F \cdot \vec d = F\hat x\cdot \Delta x\hat x =F\Delta x =F(x_1-x_0)
\end{equation}
which is a positive quantity, since $x_1 > x_0$, with our choice of coordinate system.

\begin{framed}
\textbf{Checkpoint}\\
A constant force in the positive $x$ direction, $\vec F$, acts on a box, as in Figure~\ref{fig:workenergy:work1d}. Consider the work done by $\vec F$ as the box moves from $x_1$ to $x_0$. How does it compare to the work done by $\vec F$ when moving from $x_0$ to $x_1$ (that we calculated above)?

\begin{enumerate}
\item $\vec F$ does no work on the box when it moves from $x_0$ to $x_1$.
\item The work has the same magnitude as before, but the work is now negative.
\item The work done by $\vec F$ is the same in both cases.
\end{enumerate}

\begin{framed}
\textbf{Answer}\\
\begin{enumerate}[resume]
\item
\end{enumerate}
\end{framed}
\end{framed}

\paragraph{Work in one dimension - varying force}

Suppose that instead of a constant force, $\vec F$, we have a force that changes with position, $\vec F(x)$, and can take on three different values between $x=x_0$ and $x=x_3$:
\begin{equation}
  \vec F (x)=
  \begin{cases}
    F_1\hat x & x<\Delta x \\
    F_2\hat x & \Delta x \leq x< 2\Delta x \\
    F_3\hat x & 2\Delta x \leq x
  \end{cases}
\end{equation}
as illustrated in Figure~\ref{fig:workenergy:work1d}, which shows the force on an object as it moves from position $x=x_0$ to position $x=x_3$, along three (equal) displacement vectors, $\vec d_1=\vec d_2=\vec d_3=\Delta x \hat x$.

\begin{figure}[!htbp]
\centering
\includegraphics[width=0.7\linewidth]{files/work1dvarying-1dbde50e5bef34689b7c939f0392c213.png}
\caption[]{A varying force, $\vec F(x)$, exerted on an object as it moves from position $x=x_0$ to position $x=x_3$.}
\label{fig:workenergy:work1dvarying}
\end{figure}

The total work done by the force over the three separate displacements is the sum of the work done over each displacement:
\begin{equation}
W^{tot}&=W_1+W_2+W_3\\
&=\vec F_1\cdot \vec d_1+\vec F_2\cdot \vec d_2+\vec F_3\cdot \vec d_3\\
&= F_1\Delta x +F_2\Delta x + F_3\Delta x
\end{equation}

If instead of 3 segments we had $N$ segments and the $x$ component of the force had the $N$ corresponding values $F_i$ in the $N$ segments, the total work done by the force would be:
\begin{equation}
W^{tot} = \sum_{i=0}^N\vec F_i \cdot \Delta \vec x
\end{equation}
where we introduced a vector $\Delta \vec x$ to be the vector of length $\Delta x$ pointing in the positive $x$ direction. In the limit where $\vec F(x)$ changes continuously as a function of position, we take the limit of an infinite number of infinitely small segments of length $dx$, and the sum becomes an integral:
\begin{equation}
\boxed{W^{tot} = \int_{x_0}^{x_f}\vec F(x) \cdot d\vec x}
\end{equation}
where the work was calculated in going from $x=x_0$ to $x=x_f$, and $d\vec x=dx\hat x$ is an infinitely small displacement vector (of length $dx$) in the positive $x$ direction.

\begin{framed}
\textbf{Example 7.1}\\
A block is pressed against the free end of a horizontal spring with spring constant, $k$, so as to compress the spring by a distance $D$ relative to its rest length, as shown in Figure~\ref{fig:workenergy:spring}. The other end of the spring is fixed to a wall. What is the work done by the spring force on the block in going from $x= -D$ to $x=0$? What is the work done by the block on the spring over the same displacement?

\begin{figure}[!htbp]
\centering
\includegraphics[width=0.4\linewidth]{files/spring-fb0b7b45895ba1ffc941557cc6a32aee.png}
\caption[]{A block is pressed against a horizontal spring so as to compress the spring by a distance $D$ relative to its rest length.}
\label{fig:workenergy:spring}
\end{figure}

\begin{framed}
\textbf{Solution}\\
The force exerted by the spring on the block changes continuously with position, according to Hooke's law:
\begin{equation}
\vec F(x) = -kx \hat x
\end{equation}
and points in the positive $x$ direction when the end of the spring has a negative $x$ position (with our coordinate choice illustrated in Figure~\ref{fig:workenergy:spring}, where the origin is located at the rest length of the spring). To calculate the work done by the force, we sum the work done by the force over many infinitesimally small displacements $d\vec x$ (using an integral):
\begin{equation}
W &= \int_{-D}^0 \vec F(x) \cdot d\vec x\\
&=\int_{-D}^0 (-kx \hat x) \cdot (dx \hat x)\\
&=\int_{-D}^0 -kxdx (\hat x \cdot \hat x)\\
&=-\int_{-D}^0 kx dx\\
&=-\left[\frac{1}{2}kx^2  \right]_{-D}^0\\
&=\frac{1}{2}kD^2
\end{equation}
In order to determine the work that was done by the block on the spring, we need to determine the force, $\vec F'(x)$, exerted by the block on the spring. By Newton's Third Law, this is equal in magnitude but opposite in direction to the force exerted by the spring on the block:
\begin{equation}
\vec F'(x) = -\vec F(x) = kx \hat x
\end{equation}
The work done by the block on the spring over the same displacement is:
\begin{equation}
W' &= \int_{-D}^0 \vec F'(x) \cdot d\vec x\\
&=\int_{-D}^0 (kx \hat x) \cdot (dx \hat x)\\
&=\int_{-D}^0 kx dx=-\frac{1}{2}kD^2\\
\end{equation}
which is negative. This makes sense because the force exerted by the block on the spring is in the direction opposite to the direction of displacement, so the work should be negative.
\end{framed}
\end{framed}

\paragraph{Work in multiple dimensions}

First, consider the work done by a force $\vec F$ in pulling a crate over a displacement $\vec d$, in the case where the force is directed at an angle $\theta$ above the horizontal, as shown in Figure~\ref{fig:workenergy:workangle}, and the displacement is along the $x$ axis (or rather, we chose the $x$ axis to be parallel to the displacement).

\begin{figure}[!htbp]
\centering
\includegraphics[width=0.5\linewidth]{files/workangle-c0fee3c95b822c28de3d889be8fac4c3.png}
\caption[]{A force, $\vec F$, exerted on an object as it moves from position $x=x_0$ to position $x=x_1$.}
\label{fig:workenergy:workangle}
\end{figure}

The work done by the force is given by:
\begin{equation}
W = \vec F \cdot \vec d &= Fd\cos\theta\\
&= F_{\parallel}d\\
&= Fd_{\parallel}\\
\end{equation}
where we highlighted the fact that the scalar product ``picks out'' components of vectors that are parallel to each other. $F_{\parallel} = F\cos\theta$ is the component of $\vec F$ that is parallel to $\vec d$, and $d_{\parallel}=d\cos\theta$ is the component of $\vec d$ that is parallel to $\vec F$. These are also shown in Figure~\ref{fig:workenergy:workangle}.

\begin{framed}
\textbf{Checkpoint}\\
Brent and Dean pull two crates by using ropes that make the same angle above the horizontal and with the same force. The magnitude of the crates' displacement is the same, but Dean's crate moves horizontally on the ground while Brent's crate moves up a frictionless ramp that is parallel to the rope used to pull the crate. Who did more work on the crate?

\begin{enumerate}
\item Dean because there is friction between his crate and the ground.
\item Brent.
\item They did the same amount of work.
\end{enumerate}

\begin{framed}
\textbf{Answer}\\
\begin{enumerate}[resume]
\item
\end{enumerate}
\end{framed}
\end{framed}

In general, if an object is moving along an arbitrary path, we cannot choose the $x$ axis to be parallel to the displacement or to the force. If the path can be sub-divided into straight segments over which the force is constant, as in Figure~\ref{fig:workenergy:work2d}, we can calculate the work done by the force over each segment and add the work done in each segment together to obtain the total work done by the force. Note that, in general, the work done by a force as an object moves from one position to another depends on the particular path that was taken between the two positions, since different paths will have difference lengths.

\begin{figure}[!htbp]
\centering
\includegraphics[width=0.4\linewidth]{files/work2d-99959f9f04f1f717440b6952780f6dab.png}
\caption[]{An arbitrary two dimensional path of an object from $A$ to $B$ broken into three straight segments.}
\label{fig:workenergy:work2d}
\end{figure}

\begin{framed}
\textbf{Example 7.2}\\
Compare the work done by the force of kinetic friction in sliding a crate along a horizontal surface from position $A$ (coordinates $x_A, y_A$) to position $B$ (coordinates $x_B, y_B$) using the two different paths depicted in Figure~\ref{fig:workenergy:workfriction}. Assume that the mass of the crate is $m$ and that the coefficient of kinetic friction between the crate and the ground is $\mu_k$.

\begin{figure}[!htbp]
\centering
\includegraphics[width=0.4\linewidth]{files/workfriction-1b84b5d4fa9b153b71a19e4b88afa5dc.png}
\caption[]{Two possible paths to slide a crate from position $A$ to position $B$, as seen from above.}
\label{fig:workenergy:workfriction}
\end{figure}

\begin{framed}
\textbf{Solution}\\
The force of kinetic friction is always in the direction opposite to that of motion. Thus, regardless of the path taken, the force of friction will do negative work.

Let us first calculate the work done by the force of kinetic friction along the first path (the straight line). The force of kinetic friction will have a magnitude:
\begin{equation}
f_k = \mu_k N = \mu_k mg
\end{equation}
The normal force will have the same magnitude as the weight because the crate is not moving (accelerating) in the direction perpendicular to the $xy$ plane.  The displacement vector from $A$ to $B$ can be written as:
\begin{equation}
\vec d &= (x_B-x_A)\hat x + (y_B-y_A) \hat y\\
\therefore ||\vec d|| &=d= \sqrt{(x_B-x_A)^2 + (y_B-y_A)^2}
\end{equation}
The force of kinetic friction will be in the opposite direction of the displacement vector, so the angle between the two vectors is $180{\rm \degree}$ ($\cos\theta= -1$). The work done by the force of kinetic friction is thus:
\begin{equation}
W = \vec f_k \cdot\vec d = f_k d \cos\theta = -\mu_k mg\sqrt{(x_B-x_A)^2 + (y_B-y_A)^2}
\end{equation}
and is negative, as expected.

For path 2, we break up the motion into two segments, with displacements vectors $\vec d_1$ (along $y$) and $\vec d_2$ (along $x$). We can write the two displacement vectors as:
\begin{equation}
\vec d_1 &= 0\hat x + (y_B-y_A) \hat y\\
\therefore ||\vec d_1||&=d_1=(y_B-y_A)\\
\vec d_2 &= (x_B-x_A)\hat x + 0 \hat y\\
\therefore ||\vec d_2||&=d_2=(x_B-x_A)\\
\end{equation}
Along each segment, the force of kinetic friction is anti-parallel to the displacement (note that the force of friction changes direction over the two segments), but the magnitude is $f_k=\mu_kmg$. The work done along the first segment is thus:
\begin{equation}
W_1 = \vec f_k \cdot \vec d_1 = f_k d_1 \cos\theta = -\mu_k mg(y_B-y_A)
\end{equation}
The work done along the second segment is:
\begin{equation}
W_2 = \vec f_k \cdot \vec d_2 = f_k d_2 \cos\theta = -\mu_k mg(x_B-x_A)
\end{equation}
And the total work done by the force of kinetic friction over the second path is:
\begin{equation}
W^{tot} = W_1 + W_2 = -\mu_k mg \left((x_B-x_A) + (y_B-y_A)\right)
\end{equation}
which is more work than was done along path 1. This makes sense because for both paths, the force of friction has the same magnitude and is always in the opposite direction of motion; thus, the longer the path, the more work will be done by the force.
\end{framed}
\end{framed}

\begin{framed}
\textbf{Example 7.3}\\
A box of mass $m$ is moved from the floor onto a table using two different paths, as shown in Figure~\ref{fig:workenergy:workgravity}. The table is a horizontal distance $L$ away from where the box starts and a height $H$ above the floor. Compare the work done by the weight of the box along the two possible paths.

\begin{figure}[!htbp]
\centering
\includegraphics[width=0.5\linewidth]{files/workgravity-875d9fb31a6236f7fd29d3cb33642cbf.png}
\caption[]{Two possible paths to move a box from the floor onto a table.}
\label{fig:workenergy:workgravity}
\end{figure}

\begin{framed}
\textbf{Solution}\\
We can use a coordinate system such that the origin coincides with the initial position of the box. $x$ is horizontal and $y$ is vertical, as shown in Figure~\ref{fig:workenergy:workgravity}. The weight of the box can be written as:
\begin{equation}
\vec F_g = -mg \hat y
\end{equation}
and points in the negative $y$ direction with a magnitude of $mg$. To calculate the work done by the weight along the first path, we first determine the corresponding displacement vector, $\vec d$:
\begin{equation}
\vec d = L\hat x + H\hat y
\end{equation}
and we can then determine the work:
\begin{equation}
W &= \vec F_g \cdot \vec d = (-mg \hat y) \cdot (L\hat x + H\hat y)\\
&=F_xd_x+F_yd_y= (0)(L) + (-mg)(H)\\
&= -mgH
\end{equation}
Along path 1, the work done by the weight is negative, and does not depend on the horizontal distance $L$. Let us now calculate the work done along the second path, which we break up into two segments with displacement vectors $\vec d_1$ (vertical) and $\vec d_2$ (horizontal). The displacement vectors are:
\begin{equation}
\vec d_1 &= H\hat y\\
\vec d_2 &= L\hat x
\end{equation}
The work done along the vertical segment is:
\begin{equation}
W_1 &= \vec F_g \cdot \vec d_1 = (-mg \hat y) \cdot (H\hat y)\\
&=-mgH
\end{equation}
The work done along the horizontal segment is:
\begin{equation}
W_2 &= \vec F_g \cdot \vec d_2 = (-mg \hat y) \cdot (L\hat x)\\
&=0
\end{equation}
which is zero, because the force of gravity is always vertical and thus perpendicular to the displacement vector of the horizontal segment. The total work done by the weight along the second path is:
\begin{equation}
W^{tot} = W_1 + W_2 = -mgH
\end{equation}
which is the same as the work done along path 1. As we will see, when a force is constant in magnitude and direction, the work that it does on an object in going from one position to another is independent of the path taken. This was not the case in Example~7.2, because the direction of the force of kinetic friction depends on the direction of the displacement.
\end{framed}
\end{framed}

\begin{framed}
\textbf{Checkpoint}\\
Clare and Amelia go down two different slides, as shown in Figure~\ref{fig:workenergy:slidecheckpoint}. Clare and Amelia have the same mass and the slides have the same non-zero coefficients of friction.

\begin{figure}[!htbp]
\centering
\includegraphics[width=0.5\linewidth]{files/slidecheckpoint-00df72a2d5b8afc63acc302203ecce5c.png}
\caption[]{Clare ($C$) and Amelia ($A$) go down two different slides of the same height.}
\label{fig:workenergy:slidecheckpoint}
\end{figure}

For each of the following forces, decide whether the force: does more work on Clare, does more work on Amelia, or does the same amount of work on both.

\begin{enumerate}
\item The force of gravity...
\item The force of friction...
\item The normal force from the slide...
\end{enumerate}

\begin{framed}
\textbf{Answer}\\
Gravity does the same amount of work on both, friction does more work on Amelia, and the normal force does the same amount of work on both (the normal force does zero work, since it is always perpendicular to the displacement).
\end{framed}
\end{framed}

The most general case for which we can calculate the work done by a force is the case when the force changes continuously along a path where the displacement also changes direction continuously. This is illustrated in Figure~\ref{fig:workenergy:workgeneral} which shows an arbitrary path between two points $A$ and $B$, and a force, $\vec F(\vec r)$, that depends on position ($\vec r$). In general, the work done by the force on an object that goes from $A$ to $B$ will depend on the actual path that was taken.

\begin{figure}[!htbp]
\centering
\includegraphics[width=0.5\linewidth]{files/workgeneral-84453b5d1021cea1e90c769f48cf81a0.png}
\caption[]{An arbitrary path between two points $A$ and $B$ with a force that depends on position, $\vec F(\vec r)$.}
\label{fig:workenergy:workgeneral}
\end{figure}

The strategy for calculating the work in the general case is the same: we break up the path into small straight segments with displacement vectors $d\vec l$ (Figure~\ref{fig:workenergy:dldiagram}) where we assume that the force is constant over the segment. The total work is the sum of the work over each segment:
\begin{equation}
\boxed{W = \int_A^B \vec F(\vec r) \cdot d\vec l}
\end{equation}
As usual, we use the integral symbol to indicate that you need to take an infinite number of infinitely small segments $d\vec l$ in order to calculate the sum.

\begin{figure}[!htbp]
\centering
\includegraphics[width=0.3\linewidth]{files/elementoflengthdl-6ada5f90333a0876a91e3ab41ee39190.png}
\caption[]{We divide the path into infinitesimally small segments with displacement vectors $d\vec l$.}
\label{fig:workenergy:dldiagram}
\end{figure}

You should note that this is not an integral like any other that we have seen so far: the integral is not over a single integration variable (usually we use $x$), but it is the integral (the sum!) over the specific path that we have chosen in going from $A$ to $B$. This is called a ``path integral'', and is generally difficult to evaluate.

\begin{framed}
\textbf{Example 7.4}\\
\begin{figure}[!htbp]
\centering
\includegraphics[width=0.4\linewidth]{files/workparabola-d0ed2175e176574cb5fbc15b4d5a2c74.png}
\caption[]{A parabolic path between $A$ and $B$.}
\label{fig:workenergy:workparabola}
\end{figure}

A force, $\vec F(\vec r) = \vec F(x,y) = F_x\hat x + F_y \hat y$,  is exerted on an object. The object starts at position $A$ and ends at position $B$, along a parabolic path, $y(x) = a+bx^2$, as depicted in Figure~\ref{fig:workenergy:workparabola}. What is the work done by the force, $\vec F$, along this trajectory?

\begin{framed}
\textbf{Solution}\\
In this case, the force can change with position (if $F_x$ and $F_y$ are not constant), and the direction of the path changes continuously. When we break up the path into small segments $d\vec l$, we need to incorporate the equation of the parabola to include the fact that $d\vec l$ must always be tangent to the parabola. Consider one small segment along the trajectory and the infinitesimal displacement vector $d\vec l$ at that point, as in Figure~\ref{fig:workenergy:workparabola_dr}.

\begin{figure}[!htbp]
\centering
\includegraphics[width=0.3\linewidth]{files/workparabola_dr-759577de91b461f1c08d0dda81a7adc9.png}
\caption[]{The infinitesimal displacement vector, $d\vec l$.}
\label{fig:workenergy:workparabola_dr}
\end{figure}

We can write the $x$ and $y$ components of the vector as infinitesimal distances, $dx$ and $dy$, along the $x$ and $y$ axes, respectively. The vector $d\vec l$ can thus be written:
\begin{equation}
d\vec l = dx \hat x + dy \hat y
\end{equation}
The total work done by the force is then:
\begin{equation}
W &= \int_A^B \vec F(\vec r) \cdot d\vec l\\
&=\int_A^B (F_x\hat x + F_y \hat y) \cdot (dx \hat x + dy \hat y)\\
&=\int_A^B (F_x dx + F_ydy)\\
\therefore W&= \int_A^B F_x dx + \int_A^B F_ydy
\end{equation}
where in the last line, we simply used the property that the integral of a sum is the sum of the corresponding integrals. At this point, we have two integrals over integration variables ($x$ and $y$) that are meaningful. However, we have not yet used the fact that our path is a parabola, and in general, we expect that the shape of the path is important. By saying that we are integrating (or calculating the work) over a specific path, we are really saying that $x$ and $y$ are not independent; that is, if we know the value of $x$ at some point on the path, we know the corresponding value of $y$ ($y = a+bx^2$).

Since $x$ and $y$ are not independent, we can use a ``substitution of variables'' in order to express $y$ in terms of $x$, and $dy$ in terms of $dx$:
\begin{equation}
y(x) &= a + bx^2\\
\frac{dy}{dx} &= 2bx\\
\therefore dy &= 2bxdx
\end{equation}
This allows us to convert the integral over $y$ to an integral over $x$, which also allows us to be explicit for the limits of the integral (in our example, the integral goes from $x=0$ to $x=x_1$):
\begin{equation}
W&= \int_A^B F_x dx + \int_A^B F_ydy\\
&=\int_0^{x_1} F_x dx + \int_0^{x_1} F_y(2bxdx)\\
&=\int_0^{x_1} (F_x + 2bxF_y)dx
\end{equation}
where we would need to know how $F_x$ and $F_y$ depends on $x$ and $y$ in order to actually evaluate the integral.

For example, if the force were constant ($F_x$ and $F_y$ constant), then the work done along the parabolic path would be:
\begin{equation}
W &= \int_0^{x_1} (F_x + 2bxF_y)dx\\
&=\left[F_x x + bF_yx^2  \right]_0^{x_1}\\
&=F_x x_1 + bF_yx_1^2
\end{equation}
As we mentioned earlier, \textbf{if the force is constant in magnitude and direction}, then the work done is independent of path. We can easily check this, using the displacement vector $\vec d = x_1\hat x + bx_1^2 \hat y$:
\begin{equation}
W &= \vec F \cdot \vec d = (F_x\hat x+ F_y\hat y) \cdot (x_1\hat x + bx_1^2 \hat y)\\
&=F_x x_1 + bF_yx_1^2
\end{equation}
as we found above.
\end{framed}
\end{framed}

\paragraph{Net work done}

So far, we have considered the work done on an object by a single force. If more than one force is exerted on an object, then each force can do work on the object, and we can calculate the ``net work'' done on the object by adding together the work done by each force. We will show that this is equivalent to first calculating the net force on the object, $F^{net}$ (i.e. the vector sum of the forces on the object), and then calculating the work done by the net force.

Suppose that three forces, $\vec F_1$, $\vec F_2$, and $\vec F_3$ are exerted on an object as it moves such that its displacement vector is $\vec d$. The net work done on the object is easily shown to be equivalent to the work done by the net force::
\begin{equation}
W^{net} &= W_1 + W_2 + W_3 \\
&= \vec F_1 \cdot \vec d + \vec F_2 \cdot \vec d  + \vec F_3 \cdot \vec d \\
&=(F_{1x}d_x+F_{1y}d_y+F_{1z}d_z)+ (F_{2x}d_x+F_{2y}d_y+F_{2z}d_z) + (F_{3x}d_x+F_{3y}d_y+F_{3z}d_z)\\
&=(F_{1x} + F_{2x} + F_{3x})d_x+(F_{1y} + F_{2y} + F_{3y})d_y+(F_{1z} + F_{2z} + F_{3z})d_z\\
&=\vec F^{net} \cdot \vec d
\end{equation}
where $\vec F^{net} = \vec F_1 + \vec F_2 + \vec F_3$ is the net force. The result is easily generalized to any number of forces, including if those forces change as a function of position:
\begin{equation}
W^{net} = \int_A^B F^{net}(\vec r) \cdot d\vec l
\end{equation}

\begin{framed}
\textbf{Example 7.5}\\
You push with an unknown horizontal force, $\vec F$, against a crate of mass $m$ that is located on an inclined plane that makes an angle $\theta$ with respect to the horizontal, as shown in Figure~\ref{fig:workenergy:workincline}. The coefficient of kinetic friction between the crate and the incline is $\mu_k$. You push in such a way that that crates moves at a constant speed up the incline. What is the net work done on the crate if it moves up the incline by a distance $d$?

\begin{figure}[!htbp]
\centering
\includegraphics[width=0.3\linewidth]{files/workincline-b98602f6e15a41e5f840950101a04e5d.png}
\caption[]{A crate being pushed up an incline.}
\label{fig:workenergy:workincline}
\end{figure}

\begin{framed}
\textbf{Solution}\\
Although the answer may be obvious, let's go the long way about it and calculate the work done by each force, and then sum them together to get the total work done. We start by identifying the forces exerted on the crate:

\begin{enumerate}
\item $\vec F$, the applied force, of unknown magnitude, $\vec F$.
\item $\vec F_g$, the weight of the crate, with magnitude $mg$.
\item $\vec N$, a normal force exerted by the incline.
\item $\vec f_k$, a force of kinetic friction, with magnitude $\mu_k N$, that points in the direction opposite of $\vec d$.
\end{enumerate}

These are shown in the free-body diagram in Figure~\ref{fig:workenergy:workincline_fbd}, along with our choice of coordinate system, and the displacement vector.

\begin{figure}[!htbp]
\centering
\includegraphics[width=0.3\linewidth]{files/workincline_fbd-7b9ff640523ee3e98aa1d1557bc2919d.png}
\caption[]{Free-body diagram for the crate on the incline.}
\label{fig:workenergy:workincline_fbd}
\end{figure}

With our choice of coordinate system, the displacement vector is given by:
\begin{equation}
\vec d = d (\cos\theta \hat x + \sin\theta \hat y)
\end{equation}
Before calculating the work done by each force, we need to determine the magnitude of the normal force (and thus of the force of kinetic friction). Since the crate is moving at a constant velocity, its \textbf{acceleration is zero}, so the sum of the forces must be zero. Writing out the $y$ component of Newton's Second Law allows us to find the magnitude of the normal force:
\begin{equation}
\sum F_y &= N\cos\theta -F_g - f_k\sin\theta = 0\\
\therefore mg &= N\cos\theta-\mu_kN\sin\theta = N(\cos\theta-\mu_k\sin\theta)\\
\therefore N &= \frac{mg}{\cos\theta-\mu_k\sin\theta}
\end{equation}
Writing out the $x$ component of Newton's Second Law allows us to find the magnitude of the unknown force $F$:
\begin{equation}
\sum F_x &= F - N\sin\theta - f_k\cos\theta = 0\\
\therefore F &= N\sin\theta+\mu_kN\cos\theta = N(\sin\theta+\mu_k\cos\theta)\\
&=mg\frac{\sin\theta+\mu_k\cos\theta}{\cos\theta-\mu_k\sin\theta}
\end{equation}
We now proceed to calculate the work done by each force. The work done by the normal force is identically zero, since it is perpendicular to the displacement vector. The work done by the applied force, $\vec F = F\hat x$, is:
\begin{equation}
W_F &= \vec F \cdot \vec d = (F\hat x)\cdot(d (\cos\theta \hat x + \sin\theta \hat y))\\
&=Fd\cos\theta=mg\frac{\sin\theta+\mu_k\cos\theta}{\cos\theta-\mu_k\sin\theta}d\cos\theta
\end{equation}
The work done by the force of gravity, $\vec F_g = -mg \hat y$, is:
\begin{equation}
W_g &= \vec F_g \cdot \vec d = (-mg \hat y)\cdot(d (\cos\theta \hat x + \sin\theta \hat y))\\
&=-mgd\sin\theta
\end{equation}
The work done by the force of friction, $\vec f_k$, noting that $\vec f_k$ and $\vec d$ are antiparallel:
\begin{equation}
W_f &= \vec f_k \cdot \vec d = -f_kd = -\mu_kNd\\
&=-\mu_k\frac{mg}{\cos\theta-\mu_k\sin\theta}d
\end{equation}
The net work done on the crate is thus:
\begin{equation}
W^{net}&=W_F + W_g + W_f\\
&=mg\frac{\sin\theta+\mu_k\cos\theta}{\cos\theta-\mu_k\sin\theta}d\cos\theta-mgd\sin\theta -\mu_k\frac{mg}{\cos\theta-\mu_k\sin\theta}d\\
&=mgd \left(  \frac{\sin\theta+\mu_k\cos\theta}{\cos\theta-\mu_k\sin\theta}\cos\theta - \sin\theta - \mu_k\frac{1}{\cos\theta-\mu_k\sin\theta} \right)\\
&=mgd \left(  \frac{(\sin\theta+\mu_k\cos\theta)\cos\theta - \sin\theta(\cos\theta-\mu_k\sin\theta) - \mu_k}{\cos\theta-\mu_k\sin\theta} \right)\\
&=mgd \left(  \frac{\sin\theta\cos\theta+\mu_k\cos^2\theta - \sin\theta\cos\theta+\mu_k\sin^2\theta - \mu_k}{\cos\theta-\mu_k\sin\theta} \right)\\
&=mgd \left(  \frac{\mu_k(\cos^2\theta+\sin^2\theta) - \mu_k}{\cos\theta-\mu_k\sin\theta} \right)\\
&=0
\end{equation}
where we used the fact that $\cos^2\theta+\sin^2\theta=1$. Thus we find that the net work done on the crate is zero!

\textbf{Discussion:} Of course, this makes sense, because the net force on the crate is zero, since it is not accelerating, so the net work done is also zero. As a consequence, or rather, by construction, we have the condition that if the net work done on an object is zero, then that object does not accelerate. We thus have a scalar quantity (work) that can tell us something about whether an object is changing speed. In the next section, we introduce a new quantity, ``kinetic energy'', to describe how an object's speed changes when the net work done is not zero.
\end{framed}
\end{framed}

\begin{framed}
\textbf{Olivia's Thoughts}\\
Pay close attention to the words ``on'' and ``by.'' There are a few things about this that can be tricky:

\begin{enumerate}
\item In Example~7.5, we were asked to find the \textbf{net work} done \textbf{on} the crate. Sometimes, the question won't specify that it wants you to find the net work, and will just say ``What is the work done \textbf{on} the crate?'' When you are just asked for the work done ``on'' an object, the question is implicitly asking for the \textit{net} work done on the object.
\item Just because the net work done \textbf{on} an object is zero doesn't mean that the work done \textbf{by} each of the forces is zero. This may seem obvious, but it's easy to get tripped up on a test or exam. If you are reading a question about work and it says that the object is moving at a constant speed, it's tempting to just jump ahead and say that the work must be equal to zero. However, you can only say this if it's asking you for the net work done on the object. For instance, in Example~7.5, we concluded that since the crate was moving at a constant speed, the net work was equal to zero. But if the question asked you to find the work done on the crate \textbf{by gravity}, that would mean something different. The work done \textbf{by gravity} in this case is not equal to zero (it's actually negative).
\item The work done ``on'' an object is not the same as the net work done ``by'' that object. For example, say you are in a tug-of-war and you pull the other team towards you, but you yourself do not move. The net work done \textbf{on} you is zero, but the work done \textbf{by} you is not zero. So, when you are talking about work, you should always state explicitly whether the work is being done ``on'' the object or ``by'' the object.
\end{enumerate}

\textbf{Note}: The wording won't always be like this - sometimes it will say ``How much work do you do on the box?'' instead of ``How much work is done \textbf{by} you on the box,'' so always be careful. Still, looking for key words like ``by'' and ``on'' is a good place to start.
\end{framed}

\begin{framed}
\textbf{Checkpoint}\\
A $2 {\rm kg}$ box sits on a horizontal surface. A constant horizontal force of $6 {\rm N}$ is applied to the box. The box moves with a constant acceleration of $2 {\rm m/s^2}$. Which of the following has the greatest magnitude?

\begin{enumerate}
\item The work done by the applied force.
\item The work done by friction.
\item The net work done on the box.
\end{enumerate}

\begin{framed}
\textbf{Answer}\\
\begin{enumerate}
\item
\end{enumerate}
\end{framed}
\end{framed}

\subsubsection{Kinetic energy and the work energy theorem}\label{sec:workenergy:kinetic}

At this point, you should be comfortable calculating the net work done on an object upon which several forces are exerted. As we saw in the previous section, the net work done on an object is connected to the object's acceleration; if the net force on the object is zero, then the net work done and acceleration are also zero. In this section, we derive a new quantity, kinetic energy, which allows us to connect the work done on an object with its change in speed. This will allow us to describe motion using only scalar quantities. Like the definition of work, the following derivation appears to ``come out of thin air''. Remember, though, that theorists have tried all sorts of mathematical tricks to reformulate Newton's Theory, and this is the one that worked.

Consider the most general case of an object of mass $m$ acted upon by a net force, $\vec F^{net}(\vec r)$, which can vary in magnitude and direction. We wish to calculate the  net work done on the object as it moves along an arbitrary path between two points, $A$ and $B$, in space, as shown in Figure~\ref{fig:workenergy:kepath}. The instantaneous acceleration of the object, $\vec a$, is shown along with an ``element of the path'', $d\vec l$.

\begin{figure}[!htbp]
\centering
\includegraphics[width=0.4\linewidth]{files/kepath-c0a19725a0d065d2bda035ac72018b9c.png}
\caption[]{An object moving along an arbitrary path between points $A$ and $B$ that is acted upon by a net force $\vec F^{net}$.}
\label{fig:workenergy:kepath}
\end{figure}

The net work done on the object can be written:
\begin{equation}
W^{net} = \int_A^B F^{net}(\vec r) \cdot d\vec l
\end{equation}
and is in general a difficult integral to evaluate for an arbitrary path. Our goal is to find a way to evaluate this integral by finding a function, $K$, with the property that:
\begin{equation}
\int_A^B F^{net}(\vec r) \cdot d\vec l =K_B - K_A
\end{equation}
That is, we will only have to evaluate $K$ at the end points of the path in order to determine the value of the integral. In this way, the function $K$ is akin to an anti-derivative.

In order to determine the form for the function $K$, we start by noting that, by using Newton's Second Law, we can write the integral for work in terms of the acceleration of the object:
\begin{equation}
\sum \vec F &= \vec F^{net} = m\vec a\\
\therefore \int_A^B F^{net}(\vec r) \cdot d\vec l &= \int_A^B m\vec a\cdot d\vec l =m\int_A^B \vec a\cdot d\vec l
\end{equation}
where we assumed that the mass of the object does not change along the path and can thus be factored out of the integral. Consider the scalar product of the acceleration, $\vec a$, and the path element, $d\vec l=dx\hat x  +dy\hat y + dz\hat z$, written in terms of the velocity vector:
\begin{equation}
\vec a & = \frac{d\vec v}{dt}\\
\therefore \vec a\cdot d\vec l &= \frac{d\vec v}{dt}\cdot d\vec l\\
&=\left(\frac{dv_x}{dt}\hat x+ \frac{dv_y}{dt}\hat y + \frac{dv_z}{dt}\hat z\right) \cdot (dx\hat x  +dy\hat y + dz\hat z)\\
&=\frac{dv_x}{dt}dx+\frac{dv_y}{dt}dy+\frac{dv_z}{dt}dz
\end{equation}
Any of the terms in the sum can be re-arranged so that the time derivative acts on the element of path ($dx$, $dy$, or $dz$) instead of the velocity, for example:
\begin{equation}
\frac{dv_x}{dt}dx = \frac{dx}{dt}dv_x
\end{equation}
where we recognize that $\frac{dx}{dt} = v_x$. We can thus write the scalar product between the acceleration vector and the path element as:
\begin{equation}
\vec a\cdot d\vec l&= \frac{dv_x}{dt}dx+\frac{dv_y}{dt}dy+\frac{dv_z}{dt}dz\\
&=\frac{dx}{dt}dv_x + \frac{dy}{dt}dv_y+\frac{dz}{dt}dv_z\\
&=v_xdv_x + v_ydv_y + v_zdv_z
\end{equation}
The integral for the net work done can be written as:
\begin{equation}
W^{net} &= \int_A^B F^{net}(\vec r) \cdot d\vec l =m \int_A^B (v_xdv_x + v_ydv_y + v_zdv_z)\\
&=m\int_A^B v_xdv_x +m\int_A^B  v_ydv_y + m\int_A^B v_zdv_z
\end{equation}
which corresponds to the sum of three integrals over the three independent components of the velocity vector. The components of the velocity vector are functions that change over the path and have fixed values at either end of the path. Let the velocity vector of the object at point $A$ be $\vec v_A=(v_{Ax}, v_{Ay}, v_{Az})$ and the velocity vector at point $B$ be $\vec v_B=(v_{Bx}, v_{By}, v_{Bz})$. The integral over, say, the $x$ component of velocity is then:
\begin{equation}
m\int_A^B v_xdv_x &= m\int_{v_{Ax}}^{v_{Bx}} v_xdv_x= m\left[\frac{1}{2}v_x^2  \right]_{v_{Ax}}^{v_{Bx}}\\
&=\frac{1}{2}m(v_{Bx}^2-v_{Ax}^2)
\end{equation}
We can thus write the net work integral as:
\begin{equation}
W^{net} &=m\int_A^B v_xdv_x +m\int_A^B  v_ydv_y + m\int_A^B v_zdv_z\\
&=\frac{1}{2}m(v_{Bx}^2-v_{Ax}^2) + \frac{1}{2}m(v_{By}^2-v_{Ay}^2) +\frac{1}{2}m(v_{Bz}^2-v_{Az}^2)\\
&=\frac{1}{2}m(v_{Bx}^2+v_{By}^2+v_{Bz}^2)-\frac{1}{2}m(v_{Ax}^2+v_{Ay}^2+v_{Az}^2)\\
&=\frac{1}{2}mv_B^2 - \frac{1}{2}mv_A^2
\end{equation}
where we recognized that the magnitude (squared) of the velocity is given by $v_A^2 = v_{Ax}^2+v_{Ay}^2+v_{Az}^2$. We have thus arrived at our desired result; namely, we have found a function of speed, $K(v)$, that when evaluated at the endpoints of the path allows us to calculate the net work done on the object over that path:
\begin{equation}
\boxed{K(v) = \frac{1}{2}mv^2}
\end{equation}
That is, if you know the speed at the start of the path, $v_A$, and the speed at the end of the path, $v_B$, then the net work done on the object along the path between $A$ and $B$ is given by:
\begin{equation}
\boxed{W^{net} = \Delta K = K(v_B) - K(v_a)}
\end{equation}
We call $K(v)$ the ``kinetic energy'' of the object. We can say that the net work done on an object in going from $A$ to $B$ is equal to its change in kinetic energy (final kinetic energy minus initial kinetic energy). It is important to note that we defined kinetic energy in a way that it is equal to the net work done. You may have already seen kinetic energy from past introductions to physics as a quantity that is just given; here, we instead derived a function that has the desired property of being equal to the net work done and called it ``kinetic energy''.

The relation between the net work done and the change in kinetic energy is called the ``Work-Energy Theorem'' (or Work-Energy Principle). It is the connection that we were looking for between the dynamics (the forces from which we calculate work) and the kinematics (the change in kinetic energy). Unlike Newton's Second Law, which relates two vector quantities (the vector sum of the forces and the acceleration vector), the Work-Energy Theorem relates two scalar quantities to each other (work and kinetic energy). Although we introduced the kinetic energy as a way to calculate the integral for the net work, if you know the value of the net work done on an object, then the Work-Energy Theorem can be used to calculate the change in speed of the object.

Most importantly, the Work-Energy theorem introduces the concept of ``energy''. As we will see in later chapters, there are other forms of energy in addition to work and kinetic energy. The Work-Energy Theorem is the starting point for the idea that you can convert one form of energy into another. The Work-Energy Theorem tells us how a force, by doing work, can provide kinetic energy to an object or remove kinetic energy from an object.

\begin{framed}
\textbf{Example 7.6}\\
A net work of $W$ was done on an object of mass $m$ that started at rest. What is the speed of the object after the work has been done on the object?

\begin{framed}
\textbf{Solution}\\
Using the Work-Energy Theorem:
\begin{equation}
W = \frac{1}{2}mv_f^2 - \frac{1}{2}mv_i^2
\end{equation}
where $v_i$ is the initial speed of the object and $v_f$ is its final speed. Since the initial speed is zero, we can easily find the final speed:
\begin{equation}
v_f = \sqrt{\frac{2W}{m}}
\end{equation}
\end{framed}
\end{framed}

\begin{framed}
\textbf{Example 7.7}\\
A block is pressed against the free end of a horizontal spring with spring constant, $k$, so as to compress the spring by a distance $D$ relative to its rest length, as shown in Figure~\ref{fig:workenergy:spring2}. The other end of the spring is fixed to a wall.

\begin{figure}[!htbp]
\centering
\includegraphics[width=0.4\linewidth]{files/spring-fb0b7b45895ba1ffc941557cc6a32aee.png}
\caption[]{A block is pressed against a horizontal spring so as to compress the spring by a distance $D$ relative to its rest length.}
\label{fig:workenergy:spring2}
\end{figure}

If the block is released from rest and there is no friction between the block and the horizontal surface, what is the speed of the block when it leaves the spring?

\begin{framed}
\textbf{Solution}\\
This is the same problem that we presented in Section~\ref{sec:applyingnewtonslaws:modellingwhereforcechanges} in Example~6.3, where we solved a differential equation to find the speed.

Our first step is to calculate the net work done on the object in going from $x= -D$ to $x=0$ (which corresponds to when the object leaves the spring, as discussed in Example~6.3). The forces on the object are:

\begin{enumerate}
\item $\vec F_g$, its weight, with magnitude $mg$.
\item $\vec N$, the normal force exerted by the ground.
\item $\vec F(x)$, the force from the spring, with magnitude $kx$.
\end{enumerate}

Both the normal force and weight are perpendicular to the displacement, so they will do no work. The net work done is thus the work done by the spring, which we calculated in Example~7.1 to be:
\begin{equation}
W^{net} = W_F = \frac{1}{2}kD^2
\end{equation}
By the Work-Energy Theorem, this is equal to the change in kinetic energy. Noting that the object started at rest ($v_i=0$), the final speed $v_f$ is found to be:
\begin{equation}
W^{net} &=  \frac{1}{2}mv_f^2 - \frac{1}{2}mv_i^2 =  \frac{1}{2}mv_f^2 - 0\\
\frac{1}{2}kD^2 &=\frac{1}{2}mv_f^2\\
\therefore v_f &=\sqrt{\frac{kD^2}{m}}
\end{equation}
\end{framed}
\end{framed}

\begin{framed}
\textbf{Example 7.8}\\
A block is pressed against the free end of a horizontal spring with spring constant, $k$, so as to compress the spring by a distance $D$ relative to its rest length, as shown in Figure~\ref{fig:workenergy:spring3}. The other end of the spring is fixed to a wall.

\begin{figure}[!htbp]
\centering
\includegraphics[width=0.4\linewidth]{files/spring-fb0b7b45895ba1ffc941557cc6a32aee.png}
\caption[]{A block is pressed against a horizontal spring so as to compress the spring by a distance $D$ relative to its rest length.}
\label{fig:workenergy:spring3}
\end{figure}

If the block is released from rest and the coefficient of kinetic friction between the block and the horizontal surface is $\mu_k$, what is the speed of the block when it leaves the spring?

\begin{framed}
\textbf{Solution}\\
This is the same example as the previous one, but with kinetic friction. The forces on the block are:

\begin{enumerate}
\item $\vec F_g$, its weight, with magnitude $mg$.
\item $\vec N$, the normal force exerted by the ground on the block.
\item $\vec F(x)$, the force from the spring, with magnitude $kx$.
\item $\vec f_k$, the force of kinetic friction, with magnitude $\mu_kN$.
\end{enumerate}

Both the normal force and weight are perpendicular to the displacement, so they will do no work. Furthermore, since the acceleration in the vertical direction is zero, the normal force will have the same magnitude as the weight ($N=mg$). The magnitude of the force of kinetic friction is thus $f_k = \mu_k mg$. The net work done will be the sum of the work done by the spring, $W_F$, and the work done by friction, $W_f$:
\begin{equation}
W^{net} = W_F + W_f
\end{equation}
We have already determined the work done by the spring:
\begin{equation}
W_F = \frac{1}{2}kD^2
\end{equation}
The work done by the force of kinetic friction will be negative (since it is in the direction opposite of the motion) and is given by:
\begin{equation}
W_f = \vec f_k \cdot \vec d = -f_kD = -\mu_kmgD
\end{equation}
Applying the work energy theorem, and noting that the block started at rest ($v_i=0$), the final speed $v_f$ is found to be:
\begin{equation}
W^{net} =W_F + W_f&= \frac{1}{2}mv_f^2 - \frac{1}{2}mv_i^2 \\
\frac{1}{2}kD^2-\mu_kmgD  &=\frac{1}{2}mv_f^2\\
\therefore v_f &=\sqrt{\frac{kD^2}{m}-2\mu_kgD}
\end{equation}
\textbf{Discussion:} We can think of this in terms of the concept of energy. The spring does positive work on the block, and so it increases its kinetic energy. Friction does negative work on the block, decreasing its kinetic energy. Only the spring is ``introducing'' energy into the block, as friction is removing that energy by doing negative work. Another way to think about it is that the spring is inputting energy; some of that energy goes into increasing the kinetic energy of the block, and some of it is lost by friction.

The energy that is lost to friction can be thought of as ``thermal energy'' (heat) that goes up into heating the block and the surface. Indeed, if you rub your hand against the table, you will notice that it gets warmer; you are losing some of the energy introduced to your hand by the work done by your arm into heating up the table and your hand! This shows that we can think about modelling friction using thermal energy rather than a force.
\end{framed}
\end{framed}

\subsubsection{Power}

We finish the chapter by introducing the concept of ``power'', which is the rate at which work is done on an object, or more generally, the rate at which energy is being converted from one form to another. If an amount of work, $\Delta W$, was done in a period of time $\Delta t$, then the work was done at a rate of:
\begin{equation}
\boxed{P = \frac{\Delta W}{\Delta t}}
\end{equation}
where $P$ is called the power. The SI unit for power is the ``Watt'', abbreviated $\text{W}$, which corresponds to ${\rm J/s}={\rm kg m^2/s^3}$ in base SI units. If the rate at which work is being done changes with time, then the instantaneous power is defined as:
\begin{equation}
\boxed{P = \frac{dW}{dt}}
\end{equation}
You have probably already encountered power in your everyday life. For example, your $1000 {\rm W}$ hair dryer consumes ``electrical energy'' at a rate of $1000 {\rm J}$ per second and converts it into the kinetic energy of the fan as well as the thermal energy to heat up the air. Horsepower ($\text{hp}$) is an imperial unit of power that is often used for vehicles, the conversion being $1 {\rm hp} = 746 {\rm W}$. A $100 {\rm hp}$ car thus has an engine that consumes the chemical energy released by burning gasoline at a rate of $7.46e4 {\rm J}$ per second and converts it into work done on the car as well as into heat.

\begin{framed}
\textbf{Checkpoint}\\
Two cranes lift two identical boxes off of the ground. One crane is twice as powerful as the other. Both cranes do the same amount of work on the boxes and operate at full power. Which of the following statements is true of the boxes, once the cranes have done work on them?

\begin{enumerate}
\item One box has been lifted twice as high as the other.
\item The boxes are lifted to the same height in the same amount of time.
\item The boxes are lifted to the same height, but it takes one of the boxes twice as long to get there.
\item One box is lifted twice as high as the other, but it takes the same amount of time to get there.
\end{enumerate}

\begin{framed}
\textbf{Answer}\\
\begin{enumerate}[resume]
\item
\end{enumerate}
\end{framed}
\end{framed}

\begin{framed}
\textbf{Example 7.9}\\
If a car engine can do work on the car with a power of $P$, what will be the speed of the car at some time $t$ if the car was at rest at time $t=0$?

\begin{framed}
\textbf{Solution}\\
First, we need to calculate how much total work was done on the car:
\begin{equation}
W = P t
\end{equation}
Then, using the Work-Energy Theorem, we can find the speed of the car at some time $t$:
\begin{equation}
W &= \frac{1}{2}mv_f^2 - \frac{1}{2}mv_i^2\\
Pt &= \frac{1}{2}mv_f^2 \\
\therefore v_f &= \sqrt{\frac{2Pt}{m}}
\end{equation}
\textbf{Discussion:} The model for the final speed of the car makes sense because:

\begin{itemize}
\item The dimension of the expression for $v_f$ is speed (you should check this!).
\item The speed is greater if either the time or power are greater (so the speed is larger if more work is done on the car).
\item The speed is smaller if the mass of the car is greater (the acceleration of the car will be less if the mass of the car is larger).
\end{itemize}
\end{framed}
\end{framed}

\begin{framed}
\textbf{Example 7.10}\\
You are pushing a crate along a horizontal surface at constant speed, $v$. You find that you need to exert a force of $\vec F$ on the crate in order to overcome the friction between the crate and the ground. How much power are you expending by pushing on the crate?

\begin{framed}
\textbf{Solution}\\
We need to calculate the rate at which the force, $\vec F$, that you exert on the crate does work. If the crate is moving at constant speed, $v$, then in a time $\Delta t$, it will cover a distance, $d=v\Delta t$. Since you exert a force in the same direction as the motion of the crate, the work done over that distance $d$ is:
\begin{equation}
\Delta W = \vec F \cdot \vec d = Fd\cos(0) = Fv\Delta t
\end{equation}
The power corresponding to the work done in that period of time is thus:
\begin{equation}
P = \frac{\Delta W}{\Delta t} = Fv
\end{equation}
This is quite a general result for the rate at which a force does work when it is exerted on an object moving at constant speed.
\end{framed}
\end{framed}

\begin{framed}
\textbf{Olivia's Thoughts}\\
Example~7.10 ties into what I brought up earlier. If you think to yourself: ``The velocity is constant, so the work must be zero'', the formula,
\begin{equation}
P = \frac{\Delta W}{\Delta t} = Fv
\end{equation}
wouldn't make any sense. Since $v$ is a constant velocity, the power would always be equal to zero, which of course isn't right. Again, remember that when the velocity is constant, it is only the \textbf{net work} that is equal to zero. In Example~7.10, it's asking for the power that \textbf{you} are expending by pushing on the crate (which is the same as asking for the rate of the work done \textbf{by} you \textbf{on} the crate). So, the formula does indeed make sense.
\end{framed}

\subsubsection{Summary}

The work, $W$, done on an object by a force, $\vec F$, while the object has moved through a displacement, $\vec d$, is defined as the scalar product:
\begin{equation}
W = \vec F \cdot \vec d &= Fd\cos\theta\\
&= F_xd_x+F_yd_y+F_zd_z
\end{equation}
If the force changes with position and/or the object moves along an arbitrary path in space, the work done by that force over the path is given by:
\begin{equation}
W =\int_A^B \vec F(\vec r) \cdot  d\vec l
\end{equation}
% Work allows us to quantify how a force acting on an object changes the speed of that object.

If multiple forces are exerted on an object, then one can calculate the net force on the object (the vector sum of the forces), and the net work done on the object will be equal to the work done by the net force:
\begin{equation}
W^{net} = \int_A^B \vec F^{net}(\vec r) \cdot d\vec l
\end{equation}
If the net work done on an object is zero, that object does not accelerate.

We can define the kinetic energy, $K(v)$ of an object of mass $m$ that has speed $v$ as:
\begin{equation}
K(v) = \frac{1}{2} mv^2
\end{equation}

The Work-Energy Theorem states that the net work done on an object in going from position $A$ to position $B$ is equal to the object's change in kinetic energy:
\begin{equation}
W^{net} = \Delta K = \frac{1}{2} mv_B^2 - \frac{1}{2} mv_A^2
\end{equation}
where $v_A$ and $v_B$ are the speed of the object at positions $A$ and $B$, respectively.

The rate at which work is being done is called power and is defined as:
\begin{equation}
P = \frac{dW}{dt}
\end{equation}
If a constant force $\vec F$ is exerted on an object that has a constant velocity $\vec v$, then the power that corresponds to the work being done by that force is:
\begin{equation}
P &= \frac{d}{dt} W = \frac{d}{dt}(\vec F \cdot \vec d)\\
&= \vec F \cdot \frac{d}{dt}\vec d = \vec F \cdot \vec v
\end{equation}

\begin{framed}
\textbf{Important Equations}\\
\textbf{Work:}
\begin{equation}
W &= \vec F \cdot \vec d = Fd\cos\theta\\
W &= F_xd_x+F_yd_y+F_zd_z\\
W &=\int_A^B \vec F(\vec r) \cdot  d\vec l\\
W^{net} &= \int_A^B \vec F^{net}(\vec r) \cdot d\vec l
\end{equation}
\textbf{Kinetic Energy:}
\begin{equation}
K(v) = \frac{1}{2} mv^2
\end{equation}

\textbf{Work-Energy Theorem:}
\begin{equation}
W^{net} = \Delta K = \frac{1}{2} mv_B^2 - \frac{1}{2} mv_A^2
\end{equation}
\textbf{Power:}
\begin{equation}
P &= \frac{dW}{dt}\\
P &= \vec F \cdot \vec v
\end{equation}
\end{framed}

\begin{framed}
\textbf{Important Definitions}\\
\begin{itemize}
\item \textbf{Kinetic energy:} A form of energy that an object with a mass has by virtue of having a non-zero speed. SI units: [{\textbackslash}text\{J\}]. Common variable(s): $K$.
\item \textbf{Power:} The rate at which energy is converted with respect to time. SI units: ${\rm \left[{W}\right]}$. Common variable(s): $P$.
\end{itemize}
\end{framed}

\subsubsection{Thinking about the material}

\begin{framed}
\textbf{Reflect and research}\\
\begin{itemize}
\item When was the concept of work first introduced?
\item To construct the pyramids, the ancient Egyptians used simple machines, like levers, to accomplish tasks that would not be possible otherwise. Apply what we know about work to find out how levers help people lift incredibly heavy objects.
\item After an accident, investigators use skid marks to figure out how fast the cars were going before the crash. Use your knowledge of work, figure out how they do this.
\item The Tesla Model S can accelerate from 0-100 {\textbackslash}si\{km/h\} in as little as 2.7 seconds. Calculate the power of the car in horsepower. Why is it unusual for a 7 seat sedan, like the Model S, to have such a short acceleration time? Investigate how it's possible for the Tesla to accelerate so quickly.
\end{itemize}
\end{framed}

\begin{framed}
\textbf{To try at home}\\
\begin{itemize}
\item Measure the power that you can output with your legs, and describe how you made the measurement.
\end{itemize}
\end{framed}

\begin{framed}
\textbf{To try in the lab}\\
\begin{itemize}
\item Propose an experiment to measure the thermal energy associated with a force of kinetic friction.
\item Propose an experiment to test the Work-Energy Theorem.
\end{itemize}
\end{framed}

\subsubsection{Sample problems and solutions}

\paragraph{Problems}

\begin{framed}
\textbf{Problem 7.1}\\
A ski jump can is modelled as a ramp of height $h=5 {\rm m}$, as shown in Figure~\ref{fig:workenergy:skijumpprob}. The landing area is at the same height as the bottom of the ramp. A skier of mass $m=80 {\rm kg}$ is moving at a speed $v_i=15 {\rm m/s}$ when they reach the bottom of the ramp. When the skier lands the jump, their speed is measured to be $v_f=12 {\rm m/s}$. Ignore air resistance.

\begin{figure}[!htbp]
\centering
\includegraphics[width=0.7\linewidth]{files/skijumpprob-7a4485948f1dc2b2671e28671fec477f.png}
\caption[]{A person of mass $m$ goes off a ski jump of height $h$.}
\label{fig:workenergy:skijumpprob}
\end{figure}

\begin{itemize}
\item a. What is the speed of the skier the instant they leave the ski jump, at the top of the ramp?
\item b. Use the answer from part (a) to find the work done by the force of friction between the ramp and the skier.
\end{itemize}
\end{framed}

\begin{framed}
\textbf{Problem 7.2}\\
A child of mass $m$ sits on a swing of length $L$, as in Figure~\ref{fig:workenergy:swingprob}. You push the child with a horizontal force $\vec F$. You apply the force in such a way that the child moves at a constant speed (note that $\vec F$ will not have a constant magnitude).\}

\begin{itemize}
\item a. How much work do you do to move the child from $\theta=0$ to $\theta=\theta_1$?
\item b.  Use a detailed diagram to show that the work done by $\vec F$ is equal to $mgh$, where $h$ is the change in height of the child.
\end{itemize}

\begin{figure}[!htbp]
\centering
\includegraphics[width=0.4\linewidth]{files/swingprob-d71c20d72bf5df16dc390c24266b1394.png}
\caption[]{A child on a swing is pushed from $\theta=0$ to $\theta=\theta_1$ at constant speed with a horizontal force, $\vec F$.}
\label{fig:workenergy:swingprob}
\end{figure}
\end{framed}

\paragraph{Solutions}

\begin{framed}
\textbf{Solution}\\
\begin{itemize}
\item a. We start by defining a coordinate system. We choose the $x$ axis to be horizontal and positive in the direction of motion, and we choose the $y$ axis to be vertical and the positive direction upwards.
\end{itemize}

We will determine the speed at the top of the ramp, $v_t$, using the Work-Energy Theorem:
\begin{equation}
W^{net}=\frac{1}{2}mv_f^2-\frac{1}{2}mv_t^2
\end{equation}
where $W^{net}$ is the net work done on the skier as they ``fly'' through the air. While the skier is in the air, the only force acting on them is gravity, $\vec F= -mg\hat y$. The path of the skier is a parabola, so that the displacement vector changes direction continuously. The work done by gravity is given by:
\begin{equation}
W = \int \vec F_g \cdot d\vec l
\end{equation}
where $d\vec l$ is an infinitesimal displacement along the trajectory, as shown in Figure~\ref{fig:workenergy:skiprobdisplacement}.

\begin{figure}[!htbp]
\centering
\includegraphics[width=0.7\linewidth]{files/skiprobdisplacement-89a036a287b9d6e9cdb4ce44a58e8c29.png}
\caption[]{Infinitesimal displacement along the trajectory of the jump.}
\label{fig:workenergy:skiprobdisplacement}
\end{figure}

The displacement vector will have $x$ and $y$ components:
\begin{equation}
d\vec l = dx \hat x + dy \hat y
\end{equation}
The scalar product with the force of gravity is thus:
\begin{equation}
\vec F_g \cdot d\vec l &= (-mg\hat y) \cdot (dx \hat x + dy \hat y)= -mgdy
\end{equation}
The work done by gravity can thus be converted into an integral over $y$ (for which we know the start and end values), and is given by:
\begin{equation}
W = \int \vec F_g \cdot d\vec l = \int_h^0 -mgdy = [-mgy]_h^0 = mgh
\end{equation}
The work done by gravity is positive, which makes sense, since the force of gravity is generally in the same direction as the net displacement (downwards). We did not need to take into account the specific shape of the trajectory, because the force was constant in magnitude and direction (see Example~7.4).

We can now find the speed of the skier when they leave the jump using the Work-Energy theorem:
\begin{equation}
W^{net}&=\frac{1}{2}mv_f^2-\frac{1}{2}mv_t^2\\
mgh &= \frac{1}{2}mv_f^2-\frac{1}{2}mv_t^2\\
\therefore v_t&=\sqrt{v_f^2-2gh}=\sqrt{(12 {\rm m/s})^2 - 2(9.8 {\rm m/s^2})(5 {\rm m})}=6.8 {\rm m/s}
\end{equation}

\begin{itemize}
\item b. We can again use the Work-Energy Theorem to determine the work done by friction as the skier slides up the ramp. We know that the speed of the skier at the bottom of the ramp is $v_i$, and we just found that the speed of the skier at the top of the ramp is $v_t=\sqrt{v_f^2 -2gh}$. The net work done on the skier going up the ramp is equal to:
\end{itemize}
\begin{equation}
W^{net}&=\frac{1}{2}mv_t^2-\frac{1}{2}mv_i^2\\
&=\frac{1}{2}m(v_t^2-v_i^2) = \frac{1}{2}m(v_f^2-2gh -v_i^2)\\
&=\frac{1}{2}m(v_f^2-v_i^2)-mgh
\end{equation}
The net work done is also the sum of the work done by each of the forces acting on the skier as they slide up the ramp. The forces on the skier are the force of gravity, the force of friction, and the normal force. The normal force does no work, since it is always perpendicular to the displacement. The net work is thus the sum of the work done by the force gravity, $W_g$, and the work done by the force of friction, $W_f$, over the displacement corresponding to the length of the ramp:
\begin{equation}
W^{net}=W_g+W_f
\end{equation}
The work done by gravity is:
\begin{equation}
W_g = \vec F_g \cdot \vec d = (-mg\hat y) \cdot  (d_x\hat x + h \hat y) = -mgh
\end{equation}
where $\vec d$ is the displacement vector up the ramp (unknown horizontal distance, $d_x$, and vertical distance, $h$). We can now determine the work done by the force of friction:
\begin{equation}
W^{net}&=W_g+W_f\\
\frac{1}{2}m(v_f^2-v_i^2)-mgh &=  -mgh + W_f\\
\therefore W_f &= \frac{1}{2}m(v_f^2-v_i^2) = \frac{1}{2}(80 {\rm kg})((12 {\rm m/s})^2-(15 {\rm m/s})^2)=-3240 {\rm J}
\end{equation}
And we find that the force of friction did negative work (it reduced the kinetic energy of the skier).

\textbf{Discussion:} Over the course of the jump, the skier started at the bottom of the ramp with a given kinetic energy, then lost some of that energy going up the ramp (in the form of loss to friction and negative work done by gravity). During the airborne phase, gravity did positive work and the skier gained back some of the kinetic energy that they had lost going up the ramp. Thus the net work done by the force of friction is the difference in kinetic energies between the final landing point and the beginning of the ramp, because friction is the only force that did a net amount of (negative) work over the whole trajectory (gravity did no net work over the whole trajectory). This example shows how we can start to think about energy as something that is ``conserved'', which we will explore in more detail in the next chapter.
\end{framed}

\begin{framed}
\textbf{Solution}\\
\begin{itemize}
\item a. We want to find the work done by the applied force $\vec F$. We first need to find an expression for the magnitude of $\vec F$, based on the fact that the child is not accelerating. The forces on the child are:
\item $\vec F_g$, their weight, with magnitude $mg$.
\item $\vec F_T$, the tension in the rope, which changes with the angle, $\theta$.
\item $\vec F$, the applied force, which change in magnitude as the angle, $\theta$, changes.
\end{itemize}

The forces are illustrated in Figure~\ref{fig:workenergy:swingprobfbd}.

\begin{figure}[!htbp]
\centering
\includegraphics[width=0.3\linewidth]{files/swingprobfbd-ba84c7450092567037aea8b370051104.png}
\caption[]{A free-body diagram of the forces exerted on the child.}
\label{fig:workenergy:swingprobfbd}
\end{figure}

The child is moving at a constant speed, so the net force is equal to zero. The sum of the $x$ and $y$ components of the forces are equal to zero (Newton's Second Law):
\begin{equation}
\sum F_x &= F-F_T\sin\theta =0\\
\sum F_y &= F_T\cos\theta -mg = 0
\end{equation}
Rearranging these equations gives:
\begin{equation}
F&=F_T\sin\theta\\
mg&=F_T\cos\theta
\end{equation}
We want an expression for $F$ that does not depend on $F_T$ (since $F_T$ is unknown), so we can divide one equation by the other:
\begin{equation}
\frac{F}{mg} &= \frac{F_T\sin\theta}{F_T\cos\theta}=\tan\theta\\
\therefore F(\theta)&=mg\tan\theta
\end{equation}
where we indicated that the force $\vec F(\theta)$ depends on the angle $\theta$. The work done by the force, $\vec F$, is given by:
\begin{equation}
W_F=\int_A^B\vec F(\theta) \cdot d\vec l
\end{equation}
$d\vec l$ is the ``path element'' along part of the arc of circle over which the child moves, as illustrated in Figure~\ref{fig:workenergy:swingprobdl}. We have an expression for how $\vec F$ changes in magnitude as a function of the angle $\theta$, and it would thus be convenient to perform the integral over the angle $\theta$.

\begin{figure}[!htbp]
\centering
\includegraphics[width=0.2\linewidth]{files/swingprobdl-adf6f6b05a5bd3d9b8b20443ef022b01.png}
\caption[]{A path element along the circular trajectory of the swing.}
\label{fig:workenergy:swingprobdl}
\end{figure}

We can use polar coordinate, $(r,\theta)$, instead of cartesian coordinates to describe the displacement vector, $d\vec l$. If the vector subtends an arc on the circle that makes an infinitesimal angle, $d\theta$, as illustrated, then the length of the vector $d\vec l$ is given by:
\begin{equation}
dl = L d\theta
\end{equation}
where $L$ is the radius of the circle. The vector $d\vec l$ makes an angle $\theta$ with the horizontal, and thus with the vector, $\vec F$. The dot product between $\vec F$ and $d\vec l$ can thus be written as:
\begin{equation}
\vec F(\theta) \cdot d\vec l = Fdl\cos\theta=(mg\tan\theta)(Ld\theta)\cos\theta=mgL\sin\theta d\theta
\end{equation}
We can now write the integral for the work using limit that are based on the angle $\theta$, from $\theta=0$ to $\theta=\theta_1$:
\begin{equation}
W&=\int_0^{\theta_1}mgL\sin\theta d\theta\\
&=mgL[-\cos\theta]_0^{\theta_1}=mgL(1-\cos\theta_1)
\end{equation}

\begin{itemize}
\item b. We know that the work done by $\vec F$ is $W=mgL(1 -\cos\theta_1)$. So, we want to prove that $L(1 -\cos\theta_1)$ is equal to $h$. Expanding $L(1 -\cos\theta_1)$ gives:
\end{itemize}
\begin{equation}
L(1-\cos\theta_1)&=L-L\cos\theta_1
\end{equation}
This can be illustrated on a diagram, as in Figure~\ref{fig:workenergy:swingprobgeometry}, which shows that $h$ is equal to $L -L\cos\theta$.

\begin{figure}[!htbp]
\centering
\includegraphics[width=0.4\linewidth]{files/swingprobgeometry-0ebc2108f22acceabc86359d39684055.png}
\caption[]{A diagram showing the geometry of the problem}
\label{fig:workenergy:swingprobgeometry}
\end{figure}

\textbf{Discussion:} The net force acting on the mass is equal to zero, so the net work must be equal to zero. The two forces that do work on the mass are the applied force $\vec F$, and gravity. The work done by the applied force if $mgh$, so the work done by gravity must be $-mgh$.
\end{framed}

\subsection{Chapter 8 - Potential Energy and Conservation of Energy}

\subsubsection{Overview}\label{chapter:potentialecons}

In this chapter, we continue to develop the concept of energy in order to introduce a different formulation for Classical Physics that does not use forces. Although we can describe many phenomena using energy instead of forces, this method is completely equivalent to using Newton's Three Laws. As such, this method can be derived from Newton's formulation, as we will see. Because energy is a scalar quantity, for many problems, it leads to models that are much easier to develop mathematically than if one had used forces. The chapter will conclude with a presentation of the more modern approach, using ``Lagrangian Mechanics'', that is currently preferred in physics and forms the basis for extending our description of physics to the microscopic world (e.g. quantum mechanics).

\begin{framed}
\textbf{Learning Objectives}\\
\begin{itemize}
\item Understand the difference between conservative and non-conservative forces.
\item Understand how to define potential energy for a conservative force.
\item Understand how to use potential energy to calculate work.
\item Understand the definition of mechanical energy.
\item Understand how to use conservation of mechanical energy.
\item Understand how to apply the Lagrangian formulation in a simple case.
\end{itemize}
\end{framed}

\begin{framed}
\textbf{Think About It}\\
Three roller coaster carts start at position $x=0$,  where they are all at the same height (Figure~\ref{fig:potentialecons:rollercoaster}). All of the carts start with the same velocity. At $x_1$, which roller coaster cart will be moving the fastest?

All of the roller coasters end at ground level, at $x_2$. Which roller coaster cart will be moving the fastest at $x_2$? Will all of them make it to $x_2$? Who will get there first?  Assume that the roller coaster track is frictionless.

\begin{figure}[!htbp]
\centering
\includegraphics[width=0.9\linewidth]{files/rollercoaster-b808f7c2c4d1e7bad096bad327060c99.png}
\caption[]{Three roller coasters that start at the same height and end at the same height.}
\label{fig:potentialecons:rollercoaster}
\end{figure}

\begin{framed}
\textbf{Answer}\\
All of the carts start at the same height and with the same speed, so they all start with the same mechanical energy. At $x_1$, Cart B is moving the fastest. This is because Cart B is at the lowest height, so more of its gravitational potential energy has been converted into kinetic energy. At $x_2$, Cart A and Cart B will be moving at the same speed because they are at the same height. Cart C will only make it to $x_2$ if it is going fast enough at $x=0$, since it needs to clear bump that is higher than the height it started at. Cart B will be moving the fastest at $x_2$ because throughout the duration of the ride, it is at a lower height on average than Cart A, so its gravitational potential energy is converted to kinetic energy sooner and it gets to the end in less time.
\end{framed}
\end{framed}

\subsubsection{Conservative forces}\label{sec:potentialecons:conservative}

In Section~\ref{chap:workenergy}, we introduced the concept of work, $W$, done by a force, $\vec F(\vec r)$, acting on an object as it moves along a path from position $A$ to position $B$:
\begin{equation}
\label{eq:potentialecons:workdef}
W = \int_A^B \vec F(\vec r) \cdot d\vec l
\end{equation}
where $\vec F(\vec r)$ is a force vector that, in general, is different at different positions in space ($\vec r$). We can also say that $\vec F$ depends on position by writing $\vec F(\vec r)=\vec F(x,y,z)$, since the position vector, $\vec r$, is simply the vector $\vec r = x\hat x + y \hat y+ z\hat z$. That is, $\vec F(\vec r)$ is just a short hand notation for $\vec F(x,y,z)$, and $d\vec l$ is a (very) small segment along the particular path over which one calculates the work.

The above integral is, in general, difficult to evaluate, as it depends on the specific path over which the object moved. In Example~7.2 of Section~\ref{chap:workenergy}, we calculated the work done by friction on a crate that was slid across the floor along two different paths and indeed found that the work depended on the path that was taken. In Example~7.3 of the same chapter, we saw that the work done by the force of gravity when moving a box along two different paths did not depend on the path chosen\footnote{At least for those two paths that we tried in the example.}.

We call ``conservative forces'' those forces for which the work done only depends on the initial and final positions and not on the path taken between those two positions. ``Non-conservative'' forces are those for which the work done does depend on the path taken. The force of gravity is an example of a conservative force, whereas friction is an example of a non-conservative force.

This means that the work done by a conservative force on a ``closed path'' is zero; that is, \textbf{the work done by a conservative force on an object is zero if the object moves along a path that brings it back to its starting position.}
Indeed, since the work done by a conservative force only depends on the location of the initial and final positions, and not the path taken between them, the work has to be zero if the object ends in the same place as where it started (a possible path is for the object to not move at all).

Consider the work done by gravity in raising (displacement $\vec d_1$) and lowering (displacement $\vec d_2= -\vec d_1$) an object back to its starting position along a vertical path, as depicted in Figure~\ref{fig:potentialecons:gravityvertical}.

\begin{figure}[!htbp]
\centering
\includegraphics[width=0.2\linewidth]{files/gravityvertical-f825ebbf44bae636c197ad7b8f77b4e6.png}
\caption[]{An object that has moved up and back down.}
\label{fig:potentialecons:gravityvertical}
\end{figure}

The total work done by gravity on this particular closed path is easily shown to be zero, as the work can be broken up into the negative work done as the object moves up (displacement vector $\vec d_1$) and the positive work done as the object moves down (displacement vector $\vec d_2$):
\begin{equation}
W^{tot} = \vec F_g \cdot \vec d_1 + \vec F_g \cdot \vec d_2 = -mgd + mgd = 0
\end{equation}
In order to write the path integral of the force over a closed path, we introduce a new notation to indicate that the starting and ending position are the same:
\begin{equation}
\int_A^A \vec F(\vec r) \cdot d\vec l = \oint \vec F(\vec r) \cdot d\vec l
\end{equation}
The condition for a force to be conservative is thus:
\begin{equation}
\boxed{\oint \vec F(\vec r) \cdot d\vec l = 0}
\end{equation}
since this means that the work done over a closed path is zero. The condition for this integral to be zero can be found by Stokes' Theorem:

\begin{equation}
\oint \vec F(\vec r) \cdot d\vec l = \int_S \left[\left(\frac{\partial F_z}{\partial y}-\frac{\partial F_y}{\partial z}\right)\hat x+ \left(\frac{\partial F_x}{\partial z}-\frac{\partial F_z}{\partial x}\right)\hat y + \left(\frac{\partial F_y}{\partial x}-\frac{\partial F_x}{\partial y}\right)\hat z \right]\cdot d\vec A
\end{equation}
where the integral on the right is called a ``surface integral'' over the surface, $S$, enclosed by the closed path over which the work is being calculated. Don't worry, it is way beyond the scope of this text to understand this integral or Stokes' Theorem in detail! It is however useful in that it gives us the following conditions on the components of a force for that force to be conservative (by requiring the terms in parentheses to be zero):
\begin{equation}
\label{eq:potentialecons:conservative}
\frac{\partial F_z}{\partial y}-\frac{\partial F_y}{\partial z} &= 0 \nonumber\\
\frac{\partial F_x}{\partial z}-\frac{\partial F_z}{\partial x} &= 0\nonumber\\
\frac{\partial F_y}{\partial x}-\frac{\partial F_x}{\partial y} &= 0
\end{equation}
In general:

\begin{enumerate}
\item A force can be conservative if it only depends on position in space, and not speed, time, or any other quantity.
\item A force is conservative if it is constant in magnitude and direction.
\end{enumerate}

\begin{framed}
\textbf{Checkpoint}\\
You push a crate from point $A$ to point $B$ along a horizontal surface. Is the force you exert a conservative force?

\begin{figure}[!htbp]
\centering
\includegraphics[width=0.7\linewidth]{files/crateab-3d642d5d5d5261439832e71d66ccc56a.png}
\caption[]{You push a crate from A to B along any path.}
\label{fig:potentialecons:cratepath}
\end{figure}

\begin{enumerate}
\item Yes
\item No
\item Not enough information
\end{enumerate}

\begin{framed}
\textbf{Answer}\\
\begin{enumerate}[resume]
\item
\end{enumerate}
\end{framed}
\end{framed}

\begin{framed}
\textbf{Example 8.1}\\
Is the force of gravity on an object of mass $m$, near the surface of the Earth, given by:
\begin{equation}
\vec F(x,y,z) =0\hat x + 0\hat y -mg \hat z
\end{equation}
conservative? Note that we have defined the $z$ axis to be vertical and positive upwards.\}

\begin{framed}
\textbf{Solutions}\\
The force is expected to be conservative since it is constant in magnitude and direction. We can verify this using the conditions in (\ref{eq:potentialecons:conservative}):
\begin{equation}
\frac{\partial F_z}{\partial y}-\frac{\partial F_y}{\partial z} &= \frac{\partial }{\partial y}(-mg) - 0 &= 0\\
\frac{\partial F_x}{\partial z}-\frac{\partial F_z}{\partial x} &= 0 - \frac{\partial }{\partial x}(-mg) &= 0\\
\frac{\partial F_y}{\partial x}-\frac{\partial F_x}{\partial y} &= 0 - 0 &=0
\end{equation}
and the force is indeed conservative since all three conditions are zero.
\end{framed}
\end{framed}

\begin{framed}
\textbf{Example 8.2}\\
Is the following force conservative?
\begin{equation}
\vec F(x,y,z) = \frac{-k}{r^3}\vec r = \frac{-kx}{(x^2+y^2+z^2)^\frac{3}{2}}\hat x + \frac{-ky}{(x^2+y^2+z^2)^\frac{3}{2}}\hat y + \frac{-kz}{(x^2+y^2+z^2)^\frac{3}{2}}\hat z
\end{equation}
\begin{framed}
\textbf{Solution}\\
Since the force only depends on position, it \textit{could} be conservative, so we must check using the conditions from (\ref{eq:potentialecons:conservative}):
\begin{equation}
\frac{\partial F_z}{\partial y}-\frac{\partial F_y}{\partial z} &= \frac{\partial }{\partial y}\left(\frac{-kz}{(x^2+y^2+z^2)^\frac{3}{2}}\right)-\frac{\partial }{\partial z}\left( \frac{-ky}{(x^2+y^2+z^2)^\frac{3}{2}}\right)\\
&=\frac{3kz(2y)}{2(x^2+y^2+z^2)^\frac{5}{2}}-\frac{3ky(2z)}{2(x^2+y^2+z^2)^\frac{5}{2}} = 0\\
\frac{\partial F_x}{\partial z}-\frac{\partial F_z}{\partial x} &= \frac{\partial }{\partial z}\left(\frac{-kx}{(x^2+y^2+z^2)^\frac{3}{2}}\right)-\frac{\partial }{\partial x}\left( \frac{-kz}{(x^2+y^2+z^2)^\frac{3}{2}}\right)\\
&=\frac{3kx(2z)}{2(x^2+y^2+z^2)^\frac{5}{2}}-\frac{3kz(2x)}{2(x^2+y^2+z^2)^\frac{5}{2}} = 0\\
\frac{\partial F_y}{\partial x}-\frac{\partial F_x}{\partial y} &= \frac{\partial }{\partial x}\left(\frac{-ky}{(x^2+y^2+z^2)^\frac{3}{2}}\right)-\frac{\partial }{\partial y}\left( \frac{-kx}{(x^2+y^2+z^2)^\frac{3}{2}}\right)\\
&=\frac{3ky(2x)}{2(x^2+y^2+z^2)^\frac{5}{2}}-\frac{3kx(2y)}{2(x^2+y^2+z^2)^\frac{5}{2}} = 0
\end{equation}
where we used the Chain Rule to take the derivatives. Since all of the conditions are zero, the force is conservative. As we will see, the force represented here is similar mathematically to both the force that Newton introduced in his Universal Theory of Gravity, and the force introduced by Coulomb as the electric force, which are both conservative.
\end{framed}
\end{framed}

\subsubsection{Potential energy}

In this section, we introduce the concept of ``potential energy''. Potential energy is a scalar function of position that can be defined for any conservative force in a way to make it easy to calculate the work done by that force over any path. Since the work done by a conservative force in going from position $A$ to position $B$ does not depend on the particular path taken, but only on the end points, we can write the work done by a conservative force in terms of a ``potential energy function'', $U(\vec r)$, that can be evaluated at the end points:
\begin{equation}
\boxed{-W = - \int_A^B \vec F(\vec r) \cdot d\vec l = U(\vec r_B) - U(\vec r_A) = \Delta U}
\end{equation}
where we have have chosen to define the function $U(\vec r)$ so that it relates to the \textbf{negative} of the work done for reasons that will be apparent in the next section. Figure~\ref{fig:potentialecons:potentialpath} shows an example of an arbitrary path between two points $A$ and $B$ in two dimensions for which one could calculate the work done by a conservative force using a potential energy function.

\begin{figure}[!htbp]
\centering
\includegraphics[width=0.4\linewidth]{files/potentialpath-a6e90e6b49e93ae93bfdae02b8783087.png}
\caption[]{Illustration of calculating the work done by a conservative function along an arbitrary path by taking the difference in potential energy evaluated at the two endpoints, $-W=U(\vec r_B) - U(\vec r_A)$.}
\label{fig:potentialecons:potentialpath}
\end{figure}

Once we know the function for the potential energy, $U(\vec r)$, we can calculate the work done by the associated force along any path. In order to determine the function, $U(\vec r)$, we can calculate the work that is done along a path over which the integral for work is easy (usually, a straight line).

For example, near the surface of the Earth, the force of gravity on an object of mass, $m$, is given by:
\begin{equation}
\vec F_g = -mg \hat z
\end{equation}
where we have defined the $z$ axis to be vertical and positive upwards. We already showed in Example~8.1 that this force is conservative and that we can thus define a potential energy function. To do so, we can calculate the work done by the force of gravity over a straight vertical path, from position $A$ to position $B$, as shown in Figure~\ref{fig:potentialecons:gravitydl}.

\begin{figure}[!htbp]
\centering
\includegraphics[width=0.2\linewidth]{files/gravitydl-5d65187a9c22e9b4b1279569b7a8958e.png}
\caption[]{A vertical path for calculating the work done by gravity.}
\label{fig:potentialecons:gravitydl}
\end{figure}

The work done by gravity from position $A$ to position $B$ is:
\begin{equation}
W &= \int_A^B \vec F(\vec r) \cdot d\vec l\\
&= \int_{z_A}^{z_B} ( -mg \hat z) \cdot (dz \hat z) \\
&= -mg \int_{z_A}^{z_B} dz\\
&= -mg(z_B-z_A)
\end{equation}
By inspection, we can now identify the functional form for the potential energy function, $U(\vec r)$. We require that:
\begin{equation}
-W &= U(\vec r_B) - U(\vec r_A) = U(z_B) - U(z_A)
\end{equation}
where we replaced the position vector, $\vec r$, with the $z$ coordinate, since this is a one dimensional situation. Therefore:
\begin{equation}
-W=mg(z_B-z_A)&= U(z_B) - U(z_A)\\
\therefore U(z) &= mgz + C
\end{equation}
and we have found that, for the force of gravity near the surface of the Earth, one can define a potential energy function (by inspection), $U(z) = mgz +C$.

It is important to note that, since it is only the \textbf{difference} in potential energy that matters when calculating the work done, the potential energy function can have an arbitrary constant, $C$, added to it. Thus, \textbf{the value of the potential energy function is meaningless, and only differences in potential energy are meaningful and related to the work done on an object}. In other words, it does not matter where the potential energy is equal to zero, and by choosing $C$, we can therefore choose a convenient location where the potential energy is zero.

\begin{framed}
\textbf{Checkpoint}\\
When we found the work done by gravity, we defined positive $z$ to be upwards. If we instead chose positive $z$ to be downwards, how would the potential energy function be defined?

\begin{enumerate}
\item The potential energy function would be the same, $U(z)=mgz+C$.
\item The potential energy function would be the same but negative, $U(z)= -mgz+C$
\end{enumerate}

\begin{framed}
\textbf{Answer}\\
\begin{enumerate}[resume]
\item
\end{enumerate}
\end{framed}
\end{framed}

\begin{framed}
\textbf{Checkpoint}\\
Can an object have a negative potential energy?

\begin{enumerate}
\item Yes
\item No
\end{enumerate}

\begin{framed}
\textbf{Answer}\\
\begin{enumerate}
\item
\end{enumerate}
\end{framed}
\end{framed}

\begin{framed}
\textbf{Example 8.3}\\
Calculate the work done \textbf{by the force of gravity} when a box of mass, $m$, is moved from the ground up onto a table that is a distance $L$ away horizontally and $H$ vertically, as illustrated in Figure~\ref{fig:potentialecons:table}. How much work must be done by a person moving the box?

\begin{figure}[!htbp]
\centering
\includegraphics[width=0.5\linewidth]{files/table-86ad09cede94af5d2444e9e90a6ce65a.png}
\caption[]{A box moved from the ground up onto a table.}
\label{fig:potentialecons:table}
\end{figure}

\begin{framed}
\textbf{Solution}\\
Since the force of gravity is conservative, we can use the potential energy function given by:
\begin{equation}
U(z)=mgz+C
\end{equation}
to calculate the work done by the force of gravity when the box is moved. The work done by gravity will only depend on the change in height, $H$, as the potential energy function only depends on the $z$ coordinate of an object.  We can choose the origin of our coordinate system to be the ground and choose the constant $C=0$, so that the potential energy function at the starting position of the box is:
\begin{equation}
U(z_A=0) = mg(0)= 0
\end{equation}
The potential energy function when the box is on the table, with $z=H$, is given by:
\begin{equation}
U(z_B=H) = mgH
\end{equation}
The change in potential energy, $\Delta U = U(z_B) - U(z_A)$ is equal to the negative of the work done by gravity. The work done by gravity, $W_g$, is thus:
\begin{equation}
-W_g &=  U(z_B) - U(z_A) = mgH - 0\\
\therefore W_g &= -mgH
\end{equation}
which is the same as what we found in Example~7.3 of Section~\ref{chap:workenergy}. The work done by gravity is negative, as we found previously. This makes sense because gravity has a component opposite to the direction of motion.

The work done by a person, $W_p$, to move the box can easily be found by considering the net work done on the box. While the box is moving, only the person and gravity are exerting forces on the box, so those are the only two forces performing work. Since the box starts and ends at rest, the net work done on the box must be zero (no change in kinetic energy, recall the Work-Energy Theorem):
\begin{equation}
W^{net} = 0 &= W_g + W_p\\
\therefore W_p &= -W_g = mgH
\end{equation}
\textbf{Discussion:} We find that the person had to do positive work, which makes sense, since they had to exert a force with a component in the direction of motion (upwards). It is also interesting to note that it does not matter if the person exerted a constant force or whether they varied the force that they exerted on the box as they moved it: the amount of work done by the person is fixed to be the negative of the work done by gravity.
\end{framed}
\end{framed}

\begin{framed}
\textbf{Example 8.4}\\
The force exerted by a spring that is extended or compressed by a distance, $x$, is given by Hooke's Law:
\begin{equation}
\vec F(x) = -k x\hat x
\end{equation}
where the $x$ axis is defined to be co-linear with the spring and the origin is located at the rest position of the spring. Show that the force exerted by the spring onto an object is conservative and determine the corresponding potential energy function.

\begin{framed}
\textbf{Solution}\\
Since the force depends on position, it could be conservative, which we can check with the conditions from (\ref{eq:potentialecons:conservative}):
\begin{equation}
\frac{\partial F_z}{\partial y}-\frac{\partial F_y}{\partial z} &= 0 - 0 &= 0\\
\frac{\partial F_x}{\partial z}-\frac{\partial F_z}{\partial x} &= \frac{\partial }{\partial z}(-kx)) - 0&= 0\\
\frac{\partial F_y}{\partial x}-\frac{\partial F_x}{\partial y} &= 0 - \frac{\partial }{\partial y}(-kx)) &=0
\end{equation}
and the force is indeed conservative. To determine the potential energy function, let us calculate the work done by the spring from position $x_A$ to position $x_B$:
\begin{equation}
W &=\int_A^B \vec F(\vec r) \cdot d\vec l\\
&=\int_{x_A}^{x_B} (-kx\hat x) \cdot dx \hat x\\
&=\int_{x_A}^{x_B} (-kx)dx=\left[-\frac{1}{2}kx^2  \right]_{x_A}^{x_B}\\
&=-\left( \frac{1}{2}kx_B^2-\frac{1}{2}kx_A^2 \right)
\end{equation}
Again, comparing with:
\begin{equation}
-W &= U(\vec r_B) - U(\vec r_A) = U(x_B) - U(x_A)
\end{equation}
We can identify the potential energy for a spring:
\begin{equation}
U(x) = \frac{1}{2}kx^2 + C
\end{equation}
where, in general, the constant $C$ can take any value. If we choose $C=0$, then the potential energy is zero when the spring is at rest, although it is not important what choice is made. Note that in one dimension, the potential energy function is the negative of the anti-derivative of the function that gives the $x$ component of the force.
\end{framed}
\end{framed}

\begin{framed}
\textbf{Checkpoint}\\
A conservative force acts on an object that is initially at rest. No other forces act on the object. Does the object move in a way that increases its potential energy or decreases its potential energy?

\begin{enumerate}
\item Increases.
\item Decreases.
\item It depends on the choice of $C$ for the corresponding potential energy.
\end{enumerate}

\begin{framed}
\textbf{Answer}\\
\begin{enumerate}[resume]
\item
\end{enumerate}
\end{framed}
\end{framed}

\paragraph{Recovering the force from potential energy}\label{sec:potentialecons:forcefromu}

Given a (scalar) potential energy function, $U(\vec r)$, it is possible to determine the (vector) force that is associated with it. Take, for example, the potential energy from a spring (Example~8.4):
\begin{equation}
U(x) = \frac{1}{2}kx^2 + C
\end{equation}
As you recall from Example~8.4, to find this function (in one dimension), we took the $x$ component of the spring force and (effectively) found the negative of its anti-derivative, which we defined as the potential energy function:
\begin{equation}
F(x) &= -kx\\
U(x) &= -\int F(x) dx = \int (kx) dx = \frac{1}{2}kx^2+C\\
\therefore F(x) &= -\frac{d}{dx}U(x)
\end{equation}
Thus, the force can be obtained from the negative of the potential energy function, by taking its derivative with respect to position.

In three dimensions, the situation is similar, although the potential energy function (and the components of the force vector) will generally depend on all three position coordinates, $x$, $y$, and $z$. In three dimensions, the the three components of the force vector are given by taking the gradient of the negative of the potential energy function\footnote{As you may recall from Section~\ref{app:calculus}, the gradient is a vector that points towards the direction of maximal increase in a multi-variate function.}:
\begin{equation}
\vec F(\vec r) &= -\vec\nabla U(\vec r)=-\vec\nabla U(x,y,z)\nonumber\\
\therefore F_x(x,y,z) &= -\frac{\partial }{\partial x}U(x,y,z)\nonumber\\
\therefore F_y(x,y,z) &= -\frac{\partial }{\partial y}U(x,y,z)\nonumber\\
\therefore F_z(x,y,z) &= -\frac{\partial }{\partial z}U(x,y,z)
\end{equation}

\subsubsection{Mechanical energy and conservation of energy}

Recall the Work-Energy Theorem, which relates the net work done on an object to its change in kinetic energy, along a path from point $A$ to point $B$:
\begin{equation}
W^{net}=\Delta K = K_B - K_A
\end{equation}
where $K_A$ is the object's initial kinetic energy and $K_B$ is its final kinetic energy. Generally, the net work done is the sum of the work done by conservative forces, $W^C$, and the work done by non-conservative forces, $W^{NC}$:
\begin{equation}
W^{net}=W^C+W^{NC}
\end{equation}
The work done by conservative forces can be expressed in terms of changes in potential energy functions. For example, suppose that two conservative forces, $\vec F_1$ and $\vec F_2$, are exerted on the object. The work done by those two forces is given by:
\begin{equation}
W_1 &= -\Delta U_1\\
W_2 &= -\Delta U_2
\end{equation}
where $U_1$ and $U_2$ are the changes in potential energy associated with forces $\vec F_1$ and $\vec F_2$, respectively. We can re-arrange the Work-Energy Theorem as follows\footnote{This is why we defined potential energy as negative of the work; it becomes a positive term when we move it to the same side of the equation as the kinetic energy!}:
\begin{equation}
W^{net}=W^C+W^{NC}=-\Delta U_1 - \Delta U_2 +W^{NC} &= \Delta K\\
\therefore W^{NC} = \Delta U_1 + \Delta U_2 + \Delta K
\end{equation}
That is, the work done by non-conservative forces is equal to the sum of the changes in potential and kinetic energies. In general, we can use $\Delta U$ to represent the change in the total potential energy of the object. The total potential energy is the sum of the potential energies associated with each of the conservative forces acting on the object ($\Delta U = \Delta U_1 + \Delta U_2$ above). The above expression can thus be written in a more general form:
\begin{equation}
\boxed{W^{NC}=\Delta U + \Delta K}
\end{equation}
In particular, note that if there are no non-conservative forces doing work on the object:
\begin{equation}
\boxed{\Delta K + \Delta U = 0}\\
-\Delta U = \Delta K \quad\text{if no non-conservative forces}
\end{equation}
That is, the sum of the changes in potential and kinetic energies of the object is always zero. This means that if the potential energy of the object increases, then the kinetic energy of the object must decrease by the same amount.

We can introduce the ``mechanical energy'', $E$, of an object as the sum of the potential and kinetic energies of the object:
\begin{equation}
\boxed{E = U+K}
\end{equation}
If the object started at position $A$, with potential energy $U_A$ and kinetic energy $K_A$, and ended up at position $B$ with potential energy $U_B$ and kinetic energy $K_B$, then we can write the mechanical energy at both positions and its change $\Delta E$, as:
\begin{equation}
E_A &= U_A + K_A\\
E_B &= U_B + K_B\\
\Delta E &= E_B - E_A \\
&= U_B + K_B - U_A - K_A\\
\therefore \Delta E &= \Delta U + \Delta K
\end{equation}
Thus, the change in mechanical energy of the object is equal to the work done by non-conservative forces:
\begin{equation}
W^{NC} = \Delta U + \Delta K = \Delta E
\end{equation}
and if there is no work done by non-conservative forces on the object, then the mechanical energy of the object does not change:
\begin{equation}
\Delta E &= 0\quad\text{if no non-conservative forces}\\
\therefore E &= \text{constant}
\end{equation}
This is what we generally call the ``conservation of mechanical energy''. If there are no non-conservative forces doing work on an object, its mechanical energy is conserved (i.e. constant).

The introduction of mechanical energy gives us a completely different way to think about mechanics. We can now think of an object as having ``energy'' (potential and/or kinetic), and we can think of forces as changing the energy of the object.

\begin{framed}
\textbf{Checkpoint}\\
Is the value of an an object's mechanical energy meaningful, or is it only the difference in mechanical energy that is meaningful?

\begin{enumerate}
\item Yes, the value of the mechanical energy is meaningful. At any given time, an object will have a quantifiable amount of mechanical energy.
\item No, the value is not meaningful because the value of potential energy is arbitrary. Only differences in mechanical energy are meaningful.
\item No, the value is not meaningful because both the potential and kinetic energies are arbitrary. Their values will change depending on where you set the energy to be zero.
\item It depends on which conservative forces act on the object (and therefore what ``kind'' of potential energy the object has).
\end{enumerate}

\begin{framed}
\textbf{Answer}\\
\begin{enumerate}[resume]
\item
\end{enumerate}
\end{framed}
\end{framed}

We can also think of the work done by non-conservative forces as a type of change in energy. For example, the work done by friction can be thought of as a change in thermal energy (feel the burn as you rub your hand vigorously on a table!). If we can model the work done by non-conservative forces as a type of ``other'' energy, $-W^{NC}=\Delta E^{other}$, then we can state that:
\begin{equation}
\Delta E^{other} + \Delta U + \Delta K =0
\end{equation}
which is what we usually refer to as ``conservation of energy''. That is, the total energy in a system, including kinetic, potential and any other form (e.g. thermal, electrical, etc.) is constant unless some external agent is acting on the system.

We can always include that external agent in the system so that the total energy of the system is constant. The largest system that we can have is the Universe itself. Thus, the total energy in the Universe is constant and can only transform from one type into another, but no energy can ever be added or removed from the Universe.

\begin{framed}
\textbf{Olivia's Thoughts}\\
Here's an example that may help you understand the concept of external agents and energy conservation. Say we have a mass that hangs from a spring, so that the mass oscillates up and down like a yo-yo. If we define our system to include the spring, the mass, and gravity, energy will be conserved (the energy is transformed from potential energy to kinetic energy and back again).

Now, what if someone is holding the end of the spring and they start walking so that the whole system accelerates? Energy is not conserved because the system is gaining kinetic energy, seemingly out of nowhere. The system is being acted on by an \textit{external agent} (the person). If we expand our system so that it includes the spring, the mass, gravity, \textit{and the person}, energy is conserved. Instead of the kinetic energy ``coming out of nowhere'', we can see that it is actually coming from the person converting chemical energy in their body in order to move their muscles.

\begin{figure}[!htbp]
\centering
\includegraphics[width=0.4\linewidth]{files/externalagentex-66771caacaa3f48d61e159995851589f.png}
\caption[]{A person accelerates a mass and spring by walking. If the system does not include the person, energy is not conserved. If it does include the person, energy is conserved.}
\label{fig:potentialecons:externalagentx}
\end{figure}

But what if there's an external agent acting on our new system? We can keep ``zooming out'' to include more and more external sources in the definition of our system. If you kept zooming out, eventually you would reach the point where the whole Universe was included in your system. At this point, you can't zoom out any more. This means that, if the Universe is your system, energy must always be conserved because there can't be any external agents acting on the system.
\end{framed}

\begin{framed}
\textbf{Example 8.5}\\
\begin{figure}[!htbp]
\centering
\includegraphics[width=0.4\linewidth]{files/blockspring-5094bde59d2625146e6655fc2e2cf358.png}
\caption[]{A block is launched along a frictionless surface by compressing a spring by a distance $D$. The top panel shows the spring when at rest, and the bottom panel shows the spring compressed by a distance $D$ just before releasing the block.}
\label{fig:potentialecons:blockspring}
\end{figure}

A block of mass $m$ can slide along a horizontal frictionless surface. A horizontal spring, with spring constant, $k$, is attached to a wall on one end, while the other end can move, as shown in Figure~\ref{fig:potentialecons:blockspring}. A coordinate system is defined such that the $x$ axis is horizontal and the free end of the spring is at $x=0$ when the spring is at rest. The block is pushed against the spring so that the spring is compressed by a distance $D$. The block is then released. What speed will the block have when it leaves the spring?

\begin{framed}
\textbf{Solution}\\
This is again the same example that we saw in Section~\ref{chap:ApplyingNewtonsLaws} and Section~\ref{chap:workenergy}. We will show here that it is solved very easily using conservation of energy. The forces acting on the block are:

\begin{enumerate}
\item Weight, which does no work since it is perpendicular to the block's displacement.
\item The normal force, which does no work since it is perpendicular to the block's displacement.
\item The force from the spring, which is conservative and can be modelled with a potential energy $U(x)=\frac{1}{2}kx^2$, where $x$ is the position of the end of the spring.
\end{enumerate}

The block starts at rest at position $A$ ($x= -D$), where the spring is compressed by a distance $D$, and leaves the spring at position $B$ ($x=0$), where the spring is at its rest position.

At position $A$, the kinetic energy of the block is $K_A=0$ since the block is at rest, and the potential energy from the spring force of the block is $U_A=\frac{1}{2}kD^2$. The mechanical energy of the block at position $A$ is thus:
\begin{equation}
K_A&=0\\
U_A&=\frac{1}{2}kD^2\\
\therefore E_A &= U_A + K_A = \frac{1}{2}kD^2
\end{equation}
At position $B$, the spring potential energy of the block is zero (since the spring is at rest), and all of the energy is kinetic:
\begin{equation}
K_B&=\frac{1}{2}mv_B^2\\
U_B&=0\\
\therefore E_B &= U_B+K_B=\frac{1}{2}mv_B^2
\end{equation}
Since there are no non-conservative forces doing work on the block, the mechanical energies at $A$ and $B$ are the same:
\begin{equation}
W^{NC}&=\Delta E=E_B-E_A= 0\\
\therefore E_B&=E_A\\
\frac{1}{2}mv_B^2&= \frac{1}{2}kD^2\\
 v_B &= \sqrt{\frac{kD^2}{m}}
\end{equation}
as we found previously.
\end{framed}
\end{framed}

\begin{framed}
\textbf{Example 8.6}\\
\begin{figure}[!htbp]
\centering
\includegraphics[width=0.5\linewidth]{files/blockI-932a28c0a09a5e6695d86a4eee39dcf0.png}
\caption[]{A block slides down an incline before sliding on a flat surface and stopping.}
\label{fig:potentialecons:blockI}
\end{figure}

A block of mass $m$ is placed at rest on an incline that makes an angle $\theta$ with respect to the horizontal, as shown in Figure~\ref{fig:potentialecons:blockI}. The block is nudged slightly so that the force of static friction is overcome and the block starts to accelerate down the incline. At the bottom of the incline, the block slides on a horizontal surface.
The coefficient of kinetic friction between the block and the incline is $\mu_{k1}$, and the coefficient of kinetic friction between the block and horizontal surface is $\mu_{k2}$. If one assumes that the block started at rest a distance $L$ from the bottom of the incline, how far along the horizontal surface will the block slide before stopping?

\begin{framed}
\textbf{Solution}\\
This is the same problem we solved in this Figure~\ref{fig:applyingnewtonslaws:blockI}. In that case, we solved for the acceleration of the block using Newton's Second Law and then used kinematics to find how far the block went. We can solve this problem in a much easier way using conservation of energy.

It is still a good idea to think about what forces are applied on the object in order to determine if there are non-conservative forces doing work. In this case, the forces on the block are:

\begin{enumerate}
\item The normal force, which does no work, as it is always perpendicular to the motion.
\item Weight, which does work when the height of the object changes, which we can model with a potential energy function.
\item Friction, which is a non-conservative force, whose work we must determine.
\end{enumerate}

Let us divide the motion into two segments: (1) a segment along the incline (positions $A$ to $B$ in Figure~\ref{fig:potentialecons:blockI}), where gravitational potential energy changes, and (2), the horizontal segment from positions $B$ to position $C$ on the figure. We can then apply conservation of energy for each segment.

Starting with the first segment, we can choose the gravitational potential energy to be zero when the block is at the bottom of the incline. The block starts at a height $h=L\sin\theta$ above the  bottom of the incline. The gravitational potential energy for the beginning and end of the first segment are thus:
\begin{equation}
U_A &= mgL\sin\theta\\
U_B &= 0
\end{equation}
Since the block starts at rest, its kinetic energy is zero at position $A$, and if the speed of the box is $v_B$ at position $B$, we can write its kinetic energy at both positions as:
\begin{equation}
K_A &=0\\
K_B &= \frac{1}{2}mv_B^2
\end{equation}
The mechanical energy of the object at positions $A$ and $B$ is thus:
\begin{equation}
E_A &= U_A+K_A = mgL\sin\theta\\
E_B &= U_B+K_B = \frac{1}{2}mv_B^2\\
\Delta E &= E_B - E_A = \frac{1}{2}mv_B^2 - mgL\sin\theta
\end{equation}
Finally, since we have a non-conservative force, the force of kinetic friction, acting on the first segment, we need to calculate the work done by that force. We found in Example~6.2 that the force of friction had magnitude $f_k=\mu_{k1}N=\mu_{k1}mg\cos\theta$. Since the force of friction is anti-parallel to the displacement vector, which points down the incline and has length $L$, the work done by friction is:
\begin{equation}
W^{NC}=W_f = -f_kL=-\mu_{k1}mg\cos\theta L
\end{equation}
Applying conservation of energy along the first segment, we have:
\begin{equation}
W^{NC} &= \Delta E\\
-\mu_{k1}mg\cos\theta L &= \frac{1}{2}mv_B^2 - mgL\sin\theta\\
\therefore \frac{1}{2}mv_B^2 &= mgL\sin\theta-\mu_{k1}mg\cos\theta L
\end{equation}
Note that the above equation, in words, could be read as, ``the change in kinetic energy ($\frac{1}{2}mv_B^2$) is equal to the negative change in potential energy ($mgL\sin\theta$) minus the work done by friction ($\mu_{k1}mg\cos\theta L$)''. In other words, the block had potential energy, which was converted into kinetic energy and heat (the work done by friction can be thought of as thermal energy).

We now proceed in an analogous way for the second segment, from position $B$ to position $C$. The only force that can do work along this segment (of length $x$) is the force of kinetic friction, since both the weight and normal force are perpendicular to the displacement. There are no conservative forces doing work, so there is no change in potential energy. The initial kinetic energy is $K_B$ (from above), and the final kinetic energy, $K_C$, is zero. The change in mechanical energy is thus:
\begin{equation}
\Delta E &= E_C - E_B = K_C - K_B = -K_B\\
&=-\frac{1}{2}mv_B^2\\
&=- mgL\sin\theta+\mu_{k1}mg\cos\theta L
\end{equation}
where, in the last line, we used the result from the first segment. The work done by the force of friction along the horizontal segment of (undetermined) length $x$ is:
\begin{equation}
W^{NC}=W_f = -f_kx = -\mu_{k2} N x=-\mu_{k2} mg x
\end{equation}
Finally, we can find $x$ by setting the work done by non-conservative forces equal to the change in mechanical energy:
\begin{equation}
W^{NC} &= \Delta E\\
-\mu_{k2} mg x &=- mgL\sin\theta+\mu_{k1}mg\cos\theta L \\
\therefore x&= L\frac{1}{\mu_{k2}}\left(\sin\theta - \mu_{k1}\cos\theta\right)
\end{equation}
which is the same result that we obtained in Example~6.2.

\textbf{Discussion:} By using conservation of energy, we were able to model the motion of the block down the incline in a way that was much easier than what was done in Example~6.2. Furthermore, although we modelled friction as a non-conservative force doing work, we gained some insight into the idea that this could be thought of as an energy loss. In terms of energy, we would say that the block initially had gravitational potential energy, which was then converted into kinetic energy as well as thermal energy (in the heat generated by friction).
\end{framed}
\end{framed}

\subsubsection{Energy diagrams and equilibria}\label{sec:potentialecons:ediagrams}

We can write the mechanical energy of an object as:
\begin{equation}
E = K + U
\end{equation}
which will be a constant if there are no non-conservative forces doing work on the object. This means that if the potential energy of the object increases, then its kinetic energy must decrease by the same amount, and vice-versa.

Consider a block that can slide on a frictionless horizontal surface and that is attached to a spring, as is shown in Figure~\ref{fig:potentialecons:springE} (left side), where $x=0$ is chosen as the position corresponding to the rest length of the spring. If you push on the block so as to compress the spring by a distance $D$ and then release it, the block will initially accelerate because of the spring force in the positive $x$ direction until the block reaches the rest position of the spring ($x=0$ on the diagram). When it passes that point, the spring will exert a force in the opposite direction. The block will continue in the same direction and decelerate until it stops and turns around. It will then accelerate again towards the rest position of the spring, and then decelerate once the spring starts being compressed again, until the block stops and the motion repeats. We say that the block ``oscillates'' back and forth about the rest position of the spring.

We can describe the motion of the block in terms of its total mechanical energy, $E$. Its potential energy is given by:
\begin{equation}
U(x)=\frac{1}{2}kx^2
\end{equation}
On the right of Figure~\ref{fig:potentialecons:springE} is an ``Energy Diagram'' for the block, which allows us to examine how the total energy, $E$, of the block is divided between kinetic and potential energy depending on the position of the block. The vertical axis corresponds to energy and the horizontal axis corresponds to the position of the block.

The total mechanical energy, $E=25 {\rm J}$, is shown by the horizontal red line. Also illustrated are the potential energy function ($U(x)$ in blue), and the kinetic energy, ($K=E -U(x)$, in dotted black).

\begin{figure}[!htbp]
\centering
\includegraphics[width=1\linewidth]{files/springE-eaf9354703557107d14f7d6c630c1335.png}
\caption[]{Left: The block oscillates about the rest position of the spring, between $x= -D$ and $x=D$. Right: The energy diagram for the block. This diagram is for a spring with spring constant $k=1 {\rm N/m}$.}
\label{fig:potentialecons:springE}
\end{figure}

The energy diagram allows us to describe the motion of the object attached to the spring in terms of energy. A few things to note:

\begin{enumerate}
\item At $x=\pm D$, the potential energy is equal to $E$, so the kinetic energy is zero. The block is thus instantaneously at rest at those positions.
\item At $x=0$, the potential energy is zero, and the kinetic energy is maximal. This corresponds to where the block has the highest speed.
\item The kinetic energy of the block can never be negative\footnote{Remember, the kinetic energy is given by $K=\frac{1}{2}mv^2$. Since neither mass nor the value of $v^2$ can be negative, the kinetic energy of an object can never be negative.}, thus, the block cannot be located outside the range $[ -D,+D]$, and we would say that the motion of the block is ``bound''. The points between which the motion is bound are called ``turning points''.
\end{enumerate}

An analysis of the energy diagram tells us that the block is bound between the two turning points, which themselves are equidistant from the origin. When we initially compress the spring, we are ``giving'' the block ``spring potential energy''. As the block starts to move, the potential energy of the block is converted into kinetic energy as it accelerates and then back into potential energy as it decelerates.

\begin{framed}
\textbf{Checkpoint}\\
Calculate the positions of the turning points for the situation shown in Figure~\ref{fig:potentialecons:springE}. The total energy is $25 {\rm J}$ and the spring constant is $k=1 {\rm N/m}$.

\begin{framed}
\textbf{Answer}\\
$7.1 {\rm m}$
\end{framed}
\end{framed}

By looking at only the potential energy function, without knowing that it is related to a spring, we can come to the same conclusions; namely that the motion is bound as long as the total mechanical energy is not infinite. We call the point $x=0$ a ``stable equilibrium'', because it is a local minimum of the potential energy function. If the object is displaced from the equilibrium point, it will want to move back towards that point. This can also be understood in terms of the force associated with the potential energy function:
\begin{equation}
F = -\frac{d}{dx}U(x)
\end{equation}
The local minimum occurs where the derivative of the potential function is equal to zero. Thus, the \textbf{equilibrium point is given by the condition that the force associated with the potential is zero} ($x=0$ in the case of the potential energy from a spring). The equilibrium is a stable equilibrium because the force associated with the potential energy function ($F(x)= -kx$ for the spring) points towards the equilibrium point.

The potential energy function for an object with total mechanical energy, $E$, can be thought of as a little ``roller coaster'', on which you place a marble and watch it ``roll down'' the potential energy function. You can think of placing a marble where $U(x)=E$ and releasing it. The marble would then roll down the potential energy function, just as an actual marble would roll down a real slope, mimicking the motion of the object along the $x$ axis. This is illustrated in Figure~\ref{fig:potentialecons:potential} which shows an arbitrary potential energy function and a marble being placed at a location where the potential energy is equal to $E$.

\begin{figure}[!htbp]
\centering
\includegraphics[width=0.6\linewidth]{files/potential-e092acd863f57ea581351938e0c3506c.png}
\caption[]{Arbitrary potential energy function and illustration of visualizing a marble rolling down the function by placing the marble on the potential energy function at a point where $U(x)=E$.}
\label{fig:potentialecons:potential}
\end{figure}

The motion of the marble will be bound between the two points where the potential energy function is equal to $E$. When the marble is placed as shown, it will roll towards the left, just as if it were a real marble on a track. Since the potential energy is increasing as a function of $x$ at the point where we placed the marble, the force is in the negative $x$ direction (remember, the force is the negative of the derivative of the potential energy function). With the given energy, the marble would never be able to make it to point $D$, as it does not have enough energy to ``climb up the hill''. It would roll down, through point $C$, up to point $B$, down to point $A$, and then turn around where $U(x)=E$ and return to where it started.

Locations $A$ and $C$ on the diagram are stable equilibria, because if a marble is placed in one of those locations and nudged slightly, it will come back to the equilibrium point (or oscillate about that point). Points $B$ and $D$ are ``unstable equilibria'', because if the marble is placed there and nudged, it will not immediately come back to those points. Note that if the marble were placed at point $D$ and nudged towards the right, the motion of the marble would be unbound on the right, and it would keep going in that direction.

Now, say an object's potential energy is described by the function in Figure~\ref{fig:potentialecons:potential}, and the object has total energy $E$. The object's motion along the $x$ axis will be exactly the same as the projection of the marble's motion on the $x$ axis.

\begin{framed}
\textbf{Checkpoint}\\
A force, $F(x)$, acts on an object. The potential energy function, $U(x)$, associated with the force is given by $U(x)=a(x -6)^2(x -1)(x -3)+20 {\rm J}$, where $a$ is a positive constant. $U(x)$ is plotted in Figure~\ref{fig:potentialecons:potentialcheckpoint}. Use the ``marble'' method to determine the direction of the force at $x=5$. Confirm your answer by finding the value of the force , $F(x)$, at $x=5$.

\begin{figure}[!htbp]
\centering
\includegraphics[width=0.5\linewidth]{files/potentialcheckpoint-611b04b42e2cece8657bac4b1784030f.png}
\caption[]{A potential energy function $U(x)$. The $x$-axis represents the $x$ position and the $y$-axis represents the energy.}
\label{fig:potentialecons:potentialcheckpoint}
\end{figure}

\begin{enumerate}
\item $F(x=5)= -10a$
\item $F(x=5)=10a$
\item $F(x=5)=20a$
\item $F(x=5)= -20a$
\end{enumerate}

\begin{framed}
\textbf{Answer}\\
\begin{enumerate}[resume]
\item
\end{enumerate}
\end{framed}
\end{framed}

\subsubsection{Advanced Topic: The Lagrangian formulation of classical physics}

So far, we have seen that, based on Newton's Laws, one can formulate a description of motion that is based solely on the concept of energy. A lot of research was done in the eighteenth century to reformulate a theory of mechanics that would be equivalent to Newton's Theory but whose starting point is the concept of energy instead of the concept of force. This ``modern'' approach to classical mechanics is primarily based on the research by Lagrange and Hamilton.

Although it is beyond the scope of this text to go into the details of this formulation, it is worth taking a quick look in order to get a better sense of how physicists seek to generalize theories. It is also worth noting that the Lagrangian formulation is the method by which theories are developed for quantum mechanics and modern physics.

The Lagrangian description of a ``system'' is based on a quantity, $L$, called the ``Lagrangian'', which is defined as:
\begin{equation}
\boxed{L = K - U}
\end{equation}
where $K$ is the kinetic energy of the system, and $U$ is its potential energy. A ``system'' can be a rather complex collection of objects, although we will illustrate how the Lagrangian formulation is implemented for a single object of mass $m$ moving in one dimension under the influence of gravity. Let $x$ be the direction of motion (which is vertical) such that the potential and kinetic energies of the object are given by:
\begin{equation}
U(x) &= mgx\\
K(v_x) &= \frac{1}{2}mv_x^2\\
\therefore L(x,v_x) &= \frac{1}{2}mv_x^2 - mgx
\end{equation}
where we chose the potential energy to be zero at $x=0$, and $v_x$ is the velocity of the object.

In the modern formulation of classical mechanics, the motion of the system will be such that the following integral is minimized:
\begin{equation}
S = \int Ldt
\end{equation}
where $L$ can depend on time explicitly or implicitly (through the fact that position and velocity, on which the Lagrangian depends, are themselves time-dependent). The requirement that the above integral be minimized is called the ``Principle of Least Action''\footnote{The integral, $S$, is called the ``action'' of the system.}, and is thought to be the fundamental principle that describes all of the laws of physics. The condition for the action to be minimized is given by the Euler-Lagrange equation:
\begin{equation}
\boxed{\frac{d}{dt}\left(\frac{\partial L}{\partial v_x}\right)-\frac{\partial L}{\partial x} = 0}
\end{equation}
Thus, in the Lagrangian formulation, one first writes down the Lagrangian for the system, and then uses the Euler-Lagrange equation to obtain the ``equations of motion'' for the system (i.e. equation that give the kinematic quantities, such as acceleration, for the system).

Given the Lagrangian that we found above for a particle moving in one dimension under the influence of gravity, we can determine each term in the Euler-Lagrange equation:
\begin{equation}
\frac{\partial L}{\partial v_x} &= \frac{\partial }{\partial v_x}\left(\frac{1}{2}mv_x^2 - mgx \right)=mv_x\\
\therefore\frac{d}{dt}\left(\frac{\partial L}{\partial v_x}\right) &= \frac{d}{dt} (mv_x) = ma_x\\
\frac{\partial L}{\partial x}&= \frac{\partial }{\partial x}\left(\frac{1}{2}mv_x^2 - mgx\right) = -mg\\
\end{equation}
Putting these into the Euler-Lagrange equation:
\begin{equation}
\frac{d}{dt}\left(\frac{\partial L}{\partial v_x}\right)-\frac{\partial L}{\partial x} &= 0\\
(ma_x) - (-mg) &=0\\
ma_x&=-mg\\
\therefore a_x &= -g
\end{equation}
which is exactly equivalent to using Newton's Second Law (the second last step is equivalent to $F=ma$). In the Lagrangian formulation, we do not need the concept of force. Instead, we describe possible ``interactions'' by a potential energy function. That is why you may sometimes hear of physicists talking about the ``Weak interaction'' instead of the ``Weak force'' when they are talking about one of the four fundamental interactions (forces) of Nature. This is because, in the modern formulation of physics, one does not use the concept of force, and instead thinks of potential energy functions to model what we would call a force in the Newtonian approach.

Emmy Noether, a mathematician in the early twentieth century, proved a theorem that makes the Lagrangian formulation particularly aesthetic.
Noether's theorem states that for any symmetry in the Lagrangian, there exists a quantity that is conserved. For example, if the Lagrangian does not depend explicitly on time, then a quantity, which we call energy, is conserved\footnote{If the Lagrangian does not depend on time, then we can shift the system in time and the equations of motion would be unaffected. We say that the Lagrangian is symmetric, or unaffected, by changes in time.}.

The Lagrangian that we had above for a particle moving under the influence of gravity did not depend on time explicitly, and thus energy is conserved (gravitational potential energy is converted into kinetic energy and there are no non-conservative forces). If the Lagrangian did not depend on position, then a quantity that we call ``momentum''\footnote{See Section~\ref{chap:momentumandcm}.} would be conserved. In this case, momentum in the $x$ direction was not conserved because the Lagrangian depended on $x$ through the potential energy.

\begin{framed}
\textbf{Olivia's Thoughts}\\
We saw in this chapter that describing systems in terms of energy is often easier than describing them in terms of forces. The Lagrangian gives us a way to get the same information we would get from Newton's laws (like the acceleration, etc.), but using energy as a starting point. The Lagrangian method is really useful when we are looking at motion in multiple dimensions, or when we are describing complicated systems. Using the Lagrangian is actually really simple, and just like with forces, you can pretty much approach every problem the same way. Here are the basic steps to follow:

\begin{enumerate}
\item Find two expressions for your system: one for the potential energy ($U$) and one for the kinetic energy ($K$). This often ends up being the hardest step.
\item Write down the Lagrangian, $L=K -U$, using the expressions you just found.
\item Pick a coordinate. (In one dimension, this is trivial, but it will be important once you start working in multiple dimensions). The Euler-Lagrange equation was given to you as:
\end{enumerate}
\begin{equation}
\frac{d}{dt}\left(\frac{\partial L}{\partial v_x}\right)-\frac{\partial L}{\partial x} = 0
\end{equation}
because we are working in one dimension. You can actually pick whichever coordinate you are interested in. For example, if you were interested in the motion of your object in the $y$ direction, you would pick $y$ as your coordinate and write:
\begin{equation}
\frac{d}{dt}\left(\frac{\partial L}{\partial v_y}\right)-\frac{\partial L}{\partial y} = 0
\end{equation}
\begin{enumerate}[resume]
\item Now you just have to do what the equation above tells you to do, which is to start with your Lagrangian (your $L=K -U$ equation) and take a bunch of derivatives. If you try to just plug $L$ into the Euler-Lagrange equation and do all the derivatives at once, it can get confusing. I recommend finding the components separately. I like to start by taking the partial derivative with respect to velocity, $\frac{\partial L}{\partial v_y}$, then taking its derivative with respect to time. Next, I find $\frac{\partial L}{\partial y}$ and then put it all together.
\item That's it! When you've taken the derivatives (and simplified a bit), you'll have an ``equation of motion'' that gives you information about the motion of the object. You can then use this equation however you want!
\end{enumerate}
\end{framed}

\subsubsection{Summary}

A force is conservative if the work done by that force on a closed path is zero:
\begin{equation}
\oint \vec F(\vec r) \cdot d\vec l = 0
\end{equation}
Equivalently, the force is conservative if the work done by the force on an object moving from position $A$ to position $B$ does not depend on the particular path between the two points. The conditions for a force to be conservative are given by:
\begin{equation}
\frac{\partial F_z}{\partial y}-\frac{\partial F_y}{\partial z} &= 0 \nonumber\\
\frac{\partial F_x}{\partial z}-\frac{\partial F_z}{\partial x} &= 0\nonumber\\
\frac{\partial F_y}{\partial x}-\frac{\partial F_x}{\partial y} &= 0
\end{equation}
In particular, a force that is constant in magnitude and direction will be conservative. A force that depends on quantities other than position (e.g. speed, time) will not be conservative. The force exerted by gravity and the force exerted by a spring are conservative.

For any conservative force, $\vec F(\vec r)$, we can define a potential energy function, $U(\vec r)$, that can be used to calculate the work done by the force along any path between position $A$ and position $B$:
\begin{equation}
-W = - \int_A^B \vec F(\vec r) \cdot d\vec l = U(\vec r_B) - U(\vec r_A) = \Delta U
\end{equation}
where the change in potential energy function in going from $A$ to $B$ is equal to the negative of the work done in going from point $A$ to point $B$. We can determine the function $U(\vec r)$ by calculating the work integral over an ``easy'' path (e.g. a straight line that is co-linear with the direction of the force).

It is important to note that an arbitrary constant can be added to the potential energy function, because only differences in potential energy are meaningful. In other words, we are free to choose the location in space where the potential energy function is defined to be zero.

We can break up the net work done on an object as the sum of the work done by conservative ($W^C$) and non-conservative forces ($W^{NC}$):
\begin{equation}
W^{net}&=W^{NC}+W^{C}=W^{NC}-\Delta U
\end{equation}
where $\Delta U$ is the difference in the total potential energy of the object (the sum of the potential energies for each conservative force acting on the object).

The Work-Energy Theorem states that the net work done on an object in going from position $A$ to position $B$ is equal to the object's change in kinetic energy:
\begin{equation}
W^{net}&=\frac{1}{2}mv_B^2-\frac{1}{2}mv_A^2=\Delta K
\end{equation}
We can thus write that the total work done by non conservative forces is equal to the change in potential and kinetic energies:
\begin{equation}
W^{NC}=\Delta K+\Delta U
\end{equation}
In particular, if no non-conservative forces do work on an object, then the change in total potential energy is equal to the negative of the change in kinetic energy of the object:
\begin{equation}
-\Delta U=\Delta K
\end{equation}
We can introduce the mechanical energy, $E$, of an object as:
\begin{equation}
E = U+K
\end{equation}
The net work done by non-conservative forces is then equal to the change in the object's mechanical energy:
\begin{equation}
W^{NC}=\Delta E
\end{equation}
In particular, if no net work is done on the object by non-conservative forces, then the mechanical energy of the object does not change ($\Delta E=0$). In this case, we say that the mechanical energy of the object is conserved.

The Lagrangian description of classical mechanics is based on the Lagrangian, $L$:
\begin{equation}
L = K - U
\end{equation}
which is the difference between the kinetic energy, $K$, and the potential energy, $U$, of the object. The equations of motion are given by the Principle of Least Action, which leads to the Euler-Lagrange equation (written here for the case of a particle moving in one dimension):
\begin{equation}
\frac{d}{dt}\left(\frac{\partial L}{\partial v_x}\right)-\frac{\partial L}{\partial x} = 0
\end{equation}

\begin{framed}
\textbf{Important Equations}\\
\textbf{Conditions for a force to be conservative:}
\begin{equation}
\oint \vec F(\vec r) \cdot d\vec l = 0
\end{equation}
\begin{equation}
\frac{\partial F_z}{\partial y}-\frac{\partial F_y}{\partial z} &= 0 \nonumber\\
\frac{\partial F_x}{\partial z}-\frac{\partial F_z}{\partial x} &= 0\nonumber\\
\frac{\partial F_y}{\partial x}-\frac{\partial F_x}{\partial y} &= 0
\end{equation}
\textbf{Potential energy for a conservative force:}
\begin{equation}
\Delta U&=-W\\
U(\vec r_B) - U(\vec r_A)&= - \int_A^B \vec F(\vec r) \cdot d\vec l
\end{equation}

\textbf{Work-energy theorem:}
\begin{equation}
W^{net}&=\frac{1}{2}mv_B^2-\frac{1}{2}mv_A^2=\Delta K
\end{equation}
\textbf{Work:}
\begin{equation}
W^{net}&=W^{NC}+W^{C}=W^{NC}-\Delta U\\
W^{NC}&=\Delta K+\Delta U
\end{equation}
\textbf{Energy:}
\begin{equation}
E&=U+K\\
W^{NC}&=\Delta E
\end{equation}
\textbf{Lagrange:}
\begin{equation}
L = K - U\\
\frac{d}{dt}\left(\frac{\partial L}{\partial v_x}\right)-\frac{\partial L}{\partial x} = 0
\end{equation}
\end{framed}

\begin{framed}
\textbf{Important Definitions}\\
\begin{itemize}
\item \textbf{Conservative force:} A force that does no net work when exerted over a closed path.
\item \textbf{Potential energy:} A form of energy that an object has by virtue of its position in space. The potential energy is associated with a conservative force, which is exerted in the direction that lowers the potential energy of the object. SI units: ${\rm \left[{J}\right]}$. Common variable(s): $U$.
\end{itemize}
\end{framed}

\subsubsection{Thinking about the material}

\begin{framed}
\textbf{Reflect and research}\\
\begin{itemize}
\item When did Lagrange publish his theory of classical mechanics, and what was the name of the publication?
\item What is D'Alembert's contribution to the field of classical mechanics?
\item Who first proposed the Principle of Least Action, and when?
\item What is an example of a situation not already covered that you can describe where mechanical energy is conserved?
\item Under what symmetry is angular momentum conserved?
\item Think of three renewable energy sources and describe how they use conservation of energy to produce electricity.
\item What is a Rube Goldberg machine? Look up some videos of Rube Goldberg machines, and find the coolest one!
\end{itemize}
\end{framed}

\begin{framed}
\textbf{To try at home}\\
\begin{itemize}
\item Design a small catapult or slingshot that you can build using materials found at home. Describe how these machines work using conservation of energy.
\end{itemize}
\end{framed}

\begin{framed}
\textbf{To try in the lab}\\
\begin{itemize}
\item Propose an experiment to test that energy is conserved in a system where only gravity acts.
\item Simulate the launch of a space probe out of the solar system using a gravity assist.
\item Model and investigate the craters that are created when objects are dropped into a bed of sand.
\end{itemize}
\end{framed}

\subsubsection{Sample problems and solutions}

\paragraph{Problems}

\begin{framed}
\textbf{Problem 8.1}\\
\begin{figure}[!htbp]
\centering
\includegraphics[width=0.4\linewidth]{files/massdropspring-f73fe0e55835506b4148fcc4430499ee.png}
\caption[]{A ball is dropped from rest onto a vertical spring.}
\label{fig:potentialecons:ballspring}
\end{figure}
\end{framed}

\begin{framed}
\textbf{Problem 8.2}\\
A simple pendulum consists of a mass $m$ connected to a string of length $L$. The pendulum is released from an angle $\theta_0$ from the vertical. Use conservation of energy to find an expression for the velocity of the mass as a function of the angle.

\begin{figure}[!htbp]
\centering
\includegraphics[width=0.4\linewidth]{files/pendulumvelocity-130b363d2f37b84b8531059415885194.png}
\caption[]{A pendulum is released from rest an angle $\theta_0$ from the vertical.}
\label{fig:potentialecons:pendulumvel}
\end{figure}
\end{framed}

\begin{framed}
\textbf{Problem 8.3}\\
A block of mass $m$ sits on a frictionless horizontal surface. It is attached to a wall by a spring with a spring constant $k$. The mass is pushed so as to compress the spring and then it is released (Figure~\ref{fig:potentialecons:springmassoscillate}). Use the Lagrangian formalism to find an equation of motion for the mass/spring system (i.e. use the Lagrangian to determine the acceleration of the mass).

\begin{figure}[!htbp]
\centering
\includegraphics[width=0.4\linewidth]{files/massoscillating-022f0299125fb833304e7dba4fbb0cfe.png}
\caption[]{A mass attached to a spring oscillates about the rest position of the spring.}
\label{fig:potentialecons:springmassoscillate}
\end{figure}
\end{framed}

\paragraph{Solutions}

\begin{framed}
\textbf{Solution 8.1}\\
The two forces acting on the ball are gravity and the spring force. Both are conservative, so we can use conservation of mechanical energy. We will find the energy of the ball when it is at a height $h$ above the spring, and the energy of the ball when the spring is fully compressed. Then, we will use conservation of mechanical energy to determine the compression of the spring.

Remember that the total mechanical energy is the sum of the total potential energy and the kinetic energy, $E=U+K$. Let's call the initial position of the ball $A$ and the final position of the ball $B$. You will notice that we set up our coordinate system so that $y$ is positive upwards, with $y=0$ at the point where the ball comes into contact with the spring. We choose to define both the gravitational potential energy and spring potential energy so that they are zero at $y=0$.

Since the ball starts from rest, its kinetic energy is zero at position $A$. At this point, the ball is not touching the spring, so the potential energy from the spring force is zero. The mechanical energy of the ball at position $A$ is simply equal to its gravitational potential energy:
\begin{equation}
E_A&=U_A+K_A\\
E_A&=mgh
\end{equation}
At position $B$, the ball is again at rest, so the kinetic energy of the ball is zero. Now that the ball is in contact with the spring, it will experience a force from the spring that can be modelled with a potential energy $U(y)=\frac{1}{2}ky_1^2$, where $y_1$ is the distance between the rest position of the spring and its compressed length. At point $B$ ($y= -y_1$), the ball will have both spring and gravitational potential energy, so its mechanical energy at position $B$ is given by:
\begin{equation}
E_B&=U_B+K_B=U_B\\
U_B&=mg(-y_1)+\frac{1}{2}ky_1^2\\
E_B&=-mgy_1+\frac{1}{2}ky_1^2
\end{equation}
Since mechanical energy is conserved in this system (no non-conservative forces are doing work), we can now set $E_A=E_B$ and solve for $y_1$:
\begin{equation}
E_A&=E_B\\
mgh&=-mgy_1+\frac{1}{2}ky_1^2\\
0&=\frac{1}{2}ky_1^2-mgy_1-mgh\\
\end{equation}
where in the last line we rewrote the expression as a quadratic equation. We can solve for $y_1$ with the quadratic formula:
\begin{equation}
y_1=\frac{mg\pm\sqrt{(mg)^2-4(1/2k)(-mgh)}}{k}\\
y_1=\frac{mg\pm\sqrt{mg(mg+2kh)}}{k}
\end{equation}
We now have an expression for the amount the spring is compressed, $y_1$, in terms of our known values.
\end{framed}

\begin{framed}
\textbf{Solution 8.2}\\
We are going to find a general expression for the energy of the system, and then use this expression to find the velocity at any point. There are two forces acting on the mass:

\begin{enumerate}
\item The force of tension (from the string). This force is perpendicular to the direction of motion at any point, so it does no work on the mass.
\item The force of gravity, which has a potential energy function given by $U(y)=mgy$. We choose the gravitational potential energy to be zero when the pendulum hangs vertically (when $\theta=0$ and $y=0$).
\end{enumerate}

The mechanical energy of the mass is conserved, and at any point is given by the sum of its kinetic and its gravitational potential energies:
\begin{equation}
E=mgy+\frac{1}{2}mv^2
\end{equation}
We want to find the velocity as a function of $\theta$, so we need to write $y$ in terms of $\theta$. As you may recall from Figure~\ref{fig:workenergy:swingprob}, we saw that from the geometry of the problem, we can express the height of the mass as $y=L -L\cos\theta$, or $L(1 -\cos\theta)$, where $y$ is the height as measured from the bottom point of the motion. You can refer to Figure~\ref{fig:workenergy:swingprobgeometry} to refresh your memory. The energy at any point is then:
\begin{equation}
E=mgL(1-\cos\theta)+\frac{1}{2}mv^2
\end{equation}
Conservation of energy tells us that the total energy at any point must be the same as the initial energy. So, we can use our initial conditions to find the total energy of the system. The mass starts from rest (initial kinetic energy is zero) an angle $\theta_0$ above the vertical:
\begin{equation}
E&=mgL(1-\cos\theta)+\frac{1}{2}mv^2\\
E_{initial}&=mgL(1-\cos\theta_0)
\end{equation}
Now that we have found the total energy of the system, we can write our general expression for the energy of the system at any point:
\begin{equation}
E&=mgL(1-\cos\theta)+\frac{1}{2}mv^2\\
mgL(1-\cos\theta_0)&=mgL(1-\cos\theta)+\frac{1}{2}mv^2
\end{equation}
All that's left to do is simplify the expression and rearrange for $v$:
\begin{equation}
mgL(1-\cos\theta_0)&=mgL(1-\cos\theta)+\frac{1}{2}mv^2\\
gL(1-\cos\theta_0)-gL(1-\cos\theta)&=\frac{1}{2}v^2\\
gL-gL\cos\theta_0-gL+gL\cos\theta&=\frac{1}{2}v^2\\
gL(\cos\theta-\cos\theta_0)&=\frac{1}{2}v^2\\
\therefore v&=\sqrt{2gl(\cos\theta-\cos\theta_0)}
\end{equation}
\textbf{Discussion:} We can see from this expression that the speed will be maximized when $\cos\theta$ is maximized, which will occur when $\theta=0$ (when the pendulum is vertical). This is as we expected. We can also see that we will get an imaginary number if the magnitude of $\theta$ is greater than $\theta_0$, showing that the motion is constrained between $-\theta_0$ and $\theta_0$. Finally, we showed that the velocity of the pendulum does not depend on the mass!
\end{framed}

\begin{framed}
\textbf{Solution 8.3}\\
We are going to find an equation of motion of the system using the Lagrangian method. We choose to use a one dimension coordinate system, with the $x$ axis defined to be co-linear with the spring, positive in the direction where the spring is extended, and set the origin to be located at the rest position of the spring. The kinetic energy and potential energy of the mass are given by
\begin{equation}
K&=\frac{1}{2}mv_x^2\\
U&=\frac{1}{2}kx^2
\end{equation}
since the only force exerted on the mass that can do work is the force from the spring. We have chosen the potential energy to be zero at $x=0$. The Lagrangian for this system is:
\begin{equation}
L&=K-U\\
L&=\frac{1}{2}mv_x^2-\frac{1}{2}kx^2
\end{equation}
The Euler-Lagrange equation in one dimension is:
\begin{equation}
\frac{d}{dt}\left(\frac{\partial L}{\partial v_x}\right)-\frac{\partial L}{\partial x} = 0
\end{equation}
We can calculate the terms of the Euler-Lagrange equation:
\begin{equation}
\frac{\partial L}{\partial v_x}&=\frac{\partial }{\partial v_x}\left(\frac{1}{2}mv_x^2-\frac{1}{2}kx^2\right)\\
&=mv_x\\
\therefore \frac{d}{dt}\left(\frac{\partial L}{\partial v_x}\right)&=\frac{d}{dt}(mv_x)\\
&=ma_x\\
\textrm{and}\qquad \frac{\partial L}{\partial x}&=\left(\frac{1}{2}mv_x^2-\frac{1}{2}kx^2\right)\\
&=-kx
\end{equation}
and then put them together to get:
\begin{equation}
\frac{d}{dt}\left(\frac{\partial L}{\partial v_x}\right)-\frac{\partial L}{\partial x} &= 0\\
\therefore ma_x&=-kx\\
\end{equation}
We can see that this equation of motion is equivalent to Newton's Second Law.
\end{framed}

\subsection{Chapter 9 - Gravity}

\subsubsection{Overview}\label{chapter:gravity}

In previous chapters, we have so far learned about Newton's Theory of Classical Mechanics, which allowed us to model the motion of an object based on the forces acting on the object. In this chapter, we present the theories that describe the force of gravity itself. We will see several theories of gravity and focus primarily on Newton's Universal Theory of Gravity.

\begin{framed}
\textbf{Learning Objectives}\\
\begin{itemize}
\item Understand Kepler's Laws.
\item Understand Newton's Universal Theory of Gravity.
\item Understand Gauss's law and the gravitational field.
\item Understand how to use energy to describe orbits.
\item Understand how Einstein's General Theory of Relativity differs from Newton's theory of gravity.
\end{itemize}
\end{framed}

\begin{framed}
\textbf{Think About It}\\
A person stands on a scale at the top of Mount Logan, the tallest mountain in Canada. How will her measured weight compare to her weight at sea level?\}

\begin{enumerate}
\item It will be slightly less than her weight at sea level.
\item It will be equal to her weight at sea level.
\item It will be slightly more than her weight at sea level.
\end{enumerate}

\begin{framed}
\textbf{Answer}\\
\begin{enumerate}
\item
\end{enumerate}
\end{framed}
\end{framed}

\subsubsection{Kepler's Laws}

Although humans have long been fascinated by the motion of objects in the sky, it was Johannes Kepler, in the early seventeenth century, that was the first to write down quantitative rules that described the motion of planets around the Sun. His theory was based on the extensive and detailed observations recorded by Tycho Brahe in the late sixteenth century.

Kepler proposed three laws that describe all of the data that Tycho Brahe had collected about planetary motion:

\begin{enumerate}
\item The path of a planet around the Sun is described by an ellipse with the Sun at one of its foci.
\item All planets move in such a way that the area swept by a line connecting the planet and the Sun in a given period of time is constant.
\item The ratio between the orbital periods, $T$, of two planets squared is equal to the ratio of the semi-major axes, $s$, of their orbits cubed:
\end{enumerate}
\begin{equation}
\left(\frac{T_1}{T_2}\right)^2=\left(\frac{s_1}{s_2}\right)^3
\end{equation}
We examine these three laws in more detail in the sections that follow. It should also be noted that, even though Kepler's Laws were derived for planets orbiting the Sun, they apply to any body that is orbiting any other body under the influence of gravity\footnote{In fact, they apply for any two bodies orbiting each other if the force between them is an ``inverse-square'' law, such as the gravitational and electric forces.}.

\paragraph{Kepler's First Law}

Kepler noticed that the motion of all planets followed the path of an ellipse with the Sun located at one of its foci. Figure~\ref{fig:gravity:ellipse} shows a diagram of an ellipse, along with its two foci, its semi-major axis, $s$, its semi-minor axis, $b$, and its eccentricity, and the distance between a focus and the centre of the ellipse, $c$. We define the eccentricity, $e$ of the ellipse as the ratio $e=\frac{c}{s}$. The eccentricity is a measure of how far a focus is from the centre of the ellipse. A larger eccentricity thus corresponds to a ``flatter'' ellipse. Note that a circle is just a special case of an ellipse, with both foci located at the centre of the circle. The eccentriciy of a circle is thus 0.

\begin{figure}[!htbp]
\centering
\includegraphics[width=0.5\linewidth]{files/ellipse-b667a65c43d5aca4ae37d847a50bedd1.png}
\caption[]{A ellipse, showing its two foci, its semi-major axis, $s$, its semi-minor axis, $b$, and the distance between a focus and the centre of the ellipse, $c$.}
\label{fig:gravity:ellipse}
\end{figure}

The sun is located at one of the foci. The point of closest approach to the Sun is called the ``perihelion'' of the orbit (or ``perigee'' if the orbit is around the Earth), and the point furthest from the Sun is called the ``aphelion'' of the orbit (or ``apogee'' if the orbit is around the Earth), as shown in Figure~\ref{fig:gravity:perigeeapogee}.

\begin{figure}[!htbp]
\centering
\includegraphics[width=0.7\linewidth]{files/periapogee-c6a3c9e74d40c0700ad95bdce1e3e6d3.png}
\caption[]{The orbit of the Earth around the Sun, showing the perihelion and aphelion, and the orbit of the Moon around the Earth, showing the perigee and the apogee. (Not to scale.)}
\label{fig:gravity:perigeeapogee}
\end{figure}

\begin{framed}
\textbf{Checkpoint Order the ellipses from smallest eccentricity to largest eccentricity.}\\
\begin{figure}[!htbp]
\centering
\includegraphics[width=0.7\linewidth]{files/eccellipses-23574212dfb6295ea0c1558daf1b860e.png}
\caption[]{Three ellipses, each with a different eccentricity.}
\label{fig:gravity:eccellipses}
\end{figure}

\begin{framed}
\textbf{Answer}\\
$A<C<B$
\end{framed}
\end{framed}

\paragraph{Kepler's Second Law}

Kepler's Second Law is really a statement about the speed of a planet in an elliptical orbit. It states that the area swept by a line connecting the planet and the Sun in a given period of time is fixed. This is illustrated in Figure~\ref{fig:gravity:ellipse2}, which shows the elliptical orbit of a planet around the Sun located at one of the foci, and the area swept out when the planet goes from $A$ to $B$ and from $C$ to $D$.

\begin{figure}[!htbp]
\centering
\includegraphics[width=0.45\linewidth]{files/ellipse2-0f794ab73ee184ae96b7cea9e0ddef9e.png}
\caption[]{Illustration of Kepler's Second Law, showing the area that is ``swept'' by a planet in a fixed period of time.}
\label{fig:gravity:ellipse2}
\end{figure}

Kepler's Second Law states that the two areas that are shown by the greyed out sections in the figure are the same if the planet took the same amount of time to travel between points $A$ and $B$ as it did to travel between points $C$ and $D$. Because the points $C$ and $D$ are further away from the Sun than points $A$ and $B$, the distance between points $C$ and $D$ must be smaller than the distance between points $A$ and $B$ for the two areas to be the same. This, in turn, implies that the planet must be moving slower between $C$ and $D$ than between points $A$ and $B$. The speed of a planet is thus greatest at the perihelion and smallest at the aphelion. As we will see in a later chapter, Kepler's Second Law is equivalent to the statement that the angular momentum of the planet about the Sun is conserved.

\begin{framed}
\textbf{Checkpoint}\\
Based on Kepler's second law, what can you say about the speed of a planet in a \textbf{circular} orbit?

\begin{framed}
\textbf{Answer}\\
The speed of the planet is constant.
\end{framed}
\end{framed}

\paragraph{Kepler's Third Law}

Kepler's Third Law is quantitative and relates the orbital periods ($T$) and the semi-major axes ($s$) between any two planets in orbit around the Sun:
\begin{equation}
\left(\frac{T_1}{T_2}\right)^2=\left(\frac{s_1}{s_2}\right)^3
\end{equation}
We can re-arrange this relation so that all of the quantities related to one planet are on the same side of the equal sign:
\begin{equation}
\frac{T_1^2}{s_1^3}=\frac{T_2^2}{s_2^3}=\text{constant}
\end{equation}
In other words, the ratio between the orbital period squared and the semi-major axis cubed is a constant, independent of the particular planet. In Example~9.2, we will use Newton's Universal Theory of Gravity to evaluate the constant.

\begin{framed}
\textbf{Checkpoint}\\
An object is in a circular orbit with radius $r$ and has an orbital speed $v$. If you double the radius of the circular orbit, what will be the value of the orbital speed?\}

\begin{enumerate}
\item $2v$
\item $8v$
\item $\sqrt{8}v$.
\item $\frac{1}{\sqrt{2}}v$
\end{enumerate}

\begin{framed}
\textbf{Answer}\\
\begin{enumerate}[resume]
\item
\end{enumerate}
\end{framed}
\end{framed}

\subsubsection{Newton's Universal Theory of Gravity}

Newton supposedly gained insight into the gravitational force by observing an apple falling from a tree and concluding that if it is the same force that makes apples fall at sea level and at the top of a mountain, perhaps that force can be exerted all the way up to the moon. It is rather remarkable that Newton was able to make the connection between falling apples and the motion of the moon around the Earth to find a single theory to describe both situations.

We should be clear that the theory of gravity is a different theory than Newton's ``Laws of Motion'' (Newton's Three Laws). The Laws of Motion introduce the concepts of force and inertial mass, and prescribe how to use those concepts in order to model motion using kinematics. Newton's Universal Theory of Gravity is a theory that describes the force of gravity that two bodies with (gravitational) mass exert on each other.

Newton's Universal Theory of Gravity states that if two bodies, with masses $M_1$ and $M_2$, located at positions $\vec r_1$ and $\vec r_2$, respectively, are separated by a distance, $r$, then $M_2$ will exert an attractive force on $M_1$, $\vec F_{12}$, given by:
\begin{equation}
\vec F_{12}=-G\frac{M_1M_2}{r^2}\hat r_{21}
\end{equation}
where $\hat r_{21}$ is the unit vector from $M_2$ to $M_1$:
\begin{equation}
\vec r_{21} &= \vec r_2 - \vec r_1\\
\hat r_{21} &= \frac{1}{r} \vec r_{21}
\end{equation}
as shown in Figure~\ref{fig:gravity:gvectors}. $\vec F_{12}$ should be read as ``the force on body 1 from body 2''. $G=6.67\times 10^{ -11} {\rm Nm^2/kg^2}$ is Newton's Universal Constant of Gravity. It should be noted that Newton's theory for the force of gravity written in this form only applies to either point masses (separated by a distance $r$) or spherical bodies whose centres are separated by a distance $r$ that is larger than the radius of either sphere.

\begin{figure}[!htbp]
\centering
\includegraphics[width=0.4\linewidth]{files/gvectors-91e8672d8ac9e04278a36af498a9363f.png}
\caption[]{Illustration of the vectors involved in Newton's Universal Theory of Gravity.}
\label{fig:gravity:gvectors}
\end{figure}

Originally, Newton argued that the force of gravity would be proportional to the masses of the bodies, and inversely proportional to the square of the distance between them:
\begin{equation}
F_{12}\propto \frac{M_1M_2}{r^2}
\end{equation}
and $G$ is simply the constant of proportionality.

Because of Newton's Third Law, body 1 exerts a force on body 2 that is equal in magnitude but opposite in direction:
\begin{equation}
\vec F_{12} = -\vec F_{21}
\end{equation}

\begin{framed}
\textbf{Example 9.1}\\
Calculate the magnitude of the force of gravity between yourself and a person standing $50 {\rm cm}$ from you and compare that to your weight at the surface of the Earth (the force of gravity exerted by the Earth on you).\}

\begin{framed}
\textbf{Solution}\\
If we assume that the two people have a mass of $50 {\rm kg}$, the force of gravity exerted by one on the other, if they are separated by $50 {\rm cm}$, is given by:
\begin{equation}
F&=G\frac{M_1M_2}{r^2}\\
&=(6.67\times 10^{-11} {\rm Nm^2/kg^2})\frac{(50 {\rm kg})(50 {\rm kg})}{(0.5 {\rm m})^2}\\
&=6.67\times 10^{-7} {\rm N}
\end{equation}
This is a very small force, compared to their weight, $F_g$:
\begin{equation}
F_g&=M_1g\\
&=(50 {\rm kg})(9.8 {\rm N/kg})\\
&=490 {\rm N}
\end{equation}
which is approximately 700 million times bigger.

\textbf{Discussion:} The force of gravity is a very weak force when compared to other forces in Nature, such as the electric force between two charges. Newton's Universal Constant of Gravity is very small, so the force of gravity between two objects is very small unless either of the masses involved are very large, or the distance between them is very small. In general, when modelling the motion of objects on the Earth, it is generally safe to ignore the forces of gravity between objects and only include their weight (the force of gravity from the Earth).
\end{framed}
\end{framed}

\begin{framed}
\textbf{Checkpoint}\\
The radius of the Earth is 6371 \{{\textbackslash}rm km\}. What is the order of magnitude of the Earth's mass?\}

\begin{enumerate}
\item $10^{24} {\rm kg}$
\item $10^{18} {\rm kg}$
\item $10^{19} {\rm kg}$
\item $10^{21} {\rm kg}$
\item Not enough information.
\end{enumerate}

\begin{framed}
\textbf{Answer}\\
\begin{enumerate}
\item
\end{enumerate}
\end{framed}
\end{framed}

\begin{framed}
\textbf{Example 9.2}\\
Determine the constant in Kepler's Third Law for planets orbiting the Sun, namely the value of the ratio:
\begin{equation}
\frac{s^3}{T^2}
\end{equation}
where $s$ is the semi-major axis and $T$ is the orbital period.

\begin{framed}
\textbf{Solution}\\
Since Kepler's Third Law holds for any body orbiting the Sun, we can easily determine the ratio by considering a planet of mass $m$ in a circular orbit of radius $R$ centred about the Sun. The semi-major axis of the orbit is equal to the radius of the orbit for a circular orbit ($s=R$).

If the planet is in a circular orbit about the Sun, its speed, $v$, will be constant, by Kepler's Second Law, and it will thus be executing uniform circular motion. The only force exerted on the planet is the force of gravity exerted by the Sun. Thus the force of gravity must be equal to the mass of the planet times its radial (centripetal) acceleration, $a_R$, which is given by:
\begin{equation}
a_R=\frac{v^2}{R}
\end{equation}
Newton's Second Law for the planet can be written as:
\begin{equation}
\sum F = F_g &= ma_R\\
G\frac{Mm}{R^2}&=m\frac{v^2}{R}\\
G\frac{M}{R}&=v^2
\end{equation}
where $M$ is the mass of the Sun. The speed of the planet is given by the circumference of the orbit divided by the orbital period $T$, since it is constant:
\begin{equation}
v=\frac{2\pi R}{T}
\end{equation}
Re-arranging the equation from Newton's Second Law:
\begin{equation}
G\frac{M}{R}&=v^2\\
G\frac{M}{R}&=\frac{4\pi^2 R^2}{T^2}\\
\therefore \frac{R^3}{T^2}&=G\frac{M}{4\pi^2}
\end{equation}
Thus, we find that the ratio of the cube of the orbital radius to the period squared is a constant that depends only on the mass of the Sun, which will then be the same for all planets (as it does not depend on, say, the mass of the planet that we considered).

\textbf{Discussion}: The relation above can, for example, be used to determine the mass of the Sun, since we can use geometry to determine the semi-major axis for the orbit of a planet, as Kepler did with the data from Tycho Brahe.
\end{framed}
\end{framed}

\begin{framed}
\textbf{Example 9.3}\\
The acceleration due to Earth's gravity depends on the force that the Earth exerts on an object. Using the Earth's mass and radius, determine the acceleration of falling objects due to Earth's gravity at the surface of the Earth. Also, determine the altitude where the acceleration due to Earth's gravity is half of that at the Earth's surface.

\begin{framed}
\textbf{Solution}\\
We can find the acceleration due to Earth's gravity by determining the acceleration of a mass $m$ upon which gravity is the only acting force. In other words, we model an object that is in free-fall a distance $R$ away from the centre of the Earth. Newton's Second Law can be used in one dimension (corresponding to the direction that connects the falling mass to the centre of the Earth):
\begin{equation}
\sum F &= G\frac{Mm}{R^2}=ma\\
\therefore a&=G\frac{M}{R^2}
\end{equation}
where $M=5.97\times 10^{24} {\rm kg}$ is the mass of the Earth. At the surface of the Earth, $R=R_\oplus=6.371\times 10^{6} {\rm m}$:
\begin{equation}
a&=G\frac{M}{R_\oplus^2}=(6.67\times 10^{-11} {\rm Nm^2/kg^2})\frac{(5.97\times 10^{24} {\rm kg})}{(6.371\times 10^{6} {\rm m})^2}\\
&=9.81 {\rm m/s^2}
\end{equation}
which, of course, is the value of $g$ that we have been using so far for the acceleration due to gravity near the surface of the Earth. To find the altitude at which this is reduced by half, we first find the value of $R$ that corresponds to this acceleration:
\begin{equation}
\frac{1}{2}G\frac{M}{R_\oplus^2}&=G\frac{M}{R^2}\\
\therefore R &=\sqrt{2}R_\oplus = 9.0\times 10^{6} {\rm m}
\end{equation}
which corresponds to an altitude of $h=R -R_\oplus=2640 {\rm km}$, well above the Earth's atmosphere.

\textbf{Discussion:} The acceleration of falling objects decreases as one moves further from the centre of the Earth. It is thus an approximation to assume that $g$ is a constant, although in most cases this is a very good approximation. In practice, the value of $g$ will depend both on the distance from the centre of the Earth and the composition (density) of the material in the Earth's crust below where $g$ is being measured. Precise measurements of $g$ have bee used to determine the composition of the Earth's crust and to search for mineral and oil deposits.
\end{framed}
\end{framed}

\paragraph{Weight and apparent weight}

You have probably seen images of astronauts floating around the International Space Station (ISS) or other orbiting vessels, and heard of the term ``weightlessness''  to describe their motion. The ISS is in orbit at an altitude of approximately $400 {\rm km}$, where the force of Earth's gravity is far from negligible (in Example~9.3 we showed that one needs to go to an altitude of $2640 {\rm km}$ for the force to be reduced by half of that at the surface of the Earth). The contradiction between being weightless and the fact that weight is not zero is resolved by understanding that the popular term ``weightless'' is imprecise from a physics perspective.

The correct term to use from a physics perspective is to say that the \textit{apparent weight} of the astronauts is zero when they are floating around. Weight is the magnitude of the force of gravity exerted by the Earth. Apparent weight is, for example, the force that one measures when standing on a spring scale, which is equal to the normal force exerted by the spring scale on the person. Apparent weight could also be determined by the tension in a string from which a person is suspended. The apparent weight is the sum of the forces exerted on a person minus the force of gravity. If gravity is the only force exerted on a person (or object), that person's apparent weight is zero, which is what is popularly called being weightless.

One way to feel weightless is when you are in free-fall (e.g. the first few seconds of a parachute jump from an airplane). One can think of being in orbit as continuously falling towards the centre of the Earth, but with an initial velocity in a direction such that you never actually collide with the Earth. The feeling of weightlessness will occur any time that the only force exerted on you is the force of gravity. If you are in a spacecraft in any orbit and the only force on the spacecraft is from gravity (i.e. no rockets or wings are exerting any forces), then everything in the spacecraft will have the same acceleration, since gravity is the only force acting on anything in the spacecraft, and it will appear that everything is just floating. To an outside observer, it would be clear that the spacecraft and its contents are all accelerating.

\subparagraph{Effects of Earth's rotation}

Earth's rotation affects the apparent weight of objects near the Earth's surface. Consider a person standing on a spring scale at the North pole (top free-body diagram in Figure~\ref{fig:gravity:apparentweight}). The only two forces exerted on the person are their weight, $\vec F_g$, and the normal force, $\vec N$, exerted by the spring scale. Since the person is not accelerating, the normal force and the weight have the same magnitude and opposite directions. The scale will thus read the actual weight of the person\footnote{The weight that is displayed on the scale is equal in magnitude to the normal force exerted by the scale on the person. It is the reaction force to that normal force.}.

\begin{figure}[!htbp]
\centering
\includegraphics[width=0.4\linewidth]{files/apparentweight-1bb883d1a3842318caf1d98114381ea2.png}
\caption[]{The apparent weight, given by the normal force, is different at the Earth's equator because a person's acceleration is non-zero as they rotate with the Earth.}
\label{fig:gravity:apparentweight}
\end{figure}

Consider, instead, a person standing on a spring scale at the equator (Figure~\ref{fig:gravity:apparentweight}). That person is accelerating because they are undergoing uniform circular motion as they rotate along with the Earth. Again, the only forces acting on the person are their weight and the normal force exerted by the scale. The sum of the forces must now be equal to the person's mass, $m$, times the radial acceleration, $a_r$, that is necessary for them to follow the surface of the Earth as the Earth rotates about its axis. Newton's Second Law allows us to find the magnitude of the normal force acting on the person:
\begin{equation}
\sum F &= F_g-N=ma_r=m\frac{v^2}{R}\\
\therefore N &= F_g - m\frac{v^2}{R}\\
&=G\frac{Mm}{R^2} -  m\frac{v^2}{R}\\
&=m\left(G\frac{M}{R^2} - \frac{v^2}{R}  \right)\\
&=m\left(g - \frac{v^2}{R}  \right)
\end{equation}
where $M$ is the mass of the Earth, $R$ is the radius of the Earth, and $v$ is the speed at the surface of the Earth due to the Earth's rotation.  In the last line, we used the result from Example~9.3 where we determined the value of $g$ in terms of the mass and radius of the Earth.

We see that the normal force is reduced compared to what it would be if the Earth were not rotating ($v=0$) or if one is standing at one of the poles. Your apparent weight, which you can measure by standing on a spring scale, is thus smaller at the equator than it is at the poles. The quantity in parenthesis can be thought of as a modified or ``effective'' value of $g$ at the equator.

\begin{framed}
\textbf{Checkpoint}\\
What is the effective value of $g$ at the equator?

\begin{enumerate}
\item $9.80 {\rm m/s^2}$
\item $9.78 {\rm m/s^2}$
\item $9.70 {\rm m/s^2}$
\item $9.51 {\rm m/s^2}$
\end{enumerate}

\begin{framed}
\textbf{Answer}\\
\begin{enumerate}[resume]
\item
\end{enumerate}
\end{framed}
\end{framed}

If you are circling the Earth a distance $R$ from the centre of the Earth at a constant speed $v$, it is possible for your apparent weight to be zero. Imagine standing on a scale in an aircraft that is circling the Earth and measuring your apparent weight with the spring scale. As the speed of the aircraft increases, your apparent weight, $N$, decreases according to the formula that we just found:
\begin{equation}
N=m\left(G\frac{M}{R^2} - \frac{v^2}{R}  \right)
\end{equation}
At a certain speed, $v$, your apparent weight is zero and you feel weightless:
\begin{equation}
G\frac{M}{R^2} &= \frac{v^2}{R}\\
\therefore v&= \sqrt{G\frac{M}{R} }
\end{equation}
This speed corresponds to a centripetal acceleration that is exactly equal to that due to gravity. This makes sense, since gravity is the only force that is acting on you (the normal force is now zero), which is exactly what we call being in orbit.

Earth's rotation has some interesting consequences for stationary objects. At any position on Earth that is not at the equator or the poles, the sum of the forces on any stationary object (meaning stationary relative to the Earth's surface) cannot be zero. This is because the object must rotate along with the Earth, so the net force on the object must point toward the centre of the circle about which that location on Earth is rotating.

Take, for example, a plumb line, which consists of a mass hanging from a string. The two forces acting on the mass are gravity and tension. The force of gravity must point towards the centre of the Earth. We would expect that the force of tension, and therefore the string, would point directly away from the centre of the Earth. However, we find that if the plumb line is located at some angle $\theta$ from the equator (but not at the equator or poles), as in Figure~\ref{fig:gravity:apparentweight2}, then the string will point slightly away from the centre of the Earth.  In order for the mass to remain stationary relative to the ground, it must rotate along with the Earth (radius $R$) around a circle of radius $R\cos\theta$.   Thus, the tension from the string cannot point away from the centre of the Earth, because the net force must point towards the centre of the circle of radius $R\cos\theta$.

\begin{figure}[!htbp]
\centering
\includegraphics[width=0.8\linewidth]{files/apparentweight2-c000843b2433473a77dfad5122536b17.png}
\caption[]{Away from the equator and poles, a plumb line (right) does not point towards the centre of the Earth, because the net force on the mass must provide the acceleration towards the centre of the circle (of radius $R\cos\theta$) about which the plumb line rotates due to the Earth's rotation. Note that the deflection of the plumb line is highly exaggerated.}
\label{fig:gravity:apparentweight2}
\end{figure}

\begin{framed}
\textbf{Checkpoint}\\
You cut the string of the plumb line. Where does the mass land relative to its initial latitude (the angle $\theta$ in Figure~\ref{fig:gravity:apparentweight2})?

\begin{enumerate}
\item At the same latitude.
\item Closer to the nearest pole.
\item Closer to the equator.
\end{enumerate}

\begin{framed}
\textbf{Answer}\\
\begin{enumerate}
\item
\end{enumerate}
\end{framed}
\end{framed}

\paragraph{The gravitational field}

The gravitational force exerted on a mass $m$ by a mass $M$ can be written as:
\begin{equation}
\vec F(\vec r) = -G\frac{Mm}{r^2}\hat r
\end{equation}
if we define a coordinate system with the origin located at the centre of mass $M$ so that $\vec r$ is the position of $m$ relative to $M$. We can define the ``gravitational field'', $\vec g(\vec r)$, at position, $\vec r$, due to the presence of mass $M$ as the gravitational force per unit mass exerted by $M$:
\begin{equation}
\boxed{\vec g(\vec r) = \frac{\vec F(\vec r)}{m} =  - \frac{GM}{r^2}\hat r}
\end{equation}
The word ``field'' is just a mathematical term for a function that depends on position. Since $\vec g(\vec r)$ is a vector, we would refer to it as a ``vector field''.

Defining the gravitational field makes it easy to calculate the force of gravity from $M$ on any mass $m$:
\begin{equation}
\vec F_g = m\vec g(\vec r)
\end{equation}

At the surface of the Earth, the magnitude of the gravitational field is given by:
\begin{equation}
g(R_\oplus)=\frac{GM}{R_\oplus^2}=9.81 {\rm N/kg}
\end{equation}
where $R_\oplus$ is the radius of the Earth. Of course, this also corresponds to the acceleration of objects in free-fall near the surface of the Earth, which we can find from Newton's Second Law:
\begin{equation}
\sum \vec F &= \vec F_g = m\vec a\\
m\vec g(R_\oplus)&= m\vec a\\
\therefore \vec a &= \vec g(R_\oplus)
\end{equation}
but we see here why it more precise to refer to $g$ as the ``magnitude of the gravitational field at the surface of the Earth'' rather than ``the acceleration due to Earth's gravity''. It is also worth noting that the two are only equal if the gravitational mass (on the left of the equation in the second line) is the same as the inertial mass (on the right of the equation). The gravitational mass is the mass that appears in the gravitational force defined by Newton, whereas the inertial mass is the mass that appears with the acceleration in Newton's Second Law.

\begin{framed}
\textbf{Checkpoint}\\
Two small objects with different masses, $m_1$ and $m_2$, are located a distance $r$ from a nearby star. What can you say about the magnitude of the gravitational field and the magnitude of the gravitational force on $m_1$ and $m_2$?\}

\begin{enumerate}
\item The field is different and the forces are different.
\item The field is different but the forces are the same.
\item The field is the same but the forces are different.
\item The field is the same and the forces are the same.
\end{enumerate}

\begin{framed}
\textbf{Answer}\\
\begin{enumerate}[resume]
\item
\end{enumerate}
\end{framed}
\end{framed}

Suppose that there are two large mass bodies, $M_1$ and $M_2$, and a smaller mass body, $m$. We can calculate the net gravitational force on $m$ by summing the gravitational force vectors from $M_1$ and $M_2$ separately. If the gravitational fields from $M_1$ and $M_2$ are given by $\vec g_1(\vec r)$ and $\vec g_2(\vec r)$, respectively, then the total gravitational force on $m$ is given by:
\begin{equation}
\vec F &= m\vec g_1(\vec r) + m\vec g_2(\vec r)=m(\vec g_1(\vec r)+\vec g_2(\vec r))\\
&=m \vec g(\vec r)
\end{equation}
where we have introduced the total gravitational field:
\begin{equation}
\vec g(\vec r) = \vec g_1(\vec r)+\vec g_2(\vec r)
\end{equation}
In other words, if there are multiple bodies that result in a non-negligible force of gravity, we can calculate their gravitational fields independently and sum them together to define a net gravitational field, $\vec g(\vec r)$, that models the net force of gravity from all of the bodies. The net gravitational force on a new body of mass $m'$ is then simply given by $m'\vec g(\vec r)$, and we do not need to add any more vectors together. For example, when calculating the motion of satellites that can be influenced by the force of gravity from both the Earth and the Moon, we simply need to calculate the net gravitational field from the Earth and Moon, and the motion of any satellite can then be modelled using that net gravitational field.

\begin{framed}
\textbf{Checkpoint}\\
There are two planets with equal mass located a distance $d$ apart. Position $A$ is located midway between the two planets. Position $B$ is located a distance $d/2$ from one of the planets, in the position shown in Figure~\ref{fig:gravity:fieldAB}. How does the field at $A$ compare to the field at $B$

\begin{figure}[!htbp]
\centering
\includegraphics[width=0.6\linewidth]{files/fieldAB-8b5177e687c3f0a0914470d3ce9c2cde.png}
\caption[]{Two planets are a distance $d$ apart. We are considering the gravitational field at two positions, A and B, located near the planets.}
\label{fig:gravity:fieldAB}
\end{figure}

\begin{enumerate}
\item The magnitude of the field is the same at $A$ and $B$.
\item The magnitude of the field is greater at $A$ than at $B$.
\item The magnitude of the field is greater at $B$ than at $A$.
\end{enumerate}

\begin{framed}
\textbf{Answer}\\
\begin{enumerate}[resume]
\item
\end{enumerate}
\end{framed}
\end{framed}

\paragraph{Gauss's Law}\label{sec:gravity:gauss}

Newton's Universal Theory of Gravity postulates that the force of gravity between two bodies decreases as the squared of the distance between those two bodies. Using the terminology of a field, we would say that the strength of the gravitational field from an object decreases as the inverse of the square of the distance to that object. This is an example of a what we generally call an ``inverse-square law''. The electric force between two charges is also given by an inverse-square law, and we now understand that these forces behave as if they were ``transmitted'' by waves or particles.

If a force is given by an inverse-square law, then Gauss's law gives a way to determine the strength of the field that is associated with that force. In the case of gravity, Gauss's law states that:
\begin{equation}
\oint \vec g(\vec r) \cdot d\vec A = 4\pi G M^{enc}
\end{equation}
where the integral on the left is an integral over a ``closed surface'' that completely surrounds the mass for which we want to determine the gravitational field. To evaluate the integral, imagine taking a closed surface, $S$, that surrounds the mass. The vector $d\vec A$ is defined as the vector that goes with a small element of that surface and points outwards from the closed surface. The magnitude of the vector is equal to the area of that small surface ($dA$), as illustrated in Figure~\ref{fig:gravity:gauss}. You can then take the scalar product of $d\vec A$ with the gravitational field, $\vec g(\vec r)$, at that point on the surface. If you sum all of those scalar products, you get the value of the integral on the left. Gauss's law states that the value of that integral is equal $4 \pi G$ times the total mass that is enclosed by the surface.

\begin{framed}
\textbf{Olivia's Thoughts}\\
If you want to know if a surface is closed, ask yourself if you could put water inside the surface and not be worried about it spilling out. For example, if you put water in a sphere or a cube , the water would not spill out even if you shook it around, so they are closed surfaces. A flat square is an open surface because there is no ``inside'' to even put the water in.  A bowl is an open surface because, though you can put water in it, the water could spill out.
\end{framed}

We will go into more detail about Gauss's law when we cover electromagnetism, but it is worth seeing how to use it in a simple scenario. Figure~\ref{fig:gravity:gauss} shows a spherical body of mass $M$ and radius $R$ for which we would like to determine the value of the gravitational field at a distance $r$ from the centre of the body.

\begin{figure}[!htbp]
\centering
\includegraphics[width=0.3\linewidth]{files/gauss-532f5d1a8160ab32373485383e5e63f6.png}
\caption[]{Example of a spherical Gaussian surface, $S$, of radius $r$ centred about a body of mass $M$ and radius $R$. An element of the surface, $d\vec A$ is also shown along with the gravitational field, $\vec g(\vec r)$, at that point.}
\label{fig:gravity:gauss}
\end{figure}

To do so, we draw a ``Gaussian surface'', $S$, that is a sphere with a radius $r$, and centred about the body. At any point on the surface, the area element vector $d\vec A$ points away from the centre of the spherical surface. The gravitational field vector, $\vec g(\vec r)$ will always point towards the centre of the spherical surface, as illustrated. Furthermore, by symmetry, the magnitude of $\vec g(\vec r)$ is constant along the whole Gaussian surface. Thus, the scalar product $\vec g(\vec r) \cdot d\vec A= -g(r)dA$ everywhere along the surface (it is negative because the two vectors are anti-parallel). The integral is thus given by:
\begin{equation}
\oint \vec g(\vec r) \cdot d\vec A = -g(r)\oint dA
\end{equation}
where we factored $g(r)$ out of the integral, since the magnitude of $\vec g(\vec r)$ is constant for all of the area elements $dA$ on the sphere. Remember that an integral is a sum. The integral $\oint dA$ thus means ``sum all of the area elements $dA$ over the entire surface $S$''. Thus, the integral is the total area of the spherical surface $S$\footnote{The surface area of a sphere of radius $r$ is $4\pi r^2$.}:
\begin{equation}
\oint \vec g(\vec r) \cdot d\vec A = -g(r)\oint dA =-g(r)(4\pi r^2)
\end{equation}
Inserting this into Gauss's law, we find:
\begin{equation}
\oint \vec g(\vec r) \cdot d\vec A &= 4\pi G M^{enc}\\
-g(r)(4\pi r^2) &= 4\pi G M^{enc}\\
\therefore g(r) &= - \frac{GM}{r^2}
\end{equation}
where $M^{enc}=M$ is the total mass enclosed by the Gaussian surface (in this case, the entire mass $M$ is enclosed). This is of course the result that we expected and obtained earlier from Newton's formulation. Note that Gauss's law is only easy to use if the system is highly symmetric (e.g. spherically symmetric), and that it does not give the direction of the field vector, which must be obtained from symmetry arguments.

\begin{framed}
\textbf{Olivia's Thoughts}\\
Here's an analogy to describe Gauss's law for gravity: A famous celebrity is doing an event, and they attract a certain number of fans who want to get as close to the celebrity as possible. You put up a barricade around the celebrity. The gravitational field is represented by how crowded it is somewhere along the barricade. If a second celebrity is at the event, they will attract their own fans, so there will be more people around the barricade. The number of celebrities is kind of like the enclosed mass $M^{enc}$.

A photographer is coming to the event, and you told him to stand at some location that is a distance $r$ from the celebrities. The photographer wants to know how crowded it will be when he is standing behind the barricade at that location. Gauss's law gives us a way to figure this out. If you know which celebrities are at the event ($M^{enc}$), you can determine how many people will be there (this is like finding $4\pi GM^{enc}$). Then, if you can build a barricade such that the fans are evenly distributed around it, and you know how long that barricade is ($\oint dA$), you can easily calculate how crowded it will be at some point along the barricade (you can just divide the number of people by the length of the barricade).

The barricade represents our Gaussian surface and, like a Gaussian surface, it can be whatever shape we want as long as it encloses the celebrities and passes through the point we are interested in. If we want to make sure the people are spread out evenly, the shape of the barricade is going to depend on the specific case. Let's take the example of our single spherical body. This is analagous to having one celebrity at the event.

Figure~\ref{fig:gravity:barricadeanalogy} shows two possible barricades we could build. Although we can technically build the barricade on the left, it doesn't help us because the areas closer to the celebrity will be more crowded. Instead, we want to build the barricade on the right, which is a circle of radius $r$, because the fans are evenly spread out. This is why we use a spherical Gaussian surface when we're considering the field due to a spherical body - at any point a distance $r$ from the body, the field will be the same. (Note: Remember that, unlike the barricade, the Gaussian surface isn't a physical thing, so it won't affect the gravitational field. It is just a mathematical tool that allows us to take advantage of what the field already looks like.)

\begin{figure}[!htbp]
\centering
\includegraphics[width=0.7\linewidth]{files/barricadeanalogy-98ef92293c7c725259688fc9eacd1a31.png}
\caption[]{A celebrity (black dot) attracts fans (grey dots). A photographer (dot labelled ``P'') stands behind the barricade a distance $r$ away. This shows two possible barricades we could build around the celebrity. The density of the fans is not uniform for the barricade on the left, so we would not choose that shape to evaluate the Gaussian integral.}
\label{fig:gravity:barricadeanalogy}
\end{figure}
\end{framed}

We can also use Gauss's law to determine the gravitational field \textbf{inside} of the body of mass $M$ and radius $R$. This is illustrated in Figure~\ref{fig:gravity:gauss2}, which shows a spherical Gaussian surface of radius $r$ that is \textbf{inside} of the body of mass $M$.

\begin{figure}[!htbp]
\centering
\includegraphics[width=0.3\linewidth]{files/gauss2-3bcc1d70b85158fced3e532072fa9bf9.png}
\caption[]{Example of a spherical Gaussian surface, $S$, of radius $r$ centred inside a body of mass $M$ and radius $R$.}
\label{fig:gravity:gauss2}
\end{figure}

The gravitational field inside of the body of mass $M$ is also symmetric
and  constant in magnitude across the whole surface, \%?
so that the integral is the same as before:
\begin{equation}
\oint \vec g(\vec r) \cdot d\vec A=-g(r)(4\pi r^2)
\end{equation}
However, in order to use Gauss's law, we need to determine the mass of the body that is enclosed within the spherical surface, which will be less than $M$. If we assume that the mass density, $\rho$, of the object is constant (the body is made of a uniform material), then the density is simply the mass of the object over its volume:
\begin{equation}
\rho = \frac{M}{\frac{4}{3}\pi R^3}
\end{equation}
The amount of mass enclosed by the spherical surface of radius $r$ is the density multiplied by the volume of a sphere of radius $r$:
\begin{equation}
M^{enc} = \rho \frac{4}{3}\pi r^3 = M\frac{r^3}{R^3}
\end{equation}
Applying Gauss's law, we can now find the magnitude of the gravitational field inside of the spherical body at a distance $r$ from the centre:
\begin{equation}
\oint \vec g(\vec r) \cdot d\vec A &= 4\pi G M^{enc}\\
-g(r)(4\pi r^2) &= 4\pi G M\frac{r^3}{R^3}\\
\therefore g(r) &= - \frac{G M}{R^3}r
\end{equation}
And we find that, inside a uniform spherical body of mass $M$, the gravitational field increases linearly with radius as one moves out from the centre. At the centre of the body, the gravitational field is zero.

\begin{framed}
\textbf{Checkpoint}\\
What can you say about the magnitude of the gravitational field inside a spherical \textbf{shell} of mass $M$?

\begin{enumerate}
\item It increases as you move out from the centre of the spherical shell.
\item It decreases as you move out from the centre of the spherical shell.
\item It is equal to zero.
\item It is nonzero and constant in magnitude.
\end{enumerate}

\begin{framed}
\textbf{Answer}\\
\begin{enumerate}[resume]
\item
\end{enumerate}
\end{framed}
\end{framed}

\subsubsection{Gravitational potential energy}\label{sec:gravity:potentialenergy}

Consider a large spherical body of mass $M$ with a coordinate system whose origin coincides with the centre of the spherical body (for example, the large body could be the Earth). The force, $\vec F(\vec r)$ on a body of mass $m$ (for example, a satellite), located at a position $\vec r$ is then given by:
\begin{equation}
\vec F(\vec r) = - G\frac{Mm}{r^2}\hat r=- G\frac{Mm}{r^3}\vec r
\end{equation}
where in the second equality, we use the fact that the unit vector in the direction of $\vec r$ is simply the vector $\vec r$ divided by its magnitude. We can write the force out in Cartesian coordinates:
\begin{equation}
\vec r &= x\hat x + y \hat y + z\hat z\\
r &= \sqrt{x^2+y^2+z^2} =(x^2+y^2+z^2)^\frac{1}{2} \\
\therefore \vec F(x,y,z) &= - G\frac{Mm}{(x^2+y^2+z^2)^\frac{3}{2} }(x\hat x + y \hat y + z\hat z)
\end{equation}
Mathematically, this is equivalent to the force that we considered in Example~8.2 of Section~\ref{chapter:potentialecons}, which we showed was a conservative force. The force of gravity in Newton's theory is thus a conservative force, for which we can determine a potential energy function.

In order to determine the gravitational potential energy function for the mass $m$ in the presence of a mass $M$, we calculate the work done by the force of gravity on the mass $m$ over a path where the integral for work will be ``easy'' to evaluate, namely a straight line. Figure~\ref{fig:gravity:potential} shows such a path in the radial direction, $r$, over which it will be easy to calculate the work done by the force of gravity from mass $M$ when mass $m$ moves from being a distance $r_A$ to a distance $r_B$ from the centre of mass $M$.

\begin{figure}[!htbp]
\centering
\includegraphics[width=0.5\linewidth]{files/potential-6cf2648356a2e39b57f407020da40b77.png}
\caption[]{Calculating the work done on a mass $m$ by the force of gravity exerted by mass $M$ when mass $m$ moves from a distance $r_A$ to a distance $r_B$ from the centre of mass $M$.}
\label{fig:gravity:potential}
\end{figure}

The work done by the force of gravity on $m$ in going from $r_A$ to $r_B$ is given by:
\begin{equation}
W &= \int_{r_A}^{r_B}\vec F(r) \cdot d\vec r = \int_{r_A}^{r_B} \left(- G\frac{Mm}{r^2}\hat r \right)\cdot d\vec r =\int_{r_A}^{r_B} - G\frac{Mm}{r^2}dr\\
&=\left[G\frac{Mm}{r} \right]_{r_A}^{r_B} =G\frac{Mm}{r_B} - G\frac{Mm}{r_A}
\end{equation}
The difference in potential energy in going from position $A$ to position $B$ is given by the negative of the work done by the force:
\begin{equation}
\Delta U = U(r_B) - U(r_A) = -W = G\frac{Mm}{r_A} - G\frac{Mm}{r_B}
\end{equation}
By inspection, we can identify the potential energy function for gravity:
\begin{equation}
\boxed{U(r) = -G\frac{Mm}{r} + C}
\end{equation}
which is determined only up to a constant, $C$.

A particularly useful choice of constant is $C=0$. This corresponds to choosing the potential energy to be zero only when $r$ goes to infinity. That is, the potential energy of mass $m$ is zero only when it is infinitely far away from mass $M$. The choice of constant $C$ corresponds to the (arbitrary) value of the potential energy when mass $m$ is infinitely far from mass $M$. When mass $m$ is not infinitely far away, it has \textbf{negative} potential energy (if $C=0$). This is not a problem! Remember, the only thing that is meaningful is a difference in potential energy, so the specific value of the potential energy has no meaning. The kinetic energy of an object, on the other hand, has to be positive.

Recall that if there are no other forces acting on an object, that object will move in such a way to reduce its potential energy. If the object of mass $m$ is located at some distance $r$ from the object of mass $M$, the force of gravity will attract $m$ so that $r$ decreases. As $r$ decreases in magnitude, the potential energy becomes more negative (larger in magnitude, but further away from zero), and the potential energy of $m$ will indeed decrease as it accelerates due to the force of gravity.

\paragraph{Mechanical energy with gravity}

Unless noted otherwise, we will continue our discussion of gravitational potential energy with the particular choice of constant $C=0$:
\begin{equation}
\boxed{U(r) = -G\frac{Mm}{r}}
\end{equation}
Furthermore, we will assume that $M$ is a large body, such as the Earth, which we can consider as fixed, and focus our discussion on describing the motion of mass $m$ (e.g. a satellite). If $M$ is much bigger than $m$, they will both experience a force of gravity from each other of the same magnitude (Newton's Third Law), but because $M$ is so much larger, its acceleration will be much smaller (Newton's Second Law). Thus, it is a good approximation to assume that $M$ is stationary and that only $m$ moves when $M>>m$.

We can define the total mechanical energy of mass $m$ when it has a speed $v$ (relative to $M$) and is located at a distance $r$ from the centre of mass $M$:
\begin{equation}
E = U + K = -G\frac{Mm}{r}+\frac{1}{2}mv^2
\end{equation}
where the kinetic energy term is always positive. If gravity is the only force exerted on mass $m$, then the mechanical energy, $E$, as defined above, will be a constant. The mechanical energy of an object can give us insight into the possible motion of the object.

Imagine launching a rocket straight upwards from the surface of the Earth; once all of the fuel has burnt up, the rocket's mechanical energy becomes constant as the rocket engine stops doing work on the rocket. As soon as the engine stops providing thrust, the rocket will start to slow down as the force of gravity attracts the rocket back to Earth. If the rocket is going fast enough, it will be able to completely escape the Earth's gravitational pull and travel to infinity (we assume that there are no other planets or the Sun, just the Earth exists!). If, on the other hand, the rocket's speed is too low, it will eventually stop and fall back to Earth. This is the same thing that happens to you when you try to jump vertically. If you could jump hard enough, you would be able to escape the Earth's gravitational pull!

In terms of mechanical energy, we can ask ourselves if the mechanical energy of the rocket is large enough to escape the Earth's gravitational pull. Specifically, we can ask ourselves what the value of the rocket's kinetic energy would be when it reaches infinity. The kinetic energy of the rocket is given by:
\begin{equation}
K = E - U
\end{equation}
If the rocket is infinitely far from the Earth, then its potential energy is zero, and the kinetic energy is equal to $E$.

If the mechanical energy, $E$, is negative, it is not possible for the rocket to ever make it to infinity because its kinetic energy would have to be negative. In other words, if the mechanical energy is negative, then the object of mass $m$ can never escape the gravitational pull of object $M$. We say that $m$ is ``gravitationally bound'' to $M$.

If the mechanical energy, $E$, is exactly zero, then the object's kinetic energy will become zero just as it reaches infinity. In other words, it will just barely be able to escape the gravitational pull from mass $M$. The condition for this to happen is:
\begin{equation}
E &= 0\\
K & = -U\\
\frac{1}{2}mv^2 &= G\frac{Mm}{r}\\
\therefore v_{esc} &= \sqrt{\frac{2GM}{r}}
\end{equation}
which we can interpret as a condition for the speed of the rocket. If at some distance $r$ from $M$, the rocket has the speed given by the condition above, then it will have enough kinetic energy to escape the gravitational pull of $M$. We call this speed the ``escape velocity''.

Finally, if the mechanical energy is greater than zero, then the rocket will have enough energy to escape the gravitational pull of $M$ and have a non-zero speed when it reaches infinity.

\begin{framed}
\textbf{Checkpoint}\\
What is the escape velocity from the surface of the Earth?

\begin{enumerate}
\item $4.29\times 10^{6} {\rm km/s}$
\item $1.25\times 10^{5} {\rm km/s}$
\item $11.2 {\rm km/s}$
\item $9.81 {\rm km/s}$
\end{enumerate}

\begin{framed}
\textbf{Answer}\\
\begin{enumerate}[resume]
\item
\end{enumerate}
\end{framed}
\end{framed}

\begin{framed}
\textbf{Example 9.4}\\
Show that an object of mass $m$ in a circular orbit of radius $r$ around a body of mass $M$ has half of the kinetic energy required to escape the gravitational pull of $M$.

\begin{framed}
\textbf{Solution}\\
The only force acting on the object is gravity, so it has a mechanical energy given by:
\begin{equation}
E&=U+K\\
E&=-G\frac{Mm}{r}+\frac{1}{2}mv^2
\end{equation}
In order for the object to just escape the gravitational pull of $M$, it's mechanical energy must be equal to zero:
\begin{equation}
E&=0\\
\therefore K_{esc}&=-U
\end{equation}
Since the object is in a circular orbit, we can use Newton's Second Law to find an expression for $v^2$:
\begin{equation}
F_{net}&=\frac{mv^2}{r}\\
\frac{GMm}{r^2}&=\frac{mv^2}{r}\\
\frac{GM}{r}&=v^2
\end{equation}
where in the second line we used the fact that $F_{net}$ is equal to the force of gravity exerted by $M$ on the object. The kinetic energy of the object is thus:
\begin{equation}
K&=\frac{1}{2}mv^2\\
K&=\frac{1}{2}\frac{GMm}{r}
\end{equation}
You will notice that this is very similar to our expression for $U$. In fact, we have:
\begin{equation}
K&=-\frac{1}{2}U\\
\therefore K&=\frac{1}{2}K_{esc}
\end{equation}
\textbf{Note:} We can also see that the velocity of an object in a circular orbit is equal to $\sqrt{GM/r}$, which is half the escape velocity, $v_{esc}=\sqrt{2GM/r}$
\end{framed}
\end{framed}

\subparagraph{Types of orbits}

The mechanical energy of a body of mass $m$ determines whether it is gravitationally bound to (i.e. cannot escape) the body of mass $M$. The path (orbit) that $m$ will take depends on its velocity with respect to $M$. Clearly, if the velocity of $m$ is directed at the centre of $M$, then $m$ will just collide with $M$. In all other cases, the orbit that $m$ will take depends on the mechanical energy of $m$ as well as the speed of $m$ at the point of closest approach to $M$ (see Figure~\ref{fig:gravity:conical}). The velocity of $m$ at the point of closest approach will always be perpendicular to the line joining the centres of $m$ and $M$. The different possible orbits are:

\begin{enumerate}
\item A \textbf{circular orbit} of radius $R$ (where $R$ is the distance of closest approach) if the \textbf{mechanical energy is negative} (i.e. it is bound) and the speed is exactly equal to the value necessary for the gravitational force to provide the required centripetal acceleration for uniform circular motion:
\end{enumerate}
\begin{equation}
\sum F = G\frac{Mm}{R^2} &= m\frac{v^2}{R}\\\therefore v_{circ}=\sqrt{\frac{GM}{R}}
\end{equation}
\begin{enumerate}[resume]
\item An \textbf{elliptical orbit} if the \textbf{mechanical energy is negative} and the speed at the point of closest approach is different than that required for a circular orbit.
\item A \textbf{parabolic orbit} if the \textbf{mechanical energy is exactly zero}.
\item A \textbf{hyperbolic orbit} if the \textbf{mechanical energy is bigger than zero}.
\end{enumerate}

The possible orbits are illustrated in Figure~\ref{fig:gravity:conical}, and are curves in the family of ``conic sections'', as they can be found by the intersection of a plane and a cone. All conic sections have at least one ``focus'' point (ellipses have two) that corresponds to the location of $M$.

\begin{figure}[!htbp]
\centering
\includegraphics[width=0.7\linewidth]{files/conical-4e29b968e370c802d0ed276493194a28.png}
\caption[]{The different possible orbits of $m$ due to the gravitational force of $M$ depend on the mechanical energy, $E$, of $m$. The orbits are drawn in a frame of reference where $M$ is at rest.}
\label{fig:gravity:conical}
\end{figure}

\subsubsection{Einstein's Theory of General Relativity}

Newton's Universal Theory of Gravity was extremely successful at describing the motion of planets in the solar system, and allowed for high precision astronomy. For example, precision measurements of  Uranus's orbit showed that it appeared to be inconsistent with Newton's theory, unless the gravitational influence of another planet was included in the model. This led to the discovery of the planet Neptune.

However, some issues with Newton's theory were uncovered. The orbit of Mercury was shown to be different than what Newton's theory could describe, but searches for another planet (Vulcan) were unsuccessful. In addition, Albert Einstein's theory of Special Relativity, published in 1905, was found to be incompatible with Newton's theory of gravity. One of the consequences of Special Relativity is that nothing can propagate faster than the speed of light. Newton's Universal Theory of Gravity implies that the gravitational force is transmitted instantaneously. In Newton's theory, if the Sun suddenly disappeared, Earth would immediately ``fall out'' of its orbit, and we would immediately know that the Sun has disappeared. This would violate Special Relativity because there cannot be a mechanism that would allow us to know that the Sun has disappeared faster than it would take light to propagate from the Sun. In other words, for the $8 {\rm min}$ that are required for light to travel from the Sun to the Earth, we cannot know that the Sun has disappeared: only when we literally see the Sun disappear would the Earth be ``allowed'' to fall out of its orbit.

Einstein's Theory of General Relativity is a theory developed by Einstein in order to describe gravity in a way that is consistent with Special Relativity and the propagation of light. Einstein was famous for his ``thought experiments,'' which allow us to think about some of the implications of a theory, even if the experiments would be very difficult to carry out in practice. One such thought experiment is to consider what someone would observe in an accelerating frame of reference.

Consider an observer in an elevator, as illustrated in Figure~\ref{fig:gravity:elevator}. If the elevator is stationary at the surface of the Earth (left panel), and the observer is standing on a scale, they could measure their weight, $mg$, on the scale. The two forces on the observer are their weight and the normal force, which would be equal in magnitude since the observer is not accelerating. The normal force, read out by the scale, would thus correspond to their weight. To be more precise, the normal force would be equal to $m_Gg$, where $m_G$ is the gravitational mass of the observer (that mass which is related to the force of gravity experienced by a mass).

If the elevator was instead placed in empty space, and the elevator was accelerating upwards with an acceleration of $g$ (right panel), the observer would still be able to measure their weight by stepping on the scale. The only force on the observer is the normal force from the scale, which must be equal to its mass times their acceleration $N=m_Ia=m_Ig$, since the observer is accelerating with the elevator. In this case, it is the inertial mass of the observer, $m_I$, that comes into play, so the normal force read on the scale is $m_Ig$.

\begin{figure}[!htbp]
\centering
\includegraphics[width=0.8\linewidth]{files/elevator-cb051ec5733dec3aa2b4fdbfb11190a9.png}
\caption[]{Left: A person standing on a scale in an elevator at rest at the surface of the Earth. Right: A person in an elevator that is accelerating in empty space with the same acceleration as that due to gravity at the Earth's surface. The curvature of the light beam is exaggerated.}
\label{fig:gravity:elevator}
\end{figure}

Einstein postulated that it would be impossible for the observer to distinguish whether they are at rest on the surface of the Earth, or in empty space accelerating with an acceleration of $g$. In other words, he postulated that the inertial and gravitational masses are exactly equivalent. This is what is called the ``Equivalence Principle''.

This simple statement has dramatic implications. Special Relativity requires that light will travel in a straight line in empty space. If a beam of light enters and then exits the elevator, the observer on Earth and the one accelerating in empty space must observe the same thing, since they cannot distinguish between being on Earth or accelerating in space. The observer in space, who is accelerating, will observe that the beam of light bends as it crosses the elevator (the beam travels in a straight line as observed in an inertial reference frame, so the person in the accelerating elevator would see it follow a parabolic path). The observer on Earth must thus observe the same thing, namely that light will follow a curved path in the presence of a gravitational field.

But...light must travel in a straight line in empty space. That means that if the path of a beam of light is curved near Earth, it must be because space itself is curved in the presence of a gravitational field! In other words, Einstein's Theory of General Relativity describes how the presence of mass (or energy) results in a curvature of space (and time).

Imagine a ladybug on the side of a basketball. If the ladybug starts moving in what it believes to be a straight line, it will actually move in a curved path along the surface of the ball, as in Figure~\ref{fig:gravity:ladybug}. This is like the curved path of light that we observe. If we didn't know the ball was there, we would just think that the bug was moving along a curved path. In the same way, if an observer is not aware of the curvature of spacetime, it appears that light follows a curved path.

\begin{figure}[!htbp]
\centering
\includegraphics[width=0.8\linewidth]{files/ladybuganalogy-ceee73126e2939c41982b3bdb610b1d2.png}
\caption[]{Left: A ladybug perceives itself to be moving in a straight line. Center: The basketball is curved, so the ladybugs follow curved paths. Right: What an observer would see if they didn't know the basketball was there.}
\label{fig:gravity:ladybug}
\end{figure}

Now imagine there's a second ladybug. Both bugs start at the middle of the ball and start moving towards the top of the ball in what they think is a straight line (as shown in the center panel of Figure~\ref{fig:gravity:ladybug}). When the bugs start moving, they are parallel to each other, so if the ball was not curved, the ladybugs would never meet. However, because it is curved, the ladybugs will eventually cross paths. If you were not aware that the ball was there, you would have to conclude that there was some force attracting the bugs to each other, just like if you were unaware that spacetime was curved, you would conclude that massive bodies moving towards each other are attracted by a gravitational force.

Objects that are moving in a gravitational field are actually following Newton's First Law (they are moving at constant velocity in a straight line and no force is exerted on them). It is strange and unexpected, but high precision measurements confirm that this correctly describes everything that we have measured!

Einstein's theory was able to describe the orbit of Mercury, and the prediction that gravity leads to light following a curved path was confirmed by Eddington within five years of Einstein's theory being published. Another implication of the theory is that time goes by slower in the presence of a gravitational field. Clocks on Earth run slower than clocks in orbit (where the gravitational field is weaker). This effect is taken into account when using GPS to determine your position on Earth, since this is based on comparing the time that it takes signals to arrive to your position on Earth from different satellites. This is also somewhat reasonably well described in the movie ``Interstallar'', where time is seen to pass much slower for a set of astronauts in the vicinity of a black hole, where the gravitational field is strong.

\subsubsection{Summary}

Kepler was the first to synthesize a large amount of data to quantitatively describe gravity with his three laws:

\begin{enumerate}
\item The path of a planet around the Sun is described by an ellipse with the Sun at once of its foci.
\item Planets move in such a way that the area swept by a line connecting the planet and the Sun in a given period of time is constant, independent of the location of the planet.
\item The ratio between the orbital periods, $T$, squared of two planets is equal to the ratio of the semi-major axes, $s$, of their orbits cubed:
\end{enumerate}
\begin{equation}
\left(\frac{T_1}{T_2}\right)^2=\left(\frac{s_1}{s_2}\right)^3
\end{equation}

Newton described the attractive force of gravity exerted between two bodies of mass $M_1$ and $M_2$ (which must be point masses) as:
\begin{equation}
\vec F_{12}=-G\frac{M_1M_2}{r^2}\hat r_{21}
\end{equation}
where $\vec F_{12}$ is the force on body 1 from body 2, $r$ is the distance between the two bodies, and $\vec r_{21}$ is the vector from body 2 to body 1. The motion of a body under the influence of only this force will satisfy all of Kepler's Laws, if the body is gravitationally bound.

The gravitational field, $\vec g(\vec r)$, from a body of mass $M$, is defined as the gravitational force that another body would experience per unit mass:
\begin{equation}
\vec g(\vec r)=\frac{\vec F(\vec r)}{m}=-G\frac{M}{r^2}\hat r
\end{equation}
The field can be used to determine the corresponding gravitational force, $\vec F_g$, that a body of mass $m$ would experience if located at a position $\vec r$ relative to the body of mass $M$:
\begin{equation}
F_g = m \vec g(\vec r)
\end{equation}
When describing the motion of objects near the surface of the Earth, it is thus more precise to refer to $g=9.8 {\rm N/kg}$ as the magnitude of the Earth's gravitational field at the surface of the Earth, then to refer to $g=9.8 {\rm m/s^2}$ as the acceleration due to Earth's gravity. The two are only equal if gravitational mass (the $m$ in the above equation) and inertial mass (the $m$ in Newton's Second Law) are the same.

Gauss's law, which applies to all inverse-square force laws, can be used to determine the magnitude of the gravitational field from a body of mass $M$, even if it is not a point mass:
\begin{equation}
\oint \vec g(\vec r) \cdot d\vec A = 4\pi G M^{enc}
\end{equation}

Since the force described by Newton's theory is conservative, we can define a potential energy function. The gravitational potential energy of a mass $m$ located a distance $r$ away from a mass $M$ is:
\begin{equation}
U(r) = -G\frac{Mm}{r} + C
\end{equation}
A convenient choice of the constant is $C=0$, as this corresponds to the gravitational potential energy being equal to zero when $m$ is infinitely far away from $M$.

The mechanical energy, $E$, of an object of mass $m$ that is located at a distance $r$ from an object of mass $M$, if gravity is the only conservative force exerted on $m$, is given by:
\begin{equation}
E = K + U = \frac{1}{2}mv^2 - G\frac{Mm}{r}
\end{equation}
where we have explicitly chosen $C=0$, and $v$ is the speed of $m$ relative $M$ (considered to be at rest). Furthermore, if no non-conservative forces do work on the body of mass $m$, the mechanical energy, $E$, is constant.

If the mechanical energy of $m$ is negative, it is gravitationally bound to $M$. Depending on the mechanical energy of $m$ and its velocity at the point of closest approach to $M$, the orbit of $m$ will be described by one of four conic sections (circle, ellipse, parabola, hyperbola).

Einstein's Theory of General Relativity describes gravitation as the bending of space and time caused by the presence of mass and energy. In Einstein's theory, objects follow straight (inertial) paths and do not feel a force of gravity. The curvature of space is what results in their apparent motion not being a straight line. Einstein's theory is based on the Equivalence Principle (inertial and gravitational mass are exactly equal) and the properties of how light propagates according to the Theory of Special Relativity.
{\textbackslash}end\{chapterSummary\}

\begin{framed}
\textbf{Important Equations}\\
\textbf{Kepler's Third Law:}
\begin{equation}
\left(\frac{T_1}{T_2}\right)^2=\left(\frac{s_1}{s_2}\right)^3
\end{equation}
\textbf{Gravitational force and{\textbackslash}~gravitational field:}
\begin{equation}
\vec F_{12}&=-G\frac{M_1M_2}{r^2}\hat r_{21}\\
\vec g(\vec r)&=-G\frac{M}{r^2}\hat r\\
F_g &= m \vec g(\vec r)
\end{equation}

\textbf{Gauss's Law:}
\begin{equation}
\oint \vec g(\vec r) \cdot d\vec A = 4\pi G M^{enc}
\end{equation}
\textbf{Gravitational potential energy{\textbackslash}~and mechanical energy:}
\begin{equation}
U(r) = -G\frac{Mm}{r} + C\\
E = K + U = \frac{1}{2}mv^2 - G\frac{Mm}{r}
\end{equation}
\end{framed}

\subsubsection{Thinking about the material}

\begin{framed}
\textbf{Reflect and research}\\
\begin{itemize}
\item When you look at the night sky, how can you tell the difference between a planet and a star?
\item What was the relationship between Tycho Brahe and Johannes Kepler?
\item How did Tycho Brahe collect all the data that Kepler used?
\item How much time elapsed between Kepler publishing his laws and Newton publishing his Universal Theory of Gravity?
\item What was Kepler's original intention when he synthesized Tycho Brahe's observations? What was he hoping to show?
\item What was Ptolemy's theory of gravity based upon?
\item Who was the first to suggest that planets revolved around the Sun instead of the Earth?
\item Explain how the force of gravity from the moon results in tides on both sides of the Earth.
\item Explain what an L1 Lagrange point is, and how it does not violate Kepler's Third Law.
\item How did Eddington confirm that light follows a curved path in a gravitational field?
\end{itemize}
\end{framed}

\begin{framed}
\textbf{To try in the lab}\\
\begin{itemize}
\item Theory project: Prove, based on Newton's Universal Theory of Gravity, that the motion of orbiting bodies is given by a conic section.
\item Write a computer simulation to plot the orbit of two bodies, and explore how the total mechanical energy of one object affects its motion. If the two bodies have the same mass, and both move, where is the focus of the conical section describing their respective paths?
\item Propose an experiment to model and map the position of a planet in the night sky.
\end{itemize}
\end{framed}

\subsubsection{Sample problems and solutions}

\paragraph{Problems}

\begin{framed}
\textbf{Problem 9.1}\\
Geosynchronous satellites are satellites that are placed in a circular orbit around the Earth in such a way that their orbital period is synchronized with the $24 {\rm h}$ rotation period of the Earth. The advantage of geosynchronous satellites is that they are always above the same point on Earth, which makes them useful for establishing communication networks. At what altitude must geosynchronous satellites be placed?
\end{framed}

\begin{framed}
\textbf{Problem 9.2}\\
How much energy must be expended in order to place a satellite of mass $m=1000 {\rm kg}$ in a geosynchronous circular orbit around the Earth, if the satellite is launched from the North Pole of the Earth? How much energy is this per kilogram of satellite placed in orbit?
\end{framed}

\begin{framed}
\textbf{Problem 9.3}\\
Find an expression for the gravitational field due to a thin uniform rod of mass $M$ at point $P$, which is a distance $h$ above the midsection of the rod (Figure~\ref{fig:gravity:rodfield}).

\begin{figure}[!htbp]
\centering
\includegraphics[width=0.5\linewidth]{files/rodfield-ac4b93fd8281a0c6f73aee0611dff3e2.png}
\caption[]{A thin rod of mass $M$ and length $L$ produces a gravitational field at a point $P$ located above the midsection of the rod.}
\label{fig:gravity:rodfield}
\end{figure}
\end{framed}

\paragraph{Solutions}

\begin{framed}
\textbf{Solution 9.1}\\
When a satellite orbits the Earth, the only force on the satellite is the force of gravity from the Earth. Since the satellite is in a circular orbit, that force of gravity must point towards the centre of the Earth in such a way that the satellite has the correct radial acceleration, $a_R$, to stay in uniform circular motion:
\begin{equation}
a_r=\frac{v^2}{R}
\end{equation}
where $v$ is the speed of the satellite, and $R$ is the distance between the satellite and the centre of the Earth (i.e. the centre of the circular orbit). The magnitude of the force of gravity on the satellite of mass $m$ is given by:
\begin{equation}
F = G\frac{Mm}{R^2}
\end{equation}
where $M$ is the mass of the Earth. Newton's Second Law applied to the satellite is:
\begin{equation}
\sum F_r = F &= ma_r\\
\therefore G\frac{Mm}{R^2}&=m\frac{v^2}{R}
\end{equation}
The speed of the satellite can be found from the fact that it must travel a distance of $2\pi R$ (the circumference of the orbit) in a period $T=24 {\rm h}$:
\begin{equation}
v=\frac{2\pi R}{T}
\end{equation}
which we can substitute into the equation from Newton's Second Law to find the distance $R$ (i.e. the radius of the circular orbit):
\begin{equation}
G\frac{Mm}{R^2}&=m\frac{v^2}{R}\\
G\frac{M}{R^2}&=\frac{(2\pi R)^2}{T^2R}\\ 
G\frac{M}{R^2}&=\frac{4\pi^2 R}{T^2}\\ 
\therefore R&=\sqrt[3]{G\frac{MT^2}{4\pi^2}}\\
&=\sqrt[3]{(6.67\times 10^{-11} {\rm Nm^2/kg^2})\frac{(5.97\times 10^{24} {\rm kg})(86400 {\rm s})^2}{4\pi^2}}\\
&=42.2\times 10^{6} {\rm m}
\end{equation}
which corresponds to the distance between the satellite and the centre of the Earth. To obtain the ``altitude'', $h$, namely the distance from the surface of the Earth to the satellite, we must subtract the radius of the Earth, $R_\oplus=6.371\times 10^{6} {\rm m}$ from this distance:
\begin{equation}
h = R-R_\oplus = 35.9\times 10^{6} {\rm m}
\end{equation}
Thus, geosynchronous satellites are located at an altitude of approximately $36000 {\rm km}$.

\textbf{Discussion}: Note that we could have also easily used Kepler's Third Law to determine the radius of the orbit, since we already know the period ($24 {\rm h}$), and we know the value of the constant for Kepler's Third Law from Example~9.2.
\end{framed}

\begin{framed}
\textbf{Solution 9.2}\\
We need to calculate how much work must be done for the satellite to go from being at rest at the surface of the Earth to being in a geosynchronous orbit. That work will be done by a non-conservative force (a rocket engine). The work done by the non-conservative force, $W$, is equal to the satellite's change in mechanical energy:
\begin{equation}
W = \Delta E = E_B -E_A
\end{equation}
The initial mechanical energy of the satellite, $E_A$, is given by its gravitational potential energy (it has no kinetic energy at the surface of the Earth when at the North Pole - on the equator, it would have kinetic energy due to the Earth's rotation):
\begin{equation}
E_A = K + U = 0 - G\frac{Mm}{R_\oplus}
\end{equation}
where $M=5.97\times 10^{24} {\rm kg}$ is the mass of the Earth, and $R_\oplus=6.731\times 10^{6} {\rm m}$ is the radius of the Earth.

In orbit, the energy of the rocket, $E_B$, is given by:
\begin{equation}
E_B = K + U = \frac{1}{2}mv^2 - G\frac{Mm}{R}
\end{equation}
where $R=42.2\times 10^{6} {\rm m}$ is the radius of the geosynchronous orbit (Problem~??) and $v$ is the speed of the satellite in orbit. The speed is given by:
\begin{equation}
v = \frac{2\pi R}{T}
\end{equation}
where $T=24 {\rm h}$ is the orbital period. The net work that must be done to place the satellite in orbit is thus given by:
\begin{equation}
W &= E_B - E_A = \frac{1}{2}mv^2 - G\frac{Mm}{R} - \left(- G\frac{Mm}{R_\oplus}\right)\\
&=\frac{1}{2}m\frac{4\pi^2 R^2}{T^2}+GMm\left(\frac{1}{R_\oplus}-\frac{1}{R}\right)\\
&=\frac{1}{2}(1000 {\rm kg})\frac{4\pi^2 (42.2\times 10^{6} {\rm m})^2}{(86400 {\rm s})^2}\\
&+(6.67\times 10^{-11} {\rm Nm^2/kg^2})(5.97\times 10^{24} {\rm kg})(1000 {\rm kg})\left(\frac{1}{(6.731\times 10^{6} {\rm m})}-\frac{1}{(42.2\times 10^{6} {\rm m})}\right)\\
&=5.78\times 10^{10} {\rm J}
\end{equation}
This corresponds to the energy that must be imparted to a $1000 {\rm kg}$ satellite for it to end up in a geosynchronous orbit. This corresponds to $5.78\times 10^{7} {\rm J/kg}$ as the energy required per kilogram of payload placed in geosynchronous orbit. Although we calculated work as if it were work done by a force, we can think of this work coming from stored chemical potential energy in the fuel of the rocket carrying the satellite.

\textbf{Discussion:} The energy that we found above is the minimum energy that one must provide to the satellite. In practice, in order to place a satellite in orbit, one will also need to provide enough energy to accelerate the rocket that carries the satellite up into orbit, which is typically much heavier than the satellite. If the satellite were instead launched from the equator of the Earth, the satellite would already have some initial kinetic energy due to the rotation of the Earth, and one would need to provide less energy to place it in orbit. This is the reason that most rockets are launched from near the equator (think French Guyana, Florida, Kazakhstan) in a direction that is roughly parallel with the Earth's rotation.
\end{framed}

\begin{framed}
\textbf{Solution 9.3}\\
We cannot use Gauss's law to determine the magnitude of the field because the gravitational field lacks symmetry (i.e. the field will be different at the ends of the rod than along the length of the rod). The gravitational field due to a body of mass $M$ is given by:
\begin{equation}
\vec g(\vec r)=-\frac{GM}{r^2}\hat{r}
\end{equation}

Our strategy will be to break the rod into very small segments of length $dx$. Each segment, of mass $dM$, will make a small contribution, $d\vec g$, to the gravitational field, as shown in Figure~\ref{fig:gravity:rodfieldsoln}. We will then take the sum of all these contributions to find the net field.

\begin{figure}[!htbp]
\centering
\includegraphics[width=0.5\linewidth]{files/rodfieldsoln-414a994c752a5af9ba19615e5119f9df.png}
\caption[]{A thin rod of mass $M$ and length $L$ produces a gravitational field at a point $P$ located above the midsection of the rod. Each segment of the rod $dx$ will contribute to the gravitational field.}
\label{fig:gravity:rodfieldsoln}
\end{figure}

The gravitational field due to each segment is given by:
\begin{equation}
d\vec g=-\frac{GdM}{r^2}\hat{r}
\end{equation}
The element of the field, $d\vec g$, will point in a different direction for each segment $dx$. You can conclude from Figure~\ref{fig:gravity:rodfieldsoln} that, due to symmetry, the $x$ components of the field from each segment will cancel out (for the segment $dx$ shown in the diagram, there will be an identical segment on the other side of the rod). The net field will point in the $-\hat{y}$ direction, so we are only interested in the vertical component of $d\vec g$. Using our diagram, this means that we want to find the magnitude of $dg\cos\theta$:
\begin{equation}
dg\cos\theta=\frac{G dM}{r^2}\cos\theta
\end{equation}
The magnitude of the gravitational field at point $P$ is thus given by:
\begin{equation}
g = \int dg\cos\theta =\int \frac{G dM}{r^2}\cos\theta
\end{equation}
The integral is written over $dM$, where both $r$, and $\theta$ are different for each different mass element, $dM$. We need to express any variable that changes for different mass elements in terms of a single variable of integration. We will choose $\theta$ as the variable of integration, and thus need to express $r$ and $dM$ in terms of $\theta$, $d\theta$, and other constants.

The distance, $r$, between $P$ and a mass element $dM$ located at angle $\theta$ is easily found to be:
\begin{equation}
r &= \frac{h}{\cos\theta}\\
\therefore \frac{1}{r^2} &= \frac{\cos^2\theta}{h^2}
\end{equation}

$dM$ can easily be expressed in term of $dx$ (the length of the mass element in the $x$ direction) and $\lambda$, the mass per unit length of the rod:
\begin{equation}
dM = \lambda dx = \frac{M}{L} dx
\end{equation}

We now need to express $dx$ in terms of $d\theta$. This can be found as follows, by first expressing $x$ in terms of $\theta$, and then taking the derivative of $x$ with respect to $\theta$
\begin{equation}
x &= h\tan\theta\\
\therefore \frac{dx}{d\theta}&=\frac{h}{\cos^2\theta}\\
\therefore dx  &= \frac{h}{\cos^2\theta} d\theta
\end{equation}
Now that we have found the small change in $x$ that results from a small change in $\theta$, we can write the mass element, $dM$, in terms of the $d\theta$:
\begin{equation}
dM = \frac{M}{L} dx = \frac{M}{L}  \frac{h}{\cos^2\theta} d\theta
\end{equation}
We can now write the integral in terms of $\theta$:
\begin{equation}
g = \int \frac{GdM}{r^2}\cos\theta&=G\int \frac{1}{r^2}\cos\theta dM\\
&=G\int \left( \frac{\cos^2\theta}{h^2} \right)\cos\theta \left(\frac{M}{L}  \frac{h}{\cos^2\theta}  \right)\\
&=\frac{GM}{Lh}\int\cos\theta d\theta
\end{equation}
Now that we have the integral over $\theta$, we need to set the limits to correspond to the values of $\theta$ at each end of the rod. The angle will have the same magnitude for each end of the rod, $\theta _0$, given by:
\begin{equation}
\sin\theta_0 =\frac{L}{2\sqrt{h^2 + \frac{L^2}{4}}}
\end{equation}
The magnitude of the field is thus given by:
\begin{equation}
g &=\frac{GM}{Lh}\int_{-\theta_0}^{\theta_0}\cos\theta d\theta\\
&= \frac{GM}{Lh} {\rm \left[n\theta\right]}_{-\theta_0}^{\theta_0}\\
&=\frac{2GM}{Lh} \sin\theta_0\\
&= \frac{2GM}{Lh} \frac{L}{2\sqrt{h^2 + \frac{L^2}{4}}}
\end{equation}
The gravitational field at point $P$ is thus given by:
\begin{equation}
\vec g =  -\frac{2GM}{Lh} \frac{L}{2\sqrt{h^2 + \frac{L^2}{4}}}\hat y
\end{equation}
\end{framed}

\include{ModelingWithPhysics-rotationaldynamics}

\subsection{Chapter 11 - Rotational energy and momentum}

\subsubsection{Overview}\label{chapter:angularmomentumrolling}

In this chapter, we extend our description of rotational dynamics to include the rotational equivalents of kinetic energy and momentum. We also develop the framework for describing the motion of rolling objects. We will see that many of the relations that hold for linear quantities also hold for angular quantities.

\begin{framed}
\textbf{Learning Objectives}\\
\begin{itemize}
\item Understand how to define the rotational kinetic energy of an object as well as its total kinetic energy.
\item Understand how to model rolling motion, and what slipping means in the context of rolling motion.
\item Understand how to define the angular momentum of an object and when it is conserved.
\end{itemize}

\begin{framed}
\textbf{Think About It}\\
How can you model the motion of a downwards going yo-yo?

\begin{enumerate}
\item It is similar to that of an object falling with a force of drag.
\item It is similar to that of an object rolling down an incline.
\item It is similar to that of an object sliding down an incline.
\item It is similar to that of an object rotating about a fixed axis of rotation.
\end{enumerate}

\begin{framed}
\textbf{Answer}\\
\begin{enumerate}[resume]
\item
\end{enumerate}
\end{framed}
\end{framed}
\end{framed}

\subsubsection{Rotational kinetic energy of an object}

In this section, we show how to define the rotational kinetic energy of an object that is rotating about a stationary axis in an inertial frame of reference. Consider a solid object that is rotating about an axis with angular velocity, $\vec\omega$, as depicted in Figure~\ref{fig:angularmomentumrolling:rotE}.

\begin{figure}[!htbp]
\centering
\includegraphics[width=0.4\linewidth]{files/rotE-3c26b1af2248a765fa579be7b104af95.png}
\caption[]{An object rotating about an axis that is perpendicular to the page.}
\label{fig:angularmomentumrolling:rotE}
\end{figure}

We can model the object as being composed of many point particles, each with a mass $m_i$, located at a position $\vec r_i$, with velocity $\vec v_i$ relative to the axis of rotation. We choose a coordinate system whose origin is on the axis of rotation and whose $z$ axis is co-linear with the axis of rotation, as depicted in Figure~\ref{fig:angularmomentumrolling:rotE}.

Each particle of mass $m_i$ in the object has a kinetic energy, $K_i$:
\begin{equation}
K_i = \frac{1}{2}m_iv_i^2
\end{equation}
We can sum the kinetic energy of each particle together to get the total rotational kinetic energy, $K_{rot}$, of the object:
\begin{equation}
K_{rot} = \sum_i \frac{1}{2}m_iv_i^2
\end{equation}
Although each particle will have a different velocity, $\vec v_i$, they will all have the same angular velocity, $\vec\omega$. For any particle, located a distance $r_i$ from the axis of rotation, their velocity is related to the angular velocity of the object by:
\begin{equation}
\vec v_i &= \vec \omega \times \vec r_i\\
v_i &= \omega r_i
\end{equation}
where $\vec \omega$ and $\vec r_i$ are always perpendicular to each other, since $\vec\omega$ is out of the plane of the page. Furthermore, the velocity vector, $\vec v_i$, will always be perpendicular to $\vec r_i$, since all particles are moving in circles centred about the axis of rotation.  We can thus write the total rotational kinetic energy of the object using the angular speed:
\begin{equation}
K_{rot} &= \sum_i \frac{1}{2}m_iv_i^2 = \sum_i \frac{1}{2}m_ir_i^2\omega^2= \frac{1}{2} \omega^2 \sum_i m_ir_i^2\\
&=\frac{1}{2}I\omega^2
\end{equation}
where we factored $\omega$ and the one half out of the sum, as these are the same for each particle $i$. We then recognized that the remaining sum is simply the definition of the object's moment of inertia about the axis:
\begin{equation}
I = \sum_i mr_i^2
\end{equation}

Thus, the rotational kinetic energy of an object rotating with angular speed $\omega$ about an axis that is stationary in an inertial frame of reference is given by:
\begin{equation}
\boxed{K_{rot}=\frac{1}{2}I\omega^2}
\end{equation}
where $I$ is the object's moment of inertia about that axis. The rotational kinetic energy is functionally very similar to the linear kinetic energy; instead of mass, we use the moment of inertia, and instead of speed squared, we use angular speed squared.

\paragraph{Work on a rotating object}

We can calculate the work done by a force exerted on an object rotating about a stationary axis in an inertial frame of reference. Let $\vec F$ be a force exerted at position, $\vec r$, relative to the axis of rotation at some instant in time, and let the force be exerted in the plane perpendicular to the axis of rotation, as illustrated in Figure~\ref{fig:angularmomentumrolling:work}. Because the object is rotating about the given axis, only the component of the force that is tangent to the circle about which the point where the force is exerted can do work (only the component of the force that is parallel to the displacement can do work).

The work done by the force as the object rotates by a certain angle is given by:
\begin{equation}
W = \int \vec F \cdot d\vec l = \int F_\perp dl
\end{equation}
where $d\vec l$ is a small displacement along the (circular) path followed by the point where the force is exerted, as illustrated in Figure~\ref{fig:angularmomentumrolling:work}. $F_\perp$ is the component of $\vec F$ that is perpendicular to the vector, $\vec r$, from the axis of rotation to the location where the force is exerted ($F_\perp$ is the component of $\vec F$ that is tangent to the circle).

\begin{figure}[!htbp]
\centering
\includegraphics[width=0.4\linewidth]{files/work-84fbe3453bb9868e829beb7d99b979d9.png}
\caption[]{Calculating the work done by a force on a rotating object.}
\label{fig:angularmomentumrolling:work}
\end{figure}

At some instant in time, when the force is exerted at position, $\vec r$, consider the scalar product between the torque from the force, $\vec \tau$, and an infinitesimal angular displacement, $d\vec \theta$, about the axis of rotation:
\begin{equation}
\vec\tau \cdot d\vec\theta = (\vec r \times \vec F) \cdot \left(\frac{1}{r^2} \vec r\times d\vec l\right)
\end{equation}
The vectors $\vec \tau$ and $d\vec \theta$ are parallel to the axis of rotation (because $\vec F$ and $d\vec l$ are in the plane perpendicular to the axis of rotation), so their scalar product will be equal to the product of their magnitudes. The vector $\vec r \times \vec F$ has a magnitude of:
\begin{equation}
\vec r \times \vec F = rF_\perp
\end{equation}
where $F_\perp$ is the component of the force tangent to the circle. The vector $\vec r\times d\vec l$ has a magnitude:
\begin{equation}
\vec r\times d\vec l = rdl
\end{equation}
since $\vec r$ and $d\vec l$ are always perpendicular. The scalar product $\vec\tau \cdot d\vec\theta$ is thus equal to:
\begin{equation}
\vec\tau \cdot d\vec\theta = rF_\perp \frac{1}{r^2} rdl = F_\perp dl
\end{equation}
The work done by a force when an object rotates about an axis can thus be written in terms of its torque about that axis and the corresponding angular displacement from $\theta_1$ to $\theta_2$:
\begin{equation}
W = \int_{\theta_1}^{\theta_2}\vec\tau\cdot d\vec \theta
\end{equation}

The net work done on an object through an angular displacement from $\theta_1$ to $\theta_2$ can thus be written using the net torque $\vec \tau^{net}$ exerted on the object:
\begin{equation}
W^{net} = \int_{\theta_1}^{\theta_2}\vec\tau^{net}\cdot d\vec \theta
\end{equation}
We can re-arrange this using Newton's Second Law for rotational dynamics:
\begin{equation}
\vec\tau^{net} &= I \vec\alpha\\
&= I \frac{d\vec\omega}{dt} =  I \frac{d\omega}{d\theta}\frac{d\vec\theta}{dt}=I \frac{d\omega}{d\theta} \vec\omega
\end{equation}
which allows us to write the integral over a change in angular velocity instead of angular displacement:
\begin{equation}
W^{net} &= \int_{\theta_1}^{\theta_2}\vec\tau^{net}\cdot d\vec \theta =  \int_{\theta_1}^{\theta_2}I \frac{d\omega}{d\theta} \vec\omega \cdot d\vec \theta\\
&=\int_{\omega_1}^{\omega_2}I \omega d\omega = \frac{1}{2}I\omega_2^2 - \frac{1}{2}I\omega_1^2
\end{equation}
where we used the fact that $\vec\omega$ are $d\vec\theta$ are parallel. We thus find that the Work-Energy Theorem can also be applied to find the change in rotational kinetic energy resulting from the net work done by a torque:
\begin{equation}
\boxed{W^{net}=\int_{\theta_1}^{\theta_2}\vec\tau^{net}\cdot d\vec \theta = \Delta K_{rot}}
\end{equation}

If a constant torque, $\vec\tau$, is exerted on an object that is rotating at constant angular velocity, $\vec\omega$, then the rate at which that work  is being done is given by:
\begin{equation}
P = \frac{dW}{dt} = \frac{d}{dt} \vec \tau \cdot d\vec\theta =  \vec \tau \cdot \frac{d\vec\theta}{dt} = \vec \tau \cdot \vec\omega
\end{equation}
This is very similar to the power, $P=\vec F\cdot \vec v$, with which a force does work on an object moving with constant velocity, except that instead of force we use torque, and instead of velocity, we use angular velocity.

\paragraph{Total kinetic energy of an object}

In the frame of reference of the centre of mass, an object rotating about an axis through its centre of mass with angular velocity, $\vec \omega$, will have rotational kinetic energy, $K_{rot}$, given by:
\begin{equation}
K_{rot}=\frac{1}{2}I_{CM}\omega^2
\end{equation}
where $I_{CM}$ is the moment of inertia of the object about the axis through its centre of mass.

We wish to determine the kinetic energy of the object in an inertial frame of reference where the object's centre of mass is moving with a velocity $\vec v_{cm}$; that is, in a frame where the axis of rotation is moving with the velocity of the centre of mass. We model the object as being composed of particles of mass, $m_i$, each located at position, $\vec r_i$, relative to the axis of rotation through the centre of mass. The velocity, $\vec v_i$, of a particle $i$, in this frame of reference, is given by:
\begin{equation}
\vec v_i = \vec\omega \times \vec r_i + \vec v_{CM}
\end{equation}
where $\vec\omega \times \vec r_i$ is the velocity of the particle as seen in the centre of mass (due to rotation). The kinetic energy of particle $i$, $K_i$, is given by:
\begin{equation}
K_i = \frac{1}{2}m_iv_i^2 = \frac{1}{2}m_i(\vec v_i\cdot \vec v_i)
\end{equation}
where we expressed the speed of the particle squared using a scalar product of the velocity of the particle with itself. The total kinetic energy of the object is found by summing the kinetic energies of all of the particles:
\begin{equation}
K_{tot} &= \sum \frac{1}{2}m_i(\vec v_i\cdot \vec v_i) \\
&=\frac{1}{2} \sum_i m_i (\vec\omega \times \vec r_i + \vec v_{CM}) \cdot (\vec\omega \times \vec r_i + \vec v_{CM})\\
&=\frac{1}{2} \sum_i m_i (\vec\omega \times \vec r_i)\cdot(\vec\omega \times \vec r_i ) + \frac{1}{2} \sum_i m_i (\vec v_{CM}) \cdot (\vec v_{CM}) + \sum_i m_i (\vec\omega \times \vec r_i) \cdot (\vec v_{CM})\\
&=\frac{1}{2}  \sum_i m_i \omega^2r_i^2 + \frac{1}{2} \sum_i m_i v_{CM}^2 + \sum_i m_i (\vec\omega \times \vec r_i) \cdot (\vec v_{CM})\\
&=\frac{1}{2} I_{CM}\omega ^2 + \frac{1}{2}M v_{CM}^2+\sum_i m_i (\vec\omega \times \vec r_i) \cdot (\vec v_{CM})
\end{equation}
where the first term is the rotational kinetic energy that we found earlier. The second term, called the ``translational kinetic energy'', can be thought of as the kinetic energy of the whole system with mass $M=\sum m_i$, due to the translational motion of the centre of mass. The last term is identically zero; we can re-order the scalar product and factor $\vec v_{CM}$ out of the sum:
\begin{equation}
\sum_i m_i (\vec\omega \times \vec r_i) \cdot (\vec v_{CM}) &= (\vec v_{CM}) \cdot \sum_i m_i (\vec\omega \times \vec r_i)\\
&=(\vec v_{CM}) \cdot \sum_i m_i \vec v'_{i}
\end{equation}
where $v'_{i} = \vec\omega \times \vec r_i$ is the velocity of particle $i$ in the center of mass frame of reference. But the sum:
\begin{equation}
\sum_i m_i \vec v'_{i}
\end{equation}
is the numerator for the definition of the velocity of the centre of mass, which, in the centre of mass frame of reference is identically zero!

Thus, the total kinetic energy of an object of mass, $M$, that is rotating about an axis through its centre of mass with angular velocity, $\omega$, and whose centre of mass is moving with velocity, $\vec v_{CM}$, is given by:
\begin{equation}
\boxed{K_{tot}=K_{rot}+K_{trans}=\frac{1}{2} I_{CM}\omega ^2 + \frac{1}{2}M v_{CM}^2}
\end{equation}
The total kinetic energy can be thought of as the sum of the rotational and kinetic energies.

\subsubsection{Rolling motion}

In this section, we examine how to model the motion of an object that is rolling along a surface, such as the motion of a bicycle wheel. Consider the motion of a wheel of radius, $R$, rotating with angular velocity, $\vec\omega$, about an axis perpendicular to the wheel and through its centre of mass, \textbf{as observed in the centre of mass frame}. This is illustrated in Figure~\ref{fig:angularmomentumrolling:wheelcm}.

\begin{figure}[!htbp]
\centering
\includegraphics[width=0.3\linewidth]{files/wheelcm-0e33844708b1777dab52fec6eb370889.png}
\caption[]{A wheel rotating with angular velocity $\vec\omega$ about an axis through its centre of mass.}
\label{fig:angularmomentumrolling:wheelcm}
\end{figure}

In the frame of reference of the centre of mass, each point on the edge of the wheel has a velocity, $\vec v_{rot}$, due to rotation given by:
\begin{equation}
\vec v_{rot} = \vec \omega\times \vec r
\end{equation}
where $\vec r$ is a vector (of magnitude $R$) from the centre of mass to the corresponding point on the edge of the wheel (shown in Figure~\ref{fig:angularmomentumrolling:wheelcm} for a point on the lower left of the wheel). The vector $\vec r$ is always perpendicular to $\vec \omega$, so that the speed of all points on the edge, as measured in the frame of reference of the centre of mass, is the same:
\begin{equation}
\label{eq:angularmomentumrolling:vrot}
v_{rot} = \omega R
\end{equation}
as illustrated in Figure~\ref{fig:angularmomentumrolling:wheelcm}.

Now, suppose that the whole wheel is moving, as it rolls on the ground, such that the centre of mass of the wheel moves with a velocity, $\vec v_{CM}$, as illustrated in Figure~\ref{fig:angularmomentumrolling:wheelground}.

\begin{figure}[!htbp]
\centering
\includegraphics[width=0.95\linewidth]{files/wheelground-ea054c3bfbed539816b3e6d137584dbb.png}
\caption[]{A wheel rolling without slipping on the ground, with the centre of mass moving with velocity $\vec v_{CM}$.}
\label{fig:angularmomentumrolling:wheelground}
\end{figure}

The wheel is shown at different instants in time, as the point shown in red moves around the centre of mass.\}
In the frame of reference of the ground, each point on the edge of the wheel will have a velocity $\vec v$ given by:
\begin{equation}
\vec v = \vec v_{rot} + \vec v_{CM}
\end{equation}
That is, in the frame reference of the ground, each point will have a velocity obtained by (vectorially) adding its velocity relative to the centre of mass, $\vec v_{rot}$, and the velocity of the centre of mass relative to the ground, $\vec v_{CM}$. This is illustrated in Figure~\ref{fig:angularmomentumrolling:wheelground} for one specific point, shown in red. The red vector corresponds to the velocity of the red point as the wheel rotates, and is obtained by adding the velocity of the centre of mass, $\vec v_{CM}$, and the velocity, $\vec v_{rot}$, relative to the centre of mass (shown as the dashed vector, tangent to the edge of the wheel).

Consider, specifically, the instant in time when the red point is at the bottom of the wheel, where the wheel makes contact with the ground. \textbf{If the wheel is not slipping with respect to the ground}, then the point is, at that instant, at rest relative to the ground. We call this type of motion ``rolling without slipping''; the point on the rotating object that is in contact with the ground is instantaneously at rest relative to the ground. This is the scenario illustrated in Figure~\ref{fig:angularmomentumrolling:wheelground}.

For the point in contact with the ground, the vectors $\vec v_{rot}$ and $\vec v_{CM}$ are anti-parallel, horizontal, and must sum to zero. Writing out the horizontal component of the velocity of that point (choosing the positive direction to be in the direction of the velocity of the centre of mass):
\begin{equation}
v &= -v_{rot} + v_{CM} = 0\\
\therefore v_{rot} &= v_{CM}
\end{equation}
and we find that, for rolling without slipping, the speed due to rotation about the centre of mass has to be equal to the speed of the centre of mass. The speed due to rotation about the centre of mass can be expressed using the angular velocity of the wheel about the centre of mass ((\ref{eq:angularmomentumrolling:vrot})). For rolling without slipping, we thus have the following relationship between angular velocity and the speed of the centre of mass:
\begin{equation}
\boxed{\omega R = v_{CM}}\quad \text{(rolling without slipping)}
\end{equation}
It makes sense for the angular velocity to be related to the speed of the centre of mass. The faster the wheel rotates, the faster the centre of mass will move. If the wheel is slipping with respect to the ground, then the point of contact is no longer stationary relative to the ground, and there is no relation between the angular velocity and the speed of the centre of mass. For rolling with slipping, imagine the motion of your bicycle wheel as you try to ride your bike on a slick sheet of ice.

For rolling without slipping, the magnitude of the linear acceleration of the centre of mass, $a_{CM}$, is similarly related to the magnitude of the angular acceleration of the wheel, $\alpha$, about the centre of mass:
\begin{equation}
a_{CM} &= \frac{dv_{CM}}{dt} = \frac{d}{dt}\omega R = R \frac{d\omega}{dt}\\
\therefore a_{CM} &= R\alpha
\end{equation}

\begin{framed}
\textbf{Checkpoint}\\
For rolling without slipping (Figure~\ref{fig:angularmomentumrolling:wheelground}), the speed of the point on the wheel that is in contact with the ground is 0. What is the speed of the point at the top of the wheel?

\begin{enumerate}
\item 0
\item $v_{CM}$.
\item $2v_{CM}$.
\item None of the above.
\end{enumerate}

\begin{framed}
\textbf{Answer}\\
\begin{enumerate}[resume]
\item
\end{enumerate}
\end{framed}
\end{framed}

\begin{framed}
\textbf{Example 11.1}\\
\begin{figure}[!htbp]
\centering
\includegraphics[width=0.5\linewidth]{files/diskslope-7e1ab66d6b3e0fd4d814dc76ee898212.png}
\caption[]{A disk rolling without slipping down an incline.}
\label{fig:angularmomentumrolling:diskslope}
\end{figure}

A disk of mass $M$ and radius $R$ is placed on an incline at a height $h$ above the ground. The incline makes an angle $\theta$ with respect to the horizontal, as shown in Figure~\ref{fig:angularmomentumrolling:diskslope}. If the disk starts at rest and rolls without slipping down the incline, what speed will the centre of mass have when the disk reaches the bottom of the incline?

\begin{framed}
\textbf{Solution}\\
We can use the conservation of mechanical energy to determine the speed of the centre of mass at the bottom of the incline, as there are no non-conservative forces doing work on the disk. If we choose to define gravitational potential energy such that it is zero at the bottom of the incline, we can write the total mechanical energy of the disk at the top of the incline as:
\begin{equation}
E = K+U=(0)+Mgh
\end{equation}
where the kinetic energy is zero, since the disk starts at rest\footnote{Technically, the potential energy should be taken for the height of the centre of mass, which is a distance $h_{CM}=h+R\cos\theta$ from the ground at the top of the incline, and a height $h'_{CM}=R$ at the bottom of the incline. The net difference in height for the centre of mass is thus $h_{CM} -h'_{CM} = h+R(1 -\cos\theta)$. If we assume that $h$ is much bigger than $R$, then this is negligible, otherwise, that is what we should use instead of $h$ for the potential energy.}. At the bottom of the incline, the disk will have only kinetic energy, since the potential energy at the bottom is defined to be zero. The kinetic energy of the disk will have a component from the rotation of the disk about the centre of mass, with angular speed $\omega$, and a component from the translation of the centre of mass with speed $v_{CM}$. The mechanical energy at the bottom of the incline is thus:
\begin{equation}
E' = K' + U = K'_{rot}+K'_{trans}+(0)=\frac{1}{2}I_{CM}\omega^2 + \frac{1}{2}Mv_{cm}^2
\end{equation}
Since the disk is rolling without slipping, its angular speed is related to the speed of centre of mass:
\begin{equation}
\omega = \frac{v_{CM}}{R}
\end{equation}
The moment of inertia of the disk about its centre of mass is given by:
\begin{equation}
I_{CM}=\frac{1}{2}MR^2
\end{equation}
We can thus write the mechanical energy at the bottom of the incline as:
\begin{equation}
E' &= \frac{1}{2}I_{CM}\omega^2 + \frac{1}{2}Mv_{cm}^2\\
&=\frac{1}{2}\left(  \frac{1}{2}MR^2 \right) \left(  \frac{v_{CM}}{R}\right)^2+ \frac{1}{2}Mv_{cm}^2\\
&=\frac{3}{4}Mv_{cm}^2
\end{equation}
Applying conservation of energy allows us to determine the speed of the centre of mass at the bottom of the incline:
\begin{equation}
E &= E'\\
Mgh &= \frac{3}{4}Mv_{cm}^2\\
\therefore v_{CM} &= \sqrt{\frac{4}{3}gh}
\end{equation}

\textbf{Discussion:} This example showed how we can use the conservation of energy to model the motion of an object that is rolling without slipping. The constraint of rolling without slipping allowed for the angular speed of the object to be related to the speed of its centre of mass.
\end{framed}
\end{framed}

\begin{framed}
\textbf{Checkpoint}\\
A hoop, a disk, and a sphere roll without slipping down an incline. If they are all released at the same time, in what order will they arrive at the bottom?\}

\begin{enumerate}
\item Hoop, disk, sphere.
\item Sphere, disk, hoop.
\item Disk, sphere, hoop.
\item Disk, hoop, sphere.
\end{enumerate}

\begin{framed}
\textbf{Answer}\\
\begin{enumerate}[resume]
\item
\end{enumerate}
\end{framed}
\end{framed}

\paragraph{The instantaneous axis of rotation}

When an object is rolling without slipping, we can model its motion as the superposition of rotation about the centre of mass and translational motion of the centre of mass, as in the previous section. However, because the point of contact between the rolling object and the ground is stationary, we can also model the motion as if the object were instantaneously rotating with angular velocity, $\vec \omega$, about a stationary axis through the point of contact. That is, we can model the motion as rotation only, with no translation, if we choose an axis of rotation through the point of contact between the ground and the wheel.

We call the axis through the point of contact the ``instantaneous axis of rotation'', since, instantaneously, it appears as if the whole wheel is rotating about that point. This is illustrated in Figure~\ref{fig:angularmomentumrolling:wheelinstant}, which shows, in red, the velocity vector for each point on the edge of the wheel, relative to the instantaneous axis of rotation. Because the axis of rotation is fixed to the ground, the velocity of each point about that axis of rotation corresponds to the same velocity relative to the ground that is depicted in Figure~\ref{fig:angularmomentumrolling:wheelground}.

\begin{figure}[!htbp]
\centering
\includegraphics[width=0.4\linewidth]{files/wheelinstant-2a6ca6fe77408916b27c34e10ac10053.png}
\caption[]{A wheel that is rolling without slipping, as viewed if rotating about the instantaneous axis of rotation that passes through the point of contact with the ground.}
\label{fig:angularmomentumrolling:wheelinstant}
\end{figure}

In particular, the angular velocity, $\vec \omega$, about the instantaneous axis of rotation is the same as when we model the motion as translation plus rotation about the centre of mass ,as in the previous section. Indeed, relative to the instantaneous axis of rotation, the centre of mass must still have a velocity $\vec v_{CM}$, which is given by:
\begin{equation}
\vec v_{CM} &= \vec\omega \times \vec r_{CM}\\
\therefore v_{CM} &= \omega R
\end{equation}
where $\vec r_{CM}$ is the vector from the axis of rotation to the centre of mass. This is the same condition for rolling without slipping that we found before. Similarly, the velocity of any point on the wheel, relative to the ground, is given by:
\begin{equation}
\vec v = \vec\omega \times \vec r
\end{equation}
where $\vec r$ is the vector from the axis of rotation to the point of interest (shown in Figure~\ref{fig:angularmomentumrolling:wheelinstant} for the point on the right side of the wheel). In particular, the velocity vector (in red) for any point is always perpendicular to the vector $\vec r$ for that point, which was not necessarily obvious when modelling the motion as rotation plus translation, as in Figure~\ref{fig:angularmomentumrolling:wheelground}.

\begin{framed}
\textbf{Example 11.2}\\
\begin{figure}[!htbp]
\centering
\includegraphics[width=0.5\linewidth]{files/diskslope-7e1ab66d6b3e0fd4d814dc76ee898212.png}
\caption[]{A disk rolling without slipping down an incline.}
\label{fig:angularmomentumrolling:diskslope2}
\end{figure}

A disk of mass $M$ and radius $R$ is placed on an incline at a height $h$ above the ground. The incline makes an angle $\theta$ with respect to the horizontal, as shown in Figure~\ref{fig:angularmomentumrolling:diskslope2}. What is the angular acceleration of the disk, about an axis through its centre of mass, as it rolls without slipping down the slope?

\begin{framed}
\textbf{Solution}\\
In order to determine the angular acceleration of the disk about the centre of mass, we need to model the forces that are exerted on the disk. The forces exerted on the disk are:

\begin{enumerate}
\item $\vec F_g$, the weight of the disk, exerted downwards at the centre of mass, with magnitude $Mg$.
\item $\vec N$, a normal force perpendicular to the incline, exerted by the incline at the point of contact with the disk.
\item $\vec f_s$, a force of static friction parallel to the incline, exerted by the incline at the point of contact with the disk. Without this force, the disk would simply slide down the incline without rotating.
\end{enumerate}

These forces are illustrated in Figure~\ref{fig:angularmomentumrolling:diskslope_fbd}, along with the acceleration of the centre of mass, and our choice of coordinate system (we choose the $x$ axis parallel to the acceleration of the centre of mass, to facilitate applying Newton's Second Law).

\begin{figure}[!htbp]
\centering
\includegraphics[width=0.4\linewidth]{files/diskslope_fbd-6c0bbf5637c869c1dcecbcc6a840deaf.png}
\caption[]{The forces on the disk rolling without slipping down an incline.}
\label{fig:angularmomentumrolling:diskslope_fbd}
\end{figure}

The angular acceleration of the disk about the centre of mass, $\vec \alpha$ is given by Newton's Second Law for rotational dynamics:
\begin{equation}
\vec\tau^{ext} = I_{CM}\vec\alpha
\end{equation}
where $\vec\tau^{ext}$ is the net external torque on the disk about the centre of mass (which will be in the negative $z$ direction).

The only force that can exert a torque about the centre of mass is the force of static friction. Gravity has a lever arm of zero and the normal force is anti-parallel to the vector that goes from the centre of mass to the point where the force is exerted. The net torque about the centre of mass is thus:
\begin{equation}
\vec\tau^{ext} = \vec \tau_{f_s} = \vec r_{f_s}\times \vec f_s= -Rf_s\hat z
\end{equation}
The angular acceleration will thus be in the negative $z$ direction, and the magnitude is given by:
\begin{equation}
\alpha = \frac{\tau^{ext}}{I_{CM}}=\frac{Rf_s}{\frac{1}{2}MR^2}=\frac{2f_s}{MR}
\end{equation}
However, we do not know the magnitude of the force of static friction. We can use the $x$ and $y$ components of Newton's Second Law to determine it (with acceleration of the centre of mass in the $x$ direction):
\begin{equation}
\sum F_x &= F_g\sin\theta - f_s = Ma_{CM}\\
\sum F_y &= N - F_g\cos\theta = 0
\end{equation}
Because the disk is rolling without slipping, the acceleration of the centre of mass is related to the angular acceleration of the disk:
\begin{equation}
a_{cm} = \alpha R
\end{equation}
The $x$ component of Newton's Second Law can thus be used to determine the magnitude of the force of static friction in terms of the angular acceleration:
\begin{equation}
Mg\sin\theta - f_s &= M \alpha R\\
\therefore f_s &= Mg\sin\theta - M\alpha R
\end{equation}
We can then substitute out the force of friction from our previous formula for the angular acceleration:
\begin{equation}
\alpha &= \frac{2f_s}{MR}\\
&=\frac{2Mg\sin\theta - 2M\alpha R}{MR} = \frac{2g\sin\theta}{R} - 2\alpha \\
\therefore \alpha &= \frac{2g\sin\theta}{3R}
\end{equation}

Instead of modelling the motion of the disk  as rotation about the centre of mass and translation of the center of mass, we can also model it about the instantaneous axis of rotation.

The angular acceleration about the instantaneous axis of rotation will be the same as the angular acceleration about the centre of mass. About the instantaneous axis of rotation, only the force of gravity can exert a torque, since the normal force and the force of friction both have a lever arm of zero. The torque from the force of gravity, about the instantaneous axis of rotation is:
\begin{equation}
\vec \tau_g = -F_gR\sin\theta \hat z = -MgR\sin\theta \hat z
\end{equation}
The torque from the force of gravity is equal to the moment of inertia of the disk about the instantaneous axis of rotation, $I$, multiplied by its angular acceleration:
\begin{equation}
\tau ^{ext} = \tau_g &= I\alpha\\
\therefore \alpha &= \frac{\tau_g}{I} = \frac{MgR\sin\theta}{I}
\end{equation}
The moment of inertia about the instantaneous axis of rotation is easily found using the parallel axis theorem:
\begin{equation}
I = I_{CM}+MR^2 = \frac{1}{2}MR^2 + MR^2 =\frac{3}{2}MR^2
\end{equation}
This allows us to find the angular acceleration of the disk:
\begin{equation}
\alpha &= \frac{MgR\sin\theta}{I} = \frac{MgR\sin\theta}{\frac{3}{2}MR^2}\\
&=\frac{2g\sin\theta}{3R}
\end{equation}
as we found previously, but in this case, we did not need to use Newton's Second Law to determine the force of friction.

\textbf{Discussion:} We saw that we can model the dynamics of the rolling body using either an axis through the centre of mass, or an axis through the instantaneous axis of rotation. The latter was easier in this case, because it did not require using Newton's Second Law.

By using an axis through the centre of mass to model the motion of the disk, it was clear that the force of static friction is required in order for the disk to rotate. Without the force of static friction, the disk would slide along the surface of the incline. The disk could still rotate if there is a force of kinetic friction that causes a torque that rotates the disk. If the surface were completely frictionless, the disk would simply slide down the incline, and we could model it as a sliding block. If the incline is too steep the force of static friction is no longer sufficient to provide the necessary torque required for the angular acceleration to be that which corresponds to rolling without slipping, and the disk would slip.
\end{framed}
\end{framed}

\subsubsection{Angular momentum}

In this section, we show that we can define a quantity called ``angular momentum'' as the rotational equivalent of the linear momentum.

\paragraph{Angular momentum of a particle}

The angular momentum relative to a point of rotation, $\vec L$, of a particle with linear momentum, $\vec p$, is defined as:
\begin{equation}
\boxed{\vec L = \vec r\times \vec p}
\end{equation}
where $\vec r$ is the vector from the point of rotation to the particle, and the linear momentum, $\vec p$, is defined relative to an inertial frame of reference in which the point of rotation is at rest.

Consider the time-derivative of angular momentum (where we have to use the product rule for derivatives):
\begin{equation}
\frac{d\vec L}{dt}  &= \frac{d}{dt} (\vec r\times \vec p)\\
&=\frac{d\vec r}{dt}\times \vec p + \vec r\times\frac{d\vec p}{dt}\\
&=\vec v\times \vec p + \vec r\times\frac{d\vec p}{dt}\\
\end{equation}
The first term is zero, since $\vec v$ is parallel to $\vec p$ by definition. Recall Newton's Second Law written using linear momentum:
\begin{equation}
\frac{d\vec p}{dt} = \vec F^{net}
\end{equation}
where $\vec F^{net}$ is the net force on the particle relative to the point of rotation. The rate of change of angular momentum is thus given by:
\begin{equation}
\frac{d\vec L }{dt} &= \vec r\times\frac{d\vec p}{dt}\\
&=\vec r\times\vec F^{net}
\end{equation}
where the term on the right is the net torque on the particle. Thus, the rate of change of angular momentum is given by:
\begin{equation}
\boxed{\frac{d\vec L}{dt}   = \vec \tau^{net}}
\end{equation}
which is analogous to the linear case, but we used angular momentum instead of linear momentum and net torque instead of net force. The net torque on a particle is thus equal to the rate of change of its angular momentum. In particular, the angular momentum of a particle will remain constant (not change with time) if the net torque on the particle is zero.

We can also define the angular momentum of a particle using only angular quantities:
\begin{equation}
\vec L = \vec r \times \vec p =  m \vec r \times \vec v = mr^2 \vec\omega
\end{equation}
where we factored the mass $m$ out of the momentum and used the definition $\vec \omega = 1/r^2(\vec r \times \vec v)$. We can think of $mr^2$ as the moment of inertia, $I$, of the particle and write:
\begin{equation}
\label{eq:angularmomentumrolling:liw}
\boxed{\vec L  = mr^2 \vec\omega = I \vec\omega}
\end{equation}
which is a close analogue to the definition of linear momentum, but we use moment of inertia instead of mass and angular velocity instead of velocity.

The angular momentum is thus parallel to the angular velocity of the particle about the point of rotation. If no net torque is exerted on the particle about that point, then the particle's angular momentum about that point will remain constant. We can also consider the torque and angular momentum about an axis instead of a point; in that case, we would simply take the components of torque and angular momentum that are parallel to that axis.

\begin{framed}
\textbf{Example 11.3}\\
\begin{figure}[!htbp]
\centering
\includegraphics[width=0.25\linewidth]{files/circle-d31ab27f0e6119d03652d0284fcc0b14.png}
\caption[]{A small block attached to a mass-less string moving in a horizontal circle on a table.}
\label{fig:angularmomentumrolling:circle}
\end{figure}

A small block of mass $m$ attached to a mass-less string is moving along a circle of radius $R$ on a horizontal table, as depicted from above in Figure~\ref{fig:angularmomentumrolling:circle}. If the table is frictionless: are the block's linear and/or angular momentum with respect to the axis of rotation conserved? If there is friction between the table and the block, are the block's linear and/or angular momentum with respect to the axis of rotation conserved? What can you say about the kinetic energy of the block in the two cases?

\begin{framed}
\textbf{Solution}\\
If there is no friction between the block and the table, then the forces exerted on the block are:

\begin{enumerate}
\item $\vec F_g$, the block's weight, exerted downwards, with magnitude $mg$.
\item $\vec N$, a normal force, exerted upwards, with magnitude $mg$.
\item $\vec T$, a force of tension, exerted towards the centre of the circle.
\end{enumerate}

All of these forces are perpendicular to the (tangential) displacement of the block along the circle. Thus, there can be no work done on the block and its speed, $v$, must remain constant. The kinetic energy of the block must thus remain constant.

The sum of the forces on the block must be towards the centre of the circle, since the block is in uniform circular motion. The linear momentum of the block cannot be conserved if there is a net force on the block (and clearly, the block's velocity vector changes direction as it goes around the circle).

The forces of weight and the normal force are both outside of the plane of motion, and thus cannot exert a torque along the axis of rotation. They are also equal and opposite in magnitude so the net torque from those two forces is always zero (since the net force from those forces is zero). The force of tension is always anti-parallel to the vector $\vec r$, from the axis of rotation to the particle, and cannot result in a torque about the rotation axis. Thus, the net torque on the block is zero and its angular momentum must be conserved.

If there is kinetic friction exerted by the table on the block, then there is an additional force, $\vec f_s$, exerted on the block in the direction opposite of motion (tangent to the circle, in the opposite direction from the block's velocity).

The force of friction will do negative work on the block, slowing it down and reducing its kinetic energy, which is no longer conserved. The net force on the block is non-zero, so its linear momentum is still not conserved. Finally, the force of friction, which is always perpendicular to $\vec r$, will result in a torque that reduces the angular velocity of the block. The block's angular momentum is thus no longer conserved when there is friction between the table and the block.

\textbf{Discussion:} In this example, we saw that kinetic energy, linear momentum, and angular momentum are all conserved under different conditions. Kinetic energy is conserved if no net work is done on the block. Linear momentum is conserved if the net force on the block is zero. Angular momentum is conserved if the net torque on the block is zero. By introducing angular momentum, we are able to use a new conserved quantity to help us model rotational dynamics.
\end{framed}
\end{framed}

\begin{framed}
\textbf{Example 11.4}\\
\begin{figure}[!htbp]
\centering
\includegraphics[width=0.3\linewidth]{files/line-7dfe9b02759d368dff3d32815c69cde0.png}
\caption[]{A particle moving in a straight line.}
\label{fig:angularmomentumrolling:line}
\end{figure}

A particle is moving with constant velocity $\vec v$ (in a straight line) relative to a coordinate system in an inertial frame of reference, as shown in Figure~\ref{fig:angularmomentumrolling:line}. Show that its angular momentum about the origin is conserved.

\begin{framed}
\textbf{Solution}\\
In this case, the particle is moving in a straight line, but we can still define its angular momentum relative to the origin. If $\vec r$ is the position vector of the particle relative to the origin, its angular momentum is:
\begin{equation}
\vec L = \vec r \times \vec p
\end{equation}
We can take the time derivative of the angular momentum to see if it changes with time:
\begin{equation}
\frac{d\vec L}{dt} = &= \frac{d}{dt} (\vec r\times \vec p)\\
&=\frac{d\vec r}{dt}\times \vec p + \vec r\times\frac{d\vec p}{dt}\\
&=\vec v\times \vec p + \vec r\times\frac{d\vec p}{dt}\\
\end{equation}
The first term is zero because $\vec v$ and $\vec p$ are parallel (so their cross-product must be zero). The second term is zero because the particle's momentum is constant in time (since its velocity is constant). Thus, the particle's angular momentum does not change with time, and it is conserved.

\textbf{Discussion:} Of course, we expected this result since no net torque is exerted on the particle. It is however worth highlighting that a particle does not need to be rotating for its angular momentum about a given axis to be defined or conserved; all that matters is that there is no net torque on the particle relative to that axis.
\end{framed}
\end{framed}

\paragraph{Angular momentum of an object or system}

Consider a system made of many particles of mass, $m_i$, each with a position, $\vec r_i$, and velocity, $\vec v_i$, relative to a point of rotation that is fixed in an inertial frame of reference.

We can write Newton's Second Law using the angular momentum, $\vec L_i$, for particle $i$:
\begin{equation}
\frac{d\vec L_i}{dt} = \vec \tau_i^{net}
\end{equation}
where $\vec \tau_i^{net}$ is the net torque exerted on particle $i$. We can sum each side of this equation for all of the particles in the system:
\begin{equation}
\frac{d\vec L_1}{dt} + \frac{d\vec L_2}{dt} + \frac{d\vec L_3}{dt} + \dots &= \vec \tau_1^{net} + \vec \tau_2^{net} +\vec \tau_3^{net} + \dots\\
\therefore \frac{d}{dt} \sum_i\vec L_i &= \sum_i \vec \tau_i^{net}
\end{equation}
The sum of all of the torques on all of the particles will include a sum over torques that are internal to the system and torques that are external to the system. The sum over internal torques is zero:
\begin{equation}
\sum_i \vec\tau_i^{net} = \sum_i \vec\tau_i^{int} + \sum_i \vec\tau_i^{ext} = \sum_i \vec\tau_i^{ext} = \vec\tau^{ext}
\end{equation}
where we defined, $\vec\tau^{ext}$, to be the net external torque exerted on the system. We also introduce the total angular momentum of the system, $\vec L$, as the sum of the angular momenta of the individual particles:
\begin{equation}
\vec L = \sum_i\vec L_i
\end{equation}
The rate of change of the total angular momentum of the system is then given by:
\begin{equation}
\boxed{\frac{d\vec L}{dt} = \vec \tau^{ext}}
\end{equation}

Up to this point, we did not require that the system be a solid object, so the particles in the system can move relative to each other. For example, the particles could be the Sun, planets, and everything else that is in our Solar System. The total angular momentum of all of the bodies in the Solar System (say, relative to the Sun) is conserved if there is no net torque on the solar system relative to the Sun (i.e. if there is no torque about the Sun exerted on any of the bodies in the system that is not exerted by one of the other bodies in the system).

Now, consider a solid object that is modelled as a system of many particles of mass, $m_i$, at position, $\vec r_i$, with velocity, $\vec v_i$, relative to a fixed axis of rotation. We can define the angular momentum of a single particle as equation (\ref{eq:angularmomentumrolling:liw}):
\begin{equation}
\vec L_i = m_i r_i^2 \vec \omega_i
\end{equation}
The total momentum of the system is the sum of the angular momenta of the individual particles:
\begin{equation}
\vec L &= \sum_i\vec L_i = \sum_i  m_i r_i^2 \vec \omega_i
\end{equation}
Because all of the particles are part of the same object, they must all move in unison and have the same angular velocity, $\vec\omega$, relative to the axis of rotation. We can thus define the angular momentum about the rotation axis for a solid object with angular velocity, $\vec\omega$, as:
\begin{equation}
\boxed{\vec L = \left(\sum_i  m_i r_i^2\right) \vec \omega = I\vec\omega}
\end{equation}
where we recognized that the sum in parentheses is simply the moment of inertia of the object relative to the axis of rotation. Again, it should be emphasized that this is the total angular momentum of the object about an axis of rotation, and not about a point.

Visualizing the torque and angular momentum of a system can be challenging because it almost always requires visualizing something in three dimensions. Consider a wheel (e.g. a bicycle wheel) that is spinning about horizontal axle which you hold with your hands, as illustrated in the left panel of Figure~\ref{fig:angularmomentumrolling:deltal} (without the hands). Imagine that you are holding onto the axle so that the wheel is front of you, your right hand is to the right of the wheel and your left hand is to the left of the wheel.

\begin{figure}[!htbp]
\centering
\includegraphics[width=0.4\linewidth]{files/deltal-b83960f835c131f13adeb1c0c670abe8.png}
\caption[]{A wheel rotating on an axle, with a horizontal angular velocity (left). If you try to tilt the axle as shown in the right panel, changing the angular momentum of the wheel, you will also need to exert a torque in the vertical direction (shown at the bottom right).}
\label{fig:angularmomentumrolling:deltal}
\end{figure}

We define a coordinate system as shown so that the wheel is spinning as shown in the left panel, with angular velocity (and angular momentum) in the positive $x$ direction (the top of the wheel is coming towards you).

You then try to lift your right hand while lowering your left hand in order to tilt the rotation axis, as shown in the right panel. In doing so, you change the direction of the angular momentum (and angular velocity) of the wheel such that the angular momentum, $\vec L'$, now has a vertical component, $\Delta \vec L$, as shown. The torque that is required in order to change the angular momentum is given by:
\begin{equation}
\vec \tau = \frac{d\vec L}{dt} \sim \frac{\Delta \vec L}{\Delta t}
\end{equation}
where $\Delta t$ is the time that it takes to change the axis of rotation. The torque required in order to change the axis of rotation is directed in the same direction as $\Delta \vec L$ (the positive $y$ direction). That is, you will not be able to simply tilt the axle as shown; if you want to tilt the axle, you will also need to push forward with you right hand and pull backwards with your left hand to exert the required torque (shown in the bottom right of the figure)! If you simply try to tilt the rotation axis, your right hand will be pushed towards you and your left hand away from you, as a reaction to the torque that would otherwise be required to tilt the axis!

\paragraph{Conservation of angular momentum}

In the previous section, we saw that the net external torque that is exerted on an object (or system) is equal to the rate of change of its angular momentum:
\begin{equation}
\frac{d\vec L}{dt} = \vec \tau^{ext}
\end{equation}
where the angular momentum and torque are measured about the same axis or point of rotation, fixed in an inertial frame of reference.

The total angular momentum of a system about a point of rotation is conserved (i.e. does not change with time) if there is no net external torque exerted on the system about that point. If one makes the system large enough, then all of the torques can be taken to be internal, and the angular momentum of the system is conserved. The angular momentum of the Universe about a fixed point is thus conserved.

Conservation of angular momentum is another conservation law that we derived from Newton's Second Law. In the modern formulation of physics, we understand that the conservation of angular momentum is associated with rotational symmetry of Newton's Second Law; it does not matter from which ``angle'' we model a system, we can always use Newton's Second Law. Similarly, conservation of linear momentum is associated with translational symmetry and conservation of energy is associated with the fact that Newton's Second Law does not change with time. Angular momentum is fundamentally different than linear momentum and energy, and is conserved under different conditions. The angular momentum of a system about a given axis/point is conserved if there is no net torque on the system about that axis/point.

\begin{framed}
\textbf{Example 11.5}\\
During a spin, a figure skater brings his arms close to his body and increases his angular velocity from $\omega_1$ to $\omega_2$. By what fraction did his moment of inertia decrease in doing so?

\begin{framed}
\textbf{Solution}\\
We can consider the rotation axis to be vertical through the centre of the skater. When the figure skater is spinning, there is no net external torque on him. Thus, his angular momentum is conserved as he bring his arms in. As he bring his arms in, his moment of inertia decreases, since he is bringing the mass of his arms closer to the axis of rotation. If $I_1$ and $I_2$ are the moments of inertia of the skater before and after brining his arms in, respectively, we can write the angular momentum about his axis of rotation as:
\begin{equation}
L_1 &= I_1\omega_1\\
L_2 &= I_2\omega_2
\end{equation}
Since there is no external torque on the skater, the angular momentum is the same before and after he changes his moment of inertia:
\begin{equation}
L_1 &= L_2\\
I_1\omega_1 &= I_2\omega_2\\
\therefore \frac{I_1}{I_2} &= \frac{\omega_2}{\omega_1}
\end{equation}
\textbf{Discussion:} A spinning figure skater is a good example of the conservation of angular momentum. By changing their shape, they can change their moment of inertia and thus their angular velocity.
\end{framed}
\end{framed}

\begin{framed}
\textbf{Example 11.6}\\
Show that Kepler's Second Law is equivalent to a statement about conservation of the angular momentum of a planet orbiting the Sun.

\begin{framed}
\textbf{Solution}\\
Kepler's Second Law states that in a period of time $\Delta t$, the area, $\Delta A$, that is swept out by a planet is constant, regardless of where it is along its orbit. In other words:
\begin{equation}
\frac{\Delta A}{\Delta t} = \text{constant}
\end{equation}
Figure~\ref{fig:angularmomentumrolling:kepler} shows a planet in an elliptical orbit around the sun.

\begin{figure}[!htbp]
\centering
\includegraphics[width=0.5\linewidth]{files/kepler-4f0dc3c3a6aa1990dece4a276ade8e92.png}
\caption[]{The area swept out by a planet in a period of time $dt$.}
\label{fig:angularmomentumrolling:kepler}
\end{figure}

At some point in time, the planet has a velocity vector $\vec v$ and position vector $\vec r$ relative to the Sun. In a small period of time $dt$, the planet will move along a short distance $vdt$, which we can take as a straight line if $dt$ is small enough. Let $\phi$ be the angle between the velocity and position vectors when these are tail to tail, as illustrated.

The small amount of area, $dA$, swept out by the planet in a period of time $dt$, is given by the area of the right angle triangle with height $r$ and base $vdt\sin\phi$\footnote{This is only exact in the limit of $dt\to 0$, when the small area from the extra piece outside of the ellipse vanishes.}:
\begin{equation}
dA = \frac{1}{2} r vdt\sin\phi
\end{equation}
The rate at which the area is swept out is thus:
\begin{equation}
\frac{dA}{dt} =  \frac{1}{2} r v\sin\phi
\end{equation}
Consider now the magnitude of the planet's angular momentum about the Sun:
\begin{equation}
L = rp\sin\phi = rmv\sin\phi
\end{equation}
where the mass of the planet is $m$. The rate at which the planet sweeps out the area can be written in terms of the angular momentum of the planet:
\begin{equation}
\frac{dA}{dt} &=  \frac{1}{2} r v\sin\phi = \frac{L}{2m}
\end{equation}
The only force exerted on the planet is the gravitational force from the Sun. That force is always anti-parallel to the vector $\vec r$ from the Sun to the planet, and cannot result in a torque on the planet about the Sun. Thus, the angular momentum of the planet about the Sun must be conserved, and $L$ is constant. In turn, this means that the rate at which area is swept out by the planet, which is proportional to $L$, is also constant. Thus, Kepler's Second Law is equivalent to saying that the angular momentum of a planet relative to the Sun is constant.
\end{framed}
\end{framed}

\subsubsection{Summary}

If an object is rotating with angular speed, $\omega$, about an axis that is fixed in an inertial frame of reference, the rotational kinetic energy of that object is given by:
\begin{equation}
K_{rot} = \frac{1}{2}I\omega^2
\end{equation}
where $I$ is the moment of inertia of that object about the axis of rotation.

The net work done by the net torque exerted on an object about a fixed axis or rotation in an inertial frame of reference is equal to object's change in rotational kinetic energy:
\begin{equation}
W = \int_{\theta_1}^{\theta_2}\vec \tau^{net}\cdot d\vec \theta = \frac{1}{2}I\omega_2^2 -\frac{1}{2}I\omega_1^2
\end{equation}
If a torque, $\vec \tau$, about a stationary axis is exerted on an object that is rotating with a constant angular velocity, $\vec \omega$, about that axis, then the torque does work at a rate:
\begin{equation}
P = \vec \tau \cdot \vec \omega
\end{equation}

If an object of mass, $M$, is rotating about an axis through its centre of mass, and the centre of mass of is moving with speed, $v_{CM}$, relative to an inertial frame of reference, then the total kinetic energy of the object is given by:
\begin{equation}
K_{tot} = K_{rot} + K_{trans} = \frac{1}{2}I_{CM}\omega^2+ \frac{1}{2}Mv_{CM}^2
\end{equation}
where, $\omega$, is the angular speed of the object about the centre of mass, and, $I_{CM}$, is the moment of inertia of the object about the centre of mass. The two terms in the kinetic energy come from the rotation about the centre of mass ($K_{rot}$), and the translational motion of the centre of mass ($K_{trans}$).

An object is said to be rolling without slipping on a surface if the point on the object that is in contact with the surface is instantaneously at rest relative to the surface. We can model an object that is rolling without slipping by superimposing rotational motion about the centre of mass with translational motion of the centre of mass. The angular speed, $\omega$, and the angular acceleration, $\alpha$, of the object about an axis through its centre of mass are related to the speed, $v_{CM}$, and linear acceleration, $a_{CM}$, of the centre of mass, respectively:
\begin{equation}
v_{CM} &= \omega R\\
a_{CM} &= \alpha R
\end{equation}
These conditions are equivalent to stating that the object is rolling without slipping.

When an object is rolling without slipping, we can also model its motion as if it were instantaneously rotating about an axis that goes through the point of contact between the object and the ground (the instantaneous axis of rotation). The angular speed (and acceleration) about the instantaneous axis of rotation are the same as they are when the object is modelled as rotating about its (moving) centre of mass.

An object can only be rolling without slipping if there is a force of static friction exerted by the surface on the object. Without this force, the object would slip along the surface.

We can define the angular momentum of a particle, $\vec L$, about a point in an inertial frame of reference as:
\begin{equation}
\vec L = \vec r \times \vec p
\end{equation}
where, $\vec r$, is the vector from the point to the particle, and, $\vec p$, is the linear momentum of the particle. If the particle has an angular velocity, $\vec\omega$, relative to an axis of rotation its angular momentum about that axis can be written as:
\begin{equation}
\vec L = mr^2\vec\omega = I\vec\omega
\end{equation}
where, $r$, is the distance between the particle and the axis of rotation, and $I=mr^2$, can be thought of as the moment of inertia of the particle about that axis.

We can write the equivalent of Newton's Second Law for the rotational dynamics of a particle using angular momentum:
\begin{equation}
\frac{d\vec L}{dt}=\vec\tau^{net}
\end{equation}
where, $\vec \tau^{net}$, is the net torque on the particle about the same point used to define angular momentum. That point must be in an inertial frame of reference.

The rate of change of the total angular momentum for a system of particles, $\vec L=\vec L_1 + \vec L_2 +\dots$, about a given point is given by:
\begin{equation}
\frac{d\vec L}{dt}=\vec\tau^{ext}
\end{equation}
where, $\vec\tau^{ext}$, is the net external torque on the system about the point of rotation. If the net external torque of the system is zero, then the total angular momentum of the system is constant (conserved). Again, the point of rotation must be in an inertial frame of reference\footnote{Technically, if the point is the centre of mass, then this is valid even in an accelerating frame of reference.}.

For a solid object, in which all of the particles must move in unison, we can define the angular momentum of the object about a stationary axis to be:
\begin{equation}
\vec L = I\vec \omega
\end{equation}
where, $\vec\omega$, is the angular velocity of the object about that axis, and, $I$, is the object's corresponding moment of inertia about that axis.

Many of the relations that exist between linear quantities have an analogue relation between the corresponding angular quantities, as summarized in the table below:

\begin{table}
\centering
\begin{tabular}{p{\dimexpr 0.250\linewidth-2\tabcolsep}p{\dimexpr 0.250\linewidth-2\tabcolsep}p{\dimexpr 0.250\linewidth-2\tabcolsep}p{\dimexpr 0.250\linewidth-2\tabcolsep}}
\toprule
\textbf{Name} & \textbf{Linear} & \textbf{Angular} & \textbf{Correspondence} \\
\hline
Displacement & $s$ & $\vec \theta$ & $d\vec\theta=\frac{1}{r^2} \vec r\times d\vec s$ \\
Velocity & $\vec v$ & $\vec \omega$ & $\vec\omega=\frac{1}{r^2} \vec r\times \vec v$, $v_s = \vec\omega\times \vec r$\footnote{This corresponds to the component of velocity perpendicular to $\vec r$.} \\
Acceleration & $\vec a$ & $\vec \alpha$ & $\vec\alpha=\frac{1}{r^2} \vec r\times \vec a$, $a_s = \vec\alpha\times \vec r$\footnote{This corresponds to the component of acceleration perpendicular to $\vec r$.\}} \\
Inertia & $m$ & $I$ & $I=\sum_i m_ir_i^2$ \\
Momentum & $\vec p=m\vec v$ & $\vec L = I\vec \omega$ & $\vec L = \vec r\times \vec p$ \\
Newton's Second Law & $\vec F^{ext}=m\vec a_{CM}$ & $\vec \tau^{ext} = I\vec\alpha$ & $\vec F \to \vec\tau$, $m\to I$, $\vec a \to \vec \alpha$ \\
Newton's Second Law & $\frac{d\vec p}{dt} =\vec F^{ext}$ & $\frac{d\vec L}{dt} =\vec \tau^{ext}$ & $\vec F \to \vec\tau$, $\vec p \to \vec L$ \\
Kinetic energy & $\frac{1}{2}mv^2$ & $\frac{1}{2}I\omega^2$ & $m\to I$, $v\to \omega$ \\
Power & $\vec F \cdot \vec v$ & $\vec \tau \cdot \vec\omega$ & $\vec F \to \vec\tau$, $\vec v\to \vec\omega$ \\
\bottomrule
\end{tabular}
\end{table}

\begin{framed}
\textbf{Important Equations}\\
\textbf{Rotational kinetic energy of a rotating object:}
\begin{equation}
K_{rot} = \frac{1}{2}I\omega^2
\end{equation}
\textbf{Total kinetic energy:}
\begin{equation}
K_{tot} = K_{rot} + K_{trans} = \frac{1}{2}I_{CM}\omega^2+ \frac{1}{2}Mv_{CM}^2
\end{equation}
\textbf{Work:}
\begin{equation}
W = \int_{\theta_1}^{\theta_2}\vec \tau^{net}\cdot d\vec \theta = \frac{1}{2}I\omega_2^2 -\frac{1}{2}I\omega_1^2
\end{equation}
\textbf{Power:}
\begin{equation}
P = \vec \tau \cdot \vec \omega
\end{equation}

\textbf{Angular momentum:}
\begin{equation}
\vec L = \vec r \times \vec p\\
\vec L = mr^2\vec\omega = I\vec\omega\\
\frac{d\vec L}{dt}=\vec\tau^{net}\\
\frac{d\vec L}{dt}=\vec\tau^{ext}\\
\vec L = I\vec \omega\\
\end{equation}
\end{framed}

\begin{framed}
\textbf{Important Definitions}\\
\begin{itemize}
\item \textbf{Angular momentum:} The rotational equivalent of linear momentum. Angular momentum must be defined relative to an axis of rotation. SI units: ${\rm \left[{kg\cdot m^2\cdot s^{ -1}}\right]}$. Common variable(s): $\vec L$.
\item \textbf{Rotational kinetic energy:} The rotational equivalent of translation kinetic energy. Generally, an object can have both rotational and translational kinetic energy.  SI units: ${\rm \left[{J}\right]}$. Common variables: $K_{rot}$.
\end{itemize}
\end{framed}

\subsubsection{Thinking about the material}

\begin{framed}
\textbf{Reflect and research}\\
\begin{itemize}
\item How can a bicycle move forward? Draw the external forces on the bicycle that are required for the wheels to turn.
\item Does conservation of angular momentum play a role in being able to remain upright on a bicyle? If yes, how?
\item How does an anti-lock braking system (ABS) provide better breaking for your car? What is the physics behind this?
\end{itemize}
\end{framed}

\begin{framed}
\textbf{To try at home}\\
\begin{itemize}
\item Describe how you can qualitatively confirm conservation of angular momentum.
\end{itemize}
\end{framed}

\begin{framed}
\textbf{To try in the lab}\\
\begin{itemize}
\item Propose an experiment to measure the critical angle of an incline, above which a given object cannot roll without slipping, and compare this to a model prediction.
\item Propose an experiment to test the conservation of angular momentum of a rotating object.
\item Propose an experiment to test whether an object with constant velocity can impart angular momentum to another object.
\end{itemize}
\end{framed}

\subsubsection{Sample problems and solutions}

\paragraph{Problems}

\begin{framed}
\textbf{Problem}\\
A yo-yo can be modelled as two uniform disks, of radius $R_2$, attached to either side of a smaller uniform disk of radius $R_1$, as in Figure~\ref{fig:angularmomentumrolling:yoyo}. We can assume that all three disks have a mass $m$. A mass-less string is wrapped around the smaller disk and then the yo-yo is released. What is the acceleration of the centre of mass of the yo-yo as it falls and the string unwinds?

\begin{figure}[!htbp]
\centering
\includegraphics[width=0.5\linewidth]{files/yoyo-8fcc3b35b0626567c8d6c678f4922917.png}
\caption[]{Left: Side view of the yo-yo. Right: Front view of the yo-yo, modelled as two disks of radius of $R_2$ attached to either side of a disk of radius $R_1$.}
\label{fig:angularmomentumrolling:yoyo}
\end{figure}
\end{framed}

\begin{framed}
\textbf{Problem}\\
\begin{figure}[!htbp]
\centering
\includegraphics[width=0.3\linewidth]{files/balldisk-68702f5e17e23a584057698b0fa628f4.png}
\caption[]{A projectile of mass $m$ is about to collide with a disk that can spin about its axis of symmetry. View from above.}
\label{fig:angularmomentumrolling:balldisk}
\end{figure}

A projectile of mass $m$ is fired towards a stationary disk of radius $R$ and mass $M$ that lies on a horizontal table, as depicted from above in Figure~\ref{fig:angularmomentumrolling:balldisk}. The disk is in the horizontal plane and can rotate about a vertical axis through its centre. The axle about which the disk rotates is attached to the table and cannot move. The projectile's velocity, $\vec v$, is horizontal and such that the projectile embeds itself at the edge of the disk.  What is the angular velocity of the disk, about its centre, after the projectile has embedded itself into the disk? Was the collision elastic? Was linear momentum conserved during the collision?
\end{framed}

\paragraph{Solutions}

\begin{framed}
\textbf{Solution}\\
The forces acting on the yo-yo are:

\begin{itemize}
\item $\vec F_g$, its weight, with magnitude $3mg$.
\item $\vec T$, a force of tension from the string.
\end{itemize}

The forces, where they are exerted, and our choice of coordinate system are shown in Figure~\ref{fig:angularmomentumrolling:yoyo_fbd}.

\begin{figure}[!htbp]
\centering
\includegraphics[width=0.3\linewidth]{files/yoyo_fbd-ecc661c3724d09f92dbaf7fbbadd684d.png}
\caption[]{Free body diagram for the yo-yo.}
\label{fig:angularmomentumrolling:yoyo_fbd}
\end{figure}

The yo-yo can be modelled as rolling without slipping, as if it were rolling along the string that unwinds. The torque about the centre of mass is provided by the tension in the string. The angular acceleration of the yo-yo, $\alpha$, will be related to the linear acceleration of the centre of mass, $\vec a_{CM}$, since this is rolling without slipping:
\begin{equation}
a_{CM}=\alpha R_1
\end{equation}
where $R_1$ is the radius that is analogous to rolling motion. Since the torque from the force of gravity is zero, we can write Newton's Second Law for rotational quantities as:
\begin{equation}
\vec\tau^{ext}&=I\vec\alpha\\
TR_1 &= I\alpha
\end{equation}
where $TR_1$ is the magnitude of the torque from the force of tension, since the tension is perpendicular to the vector $\vec r$ between the centre of mass and the point where the tension is exerted. The moment of inertia of the yo-yo about its centre of mass is the sum of the moments of inertia of the three disks about their axis of symmetry:
\begin{equation}
I=\frac{1}{2}MR_2^2 +\frac{1}{2} MR_2^2+\frac{1}{2}MR_1^2=\frac{1}{2}M(2R_2^2+R_1^2)
\end{equation}
We can also write Newton's Second Law in the vertical direction for the yo-yo (of mass $3M$):
\begin{equation}
\sum F_y = -F_g + T &= -3Ma_{CM}\\
-3Mg + T = -3Ma_{CM}
\end{equation}
where we $a_{CM}$ is the magnitude of the acceleration of the centre of mass (since we included the sign in the first equation).

We can eliminate the unknown force of tension from the equations by substitution. Using the equation from Newton's Second Law:
\begin{equation}
T=3M (g-a_{CM})
\end{equation}
and substituting this into the rotational equation:
\begin{equation}
TR_1 &= I\alpha\\
3M (g-a_{CM}) R_1 &=  I\alpha
\end{equation}
We can solve for $a_{CM}$ by using the condition for rolling without slipping ($\alpha R_1 = a_{CM}$):
\begin{equation}
3M (g-a_{CM}) R_1 &=  I\frac{a_{CM}}{R_1}\\
\frac{I}{R_1}a_{CM}+3MR_1a_{CM}&= 3MgR_1\\
a_{CM}\left(\frac{I}{R_1}+3MR_1\right)&= 3MgR_1\\
a_{CM}&=\frac{3MgR_1}{\frac{I}{R_1}+3MR_1}\\
&=\frac{3MgR_1}{\frac{\frac{1}{2}M(2R_2^2+R_1^2)}{R_1}+3MR_1}\\
&=\left(\frac{3R_1^2}{\frac{1}{2}(2R_2^2+R_1^2)+3R_1^2}\right)g\\
\therefore a_{CM}&=\left( \frac{3R_1^2}{R_2^2+\frac{7}{2}R_1^2}\right)g
\end{equation}
\end{framed}

\begin{framed}
\textbf{Solution}\\
We consider the projectile and disk as a system, and a rotation axis that passes through the centre of disk. There are no external torques exerted on the system about the rotation axis, so the angular momentum of the system must be conserved through the collision. Before the collision, only the projectile has angular momentum about the axis of rotation, so the magnitude of the angular momentum before the collision is:
\begin{equation}
L = rp\sin\phi
\end{equation}
where $\phi$ is the angle between the particle's momentum, $\vec p=m\vec v$, and a vector, $\vec r$, from the axis of rotation to the particle. We can calculate the particle's angular momentum just before the collision, so that $\vec r$ is the vector from the centre of the circle to the point where the particle collides (with magnitude $R$, and perpendicular to $\vec v$). The initial angular momentum of the system is thus:
\begin{equation}
L=rp=Rmv
\end{equation}
After the collision, the projectile is embedded in the disk. The resulting object has a moment of inertia given by:
\begin{equation}
I = I_{disk}+ I_{particle} = \frac{1}{2}MR^2+mR^2
\end{equation}
After the collision, the angular momentum of the disk with the embedded projectile is given by:
\begin{equation}
L'=I\omega = \left(\frac{1}{2}M+m\right)R^2\omega
\end{equation}
Using conservation of angular momentum, the angular velocity of the disk after the collision is:
\begin{equation}
L &= L'\\
Rmv &= \left(\frac{1}{2}M+m\right)R^2\omega\\
\therefore \omega &= \frac{mv}{\left(\frac{1}{2}M+m\right)R}
\end{equation}
We do not expect that mechanical energy is conserved during the collision, since the projectile embeds itself, which must cost energy. The mechanical energy before the collision is given by the kinetic energy of the projectile:
\begin{equation}
E = \frac{1}{2}mv^2
\end{equation}
After the collision, the kinetic energy is the rotational kinetic energy of the disk with embedded projectile about the axis of rotation:
\begin{equation}
E' &= \frac{1}{2}I\omega^2 = \frac{1}{2} \left(\frac{1}{2}M+m\right)R^2 \left( \frac{mv}{\left(\frac{1}{2}M+m\right)R}\right)^2\\
&= \frac{1}{2} \frac{m^2}{\left(\frac{1}{2}M+m\right)} v^2
\end{equation}
We can see that $E'$ is less than $E$, by taking their ratio:
\begin{equation}
\frac{E'}{E} &= \frac{\frac{1}{2} \frac{m^2}{\left(\frac{1}{2}M+m\right)} v^2}{\frac{1}{2}mv^2}\\
&=\frac{m}{\left(\frac{1}{2}M+m\right)}<1
\end{equation}
and we confirm that mechanical energy is not conserved in the collision (and that energy was lost since one had to deform the projectile and disk).

Linear momentum is clearly not conserved since the final linear momentum is zero, whereas before the collision, it is $\vec p=m\vec v$. The centre of mass of the disk+projectile system moves before the collision and not after. There must thus be a net external force that is exerted on the system. That force is exerted by the table onto the axle of disk, as the disk would otherwise recoil when hit with the projectile.

\textbf{Discussion:} In this example, we used conservation of angular momentum to model a collision. The collision is inelastic, because the projectile embeds itself into the disk. The linear momentum is not conserved through the collision because the axle about which the disk rotates must exert a force on the disk to prevent it from recoiling.
\end{framed}

\section{Part 2 - Oscillations, Waves, Fluids, Electricity, \& Magnetism}

\include{ModelingWithPhysics-simpleharmonicmotion}

\include{ModelingWithPhysics-waves}

\include{ModelingWithPhysics-fluidmechanics}

\subsection{Chapter 15 - Electric charges and fields}

\subsubsection{Overview}\label{chapter:chargesfields}

In this and subsequent chapters, we start to look at the theories that describe electric and magnetic phenomena. Within the framework for dynamics that was developed by Newton, we will introduce the theories of electromagnetism which describe the electric force, the magnetic force, and how these two interact. This first chapter introduces the description of the electric force, analogously to how we introduced Newton's Universal Theory of Gravity to describe the gravitational force.

\begin{framed}
\textbf{Learning Objectives}\\
\begin{itemize}
\item Understand the definition of an electric charge.
\item Understand the difference between an insulator and a conductor.
\item Understand different mechanisms for charging objects.
\item Understand Coulomb's model for the electric force.
\item Understand the definition of an electric field.
\item Understand how to calculate the electric field from a continuous distribution of charge.
\item Understand how to model an electric dipole.
\end{itemize}
\end{framed}

\begin{framed}
\textbf{Think About It}\\
If you rub a balloon against a carpet and bring it near your head, your hair will stand up and try to touch the balloon.

\begin{enumerate}
\item The electric charge of the balloon is opposite of that on your hair.
\item Your hair has no net electric charge, this is an example of charge separation and induction.
\end{enumerate}

\begin{framed}
\textbf{Answer}\\
\begin{enumerate}[resume]
\item
\end{enumerate}
\end{framed}
\end{framed}

\subsubsection{Electric charge}

You have likely experienced or heard about electric charge in your life. For example, on a dry winter day, you might find that after rubbing your bare feet on a polyester carpet, you feel a small electric shock upon touching a metallic surface such as a doorknob. You probably also know that there are positive and negative charges, and that equal charges repel each other whereas opposite charges attract. In this chapter, we develop the description of how these charges can accumulate and how they exert attractive or repulsive forces on each other.

Ordinary matter is made of atoms, which are themselves made of a small positive nucleus (containing positive protons and neutral neutrons) surrounded by a ``cloud'' of negatively charged electrons. Within a solid object, inter-atomic forces hold the atoms together, so we can model the atoms as being effectively stationary. This means that we can treat the positive nuclei as being fixed in space. The negative electrons, depending on the material, can often move from one atom to another. If an atom loses an electron to another atom, it becomes positive, whereas the atom that acquired the extra electron becomes negative.

We define the net charge on an atom (or an object) based on whether there are more protons (positive), more electrons (negative) or an equal amount (neutral). By default, atoms are neutral and have an equal number of protons and electrons. An object becomes charged when it acquires an excess (or deficit) of electrons from another object. We say that ``charge is conserved'' because the number of electrons in the system does not change, i.e. if one object became positively charged, a different object must have become negatively charged by the same amount, so that the total net charge (in the Universe) is zero.

When you rub two objects together, you allow electrons to be transferred between them. Different materials have different ``affinities'' for electrons, and the electrons will transfer to the object with the highest affinity. Returning to our earlier example, when you rub your bare feet on the polyester carpet, electrons are being removed from your feet and deposited onto the carpet. Since your feet lose electrons, you acquire a net positive charge, while the carpet acquires a net negative charge. This way of creating a net charge on an object is called ``charging by friction''. If you then touch a doorknob, electrons will jump from the doorknob and onto your (now positively charged) body, creating a spark.

The ``triboelectic series'' is a list of materials that tend to give up or acquire electrons when they are placed in close contact with each other. Some common materials from the series are shown in Figure~\ref{fig:chargesfields:triboseries}.  The greatest charge is generated by rubbing together materials that are the furthest apart from each other in the diagram. Rubbing silk on a piece of glass results in more charge than rubbing wool on the same piece of glass.

\begin{figure}[!htbp]
\centering
\includegraphics[width=1\linewidth]{files/triboseries-4b056b04414ba5a33d35300655cca63b.png}
\caption[]{A sample of a triboelectric series of materials. The materials on the right-hand side have the greatest affinity to acquire electrons.}
\label{fig:chargesfields:triboseries}
\end{figure}

\begin{framed}
\textbf{Checkpoint}\\
If you rub a glass rod with silk, which object ends up with an excess of electrons?\}

\begin{enumerate}
\item glass rod.
\item silk.
\item neither, they remain neutral.
\item both will acquire an excess of electrons.
\end{enumerate}

\begin{framed}
\textbf{Answer}\\
\begin{enumerate}[resume]
\item
\end{enumerate}
\end{framed}
\end{framed}

\paragraph{Conductors and insulators}

We can broadly classify materials into conductors (such as metals), and insulators (such as wood), depending on how easily the electrons can move around in the material. In a conductor, electrons (rather, the outer electron(s) of an atom) are only loosely bound to their nuclei, and they can thus move around the material freely. In an insulator, the electrons are tightly bound to the nuclei of their atoms and cannot easily move around. There is a third class of materials, semi-conductors, that falls somewhere between a conductor and an insulator. In a semi-conductor, electrons are typically bound to their atoms, but any additional electrons present in the material can move around as if they are in a conductor.

Within a conductor, such as a solid metallic sphere, charges can move around freely. If that sphere has a net charge, for example an excess of electrons, those excess electrons will migrate to the outer surface of the sphere. Electrons in the sphere repel each other and will quickly settle into a position where they are, on average, the furthest from all of the other electrons, which occurs if all of the electrons migrate to the surface. This is illustrated in the left panel of Figure~\ref{fig:chargesfields:conductioncharge}. If an initially neutral conducting sphere is connected to the charged sphere by a conducting wire (right panel of Figure~\ref{fig:chargesfields:conductioncharge}), some of the electrons will ``conduct'' (transfer) onto the surface of the neutral sphere, so that, on average, they are further from all other electrons. This way of adding charge to the neutral sphere is called ``charging by conduction'', and the second sphere will remain charged if the connection between spheres is broken.

\begin{figure}[!htbp]
\centering
\includegraphics[width=0.7\linewidth]{files/conductioncharge-13c1ad8aca42859d6e0e2273eeb697c0.png}
\caption[]{Charging by conduction: a neutral conducting sphere is connected to a negatively charged conducting sphere. The charges can ``spread out more'' if some of the charges move (``conduct'') from the charge sphere onto the neutral sphere.}
\label{fig:chargesfields:conductioncharge}
\end{figure}

\paragraph{Electrostatic induction}

Electrostatic induction allows one to ``induce'' a charge by using the fact that charges can move freely in a conductor. The left panel of Figure~\ref{fig:chargesfields:induction} shows a (neutral) rod made of a conducting material, with electrons distributed uniformly throughout its volume. In the right panel, a negatively charged sphere is brought next to the rod. The negative sphere repels the electrons in the rod. Since the rod is conducting, the electrons can move around easily, and so they move to the end of the rod that is furthest from the negative sphere. Those electrons will leave positive charges (corresponding to the atoms that have lost their electrons) on the side closest to the sphere. The electrons in the rod will only accumulate for as long as the force from the negative sphere is greater than the repulsive force from the electrons that have already accumulated. In practice, such an equilibrium is reached almost instantly. In equilibrium, we say that the rod is ``polarized'', or that the ``charges in the rod have separated'', although the rod is overall still neutral.

Note that we can model this as if it where positive charges that move inside of the rod instead of negative charges. The positive charges are attracted to the negative sphere, and thus accumulate on the end of the rod closest to the sphere, leaving a negative charge on the other end. The choice to call electrons ``negative'' is completely arbitrary. For convenience, we usually build models assuming that positive charges can easily move around, even if, in reality, it is almost always actually (negative) electrons that move.

\begin{figure}[!htbp]
\centering
\includegraphics[width=0.9\linewidth]{files/induction-61f57859342bff8910b8621b0315b66b.png}
\caption[]{Electrostatic induction: when a negatively charged sphere is brought close to a neutral conducting rod, the electrons in the rod, which can move freely, accumulate on the side of the rod furthest from the sphere, leaving an excess of positive charge near the sphere.}
\label{fig:chargesfields:induction}
\end{figure}

We can create a net charge on the polarized rod if we provide a conducting path for charges to leave (or enter) the rod. The Earth can be modelled as a very large reservoir of both positive and negative charges. By connecting the rod to the Earth (we say that we connect the rod to ``ground''), we provide a path for the electrons in the rod to be even further from the negatively charged sphere, and they can thus leave the rod entirely in order to go into the ground. This is illustrated in the right-hand panel of Figure~\ref{fig:chargesfields:inductioncharge}.

If we then disconnect the rod from the ground, it has now acquired an overall positive charge, as in the right hand panel. We call this ``charging by induction''. We can also think of this in terms of positive charges moving into the rod from the Earth; when we connect the rod to the ground, the positive charges in the Earth can move into the rod and get closer to the negatively charged sphere. If we disconnect the rod from the ground, the rod stays positive, just as we conclude when using a model where it is the negative charges that move\footnote{Unless magnetism is involved, it is not possible to distinguish between a flow of positive charges moving in one direction or negative charges moving in the opposite direction.
[\^41\};Others had initially observed the inverse square law for the electric force, but Coulomb was the first to formalize the theory.}.

\begin{figure}[!htbp]
\centering
\includegraphics[width=0.9\linewidth]{files/inductioncharge-7664b6a2f86d0f9e6767bf0ce16ca03e.png}
\caption[]{Charging by induction: when we connect the polarized rod to the ground, electrons can leave the rod. If we now disconnect the rod from ground, the rod is left with an overall positive charge.}
\label{fig:chargesfields:inductioncharge}
\end{figure}

\begin{framed}
\textbf{Olivia's Thoughts}\\
Before we leave this section, let's take a minute to summarize the different methods for charging an object.

\begin{enumerate}
\item Charging by friction: We start with two neutral objects. To charge the objects, we rub them together. We end up with two oppositely charged objects. To explain this, we need to know that atoms can exchange electrons when they are in close contact and that different materials have different affinities for electrons.
\item Charging by conduction: We start with one neutral object and one charged object. To charge the neutral object, we put it in contact with the charged object. Charges move from the charged object to the neutral object, so we end up with two objects that are both positive or both negative. To explain this, we need to know that like charges repel, and that charges can move around freely inside conductors.
\item Charging by induction: We start with one neutral object and one charged object. To charge the neutral object, we move the charged object close to (but not touching) the neutral object to separate the charges in the neutral object. We then use some mechanism (e.g. connecting to ground) to create a net charge. We end up with one positively charged object and one negatively charged object. Once again, we need to know that like charges repel, and that charges can move around freely in conductors.
\end{enumerate}
\end{framed}

\subsubsection{The Coulomb force}

Coulomb was the first to provide a detailed quantitative description of the force between charged objects. Nowadays, we use the (derived) SI unit of ``Coulomb'' (C) to represent charge. The ``charge'' of an object corresponds to the net excess (or lack) of electrons on the object. An electron has a charge of $-e= -1.6\times 10^{ -19} {\rm C}$. Thus, an object with a charge of $-1 {\rm C}$ has an excess of about $\frac{1}{1.6\times 10^{ -19}}=6.25\times 10^{18}$ electrons on it, which is a very large charge. If an object has an excess of electrons, it is negatively charged and we indicate this with a negative sign on the charge of the object. An object with a (positive) charge of $1 {\rm C}$ thus has a deficit of $6.25\times 10^{18}$ electrons.

Through careful studies of the force between two charged spheres, Coulomb observed[\^41] that:

\begin{itemize}
\item The force is attractive if the objects have opposite charges and repulsive if the objects have the same charge.
\item The force is inversely proportional to the squared distance between spheres.
\item The force is larger if the charges involved are larger.
\end{itemize}

This leads to Coulomb's Law for the electric force (or simply ``Coulomb's Law''), $\vec F_{12}$, exerted on a point charge $Q_1$ by another point charge $Q_2$:
\begin{equation}
\boxed{\vec F_{12}=k\frac{Q_1Q_2}{r^2}\hat r_{21}}
\end{equation}
where $\hat r_{21}$ is the unit vector from $Q_2$ to $Q_1$ and $r$ is the distance between the two charges, as illustrated in Figure~\ref{fig:chargesfields:coulombforce}. Coulomb's constant, $k=8.99\times 10^9 {\rm N\cdot m^2/C^{2}}$, is simply a proportionality constant to ensure that the quantity on the right will have units of Newtons when all other quantities are in S.I. units. In some instances, it is more convenient to use the ``permittivity of free space'', $\epsilon_0$, rather than Coulomb's constant, in which case Coulomb's Law has the form:
\begin{equation}
\vec F_{12}=\frac{1}{4\pi\epsilon_0}\frac{Q_1Q_2}{r^2}\hat r_{21}
\end{equation}
where $\epsilon_0=\frac{1}{4\pi k}=8.85\times 10^{ -12} {\rm C^2\cdot N^{ -1}\cdot m^{ -2}}$ is a more fundamental constant, as we will see in later chapters.

\begin{figure}[!htbp]
\centering
\includegraphics[width=0.5\linewidth]{files/coulombforce-596abbc328557b381fefd6d5708cb366.png}
\caption[]{Vectors involved in applying Coulomb's Law.}
\label{fig:chargesfields:coulombforce}
\end{figure}

If the two charges have positions $\vec r_1$ and $\vec r_2$, respectively, then the vector $\hat r_{21}$ is given by:
\begin{equation}
\hat r_{21} = \frac{\vec r_2 - \vec r_1}{||\vec r_2 - \vec r_1||}
\end{equation}
Coulomb's Law is mathematically identical to the gravitational force in Newton's Universal Theory of Gravity. Rather than quantity of mass determining the strength of the gravitational force, it is the quantity of charge that determines the strength of the electric force. The only major difference is that gravity is always attractive, whereas the Coulomb force can be repulsive.

\begin{framed}
\textbf{Checkpoint}\\
The Coulomb force is conservative.

\begin{enumerate}
\item True.
\item False.
\end{enumerate}

\begin{framed}
\textbf{Answer}\\
\begin{enumerate}
\item
\end{enumerate}
\end{framed}
\end{framed}

The product $Q_1Q_2$ in the numerator of Coulomb's force is positive if the two charges have the same sign (both positive or both negative) and negative if the charges have opposite signs. Again, referring to Figure~\ref{fig:chargesfields:coulombforce}, if the two charges are positive, the force on $Q_1$ will point in the same direction as $\hat r_{21}$ (since all of the scalars are positive in Coulomb's Law) and thus be repulsive. If, instead, the two charges have opposite signs, the product $Q_1Q_2$ will be negative and the force vector on $Q_1$ will point in the opposite direction from $\hat r_{21}$ and the force is attractive.

\begin{framed}
\textbf{Example 15.1}\\
Calculate the magnitude of the electric force between the electron and the proton in a hydrogen atom and compare this to the gravitational force between them.

\begin{framed}
\textbf{Solution}\\
We model this by assuming that the electron and proton are point charges a distance of $1 \overset{\circ}{\rm A}=1\times 10^{ -10} {\rm m}$ apart (1 Angstrom is about the size of the hydrogen atom). The proton and electron have the same charge with magnitude $e=1.6\times 10^{ -19} {\rm C}$, so the (attractive) electric force between them has a magnitude of:
\begin{equation}
F_e &= k\frac{Q_1Q_2}{r^2}\\
&=(9\times 10^9 {\rm N\cdot m^2/C^{2}})\frac{(1.6\times 10^{-19} {\rm C})(1.6\times 10^{-19} {\rm C})}{(1\times 10^{-10} {\rm m})^2}\\
&=2.3\times 10^{-8} {\rm N}
\end{equation}
which is a small number, but acting on a very small mass. In comparison, the force of gravity between an electron ($m_e=9.1\times 10^{ -31} {\rm kg}$) and a proton ($m_p=1.7\times 10^{ -27} {\rm kg}$) is given by:
\begin{equation}
F_g&=G\frac{m_em_p}{r^2}\\
&=(6.7\times 10^{-11} {\rm Nm^2/kg^2})\frac{(9.1\times 10^{-31} {\rm kg})(1.7\times 10^{-27} {\rm kg})}{(1\times 10^{-10} {\rm m})^2}\\
&=1.04\times 10^{-47} {\rm N}
\end{equation}
\textbf{Discussion:} As we can see, the electric force between an electron and a proton is 39 orders of magnitude larger than the gravitational force! This shows that the gravitational force is extremely weak on the scale of particles and has essentially no effect in particle physics. Indeed, the best current theory of particle physics, and the most precisely tested theory in physics, the ``Standard Model'', does not need to include gravity in order to provide a spectacularly precise description of particles. One of the big challenges in theoretical physics is nonetheless to develop a theory that integrates the gravitational force with the other forces.
\end{framed}
\end{framed}

In the following trinket, a positive and negative charge are drawn. The force each charge experiences is drawn as an arrow and calculated for a distance of $1 \overset{\circ}{\rm A}=1\times 10^{ -10} {\rm m}$ apart. Following convention, the vector arrows have their origin at the charge experiencing the force. When the trinket is run, the force arrows indicate an attractive force, i.e., the charges experience forces toward one another. If the charges are changed to like charges, e.g., \texttt{Q1.q = q} and \texttt{Q2.q = q}, the force arrows indicate repulsion between the charges.

\begin{figure}[!htbp]
\centering
\caption[]{A trinket demonstrating Coulomb Force between opposite charges.}
\label{chap:chargesfields:coulombtrinket}
\end{figure}

\begin{framed}
\textbf{Example 15.2}\\
Three charges, $Q_1=1 {\rm nC}$, $Q_2= -2 {\rm nC}$, and $q= -1 {\rm nC}$, are held fixed at the three corners of an equilateral triangle with sides of length $a=1 {\rm cm}$, with a coordinate system as shown in Figure~\ref{fig:chargesfields:chargetriangle}. What is the electric force vector on charge $q$? (Note that $1 {\rm nC}=1\times 10^{ -9} {\rm C}$).

\begin{figure}[!htbp]
\centering
\includegraphics[width=0.4\linewidth]{files/chargetriangle-6c806c6caa350d0b775548735919e542.png}
\caption[]{Three charges arranged in an equilateral triangle of side $a$.}
\label{fig:chargesfields:chargetriangle}
\end{figure}

\begin{framed}
\textbf{Solution}\\
The net electric force on charge $q$ will be the vector sum of the forces from charges $Q_1$ and $Q_2$. We thus need to determine the force vectors on $q$ from each charge using Coulomb's Law, and then add those two vectors to obtain the net force on $q$. The force vectors exerted on $q$ from each charge are illustrated in Figure~\ref{fig:chargesfields:chargetriangle_sol}.

\begin{figure}[!htbp]
\centering
\includegraphics[width=0.4\linewidth]{files/chargetriangle_sol-8952ca5b443a1c0f12a01e7a166f9cfb.png}
\caption[]{Force vectors on charge $q$.}
\label{fig:chargesfields:chargetriangle_sol}
\end{figure}

The force from charge $Q_1$ has magnitude:
\begin{equation}
F_{q1}=\left |k\frac{Q_1q}{a^2}\right |=(9\times 10^9 {\rm N\cdot m^2/C^{2}})\frac{(1\times 10^{-9} {\rm C})(1\times 10^{-9} {\rm C})}{(0.01 {\rm m})^2}=9\times 10^{-5} {\rm N}
\end{equation}
and components:
\begin{equation}
\vec F_{q1}&=-F_{q1}\cos(60 {\rm \degree})\hat x-F_{q1}\sin(60 {\rm \degree})\hat y\\
&=-(4.5\times 10^{-5} {\rm N})\hat x-(7.8\times 10^{-5} {\rm N})\hat y
\end{equation}
Similarly, the force on $q$ from $Q_2$ has magnitude:
\begin{equation}
F_{q2}=\left |k\frac{Q_2q}{a^2}\right |=(9\times 10^9 {\rm N\cdot m^2/C^{2}})\frac{(2\times 10^{-9} {\rm C})(1\times 10^{-9} {\rm C})}{(0.01 {\rm m})^2}=1.8\times 10^{-4} {\rm N}
\end{equation}
and components:
\begin{equation}
\vec F_{q2}&=-F_{q2}\cos(60 {\rm \degree})\hat x+F_{q2}\sin(60 {\rm \degree})\hat y\\
&=-(9.0\times 10^{-5} {\rm N})\hat x+(1.6\times 10^{-4} {\rm N})\hat y
\end{equation}
Finally, we can add the two force vectors together to obtain the net force on $q$:
\begin{equation}
\vec F^{net}&=\vec F_{q1}+\vec F_{q2}\\
&=-(4.5 \times 10^{-5} {\rm N})\hat x-(7.8 \times 10^{-5} {\rm N})\hat y-(9.0 \times 10^{-5} {\rm N})\hat x+(1.6 \times 10^{ 4} {\rm N})\hat y\\
&=-(13.5\times 10^{-5} {\rm N})\hat x+(8.2\times 10^{-5} {\rm N})\hat y
\end{equation}
which has a magnitude of $15.8\times 10^{ -5} {\rm N}$.

\textbf{Discussion:} In this example, we determined the net force on a charge by making use of the superposition principle; namely, that we can treat the forces exerted on $q$ by $Q_1$ and $Q_2$ independently, without needing to consider the fact that $Q_1$ and $Q_2$ exert forces on each other.
\end{framed}
\end{framed}

\subsubsection{The electric field}

We define the electric field vector, $\vec E$, in an analogous way as we defined the gravitational field vector, $\vec g$. By defining the gravitational field vector, say, at the surface of the Earth, we can easily calculate the gravitational force exerted by the Earth on any mass, $m$, without having to use Newton's Universal Theory of Gravity. As you recall, we can define the gravitational field, $\vec g(\vec r)$, at some position, $\vec r$, from a point mass, $M$, as the gravitational force per unit mass:
\begin{equation}
\vec g(\vec r) = -G \frac{M}{r^2}\hat r
\end{equation}
where $\vec r$ is a vector from the position of $M$ to where we want to know the gravitational field. As a result, the force exerted on a ``test mass'', $m$, located at position $\vec r$ relative to mass $M$ is given by:
\begin{equation}
\vec F_g=m\vec g= -G\frac{Mm}{r^2}\hat r
\end{equation}
which, of course, is the result from Newton's Theory of Gravity. As you recall, we can define the gravitational field for  any object that is not a point mass (e.g. the Earth), and use that field to find the force exerted by the Earth on any mass $m$, without having to re-calculate the gravitational field each time (which requires an integral or Gauss' Law).

We proceed in an analogous way to define the ``electric field'', $\vec E(\vec r)$, as the \textit{electric force per unit charge}. If we have a point charge, $Q$, located at the origin of a coordinate system, then the electric field from that point charge, $\vec E(\vec r)$, at some position, $\vec r$, relative to the origin is given by:
\begin{equation}
\boxed{\vec E(\vec r) = k\frac{Q}{r^2}\hat r}
\end{equation}
If we place a ``test charge'', $q$, at position $\vec r$ in space, it will experience a force given by:
\begin{equation}
\vec F_e=q\vec E=k\frac{Qq}{r^2}\hat r
\end{equation}
just as we find from Coulomb's Law.

\begin{framed}
\textbf{Checkpoint}\\
A negative charge is placed at the origin of a coordinate system. At some point in space, the electric field from that charge

\begin{enumerate}
\item points towards the origin.
\item points away from the origin.
\end{enumerate}

\begin{framed}
\textbf{Answer}\\
\begin{enumerate}
\item
\end{enumerate}
\end{framed}
\end{framed}

In Three~charges~arranged~in~an~equilateral~triangle~of~side~a., we determined the electric force on charge $q$, exerted by two other charges $Q_1$ and $Q_2$. If we now changed the value of $q$ and wanted to determine the force, we can use the electric field to simplify the process considerably. That is, we can determine the value of the electric field, $\vec E$, from $Q_1$ and $Q_2$ at the position of $q$, and then simply multiply that field vector by a charge $q$ to obtain the force on that charge, without having to add force vectors.

\begin{framed}
\textbf{Example 15.3}\\
Two charges, $Q_1=1 {\rm nC}$, and $Q_2= -2 {\rm nC}$ are held fixed at two corners of an equilateral triangle with sides of length $a=1 {\rm cm}$, with a coordinate system as shown in Figure~\ref{fig:chargesfields:chargetriangle}. What is the electric field vector at the third corner of the triangle?

\begin{figure}[!htbp]
\centering
\includegraphics[width=0.4\linewidth]{files/fieldtriangle-c6a57a30ed4678046ce1a02cb4519207.png}
\caption[]{Two charges at the corners of an equilateral triangle of side $a$.}
\label{fig:chargesfields:fieldtriangle}
\end{figure}

\begin{framed}
\textbf{Solution}\\
The net electric field at the third corner of the triangle will be the vector sum of the electric fields from charges $Q_1$ and $Q_2$. We thus need to determine the electric field vectors from each charge, and then add those two vectors to obtain the net electric field. The vectors are illustrated in Figure~\ref{fig:chargesfields:fieldtriangle_sol}.

\begin{figure}[!htbp]
\centering
\includegraphics[width=0.4\linewidth]{files/fieldtriangle_sol-d6d26a9828e420c4da7572c93d22e5b3.png}
\caption[]{Electric field vectors from two charges at the corners of an equilateral triangle of side $a$.}
\label{fig:chargesfields:fieldtriangle_sol}
\end{figure}

The electric field from charge $Q_1$ has magnitude:
\begin{equation}
E_1=\left |k\frac{Q_1}{a^2}\right |=(9\times 10^{9} {\rm N\cdot m^2/C^{2}})\frac{(1\times 10^{-9} {\rm C})}{(0.01 {\rm m})^2}=9\times 10^{4} {\rm N/C}
\end{equation}
and components:
\begin{equation}
\vec E_1&=E_1\cos(60 {\rm \degree})\hat x+E_1\sin(60 {\rm \degree})\hat y\\
&=(4.5\times 10^{4} {\rm N/C})\hat x+(7.8\times 10^{4} {\rm N/C})\hat y
\end{equation}
Similarly, the electric field from $Q_2$ has magnitude:
\begin{equation}
E_2=\left |k\frac{Q_2}{a^2}\right |=(9\times 10^{9} {\rm N\cdot m^2/C^{2}})\frac{(2\times 10^{-9} {\rm C})}{(0.01 {\rm m})^2}=1.8\times 10^{5} {\rm N/C}
\end{equation}
and components:
\begin{equation}
\vec E_2&=E_2\cos(60 {\rm \degree})\hat x-E_2\sin(60 {\rm \degree})\hat y\\
&=(9.0\times 10^{4} {\rm N/C})\hat x-(1.6\times 10^{5} {\rm N/C})\hat y
\end{equation}
Finally, we can add the two force vectors together to obtain the net force on $q$:
\begin{equation}
\vec E^{net}&=\vec E_1+\vec E_2\\
&=(4.5\times 10^{4} {\rm N/C})\hat x+(7.8\times 10^{4} {\rm N/C})\hat y+(9.0\times 10^{4} {\rm N/C})\hat x-(1.6\times 10^{5} {\rm N/C})\hat y\\
&=(13.5\times 10^{4} {\rm N/C})\hat x-(8.2\times 10^{4} {\rm N/C})\hat y
\end{equation}
which has a magnitude of $15.8\times 10^{4} {\rm N/C}$. By knowing the electric field at the empty corner of the triangle, we can now calculate the net electric force that would act on any charge placed in that location. For example, if we place a charge $q= -1 {\rm nC}$ (as in Three~charges~arranged~in~an~equilateral~triangle~of~side~a.), we can easily find the corresponding electric force:
\begin{equation}
\vec F_q &= q\vec E=(-1 {\rm nC})\left[ (13.5\times 10^{4} {\rm N/C})\hat x-(8.2\times 10^{4} {\rm N/C})\hat y \right]\\
&=-(13.5\times 10^{-5} {\rm N})\hat x+(8.2\times 10^{-5} {\rm N})\hat y
\end{equation}
as we found previously. Note that the force on $q$ is in the opposite direction of the electric field vector. This is because $q$ is negative. The \textbf{electric field at some point in space thus points in the same direction as the force that a positive test charge would experience}.

\textbf{Discussion:} In this example, we determined the net electric field by making use of the superposition principle; namely, that we can treat the electric fields from $Q_1$ and $Q_2$ independently, without needing to consider the fact that $Q_1$ and $Q_2$ exert forces on each other. By knowing the electric field at some position in space, we can easily calculate the force vector on any test charge, $q$, placed at that position. Furthermore, the sign of the charge $q$ will determine in which direction the force will point (parallel to $\vec E$ for a positive charge and anti-parallel to $\vec E$ for a negative charge).
\end{framed}
\end{framed}

\begin{framed}
\textbf{Checkpoint}\\
The electric field inside of a conductor must be zero because...

\begin{enumerate}
\item If there is an electric field, electrons will move (since it is a conductor) and arrange themselves so as to create an additional field that cancels the original field
\item If there is an electric field, protons will move (since it is a conductor) and arrange themselves so as to create an additional field that cancels the original field
\item Since electrons can move freely, they move so fast that the electric field is negligible.
\item Electric fields cannot penetrate conducting materials.
\end{enumerate}

\begin{framed}
\textbf{Answer}\\
\begin{enumerate}
\item
\end{enumerate}
\end{framed}
\end{framed}

\paragraph{Visualizing the electric field}

Generally, a ``field'' is something that has a different value at different positions in space. The pressure in a fluid under the presence of gravity is a field: the pressure is different at different heights in the fluid. Since pressure is a scalar quantity (a number), we call it a ``scalar field''. The electric field is called a ``vector field'', because it is a vector that is different at each position in space. One way to visualize the electric field is to draw arrows at different positions in space; the length of the arrow is then proportional to the strength of the electric field at that position, and the direction of the arrow represents the direction of the electric field. The electric field for a point charge is shown using this method in Figure~\ref{fig:ChargesFields:1pos}.

\begin{figure}[!htbp]
\centering
\includegraphics[width=0.3\linewidth]{files/1pos-bd746d90e839613ec799f32b8e33f122.png}
\caption[]{Electric field vectors near a point charge.}
\label{fig:ChargesFields:1pos}
\end{figure}

One disadvantage of visualizing a vector field with arrows is that the arrows take up space, and it can be challenging to visualize how the field changes magnitude and direction continuously through space. For this reason, one usually uses ``field lines'' to visualize a vector field. Field lines are continuous lines with the following properties:

\begin{itemize}
\item The direction of the vector field at some point in space is tangent to the field line at that point.
\item Field lines have a direction to indicate the direction of the field vector along the tangent (as there are two possibilities, parallel and anti-parallel).
\item The magnitude of the field is proportional to the density of field lines at that point. The more field lines near a location in space, the larger the magnitude of the field vector at that point.
\end{itemize}

An example of using field lines to represent a vector field in space is shown in Figure~\ref{fig:chargesfields:fieldlines}. The corresponding field vector is shown at two different positions in space ($A$ and $B$). At both positions, the vector is tangent to the field line at that position in space and points in the direction of the little arrow drawn at the end of the field lines. The field vector at point $A$ has a larger magnitude than the one at point $B$, since the field lines are more concentrated at point $A$ than at point $B$ (there are more field lines per unit area at that position in space, the field lines are closer together).

\begin{figure}[!htbp]
\centering
\includegraphics[width=0.4\linewidth]{files/fieldlines-3399c98223e3f2a01d954a6b769618be.png}
\caption[]{An example of determining a field vector from the continuous field lines.}
\label{fig:chargesfields:fieldlines}
\end{figure}

\begin{framed}
\textbf{Checkpoint:label: cp:chargesfields:efield}\\
It is possible for field lines to cross?\}

\begin{enumerate}
\item Yes.
\item No.
\end{enumerate}

\begin{framed}
\textbf{Answer}\\
\begin{enumerate}[resume]
\item
\end{enumerate}
\end{framed}
\end{framed}

Because the electric field vector always points in the direction of the force that would be exerted on a positive charge, electric field lines will point out from a positive charge and into a negative charge. The electric field lines for a combination of positive and negative charges is illustrated in Figure~\ref{fig:ChargesFields:2pos1neg}.

\begin{figure}[!htbp]
\centering
\includegraphics[width=0.5\linewidth]{files/2pos1neg-bb6e404661ee0b2000eae389bf928df8.png}
\caption[]{Field lines of two $+2q$ charges and one $-3q$ charge.}
\label{fig:ChargesFields:2pos1neg}
\end{figure}

\paragraph{Electric field from a charge distribution}

So far, we have only considered Coulomb's Law for point charges (charges that are infinitely small and can be considered to exist at a single point in space). We can use the principle of superposition to determine the electric field from a charged extended/continuous object by modelling that object as being made of many point charges. The electric field from that object is then the sum of the electric field from the point charges that make up that object.

Consider a charged wire that is bent into a semi-circle of radius $R$, as in Figure~\ref{fig:chargesfields:semicircle}. The wire carries a net positive electric charge, $+Q$, that is uniformly distributed along the length of the wire. We wish to determine the electric field vector at the centre of the circle.

\begin{figure}[!htbp]
\centering
\includegraphics[width=0.2\linewidth]{files/semicircle-90ec0efe2d9da30eb5c806e6cf98e1f5.png}
\caption[]{A charged wire bent into a semi-circle of radius $R$.}
\label{fig:chargesfields:semicircle}
\end{figure}

We start by choosing a very small section of wire and model that section of wire as a point charge with infinitesimal charge $dq$ (as in Figure~\ref{fig:chargesfields:semicircle_sol}). A distance $R$ from that point charge, the electric field from that point charge will have magnitude, $dE$, given by:
\begin{equation}
dE=k\frac{dq}{R^2}
\end{equation}
The electric field vector, $d\vec E$, from the point charge $dq$ is illustrated in Figure~\ref{fig:chargesfields:semicircle_sol}.

\begin{figure}[!htbp]
\centering
\includegraphics[width=0.2\linewidth]{files/semicircle_sol-cc72f82d349a967ff3aeacf8b69a4061.png}
\caption[]{Infinitesimal electric fields from point charges along the bent wire.}
\label{fig:chargesfields:semicircle_sol}
\end{figure}

Using the coordinate system that is shown, we define $\theta$ as the angle made by the vector from the origin to the point charge $dq$ and the $x$-axis. The electric field vector from $dq$ is then given by:
\begin{equation}
d\vec E = dE\cos\theta \hat x - dE\sin\theta \hat y
\end{equation}
The total electric field at the origin will be obtained by summing the electric fields from the different $dq$ over the entire semi-circle:
\begin{equation}
\vec E &= \int d\vec E = \int \left(dE\cos\theta \hat x - dE\sin\theta \hat y\right)\\
&=\left( \int dE\cos\theta \right)\hat x -\left( \int dE\sin\theta \right)\hat y\\
\therefore E_x &= \int dE\cos\theta\\
\therefore E_y &= -\int dE\sin\theta\\
\end{equation}
We are thus left with two integrals to solve for the $x$ and $y$ components of the electric field, respectively. Before jumping into solving the integrals, it is useful to think about the symmetry of the problem. Specifically, consider a second point charge, $dq'$, located symmetrically about the $x$-axis from charge $dq$, as illustrated in Figure~\ref{fig:chargesfields:semicircle_sol}. The charge $dq'$ will create a small electric field $d\vec E'$ as illustrated. When we add together $d\vec E$ and $d\vec E'$, the two $y$ components will cancel, and only the $x$ components will sum together. Similarly, for any $dq$ that we choose, there will always be another $dq'$ such that when we sum together their respective electric fields, the $y$ components will cancel. Thus, by symmetry, we can argue that the net $y$ component of the electric field, $E_y$, must be zero. We thus only need to evaluate the $x$ component of $\vec E$:
\begin{equation}
E_x = \int dE\cos\theta = \int k\frac{dq}{R^2} \cos\theta
\end{equation}
In order to solve this integral, we need to consider which variables change for different choices of the point charge $dq$. In this case, the distance $R$ is the same anywhere along the semi-circle, so only $\theta$ changes with different choices of $dq$, as $k$ is a constant. We can express $dq$ in terms of $d\theta$ and then use $\theta$ as the variable of integration (the variable that labels the different $dq$). $d\theta$ corresponds to a small change in the angle $\theta$, and is the angle that is subtended by the charge $dq$. That is, the charge $dq$ covers a small arc length, $ds$, of the semi-circle, which is related to $d\theta$ by:
\begin{equation}
ds = Rd\theta
\end{equation}
The total charge on the wire is given by $Q$, and the wire has a length $\pi R$ (half the circumference of a circle). Since the charge is distributed uniformly on the wire, the charge per unit length of any piece of wire must be constant. In particular, $dq$ divided by $ds$ must be equal to $Q$ divided by $\pi R$:
\begin{equation}
\frac{dq}{ds}&=\frac{Q}{\pi R}\\
\therefore dq &=\frac{Q}{\pi R}ds=\frac{Q}{\pi}d\theta
\end{equation}
where in the last equality we used the relation $ds=Rd\theta$. We now have all of the ingredients to solve the integral:
\begin{equation}
E_x &= \int k\frac{dq}{R^2} \cos\theta = \int_{-\pi/2}^{+\pi/2} k\frac{Q}{\pi R^2}\cos\theta d\theta\\
&= k\frac{Q}{\pi R^2}\int_{-\pi/2}^{+\pi/2}\cos\theta d\theta=k\frac{Q}{\pi R^2}\left[ \sin\theta \right]_{-\pi/2}^{+\pi/2}\\
&= k\frac{2Q}{\pi R^2}
\end{equation}
The total electric field vector at the centre of the circle is thus given by:
\begin{equation}
\vec E = k\frac{2Q}{\pi R^2} \hat x
\end{equation}
Note that if we had not realized that we did not need to solve the integral for the $y$ component, we would still find that it is zero:
\begin{equation}
E_y= -k\frac{Q}{\pi R^2}\int_{-\pi/2}^{+\pi/2}\cos\theta d\theta=-k\frac{Q}{\pi R^2}\left[ -\cos\theta \right]_{-\pi/2}^{+\pi/2}=0
\end{equation}

In order to determine the electric field at some point from any continuous charge distribution, the procedure is generally the same:

\begin{enumerate}
\item Make a \textit{good} diagram.
\item Choose a charge element $dq$.
\item Draw the electric field element, $d\vec E$, at the point of interest.
\item Write out the electric field element vector, $d\vec E$, in terms of $dq$ and any other relevant variables.
\item Think of symmetry: will any of the component of $d\vec E$ sum to zero over all of the $dq$?
\item Write the total electric field as the sum (integral) of the electric field elements.
\item Identify which variables change as one varies the $dq$ and choose an integration variable to express $dq$ and everything else in terms of that variable and other constants.
\item Do the sum (integral).
\end{enumerate}

\begin{framed}
\textbf{Olivia's Thoughts}\\
Before we get into the next few examples, I'm going to review the set-up for the  semi-circular wire, and explain why it's best to leave $dq$ in our expression as long as possible.

In this example, we were given a charged, semi-circular wire and asked to find the electric field. When asked to solve a new problem like this, many of us would start by trying to find a formula that applies to the problem. However, we were only given one formula for the electric field, $\vec E = k\frac{Q}{r^2}\hat r$, and this only applies to point charges. So, the only way that we can use this formula is to somehow make the semi-circular wire look like a bunch of point charges. To do this, we imagine that the wire looks like the left panel of Figure~\ref{fig:chargesfields:semicircle_redo}, where I have cut the wire into pieces, each of which has \textit{some} charge, $dq$.

\begin{figure}[!htbp]
\centering
\includegraphics[width=0.6\linewidth]{files/semicircle_redo-1c8b776e96f2c9eeac1372ed73b9d871.png}
\caption[]{Left: Cutting up a wire with charge $Q$ into pieces with length $ds$ and charge $dq$. Right: Visualizing the semicircular wire as being made up of point charges.}
\label{fig:chargesfields:semicircle_redo}
\end{figure}

Since the pieces are very small, we can say that they are essentially point charges. Now, the problem looks something like the right panel of Figure~\ref{fig:chargesfields:semicircle_redo}, which we know how to solve. All we have to do is find the field from each point charge, $\vec E = k\frac{Q}{r^2}\hat r \rightarrow d\vec E = k\frac{dq}{r^2}\hat r$, and add them all up using an integral: $\vec E=\int d\vec E$. From the symmetry of the problem, we find that the $y$ contributions from the different point charges will cancel, so the  integral we need to solve is $E_x=\int k\frac{dq}{R^2}\cos\theta$.

Since $R$ and $k$ are constant, we just have to integrate over $\theta$. However, our integrand doesn't contain any $d\theta$ to integrate over. That's where our pal $dq$ comes in. Now that we know we want to integrate over $\theta$, we can \textit{choose} to express $dq$ in terms of $d\theta$. Looking back at the left panel of Figure~\ref{fig:chargesfields:semicircle_redo}, how much charge, $dq$, is in each piece? Well, $dq$ is just the charge of the whole wire, $Q$, divided by the length of the wire, $\pi R$, times the length of the piece, $ds$. Since each piece is in the shape of an arc, we can get $d\theta$ into this expression by using the formula for arc length, $ds=Rd\theta$. Therefore, $dq=(Q/\pi)( ds/R)=(Q/\pi)d\theta$, and we can now solve the integral.
\end{framed}

\begin{framed}
\textbf{Example 15.4}\\
A ring of radius $R$ carries a total charge $+Q$. Determine the electric field a distance $a$ from the centre of the ring, along the axis of symmetry of the ring.\}
In order to determine the electric field, we carry out the procedure outlined above, and start by drawing a good diagram, as in Figure~\ref{fig:chargesfields:ring}, showing: our coordinate system, our choice of $dq$, the electric field element vector $d\vec E$ that corresponds to $dq$, and variables ($r$, $\theta$) to specify the position of $dq$.

\begin{figure}[!htbp]
\centering
\includegraphics[width=0.3\linewidth]{files/ring-03c4e5297f14b48cab0838925852ec16.png}
\caption[]{Determining the electric field on the axis of a ring of radius $R$ carrying charge $Q$.}
\label{fig:chargesfields:ring}
\end{figure}

\begin{framed}
\textbf{Solution}\\
In this case, the figure is challenging to draw and visualize because of the three-dimensional nature of the problem. With the specific $dq$ that we chose, the electric field element vector is given by:
\begin{equation}
d\vec E = -dE\sin\theta \hat x + 0\hat y + dE\cos\theta \hat z
\end{equation}
where $d\vec E$ has magnitude:
\begin{equation}
dE = k\frac{dq}{r^2}
\end{equation}
The $x$ and $z$ components of the total electric field will then be given by:
\begin{equation}
E_x &= -\int dE\sin\theta=-\int k\frac{dq}{r^2}\sin\theta\\
E_z &= \int dE\cos\theta=\int k\frac{dq}{r^2}\cos\theta \\
\end{equation}
In general, if we had chosen a $dq$ that is not along one of the axes of the coordinate system, the electric field element vector would have components in all three directions. However, if we consider the symmetry of the ring, we can note that once we sum together all of the electric field elements, only the $z$ components will survive. Indeed, we have shown in Figure~\ref{fig:chargesfields:ring} that for each $dq$, there will be a $dq'$ located on the opposite side of the ring that will create an electric field element that will cancel all but the $z$ component of the field element from $dq$. We thus only need to consider the $z$ components of the electric field elements when determining the total electric field:
\begin{equation}
\vec E = E_z\hat z
\end{equation}
We now have to evaluate the integral for the $z$ component of the electric field:
\begin{equation}
E_z &= \int k\frac{dq}{r^2}\cos\theta \\
\end{equation}
and determine which quantities change as we move $dq$ around the ring. In this case, both $r^2$ and $\cos\theta$ are the same for all elements on the ring, and the integral is trivial:
\begin{equation}
E_z &= k\frac{1}{r^2}\cos\theta\int dq=k\frac{Q}{r^2}\cos\theta=kQ\frac{a}{(R^2+a^2)^\frac{3}{2}}  \\
\end{equation}
where the integral $\int dq$ simply means ``sum all of the charges $dq$ together'', which is equal to $Q$, the total charge on the ring.
In the last equality, we replaced $\cos\theta$ with the variables $a$ and $R$ that are provided in the question.
\end{framed}
\end{framed}

\begin{framed}
\textbf{Olivia's Thoughts}\\
In this example, we saw why it's best to leave $dq$ in our expression until we are ready to integrate. In this case, because $r$ and $\theta$ were both constant, it was not useful to write $dq$ in terms of another variable.
\end{framed}

\begin{framed}
\textbf{Example 15.5}\\
You have rubbed a glass rod with a silk cloth such that the glass rod has acquired a positive charge. The rod has a length, $L$, a negligible cross-section, and has acquired a total positive charge, $+Q$, that is uniformly distributed along the length of the rod. What is the electric field a distance $R$ from the centre of the rod?

\begin{framed}
\textbf{Solution}\\
In order to determine the electric field, we carry out the procedure outlined above, and start by drawing a good diagram, as in Figure~\ref{fig:chargesfields:finiteline}, showing: our coordinate system, our choice of $dq$ at a distance $y$ above the centre of the rod, the electric field element vector $d\vec E$ that corresponds to $dq$, and variables ($y$, $r$, $\theta$) to specify the position of $dq$.

\begin{figure}[!htbp]
\centering
\includegraphics[width=0.3\linewidth]{files/finiteline-4685490417c1a722b87b010361357eb7.png}
\caption[]{Determining the electric field a distance $R$ from the centre of a rod of length $L$ carrying charge $Q$.}
\label{fig:chargesfields:finiteline}
\end{figure}

We define the origin to be located at the point where we want to determine the electric field, and the angle $\theta$ to be the angle between the horizontal and the position vector of $dq$. We can write the electric field element vector as:
\begin{equation}
d\vec E = dE\cos\theta \hat x - dE\sin\theta \hat y
\end{equation}
where $d\vec E$ has magnitude:
\begin{equation}
dE = k\frac{dq}{r^2}
\end{equation}
The $x$ and $y$ components of the total electric field will then be given by:
\begin{equation}
E_x &= \int dE\cos\theta=\int k\frac{dq}{r^2}\cos\theta \\
E_y &= -\int dE\sin\theta=-\int k\frac{dq}{r^2}\sin\theta\\
\end{equation}
Again, before proceeding with the integrals, we consider symmetry. Specifically, if we consider a charge $dq'$ located symmetrically about the $x$ axis from $dq$ (as illustrated in Figure~\ref{fig:chargesfields:finiteline}), we see that the $y$ component of the electric field element $d\vec E'$ that it creates will cancel the $y$ component of $d\vec E$. For each choice of $dq$, there will exist a corresponding choice $dq'$ which will result in the $y$ component of the net electric field being zero. We thus only need to evaluate the $x$ component of the total electric field:
\begin{equation}
\vec E = E_x \hat x = \left(\int k\frac{dq}{r^2}\cos\theta\right) \hat x
\end{equation}
Within the integrand, both $r$ and $\theta$ will change as we sum over the different charges $dq$ along the rod. A straightforward option to write the integral is to use $y$ as the integration constant, and to write $dq$, $r$, and $\cos\theta$ in terms of $y$. The charge $dq$ covers an infinitesimal length of the rod, $dy$. Since the rod is uniformly charged, the charge per unit length must be the same over a small length $dy$ as it is over the whole length of the rod:
\begin{equation}
\frac{dq}{dy}&=\frac{Q}{L}\\
\therefore dq &= \frac{Q}{L} dy
\end{equation}
It is often useful to introduce a constant charge per unit length, $\lambda=\frac{Q}{L}$, so that we can write the charge $dq$ as:
\begin{equation}
dq = \lambda dy
\end{equation}
We can also express $r^2$ and $\cos\theta$ in terms of $y$ (and $R$, which is constant):
\begin{equation}
r^2 &= y^2+R^2\\
\cos\theta&=\frac{R}{r}=\frac{R}{\sqrt{y^2+R^2}}
\end{equation}
Finally, we can combine this all into an integral that we can evaluate:
\begin{equation}
 E_x &= \int k\frac{dq}{r^2}\cos\theta\\
 &= k\int_{-L/2}^{L/2} \lambda \frac{1}{y^2+R^2}\frac{R}{\sqrt{y^2+R^2}} dy\\
 &= kR\lambda\int_{-L/2}^{L/2} \frac{1}{(y^2+R^2)^{\frac{3}{2}}} dy\\
 &= kR\lambda \left[  \frac{y}{R^2\sqrt{y^2+R^2}}\right]_{-L/2}^{L/2}\\
 \therefore E_x &= \frac{k\lambda}{R}\frac{L}{\sqrt{\left(\frac{L}{2}\right)^2+R^2}}
\end{equation}
If the rod were infinitely long (or very long compared to the distance $R$), the electric field becomes:
\begin{equation}
\lim_{L\to\infty}E_x=\frac{2k\lambda}{R}
\end{equation}
By using the charge per unit length, $\lambda$, we were able to easily generalize our result to that expected for an infinitely long rod with uniform charge density.

Solving the integral above in terms of the integration variable $y$ is difficult without some knowledge of integrals. For this specific integral, the easiest method to use from calculus is ``trig substitution''. We show below how we can arrive at a much easier integral if we had instead chosen the angle $\theta$ as the integration variable instead of $y$, and we will see that this is a physical illustration of the ``trig substitution method'' from calculus!

We go back to step 7 in our procedure and choose $\theta$ (instead of $y$) as the integration variable for the integral:
\begin{equation}
E_x &=\int k\frac{dq}{r^2}\cos\theta\\
\end{equation}
That is, we need to express $1/r^2$ and $dq$ in terms of $\theta$. To do this, we refer back to Figure~\ref{fig:chargesfields:finiteline}. Starting with $1/r^2$, we get:
\begin{equation}
r &= \frac{R}{\cos\theta}\\
\therefore \frac{1}{r^2}&=\frac{\cos^2\theta}{R^2}\\
\end{equation}
Now we need to find an expression for $dq$. We know that $dq=\lambda dy$, and $\lambda$ is just a constant, so we need to find $dy$ as a function of $\theta$:
\begin{equation}
y &= R\tan\theta\\
\therefore dy &= \frac{dy}{d\theta}d\theta=\frac{R}{\cos^2\theta}d\theta\\
\therefore dq &= \lambda dy =\lambda\frac{R}{\cos^2\theta}d\theta
\end{equation}
where in the second line, we took the derivative of $y=R\tan\theta$ with respect to $\theta$. With these expressions for $1/r^2$ and $dq$, the integral becomes trivial:
\begin{equation}
 E_x &= \int k\frac{dq}{r^2}\cos\theta\\
 &= k \int_{-\theta_0}^{\theta_0} \lambda\frac{R}{\cos^2\theta} \frac{\cos^2\theta}{R^2} \cos\theta d\theta\\
 &=\frac{k\lambda}{R}\int_{-\theta_0}^{\theta_0}\cos\theta d\theta\\
 &=\frac{k\lambda}{R}\left[\sin\theta \right]_{-\theta_0}^{\theta_0}\\
 &= \frac{2k\lambda}{R}\sin\theta_0
\end{equation}
where $\theta_0$ is the angle subtended by half of the rod. Referring to Figure~\ref{fig:chargesfields:finiteline}, we can easily see that:
\begin{equation}
\sin\theta_0=\frac{L/2}{\sqrt{\left(\frac{L}{2}\right)^2+R^2}}
\end{equation}
So that the total electric field is given by:
\begin{equation}
E_x &=\frac{2k\lambda}{R}\sin\theta_0=\frac{k\lambda}{R}\frac{L}{\sqrt{\left(\frac{L}{2}\right)^2+R^2}}
\end{equation}
as found before. Furthermore, in the limit of an infinitely long rod, the angle $\theta_0$ tends to $\frac{\pi}{2}$, so that the electric field becomes:
\begin{equation}
E_x=\lim_{\theta_0\to\frac{\pi}{2}}\frac{2k\lambda}{R}\sin\theta_0=\frac{2k\lambda}{R}
\end{equation}
\textbf{Discussion:} In this example, we saw how to apply the principle of superposition to determine the electric field near a finite and a infinite line of charge with constant charge per unit length. We showed that it was relatively straightforward to set up the integral in terms of $dy$, but not so easy to solve the integral. We then showed that by using $\theta$ as the integration variable, we could arrive at a much easier integral. This change of variable corresponds to a physical variable in our problem, but is also the basis for the more abstract ``trig substitution'' method used to solve integrals in calculus.
\end{framed}
\end{framed}

\begin{framed}
\textbf{Example 15.6}\\
Calculate the electric field a distance, $a$, above a infinite plane that carries uniform charge per unit area, $\sigma$.\}
In this case, we need to determine the field above an object that is two dimensional (a plane). In the previous examples (a ring, a line of charge), we modelled a one dimensional object (e.g. the line), as being made of many point charges (0-dimensional objects). We treated those point charges has having an infinitesimal length along the object so that we could sum them together to obtain the object (e.g. $dy$ was the length of the charge for the rod/line of charge).

\begin{framed}
\textbf{Solution}\\
In order to model the two-dimensional object (the plane), we model it has being the sum of many one dimensional objects. We can model a plane either as a rectangle of width, $W$, and length, $L$, as shown in the left panel of Figure~\ref{fig:chargesfields:planecharge} or as a disk of radius, $R$, as shown in the right panel. To model an infinite plane, we can then take the limit of either $L$ and $W$ going to infinity (rectangle), or of $R$ going to infinity (disk). We can model the rectangle as being the sum of many lines of \textbf{finite} length, $L$, and infinitesimal width, $dx$. Similarly, we can model the disk as the sum of infinitesimally thin rings of \textbf{finite} radius, $r$, and thickness, $dr$. In both cases, we know how to model the field from a line of charge (Example~15.5) or from a ring (Example~15.4).

\begin{figure}[!htbp]
\centering
\includegraphics[width=0.7\linewidth]{files/planecharge-ef931468f92f58cf31fa48dcf255be7e.png}
\caption[]{A two-dimensional object such as a plane modelled as a the sum of infinitely thin lines (left panel) or as the sum of infinitely thin rings (right panel).}
\label{fig:chargesfields:planecharge}
\end{figure}

We proceed by modelling the plane as a disk made up of infinitesimal rings. Our infinitesimal charge, $dq$, is thus the charge on a ring of radius $r$ and thickness $dr$, as illustrated in Figure~\ref{fig:chargesfields:disk}.

\begin{figure}[!htbp]
\centering
\includegraphics[width=0.3\linewidth]{files/disk-725ab7b22a1889b37e3080521f570aa5.png}
\caption[]{Modelling the field from a disk as the sum of fields from concentric thin rings.}
\label{fig:chargesfields:disk}
\end{figure}

We know from Determining~the~electric~field~on~the~axis~of~a~ring~of~radius~R~carrying~charge~Q. that the magnitude of the electric field a distance $a$ from the centre of the ring, along its axis of symmetry (the $z$ axis in Figure~\ref{fig:chargesfields:disk}), is given by:
\begin{equation}
dE = kdq\frac{a}{(r^2+a^2)^\frac{3}{2}}
\end{equation}
By symmetry, for all of the different infinitesimal rings that make up the disk, the field will always point along the $z$ axis. In order to determine the total field, we sum (integrate) the values of $dE$, over all of the rings, from a radius of $r=0$ to a radius $r=R$. For each ring, the value of $r$ will be different, so we need to express $dq$ in terms of $dr$ in order to perform the integral. We know that the plane has a uniform charge per unit area given by $\sigma$. The charge $dq$ of an infinitesimal ring is given by:
\begin{equation}
dq = \sigma dA=\sigma 2\pi r dr
\end{equation}
where $dA=2\pi r dr$ is the area of the infinitesimal ring of radius $r$ and thickness $dr$ (think of unfolding the ring into a rectangle of height $dr$ and length $2\pi r$, the circumference of the circle, in order to determine the area). We now have all of the ingredients in order to determine the total electric field:
\begin{equation}
E &= \int dE = \int_0^R kdq\frac{a}{(r^2+a^2)^\frac{3}{2}}  = 2\pi k a \sigma \int_0^R \frac{r}{(r^2+a^2)^\frac{3}{2}}dr\\
&=2\pi k a \sigma \left[  \frac{-1}{\sqrt{r^2+a^2}}\right]_0^R=2\pi k  \sigma\left(1-\frac{a}{R^2+a^2} \right)
\end{equation}
Finally, we can take the limit of $R\to\infty$ in order to get the electric field above an infinite plane:
\begin{equation}
E=\lim_{R\to\infty}2\pi k  \sigma\left(1-\frac{a}{R^2+a^2} \right)=2\pi k\sigma=\frac{\sigma}{2\epsilon_0}
\end{equation}
where we used $\epsilon_0$ in the last equality as the result is a little cleaner without the factors of $\pi$. Note that for an infinite plane of charge, the electric field does not depend on the distance (our variable $a$) from the plane!

\textbf{Discussion:} In this example, we showed how we can model a two-dimensional charge distribution as the sum of one-dimensional charge distributions. In particular, we showed that an infinite plane of charge can be modelled as the sum of many lines charges or of many rings of charge (we chose the latter in the above). We also found that the electric field above an infinite plane of charge does not depend on the distance from the plane; that is, the electric field is constant above an infinite plane of charge.
\end{framed}
\end{framed}

\begin{framed}
\textbf{Josh's Thoughts}\\
A common source of confusion is the process of solving for the electric field produced by continuous charges. Point charges are well defined in space as being entirely contained within a single point, while continuous charges are objects which occupy 1, 2, or 3 dimensions. The electric field produced by point charges are easily modelled by $\vec E = \frac{kQ}{r^2}\hat r$, but the electric fields produced by continuous charges must usually be obtained from an integral.

When a charge is distributed, the charge on the object must be broken down into many small charges which are written as $dq$. From there, $dq$ is rewritten in terms of a position variable over which it is convenient to integrate. Think of the position variable as a variable that you can use to distinguish charges, $dq$, located at different positions along the object.

For example, referring to Figure~\ref{fig:chargesfields:rodexample}, if I wanted to determine $E$ at the top of a rod (left-hand panel), it would be most convenient for me to integrate over $x$, but if I wanted to determine $E$ on the side of a rod, it would be most convenient to integrate over $\theta$.

\begin{figure}[!htbp]
\centering
\includegraphics[width=0.4\linewidth]{files/rodexample-76239b7407f7a41e29db62973439260f.png}
\caption[]{Calculating the electric field produced by a rod at different positions.}
\label{fig:chargesfields:rodexample}
\end{figure}

In order to determine the bounds of the integral, think of the range in position variable that is required in order to cover the entire object. I recommend paying close attention to Example~15.4, Example~15.5, and Example~15.6, and attempting questions which require integration on the Question Library.
\end{framed}

\subsubsection{The electric dipole}\label{sec:chargesfields:electricdipole}

Electric dipoles are a specific combination of a positive charge $+Q$ held at a fixed distance, $l$, from an equal and opposite charge, $-Q$, as illustrated in Figure~\ref{fig:chargesfields:dipole}.

\begin{figure}[!htbp]
\centering
\includegraphics[width=0.3\linewidth]{files/dipole-46c3107dc981a624ca87141b9e165fa2.png}
\caption[]{An electric dipole and its corresponding dipole vector, $\vec p$.}
\label{fig:chargesfields:dipole}
\end{figure}

Dipoles can be represented by their ``electric dipole vector'' (or ``electric dipole moment''), $\vec p$, defined to point in the direction \textbf{from the negative charge to the positive charge}, with magnitude:
\begin{equation}
p=Ql
\end{equation}
Dipoles arise frequently in nature. For example, a water molecule can be modelled as a dipole. In a water molecule, the two hydrogen atoms are not symmetrically arranged around the oxygen atom. The electrons tend to stay closer to the oxygen atom, so the oxygen atom has an excess of 2 electrons, while each proton has a deficit of 1 electron. This results in a separation of charge (polarization), which can be modelled as an electric dipole, as in Figure~\ref{fig:chargesfields:h20}.

\begin{figure}[!htbp]
\centering
\includegraphics[width=0.2\linewidth]{files/h20-c9e951e5ea9119ebb1b2a21e09add397.png}
\caption[]{A water molecule can be modelled as an electric dipole.}
\label{fig:chargesfields:h20}
\end{figure}

When a dipole is immersed in a uniform electric field, as illustrated in Figure~\ref{fig:chargesfields:dipoleinfield}, the net force on the dipole is zero because the force on the positive charge will always be equal to and in the opposite direction of the force on the negative charge.

\begin{figure}[!htbp]
\centering
\includegraphics[width=0.6\linewidth]{files/dipoleinfield-89cb18f52c976098c978acd723095b35.png}
\caption[]{An electric dipole in a uniform electric field.}
\label{fig:chargesfields:dipoleinfield}
\end{figure}

Although the net force on the dipole is zero, there is still a net torque about its centre that will cause the dipole to rotate (unless the dipole vector is already parallel to the electric field vector). If the dipole vector makes an angle, $\theta$, with the electric field vector (as in Figure~\ref{fig:chargesfields:dipoleinfield}), the magnitude of the net torque on the dipole about an axis perpendicular to the page and through the centre of the dipole is given by:
\begin{equation}
\tau&=\frac{l}{2}F^+\sin\theta+\frac{l}{2}F^-\sin\theta\\
&=\frac{l}{2}QE\sin\theta+\frac{l}{2}QE\sin\theta\\
&=QlE\sin\theta\\
\tau&=pE\sin\theta
\end{equation}
In Figure~\ref{fig:chargesfields:dipoleinfield}, the torque vector is into the page (the forces will make it rotate clockwise), which is the same direction as the cross product, $\vec p \times \vec E$. Note that the magnitude of the torque is also equal to the magnitude of the cross product. Thus, in general, the torque vector on a dipole, $\vec p$, from an electric field, $\vec E$, is given by:
\begin{equation}
\boxed{\vec \tau =\vec p \times \vec E}
\end{equation}
In particular, note that the torque is zero when the dipole and electric field vectors are parallel. Thus, a dipole will always experience a torque that tends to align it with the electric field vector. The dipole is thus in a stable equilibrium when it is parallel to the electric field.

\begin{framed}
\textbf{Checkpoint:label: cp:chargesfields:efield}\\
When an electric dipole is such that its dipole vector is anti-parallel to the electric field vector, the dipole is

\begin{enumerate}
\item not in equilibrium.
\item in a stable equilibrium.
\item in an unstable equilibrium.
\end{enumerate}

\begin{framed}
\textbf{Answer}\\
\begin{enumerate}[resume]
\item
\end{enumerate}
\end{framed}
\end{framed}

We can also model the behaviour of the dipole using energy. If a dipole is rotated away from its equilibrium orientation and then released, it will gain (rotational) kinetic energy as it tries to return to equilibrium, and will oscillate about the equilibrium position. When the dipole is held out of equilibrium, we can think of it has having potential energy. To determine the functional form of that potential energy function, we consider the work done in rotating the dipole from an angle $\theta_1$ to an angle $\theta_2$ (where the angle is between the dipole and the electric field vectors):
\begin{equation}
W&=\int_{\theta_1}^{\theta_2} \tau d\theta=\int_{\theta_1}^{\theta_2} -pE\sin\theta d\theta=-pE\int_{\theta_1}^{\theta_2} \sin\theta d\theta\\
&=pE[\cos\theta]_{\theta_1}^{\theta_2}=pE\cos\theta_2-pE\cos\theta_1
\end{equation}
where the negative sign in the torque is to indicate that the torque is in the opposite direction from increasing $\theta$ (in Figure~\ref{fig:chargesfields:dipoleinfield}, the torque is clockwise whereas the angle $\theta$ increases counter-clockwise). The net work done in going from position $\theta_1$ to $\theta_2$ is the negative of the change in potential energy in going from $\theta_1$ to $\theta_2$. Thus, we define the potential energy of an electric dipole, $\vec p$, in an electric field, $\vec E$, as:
\begin{equation}
\boxed{U=-pE\cos\theta=-\vec p\cdot \vec E}
\end{equation}
which has a negative sign, and we also recognize that this is equivalent to the scalar product between $\vec p$ and $\vec E$. Note that the negative sign makes sense because systems experience a force/torque that will decrease their potential energy. When the angle is zero, $\cos\theta=1$, is maximal. Since we need the position with $\theta=0$ to have the lowest potential energy, the minus sign guarantees that all values of $\theta$ other than zero will give a potential energy that is higher (greater than) $( -1) pE$. Remember that only changes in potential energy are relevant, so the minus sign should not bother you, although you should think about whether it makes sense.

\subsubsection{Summary}

Objects can acquire a net charge if they acquire a net excess or deficit of electrons. Charges are never created, they are only transferred from one object to another. One can charge an object by friction, conduction, or induction. Materials can be classified broadly as conductors, where electrons can move freely in a material, or insulators, in which electrons remain tightly bound to the atoms in the material. If a conducting object acquires a net charge, those charges will migrate to the surface of the conductor.

Coulomb was the first to quantitatively describe the electric force exerted on a point charge, $Q_1$, by a second point charge, $Q_2$, located a distance, $r$, away:
\begin{equation}
\vec F_{12}=k\frac{Q_1Q_2}{r^2}\hat r_{21}=\frac{1}{4\pi\epsilon_0}\frac{Q_1Q_2}{r^2}\hat r_{21}
\end{equation}
where $\hat r_{21}$ is the unit vector from  $Q_2$ to $Q_1$. One can write the force using either Coulomb's constant, $k$, or the permittivity of free space, $\epsilon_0$. Coulomb's force is attractive if the product $Q_1Q_2$ is negative, and repulsive if the product is positive. Thus, charges of the same sign exert a repulsive force on each other, whereas opposite charges exert an attractive force on each other.

Mathematically, Coulomb's Law is identical to the gravitational force in Newton's Universal Theory of Gravity, which implies that it is conservative. The electric field vector at some position in space is defined to be the electric force per unit charge at that position in space. That is, at some position in space where the electric field vector is $\vec E$, a charge, $q$, will experience an electric force:
\begin{equation}
\vec F=q\vec E
\end{equation}
much like a mass, $m$, will experience a gravitational force, $m\vec g$, in a position in space where the gravitational field is $\vec g$. A positive charge will experience a force in the same direction as the electric field, whereas a negative charge will experience a force in the direction opposite of the electric field. The electric field at position, $\vec r$, from a point charge, $Q$, located at the origin, is given by:
\begin{equation}
\vec E = k\frac{Q}{r^2}\hat r
\end{equation}
One can visualize an electric field by using ``field lines''. The field vector at any point in space has a magnitude that is proportional to the number of field lines at that point, and a direction that is tangent to the field lines at that point.

We can model the electric field from a continuous charged object (i.e. not a point charge) by modelling the object as being made up of many point charges. Often, it is easiest to model an $N$-dimensional object as being the sum of objects of dimension $N -1$ and an infinitesimal length in the remaining dimension. For example, we modelled a line of charge as the sum of point charges that have an infinitesimal length, and we modelled a disk of charge as the the sum of rings that have an infinitesimal thickness. In general, the strategy to model the electric field from a continuous distribution of charge is the same:

\begin{enumerate}
\item Make a \textit{good} diagram.
\item Choose a charge element $dq$.
\item Draw the electric field element, $d\vec E$, at the point of interest.
\item Write out the electric field element vector, $d\vec E$, in terms of $dq$ and any other relevant variables.
\item Think of symmetry: will any of the component of $d\vec E$ sum to zero over all of the $dq$?
\item Write the total electric field as the sum (integral) of the electric field elements.
\item Identify which variables change as one varies the $dq$ and choose an integration variable to express $dq$ and everything else in terms of that variable and other constants.
\item Do the sum (integral).
\end{enumerate}

Finally, we introduced the electric dipole, which is an object comprised of two equal and opposite charges, $+Q$ and $-Q$, held at fixed distance, $l$, from each other. One can model an electric dipole using its dipole vector, $\vec p$, defined to point in the direction from $-Q$ to $+Q$, with magnitude:
\begin{equation}
p=Ql
\end{equation}
When a dipole is immersed in a uniform electric field, $\vec E$, it will experience a torque given by:
\begin{equation}
\vec\tau=\vec p\times \vec E
\end{equation}
The torque will act such as to align the vector $\vec p$ with the electric field vector. We can define a potential energy, $U$, to model the energy that is stored in a dipole when it is not aligned with the electric field:
\begin{equation}
U=-\vec p \cdot \vec E
\end{equation}
The point of lowest potential energy corresponds to the case when $\vec p$ and $\vec E$ are parallel, whereas the point of highest potential energy is when the two vectors are anti-parallel.

\begin{framed}
\textbf{Important Equations}\\
\textbf{Coulomb's Law:}
\begin{equation}
\vec F_{12}=k\frac{Q_1Q_2}{r^2}\hat r_{21}=\frac{1}{4\pi\epsilon_0}\frac{Q_1Q_2}{r^2}\hat r_{21}
\end{equation}
\textbf{Electric field:}
\begin{equation}
\vec E = k \frac{Q}{r^2}\vec r \\
\vec E = \int d \vec E \\
\end{equation}
\textbf{Electric force:}
\begin{equation}
\vec F = q \vec E \\
\end{equation}

\textbf{Electric dipole moment:}
\begin{equation}
p = Ql \\
\end{equation}
\textbf{Torque on a dipole:}
\begin{equation}
\vec \tau = \vec p \times \vec E \\
\end{equation}
\textbf{Potential energy stored in a dipole:}
\begin{equation}
U = -\vec p \cdot \vec E \\
\end{equation}
\end{framed}

\begin{framed}
\textbf{Important Definitions}\\
\begin{itemize}
\item \textbf{Electric field:} The electric field is defined to be the electric force per unit charge. SI units: ${\rm \left[{N/C}, {\rm V/m}\right]}$. Common variable(s): $\vec E$.
\item \textbf{Coulomb's constant:} A fundamental physical constant which relates charge and distance to electric field. SI units: [ ${\rm Nm^2C^{ -2}}$]. Common variable(s): $k$.
\item \textbf{Electric dipole moment:} A vector used to represent an electric dipole. SI units: ${\rm \left[{Cm}\right]}$. Common variable(s): $\vec p$.
\item \textbf{Linear charge density:} The charge per unit length of an object. SI units: ${\rm \left[{C/m}\right]}$. Common variable(s): $\lambda$.
\item \textbf{Surface charge density:} The charge per unit area of an object. SI units: ${\rm \left[{Cm^{ -2}}\right]}$. Common variable(s): $\sigma$.
\item \textbf{Volume charge density:} The charge per unit volume of an object. SI units: ${\rm \left[{Cm^{ -3}}\right]}$. Common variable(s): $\rho$.
\end{itemize}
\end{framed}

\subsubsection{Thinking about the material}

\begin{framed}
\textbf{Reflect and research}\\
\begin{itemize}
\item Which molecule has the largest dipole moment? Why?
\item How does a laser printer exploit physical properties covered in this chapter?
\item How does a Van de Graff generator work?
\item On the 20th of May, 2019, SI base units were redefined. How does this affect Coulomb's constant?
\end{itemize}
\end{framed}

\begin{framed}
\textbf{To try at home}\\
\begin{itemize}
\item Rub your hands or feet along various household items to test their electron affinity. Which household items produce a static charge?
\item After charging your body, research the electron affinity of the surface you used to charge yourself. Knowing this, how many electrons were transferred while you charged yourself?
\end{itemize}
\end{framed}

\begin{framed}
\textbf{To try in the lab}\\
\begin{itemize}
\item Propose an experiment to measure the Coulomb's constant.
\item Propose an experiment to organize various materials based on their electron affinity.
\end{itemize}
\end{framed}

\subsubsection{Sample problems and solutions}

\paragraph{Problems}

\begin{framed}
\textbf{Problem 15.1}\\
Consider three charged rods of length $L$ which are arranged to form a triangle, as shown in Figure~\ref{fig:ChargesFields:trianglediagram}. If the charge on each rod is evenly distributed, what is the net electric field at the centre of the triangle (defined as the intersection of the bisectors)?

\begin{figure}[!htbp]
\centering
\includegraphics[width=0.4\linewidth]{files/trianglediagram-57cfb5fcb5af53ea4a1cdbd640dc0dee.png}
\caption[]{A triangle made up of charged rods}
\label{fig:ChargesFields:trianglediagram}
\end{figure}
\end{framed}

\begin{framed}
\textbf{Problem15.2}\\
Suppose that an electric dipole, with electric dipole moment, $\vec p$, is placed in a uniform electric field $\vec E$. In equilibrium, the dipole moment vector, $\vec p$, will be parallel to the electric field vector, $\vec E$. Show that, if the dipole is displaced (rotated) by a small angle away from the equilibrium, the dipole will then experience simple harmonic motion.
\end{framed}

\paragraph{Solutions}

\begin{framed}
\textbf{Solution 15.1}\\
We can model the object as the sum of three finite length wires of the length, $L$. In Example~15.5, we determined that the electric field produced at a distance, $R$, by a finite wire that subtends an angle $2\theta_0$ is given by:
\begin{equation}
E = \frac{2k\lambda}{R}\sin\theta_0
\end{equation}
The angle $\theta_0$ is easily found to be $\theta_0 = \frac{\pi}{3}$, as shown in Figure~\ref{fig:ChargesFields:trianglesolution}. The distance, $R$, is similarly found to be given by:
\begin{equation}
R = \frac{L}{2}\tan\left(\frac{\pi}{6}\right)=\frac{\sqrt{3}}{6}L
\end{equation}
\begin{figure}[!htbp]
\centering
\includegraphics[width=0.6\linewidth]{files/trianglesolution-2125f2bf1c8dc2704b3068619355f64e.png}
\caption[]{Solving for $\theta_0$ and $R$}
\label{fig:ChargesFields:trianglesolution}
\end{figure}

Thus, the magnitude of the electric field from one wire is given by:
\begin{equation}
E&= \frac{2k\lambda}{R}\sin\left(\frac{\pi}{3}\right)=\frac{\sqrt{3}k\lambda}{R}=\frac{\sqrt{3}k\lambda}{\frac{\sqrt{3}}{6}L}=\frac{6k\lambda}{L}
\end{equation}
The charge, $Q$, is evenly distributed along the rod of length, $L$, so that we can rewrite the charge density as $\lambda=\frac{Q}{L}$, which gives:
\begin{equation}
E &= \frac{6k\lambda}{L} = \frac{6kQ}{L^2}
\end{equation}
This is the magnitude of the electric field produced by each side of the triangle. The two positive wires will produce electric fields whose horizontal components ($x$-axis in Figure~\ref{fig:ChargesFields:trianglesolution}) cancel. The net electric field from the two positive, $\vec E^{pos}$, wires will then be in the negative $y$ direction:
\begin{equation}
E_{y}^{pos}=-2 \left( \frac{6kQ}{L^2}  \right) \cos\left(\frac{\pi}{3} \right)=-\frac{6kQ}{L^2}
\end{equation}
The negative wire will also produce a field, $\vec E^{neg}$, in the negative $y$ direction (the only component of the field is in the $y$ direction):
\begin{equation}
E_{y}^{neg}=- \frac{6kQ}{L^2}
\end{equation}
Thus, the total field at the centre of the triangle, when (vectorially) adding the fields from the two positive wires and the negative wire is given by:
\begin{equation}
\vec E = \left(-\frac{6kQ}{L^2} - \frac{6kQ}{L^2} \right)\hat y=- \frac{12kQ}{L^2}\hat y
\end{equation}
\end{framed}

\begin{framed}
\textbf{Solution 15.2}\\
When the dipole is rotated from the equilibrium position by an angle $\theta$, it will experience a restoring torque exerted by the electric field, with magnitude:
\begin{equation}
\tau = -pE\sin\theta
\end{equation}
where we have inserted a minus sign to indicate that this is a restoring torque, in the opposite direction of increasing angle $\theta$. The net torque is then equal to the moment of inertia times the angular acceleration of the dipole (Newton's Second Law applied for rotation):
\begin{equation}
-pE\sin\theta &= I\alpha\\
\therefore \alpha &= -\frac{pE}{I}\sin\theta\sim-\frac{pE}{I}\theta
\end{equation}
where in the last equality, we made the small angle approximation ($\sin\theta\sim\theta$). This has the form for simple harmonic motion, with angular frequency, $\omega$:
\begin{equation}
\frac{d^2\theta}{dt^2}&=-\omega^2 \theta\\
\alpha=\omega &=\sqrt{\frac{pE}{I}}
\end{equation}
\end{framed}

\subsection{Chapter 16 - Gauss' Law}

\subsubsection{Overview}\label{chapter:gauss}

In this chapter, we take a detailed look at Gauss' law applied in the context of the electric field. We have already encountered Gauss' law briefly in Section {\textbackslash}ref\{sec:gravity:gauss\} when we examined the gravitational field. Since the electric force is mathematically identical to the gravitational force, we can apply the same tools, including Gauss' law, to model the electric field as we do the gravitational field. Many of the results from this chapter are thus equally applicable to the gravitational force.

\begin{framed}
\textbf{Learning Objectives}\\
\begin{itemize}
\item Understand the concept of flux for a vector field.
\item Understand how to calculate the flux of a vector field through an open and a closed surface.
\item Understand how to apply Gauss' law quantitatively to determine an electric field.
\item Understand how to apply Gauss' law qualitatively to discuss charges on a conductor.
\end{itemize}
\end{framed}

\begin{framed}
\textbf{Think About It}\\
A neutral spherical conducting shell encloses a point charge, $Q$, located at the centre of the shell. Due to separation of charge, the outer surface of the shell will acquire a net positive charge. What is the magnitude of that charge? \}

\begin{enumerate}
\item less than $Q$.
\item exactly $Q$.
\item more than $Q$.
\end{enumerate}

\begin{framed}
\textbf{Answer}\\
\begin{enumerate}[resume]
\item
\end{enumerate}
\end{framed}
\end{framed}

\subsubsection{Flux of the electric field}\label{sec:gauss:flux}

Gauss' law makes use of the concept of ``flux''. Flux is always defined based on:

\begin{enumerate}
\item A surface.
\item A vector field (e.g. the electric field).
\end{enumerate}

and can be thought of as a measure of the number of field lines from the vector field that cross the given surface. For that reason, one usually refers to the ``flux of the electric field through a surface''. This is illustrated in Figure~\ref{fig:gauss:fluxangle} for a uniform horizontal electric field, and a flat surface, whose normal vector, $\vec A$, is shown. If the surface is perpendicular to the field (left panel), so that the field vector is parallel to $\vec A$, then the flux through that surface is maximal. If the surface is parallel to the field (right panel), then no field lines cross that surface, and the flux through that surface is zero. If the surface is rotated with respect to the electric field, as in the middle panel, then the flux through the surface is between zero and the maximal value.

\begin{figure}[!htbp]
\centering
\includegraphics[width=0.9\linewidth]{files/fluxangle-958ab1f9963cf42848c577cd247fa3d9.png}
\caption[]{Flux of an electric field through a surface that makes different angles with respect to the electric field. In the leftmost panel, the surface is oriented such that the flux through it is maximal. In the rightmost panel, there are no field lines crossing the surface, so the flux through the surface is zero.}
\label{fig:gauss:fluxangle}
\end{figure}

The vector $\vec A$ is used to represent the surface. It is defined so that the magnitude of $\vec A$ is equal to the area of the surface and the direction of $\vec A$ is perpendicular to the surface, as illustrated in Figure~\ref{fig:gauss:fluxangle}. We define the flux, $\Phi_E$, of the electric field, $\vec E$, through the surface represented by vector $\vec A$ as:
\begin{equation}
\Phi_E=\vec E\cdot \vec A=EA\cos\theta
\end{equation}
since this will have the same properties that we described above (e.g. the flux is zero when $\vec E$ and $\vec A$ are perpendicular, and the flux is proportional to number of field lines crossing the surface). Note that, because we only require that $\vec A$ is perpendicular to the surface, there are two possible choices for the direction of $\vec A$. As a result, the flux could be either positive or negative. By convention, we usually choose $\vec A$ so that the flux is positive.

\begin{framed}
\textbf{Olivia's Thoughts}\\
In the most general terms, flux is the amount of something passing through an area. In the case of the electric flux, this is the number of electric field lines passing through a surface. We can draw parallels between the electric flux and a more intuitive example of flux: the amount of water flowing through a net. Imagine submerging a net in a flowing river. The amount of water that flows through the net will depend on the strength of the current (this is like $E$), the orientation of the net relative to the flow of water ($\cos\theta$), and the size of the net ($A$). When trying to conceptualize problems about electric flux, it sometimes helps to think of the field lines as representing water currents and the surface as representing a net in order to visualize what's going on in the problem.
\end{framed}

\begin{framed}
\textbf{Checkpoint}\\
What are the S.I. units of electric flux?

\begin{enumerate}
\item ${\rm N\cdot m/C}$
\item ${\rm V\cdot m}$
\item ${\rm V/m}$
\item The units of flux depend on the dimensions of the charged object.
\end{enumerate}

\begin{framed}
\textbf{Answer}\\
\begin{enumerate}[resume]
\item
\end{enumerate}
\end{framed}
\end{framed}

\begin{framed}
\textbf{Example 16.1}\\
A uniform electric field is given by: $\vec E=E\cos\theta~\hat x+E\sin\theta~\hat y$ throughout space. A rectangular surface is defined by the four points $(0,0,0)$, $(0,0,H)$, $(L,0,0)$, $(L,0,H)$. What is the flux of the electric field through the surface?

\begin{framed}
\textbf{Solution}\\
The surface that is defined corresponds to a rectangle in the $xz$ plane with area $A=LH$. Since the rectangle lies in the $xz$ plane, a vector perpendicular to the surface will be along the $y$ direction. We choose the positive $y$ direction, since this will give a positive number for the flux (as the electric field has a positive component in the $y$ direction). The vector $\vec A$ is given by:
\begin{equation}
\vec A =A\hat y=LH\hat y
\end{equation}
The flux through the surface is thus given by:
\begin{equation}
\Phi_E&=\vec E\cdot \vec A=(E\cos\theta\hat x+E\sin\theta\hat y)\cdot(LH\hat y)\\
&=ELH\sin\theta
\end{equation}
where one should note that the angle $\theta$, in this case, is not the angle between $\vec E$ and $\vec A$, but rather the complement of that angle.

\textbf{Discussion:} In this example, we calculated the flux of a uniform electric field through a rectangle of area, $A=LH$. Since we knew the components of both the electric field vector, $\vec E$, and the surface vector, $\vec A$, we used their scalar product to determine the flux through the surface. In some cases, it is easier to work with the magnitude of the vectors and the angle between them to determine the scalar product (although note that in this example, the angle between $\vec E$ and $\vec A$ is $90{\degree} -\theta$).
\end{framed}
\end{framed}

\paragraph{Non-uniform fields}

So far, we have considered the flux of a uniform electric field, $\vec E$, through a surface, $S$, described by a vector, $\vec A$. In this case, the flux, $\Phi_E$, is given by:
\begin{equation}
\Phi_E=\vec E\cdot \vec A
\end{equation}
However, if the electric field is not constant in magnitude and/or in direction over the entire surface, then we divide the surface, $S$, into many infinitesimal surfaces, $dS$, and sum together (integrate) the fluxes from those infinitesimal surfaces:
\begin{equation}
\boxed{\Phi_E=\int \vec E\cdot d\vec A}
\end{equation}
where, $d\vec A$, is the normal vector for the infinitesimal surface, $dS$. This is illustrated in Figure~\ref{fig:gauss:fluxdA}, which shows, in the left panel, a surface for which the electric field changes magnitude along the surface (as the field lines are closer in the lower left part of the surface), and, in the right panel, a scenario in which the direction and magnitude of the electric field vary along the surface.

In order to calculate the flux through the total surface, we first calculate the flux through an infinitesimal surface, $dS$, over which we assume that $\vec E$ is constant in magnitude and direction, and then, we sum (integrate) the fluxes from all of the infinitesimal surfaces together. Remember, the flux through a surface is related to the number of field lines that cross that surface; it thus makes sense to count the lines crossing an infinitesimal surface, $dS$, and then adding those together over all the infinitesimals surfaces to determine the flux through the total surface, $S$.

\begin{figure}[!htbp]
\centering
\includegraphics[width=0.9\linewidth]{files/fluxdA-675e21c7412f32ffebdb6ac667e6b1ef.png}
\caption[]{Examples of surfaces that need to be sub-divided in order to determine the net flux through them. The surface on the left must be subdivided because the electric field changes magnitude over the surface, whereas the one on the right needs to be subdivided because the angle between $\vec E$ and $d\vec A$ is not constant (and the magnitude of $\vec E$ also changes along the surface).}
\label{fig:gauss:fluxdA}
\end{figure}

\begin{framed}
\textbf{Example 16.2}\\
An electric field points in the $z$ direction everywhere in space. The magnitude of the electric field depends linearly on the $x$ position in space, so that the electric field vector is given by: $\vec E=(ax -b)\hat z$, where $a$ and $b$ are constants. What is the flux of the electric field through a square with side length $L$ that is located in the positive $xy$ plane with one of its corners at the origin?

\begin{framed}
\textbf{Solution}\\
We need to calculate the flux of the electric field through a square of side length $L$ in the $xy$ plane. The electric field is always in the $z$ direction, so the angle between $\vec E$ and $d\vec A$ (the normal vector for any infinitesimal area element) will remain constant. We can calculate the flux through the square by dividing up the square into thin strips of length $L$ in the $y$ direction and infinitesimal width $dx$ in the $x$ direction, as illustrated in Figure~\ref{fig:gauss:fluxlinx}. We can do this because the electric field does not change with $y$, so the flux along each of these strips will be constant. If the electric field varied both as a function of $x$ and $y$, we would start with area elements that have infinitesimal dimensions in both the $x$ and the $y$ directions.

\begin{figure}[!htbp]
\centering
\includegraphics[width=0.5\linewidth]{files/fluxlinx-e8e921d01654ee11a768fd94b9df938c.png}
\caption[]{Dividing a square in the $xy$ plane into thin strips of length $L$ and width $dx$.}
\label{fig:gauss:fluxlinx}
\end{figure}

As illustrated in Figure~\ref{fig:gauss:fluxlinx}, we first calculate the flux through a thin strip of area, $dA=Ldx$, located at position $x$ along the $x$ axis. Choosing the direction of $d\vec A$ so that it gives a positive flux, we can obtain the flux through the strip:
\begin{equation}
d\Phi_E=\vec E\cdot d\vec A=EdA=(ax-b)Ldx
\end{equation}
where $\vec E\cdot d\vec A=EdA$, since the angle between $\vec E$ and $\vec A$ is zero. Summing together the fluxes from the strips, from $x=0$ to $x=L$, the total flux is given by:
\begin{equation}
\Phi_E=\int d\Phi_E=\int_0^L(ax-b)Ldx=\frac{1}{2}aL^3-bL^2
\end{equation}
\textbf{Discussion:} In this example, we showed how to calculate the flux from an electric field that changes magnitude with position. We modelled a square of side length $L$ as being made of many thin strips of length $L$ and width $dx$. We then calculated the flux through each strip and added those together to obtain the total flux through the square.
\end{framed}
\end{framed}

\paragraph{Closed surfaces}\label{sec:gauss:closedsurfaces}

One can distinguish between a ``closed'' surface and an ``open'' surface. A surface is closed if it completely defines a volume that could, for example, be filled with a liquid.  A closed surface has a clear ``inside'' and an ``outside''. For example, the surface of a sphere, of a cube, or of a cylinder are all examples of closed surfaces. A plane, a triangle, and a disk are, on the other hand, examples of ``open surfaces''.

For a closed surface, one can unambiguously define the direction of the vector $\vec A$ (or $d\vec A$) as the direction that it is perpendicular to the surface and \textbf{points towards the outside}. Thus, the sign of the flux out of a closed surface is meaningful. The flux will be positive if there is a net number of field lines exiting the volume defined by the surface (since $\vec E$ and $\vec A$ will be parallel on average) and the flux will be negative if there is a net number of field lines entering the volume (as $\vec E$ and $\vec A$ will be anti-parallel on average). The flux through a closed surface is thus zero if the number of field lines that enter the surface is the same as the number of field lines that exit the surface. When calculating the flux over a closed surface, we use a different integration symbol to show that the surface is closed:
\begin{equation}
\Phi_E=\oint \vec E\cdot d\vec A
\end{equation}
which is the same integration symbol that we used for indicating a path integral when the initial and final points are the same (see, for example, Section~\ref{sec:potentialecons:conservative}).

\begin{framed}
\textbf{Checkpoint}\\
\begin{figure}[!htbp]
\centering
\includegraphics[width=0.25\linewidth]{files/irregularfield-0ce5dc8f672cf41596e1d051d4a15239.png}
\caption[]{A non-uniform electric field flowing through an irregularly shaped closed surface.}
\label{fig:gauss:irregularfield}
\end{figure}

A non-uniform electric field $\vec E$ flows through an irregularly-shaped closed surface, as shown in Figure~\ref{fig:gauss:irregularfield}. The flux through the surface is

\begin{enumerate}
\item positive.
\item zero.
\item negative.
\end{enumerate}

\begin{framed}
\textbf{Answer}\\
\begin{enumerate}[resume]
\item
\end{enumerate}
\end{framed}
\end{framed}

\begin{framed}
\textbf{Olivia's Thoughts}\\
Consider the water flowing through a net analogy again, although now the net is a closed surface (e.g. a sphere). If there is more water flowing out of the net than into it, the flux is positive. If there is more flowing in than out, the flux is negative. If there is an equal amount of water flowing in and out, the flux is zero. If you had trouble with the last checkpoint question, try it again but now thinking about the field lines as a flow of water. Is there more water flowing in or out of the object, or is it the same?
\end{framed}

\begin{framed}
\textbf{Example 16.3}\\
A negative electric charge, $-Q$, is located at the origin of a coordinate system. Calculate the flux of the electric field through a spherical surface of radius, $R$, that is centred at the origin.

\begin{framed}
\textbf{Solution}\\
Figure~\ref{fig:gauss:fluxsphere} shows the spherical surface of radius, $R$, centred on the origin where the charge $-Q$ is located.

\begin{figure}[!htbp]
\centering
\includegraphics[width=0.3\linewidth]{files/fluxsphere-ae8a21173bcf1a55fdaa3f4d3992daec.png}
\caption[]{Calculating the flux through a spherical surface.}
\label{fig:gauss:fluxsphere}
\end{figure}

At all points along the surface, the electric field has the same magnitude:
\begin{equation}
E=\frac{1}{4\pi\epsilon_0}\frac{Q}{R^2}
\end{equation}
as given by Coulomb's law for a point charge. Although the vector $\vec E$ changes direction everywhere along the surface, it always makes the same angle ({\textbackslash}SI\{-180\}\{{\textbackslash}degree\}) with the corresponding vector $d\vec A$ at any particular location. Indeed, for a point charge, the electric field points in the radial direction (inwards for a negative charge) and is thus perpendicular to the spherical surface at all points. Since the surface is closed, the vector $d\vec A$ points outwards anywhere on the surface. Thus, at any point on the surface, we can evaluate the flux through an infinitesimal area element, $d\vec A$:
\begin{equation}
d\Phi_E=\vec E\cdot d\vec A=EdA\cos(-180\degree)=-EdA
\end{equation}
where the overall minus sign comes from the fact that $\vec E$ and $d\vec A$ are anti-parallel. The total flux through the spherical surface is obtained by summing together the fluxes through each area element:
\begin{equation}
\Phi_E=\oint d\Phi_E=\oint -EdA=-E\oint dA=-E(4\pi R^2)
\end{equation}
where we factored $E$ out of the integral, since the magnitude of the electric field is constant over the entire surface (a constant distance $R$ from the charge). In the last equality, we recognized that, $\oint dA$, simply means ``sum together all of the areas, $dA$, of the surface elements'', which gives the total surface area of the sphere, $4\pi R^2$. The flux through the spherical surface is negative, because the charge is negative, and the field lines point towards $-Q$.

Using the value that we obtained for the magnitude of the electric field from Coulomb's Law, the total flux is given by:
\begin{equation}
\Phi_E=-E(4\pi R^2)=-\frac{1}{4\pi\epsilon_0}\frac{Q}{R^2}(4\pi R^2)=-\frac{Q}{\epsilon_0}
\end{equation}
which, surprisingly, is independent of the radius of the spherical surface. Note that we used $\epsilon_0$ instead of Coulomb's constant, $k$, since the result is cleaner without the extra factor of $4\pi$.

\textbf{Discussion:} In this example, we calculated the flux of the electric field from a negative point charge through a spherical surface concentric with the charge. We found the flux to be negative, which makes sense, since the field lines go towards a negative charge, and there is thus a net number of field lines entering the spherical surface. Perhaps surprisingly, we found that the total flux through the surface does not depend on the radius of the surface! In fact, that statement is precisely Gauss' law: the net flux out of a closed surface depends only on the amount of charge enclosed by that surface (and the constant, $\epsilon_0$). Gauss' law is of course more general, and applies to surfaces of any shape, as well as charges of any shape (whereas Coulomb's Law only holds for point charges).
\end{framed}
\end{framed}

\subsubsection{Gauss' Law}

Gauss' law is a relation between the net flux through a closed surface and the amount of charge, $Q^{enc}$, in the volume enclosed by that surface:
\begin{equation}
\boxed{\oint \vec E\cdot d\vec A=\frac{Q^{enc}}{\epsilon_0}}
\end{equation}
In particular, note that Gauss' law holds true for \textbf{any} closed surface, and the shape of that surface is not specified in Gauss' law.  That is, we \textbf{can always choose the surface to use} when calculating the flux. For obvious reasons, we often call the surface that we choose a ``Gaussian surface''. But again, this surface is simply a mathematical tool, there is no actual property that makes a surface ``Gaussian''; it simply means that we chose that surface in order to apply Gauss' law.  In Example~16.3 above, we confirmed that Gauss' law is compatible with Coulomb's Law for the case of a point charge and a spherical Gaussian surface.

Physically, Gauss' law is a statement that field lines must begin or end on a charge (electric field lines originate on positive charges and terminate on negative charges). Recall, flux is a measure of the net number of lines coming out of a surface. If there is a net number of lines coming out of a closed surface (a positive flux), that surface must enclose a positive charge from where those field lines originate. Similarly, if there are the same number of field lines entering a closed surface as there are lines exiting that surface (a flux of zero), then the surface encloses no charge. Gauss' law simply states that the number of field lines exiting a closed surface is proportional to the amount of charge enclosed by that surface.

Primarily, Gauss' law is a useful tool to determine the magnitude of the electric field from a given charge, or charge distribution. We usually have to use symmetry to determine the direction of the electric field vector. In general, the integral for the flux is difficult to evaluate, and Gauss' law can only be used analytically in cases with a high degree of symmetry. Specifically, the integral for the flux is easiest to evaluate if:

\begin{enumerate}
\item \textbf{The electric field makes a constant angle with the surface}. When this is the case, the scalar product can be written in terms of the cosine of the angle between $\vec E$ and $d\vec A$, which can be taken out of the integral if it is constant:
\end{enumerate}
\begin{equation}
\oint \vec E\cdot d\vec A=\oint E\cos\theta dA=\cos\theta\oint EdA
\end{equation}
Ideally, one has chosen a surface such that this angle is 0 or $180\degree$.

\begin{enumerate}[resume]
\item \textbf{The electric field is constant in magnitude along the surface}. When this is the case, the integral can be simplified further by factoring out $E$ and simply becomes an integral over $dA$ (which corresponds to the total area of the surface, $A$):
\end{enumerate}
\begin{equation}
\oint \vec E\cdot d\vec A=\cos\theta\oint EdA =E\cos\theta\oint dA=EA\cos\theta
\end{equation}
Ultimately, the points above should dictate the choice of Gaussian surface \textbf{so that} the integral for the flux is easy to evaluate. The choice of surface will depend on the symmetry of the problem. For a point (or spherical) charge, a spherical Gaussian surface allows the flux to easily be calculated (Example~16.3). For a line of charge, as we will see, a cylindrical surface results is a good choice for the Gaussian surface. Broadly, the steps for applying Gauss' law to determine the electric field are as follows:

\begin{enumerate}
\item Make a diagram showing the charge distribution.
\item Use symmetry arguments to determine in which way the electric field vector points.
\item Choose a Gaussian surface that goes through the point for which you want to know the electric field. Ideally, the surface is such that the electric field is constant in magnitude and always makes the same angle with the surface, so that the flux integral is straightforward to evaluate.
\item Calculate the flux, $\oint \vec E\cdot d\vec A$.
\item Calculate the amount of charge located within the volume enclosed by the surface, $Q^{enc}$.
\item Apply Gauss' law,  $\oint \vec E\cdot d\vec A=\frac{Q^{enc}}{\epsilon_0}$.
\end{enumerate}

\begin{framed}
\textbf{Example 16.4}\\
An insulating sphere of radius, $R$, contains a total charge, $Q$, which is uniformly distributed throughout its volume. Determine an expression for the electric field as a function of distance, $r$, from the centre of the sphere.
Note that this is identical, mathematically, to the derivation that was done in Section~\ref{sec:gravity:gauss} for the case of gravity.

\begin{framed}
\textbf{Solution}\\
When applying Gauss' law,  we first need to think about symmetry in order to determine the direction of the electric field vector. We also need to think about all possible regions of space in which we need to determine the electric field. In particular, for this case, we need to determine the electric field both inside ($r\leq R$) and outside ($r\geq R$) of the charged sphere.

Figure~\ref{fig:gauss:spheresymmetry} shows the charged sphere of radius $R$. If we consider the direction of the electric field outside the sphere (where $\vec E_{out}$ is drawn), we realize that it can only point in the radial direction (towards or away from the centre of the sphere), as this is the only choice that preserves the symmetry of the sphere. Being a sphere, the charge looks the same from all angles; thus, the electric field must also look the same from all angles, otherwise, there would be a preferred orientation for the sphere. The same argument holds for the electric field vector inside the sphere (drawn as $\vec E_{in}$).

\begin{figure}[!htbp]
\centering
\includegraphics[width=0.4\linewidth]{files/spheresymmetry-cfcd411403bc4dc5e77a998a9afcd219.png}
\caption[]{For a spherical charge distribution, the electric field inside and outside must point in the radial direction, by symmetry.}
\label{fig:gauss:spheresymmetry}
\end{figure}

We now need to choose a Gaussian surface that will make the flux integral easy to evaluate. Ideally, we can find a surface over which the electric field makes the same angle with the surface and over which the electric field is constant in magnitude. Again, based on the symmetry of the charge distribution, it is clear that a spherical surface of radius, $r$, will satisfy these properties.

We start by applying Gauss' law outside the charge (with $r\geq R$) to determine the electric field, $\vec E_{out}$. Figure~\ref{fig:gauss:spherein} shows our choice of spherical Gaussian surface (labelled $S$) of radius, $r$, which is concentric with the spherical charge distribution of radius, $R$, and total charge, $+Q$.

\begin{figure}[!htbp]
\centering
\includegraphics[width=0.4\linewidth]{files/spherein-1460112cc9e975744df1cf30fe343f7c.png}
\caption[]{A spherical Gaussian surface to determine the electric field outside of a sphere of radius, $R$, holding charge, $+Q$.}
\label{fig:gauss:spherein}
\end{figure}

In order to apply Gauss' law,  we need to calculate:

\begin{itemize}
\item the net flux through the surface.
\item the charge in the volume enclosed by the surface.
\end{itemize}

The net flux through the surface is found in the same way as in Example~16.3, and is given by:
\begin{equation}
\Phi_E&=\oint \vec E\cdot d\vec A=\oint E dA= E\oint dA=E(4\pi r^2)
\end{equation}
where our choice of spherical surface led to $\vec E\cdot d\vec A=EdA$, since $\vec E$ and $d\vec A$ are always parallel. Furthermore, by symmetry, the electric field must be constant in magnitude along the whole surface, or the spherical symmetry would be broken. This allowed us to factor the $E$ out of the integral, leaving us with, $\oint dA$, which is simply the area of our Gaussian spherical surface, $4\pi r^2$.

The Gaussian surface with $r\geq R$ encloses the whole charged sphere, so the charge enclosed is simply the charge of the sphere, $Q^{enc}=Q$. Applying Gauss' law allows us to determine the magnitude of the electric field:
\begin{equation}
\oint \vec E\cdot d\vec A&=\frac{Q^{enc}}{\epsilon_0} \\
E(4\pi r^2) &= \frac{Q}{\epsilon_0}\\
\therefore E&= \frac{1}{4\pi\epsilon_0}\frac{Q}{r^2}
\end{equation}
which is the same as the electric field a distance $r$ from a point charge. Thus, from the outside, a spherical charge distribution leads to the same electric field as if the charge were concentrated at the centre of the sphere.

Next, we determine the magnitude of the electric field inside the charged sphere. In this case, we choose a spherical Gaussian surface of radius $r\leq R$, that is concentric with the sphere, as illustrated by the surface labelled, $S$, that is shown in Figure~\ref{fig:gauss:sphereout}.

\begin{figure}[!htbp]
\centering
\includegraphics[width=0.4\linewidth]{files/sphereout-6866d16fabe96bb0b0685aa3085b2ab7.png}
\caption[]{A spherical Gaussian surface to determine the electric field inside of a sphere of radius, $R$, holding charge, $+Q$.}
\label{fig:gauss:sphereout}
\end{figure}

The flux integral is trivial again, since the electric field always makes the same angle with the Gaussian surface, and the magnitude of the electric field is constant in magnitude along the surface:
\begin{equation}
\Phi_E&=\oint \vec E\cdot d\vec A=\oint E dA= E\oint dA=E(4\pi r^2)
\end{equation}
In this case, however, the charge in the volume enclosed by the Gaussian surface is less than $Q$, since the whole charge is not enclosed. We are told that the charge is distributed uniformly throughout the spherical volume of radius $R$. We can thus define a volume charge density, $\rho$, (charge per unit volume) for the sphere:
\begin{equation}
\rho=\frac{Q}{V}=\frac{Q}{\frac{4}{3}\pi R^3}
\end{equation}
The volume enclosed by the Gaussian surface is $\frac{4}{3}\pi r^3$, thus the charge, $Q^{enc}$, contained in that volume is given by:
\begin{equation}
Q^{enc}=\frac{4}{3}\pi r^3 \rho=\frac{4}{3}\pi r^3 \frac{Q}{\frac{4}{3}\pi R^3}=Q\frac{r^3}{R^3}
\end{equation}
Finally, we apply Gauss' law to find the magnitude of the electric field inside the sphere:
\begin{equation}
\oint \vec E\cdot d\vec A&=\frac{Q^{enc}}{\epsilon_0} \\
E(4\pi r^2) &=\frac{Q}{\epsilon_0}\frac{r^3}{R^3}\\
\therefore E&= \frac{Q}{4\pi\epsilon_0R^3}r
\end{equation}
Note that the electric field increases linearly with radius inside of the charged sphere, and then decreases with radius squared outside of the sphere. Also, note that at the centre of the sphere, the electric field has a magnitude of zero, as expected from symmetry.

\textbf{Discussion:} In this example, we showed how to use Gauss' law to determine the electric field inside and outside of a uniformly charged sphere. We recognized the spherical symmetry of the charge distribution and chose to use a spherical surface in order to apply Gauss' law.  This, in turn, allowed the flux to be easily calculated. We found that outside the sphere, the electric field decreases in magnitude with radius squared, just as if the entire charge were concentrated at the centre of the sphere. Inside the sphere, we found that the electric field is zero at the centre, and increases linearly with radius.
\end{framed}
\end{framed}

\begin{framed}
\textbf{Olivia's Thoughts}\\
In Section~\ref{chapter:gravity}, I provided an analogy to explain Gauss' law for gravity. If that analogy worked for you, I recommend you revisit it, as the same principles apply here. I'm now going to describe another way to think about Gauss' law that will be useful as you become more familiar with field lines.

Figure~\ref{fig:gauss:sphere_fieldlines} shows the field lines coming from a positively charged sphere, like the one in Example~16.4.

\begin{figure}[!htbp]
\centering
\includegraphics[width=0.3\linewidth]{files/sphere_fieldlines-65b7b440bddcf87b2453dc79de3bf357.png}
\caption[]{The field lines for a positively charged sphere.}
\label{fig:gauss:sphere_fieldlines}
\end{figure}

Let's see what happens when we put a spherical Gaussian surface around the charged sphere. Figure~\ref{fig:gauss:2surfaces_1} shows two surfaces of different sizes.

\begin{figure}[!htbp]
\centering
\includegraphics[width=0.6\linewidth]{files/2surfaces_1-d3a5fe6e84b1b1eae3399078f9dfcc82.png}
\caption[]{Two Gaussian surfaces with different radii around a charged sphere.}
\label{fig:gauss:2surfaces_1}
\end{figure}

To make it a bit clearer, I'll now take the charge out of the image (Figure~\ref{fig:gauss:2surfaces_2}), and use dots to show where the field lines passed through each surface.

\begin{figure}[!htbp]
\centering
\includegraphics[width=0.5\linewidth]{files/2surfaces_2-7b83365589a0dc3bba79cef83eaf9f3f.png}
\caption[]{The Gaussian surfaces from Figure~\ref{fig:gauss:2surfaces_1}, where the dots show where the field lines passed through each surface.}
\label{fig:gauss:2surfaces_2}
\end{figure}

Gauss' law tells us that, because each surface enclosed the same amount of charge, the same number of field lines will pass through each surface. Therefore, these spheres should have the same number of dots on them. The difference between them is the \textit{density} of the dots. We have learned that the closer the field lines are to each other, the stronger the electric field is. So, if we wanted to know the magnitude of the electric field at some point on one of these surfaces, we would calculate the number field lines passing through it, and then divide by the area of the sphere to get the density of the field lines.

Let's revisit the equation for Gauss' law,
\begin{equation}
\oint \vec E\cdot d\vec A=\frac{Q^{enc}}{\epsilon_0}
\end{equation}
The right hand side gives us the number of field lines (the flux). If we can construct a surface so that the field lines passing through it are evenly distributed (and they are perpendicular to the surface), the left hand side becomes $\oint \vec E\cdot d\vec A= EA$. To solve for the electric field, we write $E=Q^{enc}/(\epsilon_0A)$, which is effectively saying that the electric field is equal to the number of field lines divided by the area of the surface.
\end{framed}

\begin{framed}
\textbf{Checkpoint}\\
\begin{figure}[!htbp]
\centering
\includegraphics[width=0.3\linewidth]{files/spherecube-975b218a76380121f84eee937fcc1c53.png}
\caption[]{A charged spherical shell with a cubic device inside of it.}
\label{fig:gauss:spherecube}
\end{figure}

A thin charged spherical shell carries a uniformly distributed charge of $+Q$. If we place a cube inside the shell, as shown in Figure~\ref{fig:gauss:spherecube}, what is the total flux out of the surface of the cube?

\begin{enumerate}
\item $\frac{Q}{12\pi}{\rm Vm}$.
\item $\frac{Q}{2\pi} {\rm Vm}$.
\item $\frac{Q}{6} {\rm Vm}$.
\item $0 {\rm Vm}$.
\end{enumerate}

\begin{framed}
\textbf{Answer}\\
\begin{enumerate}[resume]
\item
\end{enumerate}
\end{framed}
\end{framed}

\begin{framed}
\textbf{Example 16.5}\\
An infinitely long straight wire carries a uniform charge per unit length, $\lambda$. What is the electric field at a distance, $R$, from the wire?\}
The first thing we want to do is determine the direction of the electric field vector. We start by making a diagram of the charge distribution, as in Figure~\ref{fig:gauss:linecharge}. To choose the direction, we can use symmetry arguments. In this case,  our choice of field direction must preserve rotational symmetry. This means that if we are in the plane perpendicular to the wire (i.e. the side view in Figure~\ref{fig:gauss:linecharge}), the electric field should look the same from all directions (e.g. it shouldn't matter if we look at it from the left side or the right side). With this in mind, there are three options for the electric field. It could be either:

\begin{enumerate}
\item in the radial direction (point to/from the centre of the wire).
\item such that electric field lines form concentric circles with the wire.
\item co-linear with the wire.
\end{enumerate}

\begin{framed}
\textbf{Solution}\\
In all three possibilities above, you would not be able to infer that one particular direction in the plane perpendicular to the wire is preferred. All three possibilities preserve the rotational symmetry of the wire (the wire looks the same from all directions in the plane perpendicular to the wire).

We expect that a a negative charge would be attracted to the wire, so the electric field should have at least some radial component. We can thus eliminate the second and third options. The electric field will must then look like that illustrated in Figure~\ref{fig:gauss:linecharge}.

\begin{figure}[!htbp]
\centering
\includegraphics[width=0.4\linewidth]{files/linecharge-4014db5a513bae16bff2ff03bbbcfb09.png}
\caption[]{An infinite line of charge carrying uniform charge per unit length, $\lambda$. The left panel shows a side view and the right panel a view from above. The electric field must be in the radial direction or there would be a preferred direction.}
\label{fig:gauss:linecharge}
\end{figure}

Next, we need to choose a Gaussian surface in order to apply Gauss' law. A convenient choice is a cylinder (a ``pill box'') of radius $R$ and length $L$, as shown in Figure~\ref{fig:gauss:fluxlinecharge}, as this goes through a point that is a distance $R$ from the wire (where we are asked for the electric field). At all points on the cylindrical surface, the electric field vector is either perpendicular or parallel to the surface.

\begin{figure}[!htbp]
\centering
\includegraphics[width=0.4\linewidth]{files/fluxlinecharge-5050ac892660d0dcaa40a8b7c9eb94a3.png}
\caption[]{A cylindrical Gaussian surface is used to calculate the flux from an infinite line of charge.}
\label{fig:gauss:fluxlinecharge}
\end{figure}

We can think of the cylindrical surface as being composed of three surfaces: two disks on either end (the lids of the pill box), and the curved surface that makes up the side of the cylinder. The flux through the entire cylindrical surface will be the sum of the fluxes through the two lids plus the flux through the side:
\begin{equation}
\oint \vec E\cdot d\vec A = \int_{side} \vec E\cdot d\vec A + \int_{lid}\vec E\cdot d\vec A + \int_{lid}\vec E\cdot d\vec A
\end{equation}
where you should note that the closed integral ($\oint$) was separated into three normal integrals ($\int$) corresponding to the three ``open'' surfaces that make up the closed surface. Again, remember that the flux is proportional to the net number of field lines exiting/entering the closed surface, so it makes sense to count those lines over the three open surfaces and add them together to get the total number for the closed surface.

The flux through the lids is identically zero, since the electric field is perpendicular to $d\vec A$ everywhere on the lids. The total flux is therefore equal to the flux through the curved side surface, for which the electric field vector is always parallel to $d\vec A$, and for which the electric field vector is constant in magnitude:
\begin{equation}
\oint \vec E\cdot d\vec A = \int_{side} \vec E\cdot d\vec A =\int_{side} EdA=E\int_{side}dA=E(2\pi R L)
\end{equation}
where we recognized that the side surface can be unfolded into a rectangle of height $L$ and width $2\pi R$, corresponding to the circumference of the cylinder, so that the area of the side of the cylinder is given by $A=2\pi R L$.

Next, we determine the charge inside the volume enclosed by the surface. Since the cylinder encloses a length $L$ of wire, the enclosed charge is given by:
\begin{equation}
Q^{enc}=\lambda L
\end{equation}
where $\lambda$ is the charge per unit length on the wire. Putting this all together into Gauss' law gives us the electric field at a distance $R$ from the wire:
\begin{equation}
\oint \vec E\cdot d\vec A&=\frac{Q^{enc}}{\epsilon_0} \\
E(2\pi R L) &= \frac{\lambda L}{\epsilon_0}\\
\therefore E&= \frac{\lambda}{2\pi\epsilon_0R}
\end{equation}
Note that this is the same result that we obtained in Example~15.5, when we took the limit of the finite line of charge having infinite length.

\textbf{Discussion:} In this example, we applied Gauss' law to determine the electric field at a distance from an infinitely long charged wire. We used symmetry to argue that the field should be radial and in the plane perpendicular to the wire, and recognized that a cylindrical Gaussian surface would exploit the symmetry so that the flux can easily be calculated. We obtained the same result as we did from integrating Coulomb's Law in Example~15.5. However, using Gauss' law was much less work than integrating Coulomb's Law.
\end{framed}
\end{framed}

\begin{framed}
\textbf{Checkpoint}\\
Why is it difficult to apply Gauss' law to a finite wire?

\begin{enumerate}
\item It is easy to apply Gauss' law to a finite wire.
\item Because the flux of a finite wire is undefined.
\item Because we do not know the charge density of a finite wire.
\item Because the symmetry argument does not hold.
\end{enumerate}

\begin{framed}
\textbf{Answer}\\
\begin{enumerate}[resume]
\item
\end{enumerate}
\end{framed}
\end{framed}

\begin{framed}
\textbf{Josh's Thoughts}\\
Gauss' law requires us to choose a ``Gaussian'' surface, but which surface should we choose? Generally, it is useful to choose a surface such that the flux can easily be determined, ideally without having to actually do an integral. If symmetry can be exploited such that $\vec E$ has a constant magnitude and direction relative to $d\vec A$ at every location of the Gaussian surface, then $\int \vec E \cdot d\vec A$ will be equal to $E A$. This is why Gaussian surfaces are often of the same shape as the charged object they are enclosing.

For example, if I need to enclose a cylindrical charge, it would be reasonable to enclose the charge with a cylindrical Gaussian surface, as shown in Figure~\ref{fig:gauss:choosecylinder}

\begin{figure}[!htbp]
\centering
\includegraphics[width=0.3\linewidth]{files/choosecylinder-eda869a34c5238cae536651ec6a93e68.png}
\caption[]{A cylindrical Gaussian surface to enclose a cylindrical charge.}
\label{fig:gauss:choosecylinder}
\end{figure}

When dealing with point charges which have no shape and are thus spherically symmetric, it makes sense to choose a spherical Gaussian surface, as shown in Figure~\ref{fig:gauss:choosesphere}, since the electric field is in the radial direction for a point charge.

\begin{figure}[!htbp]
\centering
\includegraphics[width=0.2\linewidth]{files/choosesphere-bfbaa81e8b4c4b6ad75bf006d36063dd.png}
\caption[]{. A spherical Gaussian surface to enclose a point charge.}
\label{fig:gauss:choosesphere}
\end{figure}

Finally, there are some cases of less than ideal choices for the Gaussian surfaces. While never wrong, they may require rather complicated integrals to determine the flux. These cases will still provide a correct answer if the situation is modelled correctly.

Suppose that I enclose a spherical charge with a cylindrical Gaussian surface, as shown in Figure~\ref{fig:gauss:dontchoosecylinder}. The electric field will be stronger near the middle of the cylinder's length than at the centre of its end caps, which means that $\vec E$ is not constant in $\int \vec E \cdot d \vec A$, so the integral cannot be simplified to $EA$. A better choice for a Gaussian surface in this case would be a sphere, which exploits the symmetry of the charge distribution and results in a $\vec E$ of constant magnitude everywhere along the surface. Figure~\ref{fig:gauss:fluxdA} and Figure~\ref{fig:gauss:fluxlinx} give other examples of when we cannot assume $\Phi$ to be equal to $EA$.

\begin{figure}[!htbp]
\centering
\includegraphics[width=0.2\linewidth]{files/dontchoosecylinder-ab9276b288b9db308f37157107ffbc73.png}
\caption[]{. A cylindrical surface is not a good choice to enclose a spherical charge.}
\label{fig:gauss:dontchoosecylinder}
\end{figure}
\end{framed}

\begin{framed}
\textbf{Example 16.6}\\
Determine the electric field above an infinitely large plane of charge with uniform surface charge per unit area, $\sigma$.

\begin{framed}
\textbf{Solution}\\
Figure~\ref{fig:gauss:plane} shows a portion of the infinite plane. The electric field vector must be perpendicular to the plane or a preferred direction could otherwise be inferred from the direction of the electric field. We can also argue that the horizontal components of the electric field will cancel everywhere above the plane, since the plane is infinite. The electric field will point away from the plane if the charge is positive, and towards the plane if the charge is negative.

\begin{figure}[!htbp]
\centering
\includegraphics[width=0.4\linewidth]{files/plane-d0a96e90ca3e3d639d338b307ecccab8.png}
\caption[]{The electric field above an infinite plane with uniform charge per unit area, $\sigma$, must be perpendicular to the plane.}
\label{fig:gauss:plane}
\end{figure}

A cylindrical or box-shaped Gaussian surface would both lead to the flux integral being easy to calculate, as illustrated in Figure~\ref{fig:gauss:fluxplane}. Indeed, since the electric field is perpendicular to the plane, only the parts of the surface that are parallel to the plane (the lids on the cylinder, the two horizontal planes in the box) will have a net flux through them.

\begin{figure}[!htbp]
\centering
\includegraphics[width=0.8\linewidth]{files/fluxplane-699ae1b062d45554f05fd11d04c847bd.png}
\caption[]{A cylindrical surface or a box are both good choices for a Gaussian surface above a plane, since only the parts of the surface parallel to the plane will have net flux through them.}
\label{fig:gauss:fluxplane}
\end{figure}

Let us choose a box (right panel of Figure~\ref{fig:gauss:fluxplane}) of length, $L$, with a square cross-section of side, $a$. We place the box such that the plane intersects the centre of the box (although this is not required, since we already know that the electric field will not depend on distance from the plane). The flux through the box is simply the flux through the two horizontal planes (of area $a^2$):
\begin{equation}
\oint \vec E\cdot d\vec A&= \int_{top} EdA+\int_{bottom}EdA=2Ea^2
\end{equation}
The box encloses a section of the plane with area $a^2$, so that the net charge enclosed by the surface is:
\begin{equation}
Q^{enc}=\sigma a^2
\end{equation}
Applying Gauss' law allows us to determine the magnitude of the electric field:
\begin{equation}
\oint \vec E\cdot d\vec A&=\frac{Q^{enc}}{\epsilon_0} \\
2Ea^2&= \frac{\sigma a^2}{\epsilon_0}\\
\therefore E&= \frac{\sigma}{2\epsilon_0}
\end{equation}
which is the same result that we found in Example~15.6.

\textbf{Discussion:} In this example, we used Gauss' law to determine the electric field above an infinite plane. We found that we had a choice of Gaussian surfaces (cylinder, box) that allowed us to apply Gauss' law. We found the same result that we had found in Example~15.6 where we had integrated Coulomb's Law (twice, once for a ring of charge, then for a disk, then took the limit of the disk radius going to infinity). Again, we see that in configurations with a high degree symmetry, Gauss' law can be very straightforward to apply.
\end{framed}
\end{framed}

\subsubsection{Charges in a conductor}\label{sec:gauss:conductors}

We can use Gauss' law to understand how charges arrange themselves on a conductor. Consider an infinite plane that carries a total charge per unit area, $\sigma$, similar to what we considered in Example~16.6. In this case, we explicitly consider the plane to be a conductor and to have a finite thickness. If we zoom into the plane, we can illustrate that the charges are located on the surface of the plane, as illustrated in Figure~\ref{fig:gauss:fluxconductingplane}, where the plane is seen edge on. Thus, the \textbf{charge density at the surface is half of the total charge density} of the plane.

\begin{figure}[!htbp]
\centering
\includegraphics[width=0.4\linewidth]{files/fluxconductingplane-1c0be40aaf0892d7a87ea45a940a18c3.png}
\caption[]{Cross-section of a conducting plane where the charges migrate to the surface. A box-shaped Gaussian surface is also shown as seen from the side (the third dimension of the box is perpendicular to the plane of the page).}
\label{fig:gauss:fluxconductingplane}
\end{figure}

To determine the electric field near the plane, we choose a Gaussian surface that is a box (as in Example~16.6), but require the lower end of the box to go through the plane, as illustrated in Example~16.6. With this choice of Gaussian surface, only the top surface (area $a^2$) will have flux through it, since the \textbf{electric field inside a conductor must be zero}\footnote{Since charges can freely move in a conductor, they will move until there is no reason to move. Eventually, the charges accumulate in such a way that the net field in the conductor is zero. For a plane, this means that half of the charges will move to each side, as illustrated.}. The total flux is given by:
\begin{equation}
\oint \vec E\cdot d\vec A&= \int_{top} EdA=Ea^2
\end{equation}
The charge enclosed is given by:
\begin{equation}
Q^{enc}=\frac{\sigma}{2}a^2
\end{equation}
where we used the fact that only half of the charges are inside the volume enclosed by our Gaussian surface, so that the charge per unit area is half ($\frac{\sigma}{2}$) of that for the entire plane. Applying Gauss' law, we find that the electric field is given by:
\begin{equation}
\oint \vec E\cdot d\vec A&=\frac{Q^{enc}}{\epsilon_0} \\
Ea^2&= \frac{\sigma a^2}{2\epsilon_0}\\
\therefore E&= \frac{\sigma}{2\epsilon_0}\quad \text{(Field above an infinite plane)}
\end{equation}
as in Example~16.6, but now the factor of two comes from having half of the charge density, whereas before it was because two of the faces of the box had non-zero flux. We can generalize this result to determine the electric field near the surface of any conductor. Very close to the surface of any object, one can consider the surface as being similar to an infinite plane. If that surface carries charge per unit area, $\sigma$, then the electric field just above the surface is given by:
\begin{equation}
E&= \frac{\sigma}{\epsilon_0} \quad \text{(Field near a conducting surface)}
\end{equation}
In this case, there is no factor of two because the charge density in this equation is the charge density of the conductor (not the charge density one side of the surface). In the previous equation, the charge density on the surface of the conducing plane was $\frac{\sigma}{2}$.

Consider, now, a neutral spherical conducting shell, as shown from the side in the left panel of Figure~\ref{fig:gauss:sphereshell}. When a charge, $+Q$, is placed at the centre of the shell (right panel), charges inside the shell will move until the field inside the conducting material of the shell is identically zero. The negative charges will move towards the inner surface (as they are attracted to $+Q$) and positive charges will be repelled onto the outer surface, under the influence of the electric field created by $+Q$ (shown in the diagram as $\vec E_{Q}$). Eventually, the separation of charges will lead to an electric field (shown in the diagram as $\vec E_{\sigma}$) in the opposite direction. The charges will stop moving once the total electric field in the conductor is zero (when the two fields cancel exactly everywhere in the conductor).

\begin{figure}[!htbp]
\centering
\includegraphics[width=0.7\linewidth]{files/sphereshell-9b7f9182300681b8ecbe7b3c10387400.png}
\caption[]{Left: a neutral conducting spherical shell (seen edge on). Right: A positive charge, $+Q$, placed at the centre of the shell. Charges in the shell will separate in order to keep the electric field inside the conductor zero.}
\label{fig:gauss:sphereshell}
\end{figure}

We can use Gauss' law to determine the amount of charge that has accumulated on the inner surface. Consider the Gaussian spherical surface, $S_1$, in Figure~\ref{fig:gauss:sphereshell}, that is concentric with the shell and has a radius such that the surface is just inside the shell. Since the electric field is zero inside the shell, the flux out of the Gaussian surface must be zero. By Gauss' law, the amount of charge enclosed by the surface must also be zero. Thus, a total charge, $-Q$, will have accumulated on the inner surface of the conductor (since $Q^{enc}= -Q+Q=0$). Because one cannot just create charge from nothing, there must be an equal amount of opposite charge, $+Q$, on the outer surface of the shell. This is true of any conducting material with a cavity inside of it: if you place a charge $+Q$ in the cavity, a charge $-Q$ will accumulated on the inner surface and a charge $+Q$ will accumulate on the outer surface.

Now consider the flux out of the surface $S_2$, which is outside of the shell. The net charge enclosed will be $Q^{enc}=+Q -Q+Q=+Q$. If we say that the radius of $S_2$ is $r$,  then the flux out of the spherical surface is given by:
\begin{equation}
\oint \vec E\cdot d\vec A &= E(4\pi r^2)\\
\end{equation}
and the electric field, from Gauss' law, is simply that of a point charge, $+Q$:
\begin{equation}
E&=\frac{1}{4\pi\epsilon_0}\frac{Q}{r^2}
\end{equation}
and the shell has no effect on the field. Right at the surface of the shell (outer radius $R$), the surface charge density is given by:
\begin{equation}
\sigma=\frac{Q}{4\pi r^2}
\end{equation}
Above, we found the electric field at the surface of a conductor that carries charge per unit area, $\sigma$, to be:
\begin{equation}
E&= \frac{\sigma}{\epsilon_0}
\end{equation}
which is clearly the same result that we obtained using the spherical surface, $S_2$:
\begin{equation}
E&= \frac{\sigma}{\epsilon_0}=\frac{1}{4\pi\epsilon_0}\frac{Q}{r^2}
\end{equation}
Note that we found the electric field using Gauss' law only in this last case, and found it to be equal to the electric field that one obtains from Coulomb's law. Thus, Gauss' law only works if the field has an ``inverse square law'' dependence. If Gauss' law does not provide the correct electric field, then the force does not depend on $1/r^2$. Gauss' law can be used to make extremely stringent tests of whether the force goes as $1/r^2$ or deviates from this model.

\subsubsection{Interpretation of Gauss' law and vector calculus}\label{sec:gauss:interpretation}

In this section, we provide a little more theoretical background and intuition on Gauss' law, as well as its connection to vector calculus (which is beyond the scope of this textbook, but interesting to have a feeling for). Very generally, Gauss' law is a statement that connects a property of a vector field to the ``source'' of that field. We think of mass as the source for the gravitational field, and we think of charge as the source for the electric field. The property of the field that we considered in this case was its ``flux out of a closed surface''.

Recall that determining the flux of a field out of a closed surface is equivalent to counting the net number of field lines that exit that closed surface. Field lines must start on a positive charge and must end on a negative charge. Thus, if there is a net number of field lines exiting the surface, there must be a positive charge in the volume defined by the surface (a ``source'' of field lines). If there is a net number of field lines entering the surface, then the volume defined by the surface must enclose a negative charge (a ``sink'' of field lines). Gauss' law is simply a statement that the number of field lines entering/exiting a closed surface is proportional to the amount of charge enclosed in that volume.

The flux out of a closed surface is tightly connected to the vector calculus concept of ``divergence'', which describes whether field lines are diverging (spreading out or getting closer together). When a point charge is present, field lines will emanate radially from that point charge; in other words, they will diverge. We say that the electric field has non-zero divergence if there is a source of the electric field in that position of space. The key difference between the concept of divergence and that of ``flux out of a closed surface'', is that divergence is a local property of the field (it is true at a point), whereas the flux out of a surface must be calculated using a finite volume and makes it challenging to define the field at a specific position. Gauss' law defined using flux is thus not as useful for describing how the field changes at specific positions, and is usually limited to situations with a high degree of symmetry.

The divergence, $\nabla \cdot \vec E$, of a vector field, $\vec E$, at some position is defined as:
\begin{equation}
\nabla \cdot \vec E=\frac{\partial E}{\partial x}+\frac{\partial E}{\partial y}+\frac{\partial E}{\partial z}
\end{equation}
and corresponds to the sum of three partial derivatives evaluated at that position in space. Gauss' theorem (also called the divergence theorem) states that:
\begin{equation}
\int_V \nabla \cdot \vec E = \oint_S \vec E \cdot d\vec A
\end{equation}
where the subscript on the integral indicates whether the sum (integral) should be carried out over a volume, $V$, or over a closed surface, $S$, as we have practised in this chapter. While it is not important at this level to understand the theorem in detail, the point is that one can convert a ``flux over a closed surface'' into an integral of the divergence of the field. In other words, we can convert a global property (flux) to a local property (divergence). Gauss' law in terms of divergence can be written as:
\begin{equation}
\boxed{\nabla \cdot \vec E = \frac{\rho}{\epsilon_0}}\quad \text{(Local version of Gauss' law)}
\end{equation}
where $\rho$ is the charge per unit volume at a specific position in space. This is the version of Gauss' law that is usually seen in advanced textbooks and in Maxwell's unified theory of electromagnetism. This version of Gauss' law relates a local property of the field (its divergence) to a local property of charge at that position in space (the charge per unit volume at that position in space). If we integrate both sides of the equation over volume, we recover the original formulation of Gauss' law: the left hand side, by the divergence theorem, leads to flux when integrated over volume, whereas on the right hand side, the integral over volume of charge per unit volume, $\rho$, will give the total charge enclosed in that volume, $Q^{enc}$:
\begin{equation}
\int_V  \left(\nabla \cdot \vec E \right)dV&= \int_V \left(\frac{\rho}{\epsilon_0}\right) dV\\
\oint_S \vec E \cdot d\vec A &=\frac{Q^{enc}}{\epsilon_0}
\end{equation}

\subsubsection{Summary}

We can define the \textbf{flux} of a uniform and constant vector field, $\vec E$, through a flat surface, as:
\begin{equation}
\Phi_E = \vec E \cdot \vec A = EA\cos\theta
\end{equation}
where $\vec A$ is a vector that is perpendicular to the surface with a magnitude equal to the area of that surface, and, $\theta$, is the angle between $\vec A$ and $\vec E$.

The flux of a field through a surface is proportional to the number of field lines that cross that surface. If the surface is parallel to the field ($\vec A$ and $\vec E$ are thus perpendicular), the flux through that surface is zero (no field lines cross the surface, the scalar product is zero). If $\vec E$ and $\vec A$ change over the surface ($\vec E$ and/or $\vec A$ change magnitude and/or direction relative to each other along the surface), then we treat the surface as being made of infinitesimal surface elements over which the two vectors are constant. We define a vector $d\vec A$ to be perpendicular to the surface element with an infinitesimal area, $dA$. The total flux is then obtained by summing the fluxes through each surface element:
\begin{equation}
\Phi_E=\int \vec E \cdot d\vec A=\int EdA\cos\theta
\end{equation}
Note that the direction of the vector $d\vec A$ (or $\vec A$) is ambiguous, as one can choose either of two directions perpendicular to a surface. Usually, one chooses the direction of $\vec A$ so that the flux is positive (i.e. $\vec A$ has a component parallel to $\vec E$). However, if the surface is ``closed'' (that is, it defines a volume), then we always choose the direction of $d\vec A$ so that it points outwards from the surface (since the surface encloses a volume, one can define an ``inside'' and an ``outside'').

In the case of the electric field, Gauss' law relates the flux of the electric field from a closed surface to the amount of charge, $Q^{enc}$, contained in the volume enclosed by that surface:
\begin{equation}
\oint \vec E \cdot d\vec A = \frac{Q^{enc}}{\epsilon_0}
\end{equation}
Physically, Gauss' law is a statement that field lines must begin or end on a charge (electric field lines originate on positive charges and terminate on negative charges). If there is a net number of lines coming out of a closed surface (a positive flux), that surface must enclose a positive charge from where those field lines originate. Similarly, if there are the same number of field lines entering a closed surface as there are lines exiting that surface (a flux of zero), then the surface encloses no charge. Gauss' law states that the number of field lines exiting a closed surface is proportional to the amount of charge enclosed by that surface.

Gauss' law is useful to determine the electric field. However, this can only be done analytically for charge distributions with a very high degree of symmetry. This is because the flux integral is not usually easy to evaluate unless:

\begin{enumerate}
\item \textbf{The electric field makes a constant angle with the surface}. When this is the case, the scalar product can be written in terms of the cosine of the angle between $\vec E$ and $d\vec A$, which can be taken out of the integral if it is constant:
\end{enumerate}
\begin{equation}
\oint \vec E\cdot d\vec A=\oint E\cos\theta dA=\cos\theta\oint EdA
\end{equation}
\begin{enumerate}[resume]
\item \textbf{The electric field is constant in magnitude along the surface}. When this is the case, the integral can be simplified further by factor out $E$, and simply becomes an integral over $dA$ (which corresponds to the total area of the surface, $A$):
\end{enumerate}
\begin{equation}
\oint \vec E\cdot d\vec A=\cos\theta\oint EdA =E\cos\theta\oint dA=EA\cos\theta
\end{equation}
Note that Gauss' law does not specify a closed surface over which to calculate the flux; it holds for any surface. We can thus choose a surface that will make the flux integral easy to evaluate - we call this choice a ``Gaussian surface'' (not because it has some special property, but because we chose that surface to apply Gauss' law). A procedure for applying Gauss' law to determine the electric field at some point in space can be written as:

\begin{enumerate}
\item Make a diagram showing the charge distribution.
\item Use symmetry arguments to determine in which way the electric field vector points.
\item Choose a Gaussian surface that goes through the point for which you want to know the electric field. Ideally, the surface is such that the electric field is constant in magnitude and always makes the same angle with the surface, so that the flux integral is straightforward to evaluate.
\item Calculate the flux, $\oint \vec E\cdot d\vec A$.
\item Calculate the amount of charge in the volume enclosed by the surface, $Q^{enc}$.
\item Apply Gauss' law, $\oint \vec E\cdot d\vec A=\frac{Q^{enc}}{\epsilon_0}$.
\end{enumerate}

We showed how Gauss' law can be used to understand and quantify how charges arrange themselves on a conductor, in such a way that the electric field is zero everywhere in the conductor. Finally, we briefly introduced a more modern version of Gauss' law that uses divergence instead of flux:
\begin{equation}
\nabla \cdot \vec E &= \frac{\rho}{\epsilon_0}
\end{equation}
This last version has the advantage that it relates a local property of the field (divergence) to a local property of charge (charge density at some position in space).

\begin{framed}
\textbf{Important Equations}\\
\textbf{Gauss' Law:}
\begin{equation}
\Phi & = \frac{Q_{enc}}{\epsilon_0}\\
\Phi &= \int \vec E \cdot d \vec A
\end{equation}
\end{framed}

\begin{framed}
\textbf{Important Definitions}\\
\begin{itemize}
\item \textbf{Electric flux:} A measure of the number of electric field lines crossing a surface. SI units: ${\rm \left[{Vm}\right]}$. Common variable(s): $\Phi_E$.
\end{itemize}
\end{framed}

\subsubsection{Thinking about the material}

\begin{framed}
\textbf{Reflect and research}\\
\begin{itemize}
\item Could Gauss' law be applied to magnetism? Why or why not?
\item What else has Gauss done?
\item Are there other interaction for which Gauss' law can be applied?
\item What are Maxwell's equations?
\item How are measurements of flux used in environmental research?
\item How does one use Gauss' law to test the $1/r^2$ dependence of Coulomb's Law?
\end{itemize}
\end{framed}

\begin{framed}
\textbf{To try in the lab}\\
\begin{itemize}
\item Propose an experiment to measure the charge of an object using Gauss' law.
\item Propose an experiment to measure the electric field of a charged object, then compare your experimental results to the theoretical results predicted calculated by Gauss' law.
\item Simulate the surface charge distribution on the inside and outside of a conducting cubic shell which encloses a point charge.
\end{itemize}
\end{framed}

\subsubsection{Sample problems and solutions}

\paragraph{Problems}

\begin{framed}
\textbf{Problem 16.1}\\
Consider a charged sphere of radius $R$, which has a non-uniform charge density that varies as $\rho(r) = ar^2$.

\begin{itemize}
\item a.  What is the total charge of the sphere?
\item b. What is the electric field as a function of distance from the centre of the sphere outside the sphere, $r>R$?
\item c. What is the electric field as a function of distance from the centre of the sphere inside the sphere, $r\leq R$?
\end{itemize}
\end{framed}

\begin{framed}
\textbf{Problem 16.2}\\
A long insulating wire with charge per unit length $+\lambda$ is surrounded by a coaxial conducting shell with charge per unit length $-2\lambda$. You can assume that both the wire and the shell are infinitely long. The inner surface of the shell has a circular cross section of radius $R$. The cross section of the outer surface is an equilateral triangle with side length $L$.

\begin{itemize}
\item a. What is the average surface charge density, $\sigma_{in}$, on the inner surface of the conducting shell?
\item b. What is the average surface charge density, $\sigma_{out}$, on the outer surface of the conducting shell?
\end{itemize}

\begin{figure}[!htbp]
\centering
\includegraphics[width=0.8\linewidth]{files/coaxial_triangle-ae2491e601ad23ad3a7ca4bfe406a2d7.png}
\caption[]{Left: Wire surrounded by a conducting shell with a cylindrical inner surface and triangular outer surface. Right: Cross section of the set-up.}
\label{fig:gauss:coaxial_triangle}
\end{figure}
\end{framed}

\paragraph{Solutions}

\begin{framed}
\textbf{Solution 16.1}\\
\begin{itemize}
\item a. In order to determine the total charge of the sphere, we divide the sphere into shells of radius $r$ and infinitesimal thickness $dr$. The volume, $dV$, of a shell is given by its surface area multiplied by its thickness:
\end{itemize}
\begin{equation}
dV = 4\pi r^2 dr
\end{equation}
We can obtain the charge, $dQ$, of each shell by using the charge per unit volume, $\rho(r)$:
\begin{equation}
dQ = \rho(r) dV = ar^24\pi r^2 dr = 4a\pi r^4 dr
\end{equation}
The total charge of the sphere is found by summing the charges from each shell over the radius of the sphere:
\begin{equation}
Q=\int dQ =\int_0^R4a\pi r^4 dr=\frac{4}{5}a\pi R^5
\end{equation}
{\textbackslash}item Outside of the sphere, we can use a spherical Gaussian surface of radius $r$, so that the flux is given by:
\begin{equation}
\oint \vec E\cdot d\vec A=E\cdot A=4\pi r^2 E
\end{equation}
The entire charge of the sphere is enclosed. Applying Gauss' law, we can determine the electric field outside the sphere:
\begin{equation}
\oint \vec E\cdot d\vec A&= \frac{Q^{enc}}{\epsilon_0}\\
4\pi r^2 E&= \frac{4a\pi R^5}{5\epsilon_0}\\
\therefore E(r)&=\frac{aR^5}{5\epsilon_0r^2}
\end{equation}
and we see that the electric field decreases as the radius squared. When we are outside of the sphere, it behaves the same way as a point charge with $Q=(4/5)a\pi R^5$.

\begin{itemize}
\item b. Inside the sphere, we still use a Gaussian spherical surface of radius $r$, so that the flux is given by:
\end{itemize}
\begin{equation}
\oint \vec E\cdot d\vec A=4\pi r^2 E
\end{equation}
However, inside the sphere, the Gaussian surface only encloses the charge up to a radius of $r$. Similarly to part (a), we can find the enclosed charge by integration:
\begin{equation}
Q^{enc}=\int dQ =\int_0^r 4a\pi r^4 dr=\frac{4}{5}a\pi r^5
\end{equation}
Applying Gauss' law:
\begin{equation}
\oint \vec E\cdot d\vec A&= \frac{Q^{enc}}{\epsilon_0}\\
4\pi r^2 E&= \frac{4a\pi r^5}{5\epsilon_0}\\
\therefore E(r)&=\frac{ar^3}{5\epsilon_0}
\end{equation}
and we find that the electric field is zero at the centre of the sphere and increases with $r^3$ inside the sphere.
\end{framed}

\begin{framed}
\textbf{Solution 16.2}\\
\begin{itemize}
\item a. To find the surface charge density on the inner surface, we draw a cylindrical Gaussian surface, $S_1$, whose cross-sectional radius is slightly larger than $R$, as in Figure~\ref{fig:gauss:coaxial_triangle_side}. We know that the electric field inside of the (conducting) shell is zero, so that the flux out of $S_1$ will be zero. By Gauss' law, we find:
\end{itemize}
\begin{equation}
\oint \vec E \cdot d \vec A &=\frac{Q_{enc}}{\epsilon_0}\\
0 &=\frac{Q_{enc}}{\epsilon_0}\\
\therefore Q_{enc}&=0
\end{equation}
Since $Q_{enc}=0$ and the wire has a constant charge per unit length $\lambda$, the charge per unit length on the inner surface of the shell must be $-\lambda$, so that the net charge charge per unit length is zero. The surface charge density, $\sigma_{in}$, is the linear charge density divided by the circumference of the cross section:
\begin{equation}
\sigma_{in}&=\frac{\lambda}{2\pi R}
\end{equation}

\begin{itemize}
\item b. For the outer surface, we consider a Gaussian surface with a triangular cross section, $S_2$. Note that the shape of the cross section is not important in this case (i.e. we could have used a cylindrical Gaussian surface), since we are only concerned with the enclosed charge and not the magnitude of the electric field.
\end{itemize}

The charge per unit length of the wire is $\lambda$ and the charge per unit length of the conducting shell was given as $-2\lambda$. The enclosed charge per unit length is thus
\begin{equation}
\lambda_{enc}=+\lambda-2\lambda=-\lambda
\end{equation}
We know from part (a) that the net charge per unit length of the wire and inner surface is 0, so the charge per unit length of the outer surface must be $-\lambda$. The average surface charge density, $\sigma_{out}$, is the linear charge density divided by the length of the three sides:
\begin{equation}
\sigma_{out}=-\frac{\lambda}{3L}
\end{equation}

\begin{figure}[!htbp]
\centering
\includegraphics[width=0.4\linewidth]{files/coaxial_triangle_sid-52bfd62b5da7465dab59e3bae2b2561e.png}
\caption[]{Cross section of the wire/conducting shell set up, showing the Gaussian surfaces.}
\label{fig:gauss:coaxial_triangle_side}
\end{figure}
\end{framed}

\include{ModelingWithPhysics-electricpotential}

\include{ModelingWithPhysics-current}

\subsection{Chapter 19 - Electric circuits}

\subsubsection{Overview}\label{chapter:circuits}

In this chapter, we develop the tools to model electric circuits. This will allow us to determine the current and voltages across different components, such as resistors and capacitors, within a circuit. We will also discuss how a battery can provide a current at a fixed potential difference, and how one can construct devices to measure current and voltages.

\begin{framed}
\textbf{Learning Objectives}\\
\begin{itemize}
\item Understand how a battery works.
\item Understand Kirchhoff's rules and how to apply them.
\item Understand how to model a circuit with resistors and/or capacitors.
\item Understand how an ammeter and voltmeter function, and how to model them.
\end{itemize}
\end{framed}

\begin{framed}
\textbf{Think About It}\\
If two outlets in your house are connected to the same circuit, are the outlets connected in series or in parallel?

\begin{enumerate}
\item series
\item parallel
\end{enumerate}

\begin{framed}
\textbf{Answer}\\
\begin{enumerate}[resume]
\item
\end{enumerate}
\end{framed}
\end{framed}

\subsubsection{Batteries and simple circuits}

:label: sec:circuits:batteries
A battery is an electric component that provides a constant electric potential difference (a fixed voltage) across its terminals. Luigi Galvani was the first to realize that certain combinations of metals placed into contact with each other can lead to an electric potential difference (or rather, he realized that they can make the legs of a dead frog twitch, which we now understand to be from the potential difference due to the metals). Effectively, Galvani created the first ``electrochemical cell''. Alessandro Volta then combined several of these cells together to form the ``voltaic pile'', which is what we would now call a battery (a battery is technically a combination of several cells, or a ``battery of cells'', although one often uses the term battery even if only a single electric cell is involved).

\paragraph{The electrochemical cell}

An electric cell can be constructed from metals that have different affinities to be dissolved in acid. A simple cell, similar to that originally made by Volta, is comprised of two metal electrodes placed in a liquid called electrolyte, as illustrated in Figure~\ref{fig:circuits:electriccell}. Common materials used are carbon (Volta used silver) and zinc for the electrodes, and sulfuric acid for the electrolyte. Before the cell is constructed, the electrodes and the electrolyte are all electrically neutral.

\begin{figure}[!htbp]
\centering
\includegraphics[width=0.7\linewidth]{files/electriccell-b624e62206c62803c9f633d00bae9bb0.png}
\caption[]{A simple electric cell, where zinc ions dissolve in sulfuric acid leaving electrons on the metal.}
\label{fig:circuits:electriccell}
\end{figure}

% TODO Is it incorrect to show the electrons entering the solution?

Once the zinc is immersed in the electrolyte, the zinc atoms tend to dissolve into the electrolyte in the form of zinc ions (doubly charged, Zn\textsuperscript{2+}). This leaves an excess of electrons on the zinc electrode, resulting in a net negative electric charge. Similarly, the positively charged zinc ions attract electrons from the carbon electrode into the solution, leaving the carbon electrode positively charged. Very quickly, an equilibrium is reached, since at some point, the negative charge of the zinc electrode will electrically attract positive zinc ions, preventing any more zinc ions from dissolving into the solution. Similarly, as the carbon electrode builds a positive charge, that charge will eventually prevent electrons from ``jumping'' into the solution. At this point, there will be a fixed electric potential difference between the two electrodes (terminals) of the battery.

If the two electrodes are connected together through a resistor, the electrons will leave the zinc electrode, cross the resistor, and end up on the positive carbon electrode. This will leave space for more electrons on the zinc electrode, so more zinc ions will dissolve into the solution. Thus, a circuit is formed, where electron travel up the zinc electrode, through the resistor and back down the carbon electrode. At the same time, more and more zinc ions dissolve into the electrolyte, until the zinc electrode is completely dissolved. In practice, the zinc ions travel through the solution and plate onto the carbon electrode (the electrons do not quite ``jump'' into the electrolyte, rather, it is the zinc ions that move in the electrolyte). Since the charge on the electrodes is continuously replenished, the potential difference between the electrodes remains constant even as current is flowing.

The electric cell will stop working once the zinc electrode has completely dissolved (this is what happens when your battery is dead). Note that there is also a maximum current that the cell can supply, which depends on the rate at which the zinc can dissolve into the electrolyte and plate onto the carbon electrode. If the electrodes of the cell are connected with a very low resistance resistor, the resulting current will be too large for the potential difference to be maintained. Most electric cells work in similar ways, although the chemical reactions can be much more complex. Sometimes, the chemical reaction is reversible; one could use a different battery to apply a negative voltage to the carbon electrode to reverse the reaction and plate the zinc back onto the zinc electrode, thus ``recharging the battery'' (and converting electric energy back into stored chemical potential energy).

\paragraph{The ideal battery in a circuit}

As we proceed, we will use the term ``battery'' loosely to refer to a device (such as an electric cell or collection of cells) that can provide a fixed potential difference between two terminals (or electrodes). Figure~\ref{fig:circuits:batterysymbol} shows the circuit diagram for a battery, consisting of two (or four) vertical bars, with the larger bar indicating the positive terminal of the battery.

\begin{figure}[!htbp]
\centering
\includegraphics[width=0.6\linewidth]{files/batterysymbol-274afa3da108a9466ac4272b2cd962c5.png}
\caption[]{Circuit diagram symbols that can be used for a battery.}
\label{fig:circuits:batterysymbol}
\end{figure}

Figure~\ref{fig:circuits:resistorsymbol} shows the circuit diagram symbols that are used for a resistor (different symbols are used in North American and in Europe).

\begin{figure}[!htbp]
\centering
\includegraphics[width=0.6\linewidth]{files/resistorsymbol-cdfd7783ebc9bea4d3ca2eec6411d202.png}
\caption[]{Circuit diagram symbols for a resistor, using the North American convention (left), and the European convention (right).}
\label{fig:circuits:resistorsymbol}
\end{figure}

Figure~\ref{fig:circuits:batteryresistor} shows a circuit diagram for a very simple circuit consisting of a single $9 {\rm V}$ battery connected to a $2 {\rm \Omega}$ resistor. When drawing a circuit diagram (or making a real circuit), one connects the various components together (e.g. batteries and resistors) with \textbf{segments of wire that have zero resistance}, even if, in practice, wires always have some resistance. However, since the wires are connected in series with resistors (or other components that have a resistance), one can always include the resistance of the wires by adding it to the resistance of the other components. For example, in Figure~\ref{fig:circuits:batteryresistor}, if the wires have a total resistance of $1 {\rm \Omega}$, we could simply model the circuit as if the resistor had a resistance of $3 {\rm \Omega}$ instead of $2 {\rm \Omega}$. In practice, this is usually accounted for when a circuit diagram is made (i.e. any resistors include the resistance of the wires connected to it).

\begin{figure}[!htbp]
\centering
\includegraphics[width=0.35\linewidth]{files/batteryresistor-2b2416149cc3b585879ef82a83e4ac22.png}
\caption[]{A simple circuit, showing a $9 {\rm V}$ battery and a $2 {\rm \Omega}$ resistor. For ease in analyzing circuits, we suggest drawing a ``battery arrow" above batteries that goes from the negative to the positive terminal.}
\label{fig:circuits:batteryresistor}
\end{figure}

The circuit in Figure~\ref{fig:circuits:batteryresistor} is simple to analyze. In this case, whichever charges exit one terminal of the battery must pass through the resistor and then enter the other terminal of the battery. We \textbf{always use conventional current} to analyze a circuit. Thus, we model the circuit as if positive charges exit the positive terminal of the battery, go through the resistor, and then enter the negative terminal of the battery.

We recommend that you always draw a ``battery arrow'' for each battery in a circuit diagram to indicate the direction in which the electric potential increases and in which direction the conventional current would exit the battery if a simple resistor were connected across the battery. We also indicate the current that is flowing in any wire of the circuit by drawing an arrow in the direction of current on that wire (labelled $I$ in Figure~\ref{fig:circuits:batteryresistor}).

% In complex circuits, the current may not necessarily flow in the same direction as the battery arrow, and the battery arrow makes it easier to analyze those circuits.

It is helpful to think about the value of the electric potential along different parts of a circuit. Below (Figure~\ref{fig:circuits:batteryresistor_colour}), we have taken the circuit from Figure~\ref{fig:circuits:batteryresistor} and highlighted regions where the electric potential is constant.

\begin{figure}[!htbp]
\centering
\includegraphics[width=0.4\linewidth]{files/batteryresistor_colo-947aa7e3376cf4c2c40889a8b3e04443.png}
\caption[]{The same circuit as in Figure~\ref{fig:circuits:batteryresistor} showing the two regions over which the electric potential is constant.}
\label{fig:circuits:batteryresistor_colour}
\end{figure}

Since the wires have no resistance, the electric potential is constant along a wire. In other words, because the wire has no resistance, the charges/current cannot dissipate any power in the wire ($P=I^2R$), and the charges do not ``lose'' any potential energy (and so the potential cannot change). The only place where the charges can dissipate energy is inside the resistor. Once the charges have crossed the resistor, the electric potential in the wire is again constant until they reach the other terminal of the battery. Thus, in this simple circuit, the electric potential difference across the resistor is the same as the potential difference across the terminals of the battery.  This is shown by the coloured areas in Figure~\ref{fig:circuits:batteryresistor_colour}. If we choose $0 {\rm V}$ to be defined at the negative terminal of the battery, then the potential is $9 {\rm V}$ everywhere in the red area (to the right of the resistor), and $0 {\rm V}$ everywhere in the grey area (to the left of the resistor).

We can apply Ohm's Law (the macroscopic version) to the resistor and determine the current in the circuit, since we know the potential difference across the resistor:
\begin{equation}
\Delta V&=RI\\
\therefore I&=\frac{\Delta V}{R}=\frac{(9 {\rm V})}{(2 {\rm \Omega})}=4.5 {\rm A}
\end{equation}

It is helpful to think of circuits in terms of energy. Charges move along the circuit and their potential energy changes as they go through components, while it remains constant as they move through a wire. If a positive charge enters the negative terminal of a battery and exits the positive terminal, its potential energy will have increased. If that charge then enters a resistor, its potential energy will decrease as it moves through the resistor, since the charge will ``use'' its potential energy to heat up the resistor. Batteries provide the energy to ``push'' the charges through the resistors in the circuit by converting chemical potential energy into the electrical potential energy of the charges.

It is also useful to make the analogy with fluid dynamics; one can think of the battery as a pump that is continuously pushing a viscous incompressible fluid through a pipe with a narrow section, as illustrated in Figure~\ref{fig:circuits:watercircuit}. The wide section of the pipe is akin to the wires with no resistance, and the narrow section is akin to the resistor. The pressure difference generated by the pump is analogous to the voltage produced by the battery, and the flow rate of the liquid is analogous to the electric current. The pressure in the pipe does not drop in the wide section, if there is no resistance. The entire pressure drop of the fluid is across the narrow section, just as the voltage only drops across the resistor.

\begin{figure}[!htbp]
\centering
\includegraphics[width=0.5\linewidth]{files/watercircuit-c2d85239697efc723170b761c8b1073a.png}
\caption[]{A fluid dynamics analogue of the circuit in Figure~\ref{fig:circuits:batteryresistor}, where a pump plays the role of the battery, and a narrow pipe that of a resistor.}
\label{fig:circuits:watercircuit}
\end{figure}

\begin{framed}
\textbf{Example 19.1}\\
Two resistors, of $2 {\rm \Omega}$ and $4 {\rm \Omega}$, respectively, are connected in series to a $12 {\rm V}$ battery. What is the current through each of the resistors, and what is the voltage across each resistor?

\begin{framed}
\textbf{Solution}\\
We start by making a circuit diagram, as in Figure~\ref{fig:circuits:tworesistors}, showing the resistors, the current, $I$, the battery and the battery arrow. Note that since this is a closed circuit with only one path, the current through the battery, $I$, is the same as the current through the two resistors.

\begin{figure}[!htbp]
\centering
\includegraphics[width=0.42\linewidth]{files/tworesistors-e49d8c6bd8a4e07e755b86fd2da0ea23.png}
\caption[]{Two resistors connected in series with a battery.}
\label{fig:circuits:tworesistors}
\end{figure}

If we choose the potential on the negative side of the battery to be $0 {\rm V}$, then points ${\rm a}$ and ${\rm e}$ on the diagram are at a potential of $0 {\rm V}$, since potential cannot change in a wire with no resistance. Similarly, the points at ${\rm b}$ and ${\rm c}$ are at a potential of $12 {\rm V}$ (relative to points ${\rm a}$ and ${\rm e}$). At point ${\rm d}$, between the two resistors, the potential will be between $0 {\rm V}$ and $12 {\rm V}$, since the potential will ``drop'' as the current goes through the $2 {\rm \Omega}$ resistor.

The easiest way to determine the current through this simple circuit is to combine the two resistors into a single effective resistor with resistance:
\begin{equation}
R_{eff}=(2 {\rm \Omega})+(4 {\rm \Omega})=6 {\rm \Omega}
\end{equation}
so that the circuit can be simplified to that shown in Figure~\ref{fig:circuits:batteryresistor2}:

\begin{figure}[!htbp]
\centering
\includegraphics[width=0.42\linewidth]{files/batteryresistor2-ec040ff635391cdcdeeaef863ed15370.png}
\caption[]{The resistors from the circuit in Figure~\ref{fig:circuits:tworesistors} have been combined in series to simplify the circuit.}
\label{fig:circuits:batteryresistor2}
\end{figure}

The potential difference across the effective resistor is the same as that across the battery (between points ${\rm e}$ and ${\rm c}$), so that Ohm's Law can be applied to the effective resistor to determine the current that traverses it:
\begin{equation}
\Delta V &= R_{eff}I\\
\therefore I&=\frac{\Delta V}{R_{eff}}=\frac{(12 {\rm V})}{(6 {\rm \Omega})}=2 {\rm A}
\end{equation}
This current is the same that traverses each individual resistor, since it is the same as the current that goes through the battery. Referring back to the full circuit (Figure~\ref{fig:circuits:tworesistors}), we can now use Ohm's Law to calculate the voltage drop across each resistor, since we know the current through each resistor. The voltage across the $2 {\rm \Omega}$ resistor is given by:
\begin{equation}
\Delta V_{2\Omega}=RI=(2 {\rm \Omega})(2 {\rm A})=4 {\rm V}
\end{equation}
and the voltage across the $4 {\rm \Omega}$ resistor is given by:
\begin{equation}
\Delta V_{4\Omega}=RI=(4 {\rm \Omega})(2 {\rm A})=8 {\rm V}
\end{equation}
Note that the sum of these two voltages is equal to the voltage increase across the battery, by conservation of energy. Consider the electric potential at different points in Figure~\ref{fig:circuits:tworesistors} as you move clockwise around the loop starting at point ${\rm a}$. If the electric potential is defined to be $0 {\rm V}$ at the negative end of the battery (points ${\rm a}$ and ${\rm e}$), the potential at point ${\rm d}$ (between the resistors) is the potential at point ${\rm e}$ plus the potential difference across the $4 {\rm \Omega}$ resistor:
\begin{equation}
V_d = V_e+\Delta V_{4\Omega}=(0 {\rm V})+(\Delta V_{4\Omega})=8 {\rm V}
\end{equation}
If we then add the potential difference across the $2 {\rm \Omega}$ resistor to the potential at point ${\rm d}$, we find  that the potential at point ${\rm c}$ is $V_c=V_d+\Delta V_{2\Omega}=12 {\rm V}$, as expected, since this corresponds to the potential at the positive terminal of the battery.

\textbf{Discussion:} In this example, we showed how one can model a circuit by combining resistors together into effective resistors to simplify the circuit. We also showed how the potential differences across different components in a circuit must add up to zero (the voltage drops across the resistors must sum to the voltage increase across the battery).
\end{framed}
\end{framed}

\begin{framed}
\textbf{Checkpoint}\\
What is the voltage across the combination of a $3 {\rm V}$ battery connected in series with a $6 {\rm V}$ battery, where the negative terminal of the $6 {\rm V}$ battery faces the positive terminal of the $3 {\rm V}$ battery?

\begin{enumerate}
\item $9 {\rm V}$.
\item $6 {\rm V}$.
\item $3 {\rm V}$.
\item $0 {\rm V}$.
\end{enumerate}

\begin{framed}
\textbf{Answer}\\
\begin{enumerate}
\item
\end{enumerate}
\end{framed}
\end{framed}

\paragraph{The real battery in a circuit}

So far, we have modelled batteries as ``ideal'' devices that provide a fixed potential difference. In reality, this neglects the fact that the materials that make the battery will themselves have a resistance. For example, if electrons want to leave the zinc rod in the electric cell illustrated in Figure~\ref{fig:circuits:electriccell}, they will lose some energy as they pass through the zinc. Thus, when modelling a real battery in a circuit, it is important to include its ``internal resistance'', as a resistor in series with the potential difference. This is illustrated in Figure~\ref{fig:circuits:realbattery}, which shows the two terminals of a real battery, an ideal battery (with a fixed potential difference, $\Delta V_{ideal}$), and its internal resistance, $r$ (which can be drawn on either side of the battery).

\begin{figure}[!htbp]
\centering
\includegraphics[width=0.8\linewidth]{files/realbattery-25df902f5102763e8da286f2d44bddb1.png}
\caption[]{Model of a real battery, showing an ideal battery in series with a resistor to model the internal resistance of the battery.}
\label{fig:circuits:realbattery}
\end{figure}

It is important to note that the potential difference across the terminals of the real battery is only equal to the potential difference across the ideal battery \textbf{if there is no current flowing through the battery}. If there is a current, $I$, flowing through the internal resistance, the electric potential will decrease by an amount $Ir$ across the internal resistance, and the voltage across the real terminals will be $\Delta V_{ideal} -Ir$.

\begin{framed}
\textbf{Example 19.2}\\
When no resistance is connected across a real battery, the potential difference across its terminals is measured to be $6 {\rm V}$. When an $R=2 {\rm \Omega}$ resistor is connected across the battery, a current of $2 {\rm A}$ is measured through the resistor. What is the internal resistance, $r$, of the battery, and what is the voltage across its terminals when the $R=2 {\rm \Omega}$ resistor is connected?

\begin{framed}
\textbf{Solution}\\
The real battery can be modelled as an ideal battery with potential difference, $\Delta V_{ideal}$, in series with an internal resistance, $r$. While we do not know the value of the internal resistance, we are told that the potential difference across the terminals of the real battery is $6 {\rm V}$ \textbf{when no current flows through it}. Since no current flows through the internal resistance, the voltage does not drop across the internal resistance, and the voltage across the terminals of the real battery (e.g. Figure~\ref{fig:circuits:realbattery}) must thus be equal to the voltage across the terminals of the ideal battery, so that $\Delta V_{ideal}=6 {\rm V}$.

With this information, we can make a circuit diagram for the case when the $2 {\rm \Omega}$ resistor is connected across the terminals of the real battery, as in Figure~\ref{fig:circuits:realbatterycircuit}.

\begin{figure}[!htbp]
\centering
\includegraphics[width=0.42\linewidth]{files/realbatterycircuit-c0da95c1617d5d17aee894d81382d86a.png}
\caption[]{A circuit showing a real battery (with internal resistance $r$) in series with a resistor.}
\label{fig:circuits:realbatterycircuit}
\end{figure}

The terminals of the real battery are located at points ${\rm a}$ and ${\rm c}$ of the diagram, whereas the terminals of the ideal battery correspond to points ${\rm a}$ and ${\rm b}$. When no current flows through the internal resistor, $r$, there is no voltage drop across that resistor and the potential at ${\rm b}$ will be equal to the potential at ${\rm c}$, as we argued above.

The circuit in Figure~\ref{fig:circuits:realbatterycircuit} is now identical to that analyzed in Example~19.1, and can be treated the same way. We can combine the $2 {\rm \Omega}$ resistor with the internal resistance, $r$, in series to obtain an effective resistor, $R_{eff}=r+R$. The voltage drop across the effective resistor will be the same as the potential difference across the ideal battery, and we can make use of Ohm's Law to find the internal resistance, $r$:
\begin{equation}
\Delta V_{ideal}&= R_{eff}I=(r+R)I\\
\therefore r &= \frac{\Delta V_{ideal}}{I}-R=\frac{(6 {\rm V})}{(2 {\rm A})}-(2 {\rm \Omega})=1 {\rm \Omega}
\end{equation}
Now that we know the internal resistance, we can determine the voltage drop across the internal resistor, using Ohm's Law:
\begin{equation}
\Delta V_r = rI=(1 {\rm \Omega})(2 {\rm A})=2 {\rm V}
\end{equation}
The voltage drop across the real terminals of the battery (between points ${\rm a}$ and ${\rm c}$), is thus given by:
\begin{equation}
\Delta V_{real}=\Delta V_{ideal}-\Delta V_r=(6 {\rm V})-(2 {\rm V})=4 {\rm V}
\end{equation}
Again, you can verify that the voltage drops across the two resistors will sum to the total voltage drop across the terminals of the ideal battery.

\textbf{Discussion:} Modelling real batteries is not so different from modelling ideal batteries, since one only needs to include an internal resistance into the circuit. The key difference with a real battery is that the voltage across its real terminals depends on what is connected to the battery. In the example above, the battery has a voltage of $6 {\rm V}$ across its (real) terminals when nothing is connected, but the voltage drops to $4 {\rm V}$ when a $2 {\rm \Omega}$ resistor is connected.
\end{framed}
\end{framed}

\begin{framed}
\textbf{Checkpoint}\\
Suppose that you would like to measure the ideal voltage of a real battery by connecting a measurement device (a voltmeter) across its terminals. In order to get the most accurate reading, should you choose a voltmeter with a high resistance, or a voltmeter with a low resistance?

\begin{enumerate}
\item High resistance.
\item Low resistance.
\item It doesn't matter if the voltmeter has a high or low resistance.
\end{enumerate}

\begin{framed}
\textbf{Answer}\\
\begin{enumerate}
\item
\end{enumerate}
\end{framed}
\end{framed}

\subsubsection{Kirchhoff's rules}

Kirchhoff's rules correspond to concepts that we have already covered, but allow us to easily model more complex circuits, for instance, those where there is more than one path for the current to take. Kirchhoff's rules refer to ``junctions'' and ``loops''. Junctions and loops depend only on the shape of the circuit, and not on the components in the circuit. Figure~\ref{fig:circuits:3loopempty} shows a circuit with no components in order to illustrate what is meant by a junction and a loop.

\begin{figure}[!htbp]
\centering
\includegraphics[width=0.4\linewidth]{files/3loopempty-0e9cf961ed755afb96b62a98e97680bd.png}
\caption[]{A circuit that has 3 loops and 2 junctions.}
\label{fig:circuits:3loopempty}
\end{figure}

The locations at points ${\rm d}$ and ${\rm c}$ are considered ``junctions'', because there are more than 2 segments of wire connected to that point. The points at locations ${\rm a}$, ${\rm b}$, ${\rm e}$ and ${\rm f}$ only have two segments of wire connected to them. The circuit in Figure~\ref{fig:circuits:3loopempty} thus has 2 junctions.

A loop is a closed path that one can trace around the circuit without passing over the same segment of wire twice. The circuit in Figure~\ref{fig:circuits:3loopempty} has 3 such loops, which we can identify using the letters at the various nodes of the circuit:

\begin{enumerate}
\item ${\rm abcda}$
\item ${\rm abcefda}$
\item ${\rm dcefd}$
\end{enumerate}

Note that it does not matter where one starts on the loop, only that one can identify how many different loops are present in the circuit.

% %%The loops are shown more explicitly in [](#fig:circuits:looplabel).%%%
% %%```{figure} figures/Circuits/circuitspng/looplabel.png
% %%:label: fig:circuits:looplabel
% %%:width: 40%
% %%:align: center
% %%:alt: The three loops in the circuit.
% %%The three loops in the circuit.
% %%```

\begin{framed}
\textbf{Checkpoint}\\
\begin{figure}[!htbp]
\centering
\includegraphics[width=0.5\linewidth]{files/7loop-7d21110540d72f12bf874c92522b5941.png}
\caption[]{Circuit layout}
\label{fig:circuits:7loop}
\end{figure}

How many loops and junctions does the circuit in Figure~\ref{fig:circuits:7loop} have?

\begin{enumerate}
\item The circuit has five loops and four junctions
\item The circuit has three loops and eight junctions
\item The circuit has seven loops and four junctions.
\item The circuit has four loops and four junctions.
\end{enumerate}

\begin{framed}
\textbf{Answer}\\
\begin{enumerate}[resume]
\item
\end{enumerate}
\end{framed}
\end{framed}

\paragraph{Junction rule}

The junction rule states that: \textbf{The current entering a junction must be equal to the current exiting a junction.}

This is in fact a simple statement about conservation of charge. If charges are flowing into a junction (from one or more segments of wire in that junction), then the same amount of charges must flow back out of the junction (through one or more different segments of wire).

Consider the junction illustrated in Figure~\ref{fig:circuits:junction}, comprised of 5 segments of wire, each carrying a different current. As shown, currents $I_1$, $I_3$, and $I_4$ flow into the junction, whereas currents $I_2$ and $I_5$ flow out of the junction.

\begin{figure}[!htbp]
\centering
\includegraphics[width=0.4\linewidth]{files/junction-4828061de9aa07ebe924a9a7d3495dbf.png}
\caption[]{A junction with 5 segments and 5 currents.}
\label{fig:circuits:junction}
\end{figure}

The junction rule states that the current entering the junction must equal the current coming out of the junction. This allows us to relate the currents to each other in an equation:
\begin{equation}
\text{incoming currents}&=\text{outgoing currents}\\
I_1+I_3+I_4 &=I_2+I_5
\end{equation}

\paragraph{Loop rule}

The loop rule states that: \textbf{The net voltage drop across a loop must be zero.}

This is a statement about conservation of energy, that we already noted in Example~19.1. Once you have identified a specific loop, if you trace a closed path around the loop, the electric potential must be the same at the end of the path as at the beginning of the path (since it is literally the same point in space). This means that if there is a voltage drop along the path (e.g. due to one or more resistors), then there must be equivalent voltage increases somewhere else on the path (e.g. due to one or more batteries). If this were not the case, it would be possible to have a path where charges could gain a net amount of energy by going around that path, which they could keep doing indefinitely and create an infinite amount of energy; instead, if charges gain potential energy in a battery, they must then lose exactly the same amount of energy inside one or more resistors along the path.

Figure~\ref{fig:circuits:loop} shows a loop (which could be part of a larger circuit) to which we can apply the loop rule. The loop contains two batteries, facing in opposite directions (which would not normally be a good use of batteries), as illustrated by the battery arrows.

\begin{figure}[!htbp]
\centering
\includegraphics[width=0.55\linewidth]{files/loop-ad77847ada097ce589459094edc79d1b.png}
\caption[]{A loop with 2 batteries and 3 resistors.}
\label{fig:circuits:loop}
\end{figure}

The procedure for applying the loop rule is as follows:

\begin{enumerate}
\item Identify the loop, including starting position and direction.
\item Start at the beginning of the loop, and trace around the loop.
\item Each time a battery is encountered, \textbf{add the battery voltage if you are tracing the loop in the same direction as the corresponding battery arrow}, subtract the voltage otherwise.
\item Each time a resistor is encountered, \textbf{subtract the voltage across that resistor ($RI$, from Ohm's Law) if tracing the loop in the same direction as the current}, add the the voltage otherwise.
\item Once you have traced back to the starting point, the resulting sum must be zero.
\end{enumerate}

To illustrate the procedure, we trace out the loop ${\rm abcedfga}$ in Figure~\ref{fig:circuits:loop}. We thus start at point ${\rm a}$ and trace the loop in the counter-clockwise direction.

\begin{itemize}
\item Between points ${\rm a}$ and ${\rm b}$ we encounter a battery, and we are tracing in the \textbf{opposite direction of that battery's arrow}, so we subtract the voltage from that battery: $-\Delta V_1$.
\item Between points ${\rm b}$ and ${\rm c}$, we encounter a battery, and we are tracing in the \textbf{same direction as that battery's arrow}, so we add the voltage from that battery: $+\Delta V_2$.
\item Nothing happens to the potential along the wire from ${\rm c}$ to ${\rm d}$.
\item Between points ${\rm d}$ and ${\rm e}$, we encounter a resistor, and we are tracing in the \textbf{same direction as the current through that resistor}, so subtract the voltage across that resistor: $-R_1I$).
\item Similarly, we subtract the voltages across resistors $R_2$ and $R_3$, as we are tracing in the \textbf{same direction as the current through those resistors}: $-IR_2 -IR_3$.
\item We are back at the beginning of the loop, so the terms must sum to zero.
\end{itemize}

We can now use the loop rule, which states that the sum of the above voltages must be zero:
\begin{equation}
-\Delta V_1 + \Delta V_2 - R_1I - R_2I - R_3I = 0\quad \text{(loop abcdefga)}
\end{equation}
This equation then gives us a relation between the various quantities (current, resistors, battery voltages) in the circuit which can be used to model the circuit.

\begin{framed}
\textbf{Checkpoint}\\
Suppose that the equation describing loop ${\rm abcdefga}$ (Figure~\ref{fig:circuits:loop}) was obtained from a different starting position and the loop was traced in the opposite direction. Would this produce a different equation?

\begin{enumerate}
\item Yes, the equation would be incorrect if the loop is traced in the direction opposite to the flow of current.
\item Yes, the equation must start from the point ${\rm a}$ because the creator of the circuit assumes the person calculating current and voltage will begin at point ${\rm a}$.
\item Yes, there is no incorrect starting point, but choosing to trace the circuit in the direction opposite to the flow of current would produce an incorrect equation.
\item No, there is no incorrect direction or starting point.
\end{enumerate}

\begin{framed}
\textbf{Answer}\\
\begin{enumerate}[resume]
\item
\end{enumerate}
\end{framed}
\end{framed}

\begin{framed}
\textbf{Olivia's Thoughts}\\
One way to conceptualize Kirchhoff's loop rule is to draw an analogy to gravity. Imagine that the circuit in Figure~\ref{fig:circuits:loop} operates like a roller coaster, as shown in Figure~\ref{fig:circuits:rollercoaster}. Each battery is like the mechanical lift that brings the cart to the top of the hill. The battery arrow points in the uphill direction.

\begin{figure}[!htbp]
\centering
\includegraphics[width=0.5\linewidth]{files/rollercoaster-30ac3b258fc22d96ce9b0086f3ceaa74.png}
\caption[]{The circuit from Figure~\ref{fig:circuits:loop} using the roller coaster analogy.}
\label{fig:circuits:rollercoaster}
\end{figure}

This time, we'll start at point ${\rm b}$ and go counter clockwise. In this analogy, Kirchhoff's loop rule says that when the roller coaster cart comes back around to ${\rm b}$, it needs to have the same gravitational potential it started with (since it will be at the same elevation). Let's follow its journey around the loop. It starts at ${\rm b}$, then goes through a battery and gets lifted uphill. The cart moves along the track (in the direction of the current) and loses potential energy whenever it goes down a hill (meaning through a resistor). When the cart comes around the left side of the circuit, it encounters another battery. This time, the cart is moving \textit{opposite} to the battery arrow, so the cart goes downhill instead of getting lifted uphill. When it gets back to ${\rm b}$, it has the same potential it started with. Note that, if we are moving in the direction opposite to the current, this is like going backwards through the roller coaster. Instead of going downhill through each resistor, you would have to go uphill, and thus gain potential energy.
\end{framed}

\subsubsection{Applying Kirchhoff's rules to model circuits}

In this section, we show how to model a circuit using Kirchhoff's rules. In general, one can consider a circuit to be fully modelled if one can determine the current in each segment of the circuit. We will show how one can apply the same procedure to model any circuit that contains batteries and resistors. The procedure is as follows:

\begin{enumerate}
\item Make a good diagram of the circuit.
\item Simplify any resistors that can easily be combined into effective resistors (in series or in parallel).
\item Make a new diagram with the effective resistors, showing battery arrows, and labelling all of the nodes so that loops can easily be described.
\item Make a \textbf{guess} for the directions of the current in each segment.
\item Write the junction rule equations.
\item Write the loop equations.
\item This will lead to $N$ independent equations that one can solve for the $N$ different currents in the circuit.
\item Once you have determined all of the currents, if some of them are negative numbers, switch the direction of those currents in the diagram (they will be negative if you guessed the direction incorrectly).
\end{enumerate}

We will illustrate the procedure on the circuit shown in Figure~\ref{fig:circuits:bigcircuit}, for which we would like to know the current through each resistor and each battery. The circuit contains 5 resistors ($R_1$-$R_5$), 2 real batteries (with ideal voltages $\Delta V_1$ and $\Delta V_2$), and 2 additional resistors to model the internal resistances of the real batteries ($r_1$, $r_2$)

\begin{figure}[!htbp]
\centering
\includegraphics[width=0.42\linewidth]{files/bigcircuit-d85731ab6ffaa6e5a59fc67d56856ba0.png}
\caption[]{A circuit that can be simplified and then solved with Kirchhoff's rules.}
\label{fig:circuits:bigcircuit}
\end{figure}

\begin{framed}
\textbf{Checkpoint}\\
How many different currents are in the circuit shown in Figure~\ref{fig:circuits:bigcircuit}?

\begin{enumerate}
\item 3
\item 4
\item 5
\item 6
\end{enumerate}

\begin{framed}
\textbf{Answer}\\
\begin{enumerate}[resume]
\item
\end{enumerate}
\end{framed}
\end{framed}

\textbf{Simplifying the resistors (step 2):} In this circuit, resistors $r_2$, $R_1$ and $R_2$ are in series, so that they can be combined into an effective resistor, $R_6$:
\begin{equation}
R_6=r_2+R_1+R_2
\end{equation}
With this simplification, we obtain the circuit illustrated in Figure~\ref{fig:circuits:bigcircuit_simp1}

\begin{figure}[!htbp]
\centering
\includegraphics[width=0.42\linewidth]{files/bigcircuit_simp1-c3aa371fb88c44134be8c89020416c77.png}
\caption[]{The resistors $r_2$, $R_1$ and $R_2$ in series from the circuit in Figure~\ref{fig:circuits:bigcircuit} have been combined into the effective resistor, $R_6$, to simplify the circuit.}
\label{fig:circuits:bigcircuit_simp1}
\end{figure}

Next, we note that resistors $R_4$ and $R_5$ are in parallel and can be easily combined into a resistor, $R_7$:
\begin{equation}
R_7=\frac{R_4R_5}{R_4+R_5}
\end{equation}
which leads to the circuit illustrated in Figure~\ref{fig:circuits:bigcircuit_simp2}.

\begin{figure}[!htbp]
\centering
\includegraphics[width=0.42\linewidth]{files/bigcircuit_simp2-99756f76f2fb3356a4713b80fb3fb893.png}
\caption[]{The resistors $R_4$ and $R_5$ in parallel from the circuit in Figure~\ref{fig:circuits:bigcircuit_simp1} have been combined into the effective resistor, $R_7$, to simplify the circuit.}
\label{fig:circuits:bigcircuit_simp2}
\end{figure}

Finally, we note that $r_1$ and $R_7$ are in series and can be combined into an effective resistor, $R_8$:
\begin{equation}
R_8=r_1+R_7=r_1+\frac{R_4R_5}{R_4+R_5}
\end{equation}
leading to the simplified circuit illustrated in Figure~\ref{fig:circuits:bigcircuit_simp3}, which we have labelled with nodes and battery labels.

\begin{figure}[!htbp]
\centering
\includegraphics[width=0.45\linewidth]{files/bigcircuit_simp3-52981a5029c13508d012c4743b946805.png}
\caption[]{The resistors $r_1$ and $R_7$ from the circuit in Figure~\ref{fig:circuits:bigcircuit_simp2} have been combined into an effective resistor, $R_8$, to simplify the circuit.}
\label{fig:circuits:bigcircuit_simp3}
\end{figure}

\textbf{Guessing the directions of the currents (step 4):} Before we can write the equations from Kirchhoff's rules, we need to label the currents in the circuit diagram. In general, it is not always obvious in which way the currents will go, so we make a guess that we can fix later if we guessed wrong.

In order to guess the current directions, choose one point on the circuit and move along a segment. Label the current in that segment and continue moving through the circuit, splitting up the current when a junction is encountered. Make sure to only have one current per segment. We guess the currents as follows, referring to Figure~\ref{fig:circuits:bigcircuit_simp}:

\begin{itemize}
\item We start at point ${\rm a}$ and move upwards to point ${\rm f}$. We will call the current in that segment, $I_1$.
\item Since there is no junction, the current $I_1$ continues through the resistor $R_8$ to point ${\rm e}$.
\item There is a junction at point ${\rm e}$, so we split the current $I_1$ into currents $I_2$ (towards point ${\rm d}$), and $I_3$ (downwards to point ${\rm b}$).
\item We follow current $I_2$ first; $I_2$ flows from ${\rm e}$ to ${\rm d}$, then down to ${\rm c}$, through the battery $\Delta V_2$, and to point ${\rm b}$, where there is again junction.
\item We follow current $I_3$, which just flows down to the junction at point ${\rm b}$, where it ``meets up'' with current $I_2$.
\item Currents $I_2$ and $I_3$ both flow into the junction at point ${\rm b}$, and the current flowing out of the junction, through the battery $\Delta V_1$, and towards point ${\rm a}$ is, again, $I_1$, since this current then flows up to point ${\rm f}$.
\item All segments of wire have a labelled current, so we are done guessing currents.
\end{itemize}

You can proceed in an analogous way for any circuit. The final circuit, with currents labelled, is shown in Figure~\ref{fig:circuits:bigcircuit_simp}:

\begin{figure}[!htbp]
\centering
\includegraphics[width=0.45\linewidth]{files/bigcircuit_simp-74458b91bfc85d701204ec05fac2eb8a.png}
\caption[]{Final and labelled circuit diagram that is simplified from the one in Figure~\ref{fig:circuits:bigcircuit}.}
\label{fig:circuits:bigcircuit_simp}
\end{figure}

We can now proceed using Kirchhoff's rules to solve for the values of the currents in the circuit. It is useful to note that there are 3 unknown currents in this circuit; we thus hope that Kirchhoff's rules will give us 3 independent equations.

\textbf{Applying the junction rule (step 5):} In the circuit from Figure~\ref{fig:circuits:bigcircuit_simp}, there are two junctions (at points ${\rm b}$ and ${\rm e}$), so we will get two equations from the junction rule. To apply the junction rule, the sum of the currents coming into the junction must be equal to the currents going out of the junction:
\begin{equation}
\text{incoming currents}&=\text{outgoing currents}&\\
I_2+I_3 &= I_1 \quad &\text{(junction b)}\\
I_1 &= I_2+I_3 \quad &\text{(junction e)}\\
\end{equation}
Note that the two equations are not independent (in fact, they are the same). It is generally the case that if there $N$ junctions, one will obtain less than $N$ independent equations (usually, $N -1$ equations will be independent). In this case, the two junctions only gave us one equation.

\textbf{Applying the loop rule (step 6):} This circuit contains 3 different loops: ${\rm abcdefa}$, ${\rm abefa}$, and ${\rm bcdeb}$, which will lead to 3 equations from the loop rule. We expect that these equations will not be independent, since this would lead to 4 equations and 3 unknowns when combined with the junction rule equation. Let us start with the loop ${\rm abcdefa}$:

\begin{itemize}
\item From ${\rm a}$ to ${\rm b}$, we trace through the battery in the \textbf{opposite direction from the battery arrow}: $-\Delta V_1$.
\item From ${\rm b}$ to ${\rm c}$, we trace through the battery in the \textbf{same direction as the battery arrow}: $+\Delta V_2$.
\item From ${\rm c}$ through ${\rm d}$ and through to ${\rm e}$ we go through the resistor $R_6$ in the \textbf{opposite direction from the current}, $I_2$, in that resistor: $+I_2R_6$.
\item From ${\rm e}$ to ${\rm f}$, we go through the go through the resistor $R_8$ in the \textbf{opposite direction from the current}, $I_1$, in that resistor: $+I_1R_8$.
\item And we are back at the starting point, so the sum of the above terms is equal to zero.
\end{itemize}

which gives the equation:
\begin{equation}
-\Delta V_1+\Delta V_2+I_2R_6+I_1R_8=0\quad\text{(loop abcdefa)}
\end{equation}
Similarly, for the loop ${\rm abefa}$, we obtain:
\begin{equation}
-\Delta V_1+I_3R_3+I_1R_8=0\quad\text{(loop abefa)}
\end{equation}
and for loop ${\rm bcdeb}$:
\begin{equation}
\Delta V_2+I_2R_6-I_3R_3=0\quad\text{(loop bcdeb)}
\end{equation}
Although it appears that we have obtained 3 additional equations, only two of these are independent. For example, if you sum the second and third equations (loops ${\rm abefa}$, and ${\rm bcdeb}$), you simply obtain the first equation (loop ${\rm abcdefa}$). In general, if there are $N$ different loops, one will obtain less than $N$ independent equations (usually $N -1$ independent equations, as we did here).

At this point, after choosing one of the junction equations, and two of the loop equations, we have 3 independent equations that we can solve for the 3 unknown currents{\textbackslash}footnote\{The 3 unknowns do not necessarily have to be the currents, and could be any combination of the currents, battery voltage and resistors. As long as there at most 3 unknown quantities, this circuit can be solved.\}:
\begin{equation}
I_1 &= I_2+I_3 \quad &\text{(junction e)}\\
-\Delta V_1+\Delta V_2+I_2R_6+I_1R_8&=0\quad&\text{(loop abcdefa)}\\
-\Delta V_1+I_3R_3+I_1R_8&=0\quad&\text{(loop abefa)}
\end{equation}
It is only a matter of some simple math to solve for the 3 unknowns from these 3 equations (which we carry out in the example below).

\begin{framed}
\textbf{Example 19.3}\\
Referring to the circuit in Figure~\ref{fig:circuits:bigcircuit_vals}, what is the voltage across the real terminal of the battery with ideal voltage $\Delta V_1$ (the voltage between points ${\rm a}$ and ${\rm b}$)? What is the current through resistor $R_5$?

\begin{framed}
\textbf{Solution}\\
\begin{figure}[!htbp]
\centering
\includegraphics[width=0.6\linewidth]{files/bigcircuit_vals-5027390ef3ef30e202afcf35ff92a3fc.png}
\caption[]{The same circuit as in Figure~\ref{fig:circuits:bigcircuit}, with values filled in.}
\label{fig:circuits:bigcircuit_vals}
\end{figure}

Since this circuit is the same that we just analyzed, we know that it can be simplified into the circuit shown in Figure~\ref{fig:circuits:bigcircuit_vals_simp}, with resistors:
\begin{equation}
R_6&=r_2+R_1+R_2=(1 {\rm \Omega})+(3 {\rm \Omega})+(4 {\rm \Omega})=8 {\rm \Omega}\\
R_8&=r_1+\frac{R_4R_5}{R_4+R_5}=(1 {\rm \Omega})+\frac{(2 {\rm \Omega})(2 {\rm \Omega})}{(2 {\rm \Omega})+(2 {\rm \Omega})}=2 {\rm \Omega}
\end{equation}

\begin{figure}[!htbp]
\centering
\includegraphics[width=0.45\linewidth]{files/bigcircuit_vals_simp-425edb9cdde8849007b8f6c25e9b96b5.png}
\caption[]{Simplified version of the circuit in Figure~\ref{fig:circuits:bigcircuit_vals}.}
\label{fig:circuits:bigcircuit_vals_simp}
\end{figure}

From above, we know that this leads to the following three equations:
\begin{equation}
I_1 &= I_2+I_3 \quad &\text{(junction e)}\\
-\Delta V_1+\Delta V_2+I_2R_6+I_1R_8&=0\quad&\text{(loop abcdefa)}\\
-\Delta V_1+I_3R_3+I_1R_8&=0\quad&\text{(loop abefa)}
\end{equation}
In order to solve these types of equations, it is usually convenient to place the battery voltages on the right hand side, and the resistor voltages on the left hand side. Although it is generally bad practice to fill numbers into the equations before solving them, it is almost always a good idea when solving the $N$ equations for the $N$ currents. Furthermore, in order to make the equations legible, it is also useful to not write in the units (which is very bad practice in general!). Thus, filling in the values for the resistors and the battery voltages, moving the voltages to the right hand side, we obtain the following system of equations:
\begin{equation}
I_1-I_2-I_3&=0  \quad &\text{(junction e)}\\
2I_1+8I_2&=8 \quad&\text{(loop abcdefa)}\\
2I_1+4I_3&=12 \quad&\text{(loop abefa)}
\end{equation}
Subtracting the second equation from the third equation (to eliminate $I_1$):
\begin{equation}
4I_3-8I_2&=4\\
\therefore I_3&=1+2I_2
\end{equation}
Substituting this into the junction equation:
\begin{equation}
I_1-I_2-I_3&=0\\
I_1-I_2-1-2I_2&=0\\
\therefore I_2=\frac{1}{3}(I_1-1)
\end{equation}
Finally, substituting this into the equation from loop ${\rm abcdefa}$, allows us to determine $I_1$ and the other two currents:
\begin{equation}
2I_1+8I_2&=8\\
2I_1+8\left(\frac{1}{3}(I_1-1) \right)&=8\\
\therefore I_1&=\frac{16}{7}=2.29 {\rm A}\\
\therefore I_2&=\frac{1}{3}(I_1-1)=0.43 {\rm A}\\
\therefore I_3&=1+2I_2=1.86 {\rm A}\\
\end{equation}
In this case, the currents are all positive, so the diagram in Figure~\ref{fig:circuits:bigcircuit_vals_simp} is correct and we do not need to reverse the direction of any of the currents.

We can now determine the potential difference across the real terminals of the battery $\Delta V_1$. The current through the battery is $I_1=2.29 {\rm A}$, which cause a voltage drop, $\Delta V_{r1}$, across its internal resistance, $r_1$ of:
\begin{equation}
\Delta V_{r1}=I_1r_1=(2.29 {\rm A})(1 {\rm \Omega})=2.29 {\rm V}
\end{equation}
The voltage across the real terminals of the battery is then:
\begin{equation}
\Delta V_{real}=\Delta V_1-\Delta V_{r1}=(12 {\rm V})-(2.29 {\rm V})=9.7 {\rm V}
\end{equation}

The current through the resistor $R_5$ (Figure~\ref{fig:circuits:bigcircuit_vals}) requires a little more thought, since we calculated the current, $I_1$ through the effective resistor $R_8$, which we must now ``break apart''. Figure~\ref{fig:circuits:bigcircuit_vals_r8} shows the components of $R_8$.

\begin{figure}[!htbp]
\centering
\includegraphics[width=0.5\linewidth]{files/bigcircuit_vals_r8-95d96b76bef1e0d43ff22f54d3083840.png}
\caption[]{The components of the effective $R_8$ resistor from Figure~\ref{fig:circuits:bigcircuit_vals_simp}. The current, $I_1$, coming from the battery goes through $r_1$ and then splits up.}
\label{fig:circuits:bigcircuit_vals_r8}
\end{figure}

The current $I_1$, that goes through the $\Delta V_1$ battery also goes through the $r_1$ internal resistance of the battery. That current then splits up into currents, $I_4$ and $I_5$, to go through the resistors $R_4$ and $R_5$. Although it should be obvious that half of $I_1$ will go through each resistor (since these are equal), we can determine this from applying Kirchhoff's rules to the combination of resistors in Figure~\ref{fig:circuits:bigcircuit_vals_r8}:
\begin{equation}
I_1&=I_4+I_5 \quad&\text{(junction)}\\
I_5R_5-I_4R_4&=0\quad&\text{(clockwise loop)}
\end{equation}
From the loop equation, we have:
\begin{equation}
I_5=\frac{R_4}{R_5}I_4=I_4
\end{equation}
since $R_4=R_5=2 {\rm \Omega}$. Since $I_4=I_5$, the junction equation gives:
\begin{equation}
I_5=\frac{1}{2}I_1=1.15 {\rm A}
\end{equation}
By solving for $I_4$ and $I_5$, we have now determined all of the currents through all of the segments of the original circuit in Figure~\ref{fig:circuits:bigcircuit_vals}.

\textbf{Discussion:} In this example, we showed how one can use a simplified circuit to solve the current through the effective resistors in the simplified circuit. Once those currents are known, we showed that it is straightforward to determine the currents through individual resistors that have been combined into effective resistors.
\end{framed}
\end{framed}

\begin{framed}
\textbf{Josh's Thoughts}\\
Solving a circuit can be daunting, especially if the diagram is drawn in an unfamiliar  way. While the circuits in this chapter are designed to be as easy to read as possible, many circuits are much more strange. For example, here is a circuit which you may come across:

\begin{figure}[!htbp]
\centering
\includegraphics[width=0.4\linewidth]{files/circuit1Josh-7520d0c8826bca1fc6e734a164b2bfb1.png}
\caption[]{A weird looking circuit.}
\label{fig:circuits:circuit1Josh}
\end{figure}

The circuit in Figure~\ref{fig:circuits:circuit1Josh} May look like it is a difficult circuit to solve, but the diagram can be re-drawn to reveal the simplicity of the circuit:

\begin{figure}[!htbp]
\centering
\includegraphics[width=0.4\linewidth]{files/circuit2Josh-4cb1604ed35f8db5448b4c644cbb8f91.png}
\caption[]{A much less weird looking circuit.}
\label{fig:circuits:circuit2Josh}
\end{figure}

What used to be a strange kite shape is now just a parallel circuit, which can be further simplified by calculating the effective resistance:
\begin{equation}
R_{eff} &= (R_1^{-1}+R_2^{-1}+(R_3+R_4)^{-1})^{-1}
\end{equation}
Which gives a series circuit with only one resistor:

\begin{figure}[!htbp]
\centering
\includegraphics[width=0.4\linewidth]{files/circuit3Josh-a39cdccc8edb6d8a92b6f2e1d82a2f3a.png}
\caption[]{A simple circuit.}
\label{fig:circuits:circuit3Josh}
\end{figure}

Circuits can be drawn in many unique or potentially confusing ways, but knowing how to read the circuit and re-draw it can help make the diagram more legible and the circuit easier to solve.
\end{framed}

\subsubsection{Measuring current and voltage}

In this section, we describe how one can build devices to measure current and voltage. A device that measures current is called an ``ammeter'' and a device that measured voltage is called a ``voltmeter''. Nowadays, these are usually found within the same physical device (a ``multimeter''), which can also measure resistance (by measuring voltage and current, resistance can easily determined). We will limit our description to the design of simple analogue ammeters and voltmeters.

As we will see in Chapter {\textbackslash}ref\{chapter:magneticforce\}, it is straightforward to build a device that can measure very small amounts of current, by running the current through a coil in a magnetic field so that the coil can deflect a needle that indicates the amount of current. Such a device is called a ``galvanometer'' and is usually limited to measuring very small current (of order {\textbackslash}si\{mA\}). In this section, we describe how one can use a galvanometer in order to build both ammeters to measure large currents and voltmeters.

\paragraph{The ammeter}

An ammeter is built by placing a galvanometer in parallel with a ``shunt'' resistor, $R_s$. The shunt resistor is a small resistor that ``shunts'' (deflects) the current away from the galvanometer, so that most of the current goes through the shunt resistor. This is illustrated in Figure~\ref{fig:circuits:ammeter}, which shows the galvanometer (circle with the ${\rm G}$ inside), the internal resistance of the galvanometer, $R_G$, and the shunt resistor, $R_S$. The actual ammeter would be contained in a box and have two connectors (shown as ${\rm A}$ and ${\rm B}$ in the figure).

\begin{figure}[!htbp]
\centering
\includegraphics[width=0.55\linewidth]{files/ammeter-db19cc75fe976fb35097957e7bff97e2.png}
\caption[]{Constructing an ammeter from a galvanometer by placing a ``shunt'' resistor in parallel with the galvanometer.}
\label{fig:circuits:ammeter}
\end{figure}

By modelling the ammeter, we can determine the total current, $I$, that we would like to measure using the known values of the resistors and the current, $I_G$, measured by the galvanometer. Considering any of the two junctions, and a clockwise loop, we have:
\begin{equation}
I&=I_G+I_S \quad&\text{(junction)}\\
I_GR_G-I_SR_S&=0\quad&\text{(clockwise loop)}\\
\therefore I_S&=\frac{R_G}{R_S}I_G\\
\therefore I &= I_G+I_S=\left(1+\frac{R_G}{R_S}\right) I_G
\end{equation}
which allows us to determine the current $I$ from the current $I_G$, measured by the galvanometer. We also see that most of the current goes through the shunt (since $R_S$ is chosen to be smaller than $R_G$). The ammeter will have a total resistance, $R_A$, given by:
\begin{equation}
R_A=\frac{R_GR_S}{R_G+R_S}
\end{equation}
In order to measure the current through a specific segment of a circuit, an ammeter must be placed in series with that segment (so that the current that we want to measure will pass through the ammeter). Figure~\ref{fig:circuits:ammeterR} shows how to connect an ammeter (circle with the letter ${\rm A}$) in order to measure the current through a resistor, $R$.

\begin{figure}[!htbp]
\centering
\includegraphics[width=0.4\linewidth]{files/ammeterR-6b1f56c0a866644a94700534fcf4da05.png}
\caption[]{An ammeter is placed in series with a resistor to measure the current through the resistor.}
\label{fig:circuits:ammeterR}
\end{figure}

\paragraph{The voltmeter}

A voltmeter is constructed by placing a large resistor, $R_V$, in series with a galvanomenter (that has internal resistance $R_G$), as illustrated in Figure~\ref{fig:circuits:voltmeter}. The voltmeter is designed to measure the potential difference between the terminals of the voltmeter (labelled ${\rm A}$ and ${\rm B}$ in the Figure).

\begin{figure}[!htbp]
\centering
\includegraphics[width=0.5\linewidth]{files/voltmeter-49ef0913942c401600576df4ae715547.png}
\caption[]{Constructing an voltmeter from a galvanometer by placing a resistor in series with the galvanometer.}
\label{fig:circuits:voltmeter}
\end{figure}

Given the values of the resistors, and the current measured by the galvanometer, one can easily determine the potential difference between points ${\rm A}$ and ${\rm B}$, since the current measured by the galvanometer goes directly through each resistor:
\begin{equation}
\Delta V = V_B-V_A=-I_G(R_V+R_G)
\end{equation}
In order to measure a potential difference across a component, the voltmeter must be placed in parallel with the component. Figure~\ref{fig:circuits:voltmeterR} shows how to connect a voltmeter (circle with the letter ${\rm V}$) in order to measure the voltage across a resistor, $R$.

\begin{figure}[!htbp]
\centering
\includegraphics[width=0.3\linewidth]{files/voltmeterR-64b16d12078075d9e8cbf6f8dd3935e8.png}
\caption[]{A voltmeter is placed in parallel with a resistor to measure the voltage across the resistor.}
\label{fig:circuits:voltmeterR}
\end{figure}

When using an ammeter or a voltmeter, you will notice that these usually have buttons or dials to choose the range of currents or voltages to be measured. All the dial does is change the value of the shunt or series resistor in order to maintain a given maximum current through the galvanometer. An ohmmeter, to measure resistance, is simply an ammeter with a built-in fixed potential difference (so that by measuring current across a known potential difference, the resistance of the component can be determined).

\begin{framed}
\textbf{Example 19.4}\\
Two resistors with a resistance of $1 {\rm k\Omega}$ are placed in series with a $12 {\rm V}$ battery. A voltmeter with a total resistance of $R_V=10 {\rm k\Omega}$ is used to measure the voltage across one of the resistors. What reading does the voltmeter show?

\begin{framed}
\textbf{Solution}\\
Since the two resistors have the same resistance, and are in series with the battery, when no voltmeter is connected, the voltage across either resistor is easily shown to be $6 {\rm V}$. However, by connecting the voltmeter across one of the resistors, we modify the circuit, and we should expect the voltage that is read to be different than $6 {\rm V}$ (can you tell if it will be larger or smaller?). The circuit, with the voltmeter connected, is shown in Figure~\ref{fig:circuits:voltmeter2R}.

\begin{figure}[!htbp]
\centering
\includegraphics[width=0.35\linewidth]{files/voltmeter2R-b7592cd3b8ef50a3b49d46c3cf69f11e.png}
\caption[]{When using a voltmeter, the circuit is modified.}
\label{fig:circuits:voltmeter2R}
\end{figure}

We can model this circuit quite easily by combining the voltmeter (modelled as a resistor) in parallel with one of the resistors:
\begin{equation}
R_{eff}&=\frac{R_VR}{R_V+R}\\
&=\frac{(10 {\rm k\Omega})(1 {\rm k\Omega})}{(10 {\rm k\Omega})+(1 {\rm k\Omega})}\\
&=\frac{10}{11}{\rm k\Omega}=0.91 {\rm k\Omega}
\end{equation}
The sum of the voltage drops across the effective resistor and the other resistor must equal the potential difference across the battery (Kirchhoff's loop rule):
\begin{equation}
R_{eff}I+RI&=\Delta V\\
\therefore I &= \frac{\Delta V}{R_{eff}+R}=\frac{(12 {\rm V})}{(0.91 {\rm k\Omega})+(1 {\rm k\Omega})}=6.29e-3 {\rm A}
\end{equation}
The voltage drop across the effective resistor is the same as the reading on the voltmeter:
\begin{equation}
\Delta V_{voltmeter}=IR_{eff}=(6.29e-3 {\rm A})(0.91 {\rm k\Omega})=5.7 {\rm V}
\end{equation}
and the voltmeter reads a smaller voltage than there would be without the voltmeter.

\textbf{Discussion:} In this example, we saw that by using a voltmeter to measure a voltage in a circuit, we actually disturb the circuit. By placing the voltmeter in parallel with one resistor, we created an effective resistor with a resistance that is lower than the resistance of either the voltmeter or the resistor. This lowered the total resistance of the circuit, which increased the current. A larger current through the second resistor (without the voltmeter) leads to a larger voltage drop than $6 {\rm V}$ across that resistor. Thus, the voltage drop across the resistor with the voltmeter will be less than $6 {\rm V}$, as we found, since the two voltage drops need to add to $12 {\rm V}$. In general, when using a voltmeter, one needs a voltmeter with a very high resistance in order to minimize the disturbance to the circuit (if the voltmeter has a high resistance, only a small amount of current will be shunted from the resistor). In practice, voltmeters have resistance that are typically of the order of $1 {\rm M\Omega}$.
\end{framed}
\end{framed}

\subsubsection{Modelling circuits with capacitors}

{\textbackslash}begin\{review\}
* Section {\textbackslash}ref\{sec:potential:capacitors\} on capacitors.

{\textbackslash}end\{review\}
So far, we have modelled circuits where the current does not change with time. When a capacitor is included in a circuit, the current will change with time, as the capacitor charges or discharges. The circuit shown in Figure~\ref{fig:circuits:RCcircuit} shows an ideal battery{\textbackslash}footnote\{The model still holds for a real battery, since the internal resistance of the battery can just be included into the resistance of the resistor, $R$.\} ($\Delta V$), in series with a resistor ($R$), a capacitor ($C$, two vertical bars) and a switch (${\rm S}$) that is open.

\begin{figure}[!htbp]
\centering
\includegraphics[width=0.4\linewidth]{files/RCcircuit-ed0f38181d5ef64b678a6171d86d88e5.png}
\caption[]{A simple circuit with a resistor, battery, and capacitor.}
\label{fig:circuits:RCcircuit}
\end{figure}

When the switch is open, current cannot flow through the circuit. If we assume that the capacitor has no charge on it, once we close the switch, current will start to flow and charges will accumulate on the capacitor. Electrons will leave the negative terminal of the battery, flow through the resistor and accumulate on the left side of the capacitor, which acquires a negative charge. This pushes electrons off of the right hand side of the capacitor, which then becomes positively charged. The electrons from the positive side of the capacitor then flow into the positive side of the battery, completing the circuit.

Eventually, the charges on the capacitor will build up to a point were they prevent any further flow of current. Once the left side of the capacitor is at the same potential as the left side of the battery, current will cease to flow. That is, eventually, the potential difference across the capacitor will be equal to that across the battery, and we can think of this as a circuit used to charge a capacitor. The current is high when the switch is first opened, but eventually goes down to zero as the capacitor charges. The current is thus time-dependent.

We can model this simple circuit (with the switch closed) using Kirchhoff's loop rule. The sum of the voltages across each component must sum to zero:
\begin{equation}
\Delta V - IR - \frac{Q}{C} = 0
\end{equation}
where we used the fact that the charge, $Q$, on a capacitor is related to the potential difference, $\Delta V_C$, across the capacitor by $Q=C\Delta V_C$. The current, $I$, is the rate at which charges flow through the circuit, and is thus equal to rate at which charges accumulate on the capacitor:
\begin{equation}
I=\frac{dQ}{dt}
\end{equation}
Substituting this into the loop equation, we obtain a separable differential equation for the charge on the capacitor as a function of time, $Q(t)$:
\begin{equation}
\Delta V - IR - \frac{Q}{C} &= 0\\
\Delta V - \frac{dQ}{dt}R - \frac{Q}{C} &= 0\\
\Delta V - \frac{Q}{C} &= \frac{dQ}{dt}R\\
C\Delta V - Q &= RC\frac{dQ}{dt}\\
\therefore \frac{dt}{RC}&=\frac{dQ}{C\Delta V - Q }
\end{equation}
This is similar to differential equations that we have solved previously (in fact, it is the same equation as in Example~6.4 where we looked at the effect of velocity-dependent drag). The solution to the equation, assuming that the switch is closed at $t=0$, is given by an exponential:
\begin{equation}
Q(t) = C\Delta V\left( 1 - e^{-\frac{t}{RC}} \right)
\end{equation}
Thus, the charge on the capacitor starts at zero when the switch is closed, and grows asymptotically until it reaches a value of $Q=C\Delta V$, which corresponds to the capacitor having the same potential difference across it as the battery. The value $\tau=RC$ is called the ``time constant'' of the RC circuit, and corresponds to the time at which the capacitor will reach about $(1 -e^{ -1})=63\%$ of its maximal charge. The current as a function of time is given by:
\begin{equation}
I(t)=\frac{dQ}{dt}=\frac{\Delta V}{R}e^{-\frac{t}{RC}}
\end{equation}
and we can see that at time $t=0$ the current is the same as if there were no capacitor present, and the current then decreases exponentially until it reaches zero.

\subsubsection{Summary}

Batteries are usually formed from a collection of electrochemical cells. Batteries provide a constant electric potential difference across their terminals, usually sustained by a chemical reaction, as long as the current through the battery is not too large (or the chemical reactions cannot be sustained).

An ideal battery has no resistance and can be modelled as a simple potential difference in a circuit. A real battery includes an internal resistance and be modelled in a circuit as an ideal battery in series with a resistor. The voltage across the terminals of a real battery is equal to the voltage across the terminals of the ideal battery only when no current flows through the internal resistance.

Circuits are modelled using circuit diagram that include components (such as batteries and resistors) and wires. Wires are always modelled as having no resistance, since their resistance can be included by placing the appropriate resistor along the wire. The electric potential is always constant along a wire with no resistance. When modelling a circuit, \textbf{one always models the direction of conventional current}; that is, current is always indicated as the direction in which positive charges flow (even if in reality, it is negative electrons that flow in the opposite direction).

Circuits should be thought of in terms of conservation of energy. Components produce a potential difference between sections of wire. Batteries correspond to an increase in potential (if going from the negative to the positive terminal), whereas resistors corresponds to a decrease in potential (if going in the same direction as current through the resistor).

Kirchhoff's rules allow us to model complex circuits. The junction rule states that: \textbf{The current entering a junction must be equal to the current exiting a junction.} This is a statement about conservation of charge. If charges are flowing into a junction, then the same amount of charges must flow back out of the junction per unit time.

The loop rule states that: \textbf{The net voltage drop across a loop must be zero.} This is a statement about conservation of energy indicating that as the potential energy of a positive charge increases as it goes through a battery, it will decrease by the same amount if it goes through a resistor that is connected to the terminals of that battery.

In order to \textbf{apply the loop rule}, we strongly suggest using the following procedure, after having made a clear, labelled diagram showing battery arrows and currents in the circuit:

\begin{enumerate}
\item Identify the loop, including starting position and direction.
\item Start at the beginning of the loop, and trace around the loop.
\item Each time a battery is encountered, \textbf{add the battery voltage if you are tracing the loop in the same direction as the corresponding battery arrow}, subtract the voltage otherwise.
\item Each time a resistor is encountered, \textbf{subtract the voltage across that resistor ($RI$, from Ohm's Law) if tracing the loop in the same direction as the current}, add the the voltage otherwise.
\item Once you have traced back to the starting point, the resulting sum must be zero.
\end{enumerate}

In general, we suggest the following procedure in order to use Kirchhoff's rules to model any circuit:

\begin{enumerate}
\item Make a good diagram of the circuit.
\item Simplify any resistors that can easily be combined into effective resistors (in series or in parallel).
\item Make a new diagram with the effective resistors, showing battery arrows, and labelling all of the nodes so that loops can easily be described.
\item Make a \textbf{guess} for the directions of the current in each segment.
\item Write the junction rule equations. Usually, you will get $M -1$ independent equations for $M$ loops.
\item Write the loop equations. Usually, you will get $M -1$ independent equations for $M$ loops.
\item This will lead to $N$ independent equations that one can solve for the $N$ different currents in the circuit.
\item Once you have determined all of the currents, if some of them are negative numbers, switch the direction of those currents in the diagram (they will be negative if you guessed the direction incorrectly).
\end{enumerate}

Current and voltage measuring devices (ammeters and voltmeters, respectively) can be constructed from a galvanometer, which measures small currents. An ammeter is constructed by placing a small shunt resistor in parallel with the galvanometer so that most of the current passes through the shunt resistor. The resulting ammeter must be placed in series with a component in order to measure the current through that component.

A voltmeter is constructed by placing a resistor in series with the galvanometer in order to reduce the current through the galvanometer. The resulting voltmeter must be placed in parallel with a component in a circuit in order to measure the voltage across that component. Note that because voltmeters and ammeters have a non-zero resistance, they will affect the circuit once they are connected.

When a capacitor is placed in a circuit, the current in the circuit will no longer be constant in time. If an uncharged capacitor with capacitance, $C$, is placed in a series circuit with a battery and a resistor of resistance, $R$, the capacitor will charge until the voltage across the capacitor is equal to that across the battery. Once the capacitor is charged, current ceases to flow in the circuit. The charges on a capacitor accumulate with a rate that decays exponentially; it will take an infinite amount of time for the capacitor to become fully charged. It will be charged to about 63\% of maximum charge after a period of time, $\tau=RC$, called the time constant of the capacitor.

\begin{framed}
\textbf{Important Equations}\\
\textbf{Ohm's Law:}
\begin{equation}
\Delta V &= IR
\end{equation}
\textbf{Junction Rule:}
\begin{equation}
\sum I_{in} &= \sum I_{out}
\end{equation}
\textbf{Loop Rule:}
\begin{equation}
\sum_{loop} \Delta V = 0
\end{equation}
\end{framed}

\subsubsection{Thinking about the material}

\begin{framed}
\textbf{Reflect and research}\\
\begin{itemize}
\item When did Galvani and Volta experiment with electric cells?
\item What is the largest voltage that Volta obtained with his voltaic pile?
\item How does one charge a rechargeable battery? What would the circuit look like?
\end{itemize}
\end{framed}

\begin{framed}
\textbf{To try at home}\\
\begin{itemize}
\item Research circuit diagrams of appliances you have at home.
\end{itemize}
\end{framed}

\begin{framed}
\textbf{To try in the lab}\\
\begin{itemize}
\item Propose an experiment to measure the change in current of an RC circuit as a capacitor builds up and releases charge.
\item Propose an experiment to determine the RC constant for a capacitor charging circuit.
\item Propose an experiment to measure the resistance of a voltmeter and compare your results with the manufacturer's.
\end{itemize}
\end{framed}

\subsubsection{Sample problems and solutions}

\paragraph{Problems}

\begin{framed}
\textbf{Problem 19.1}\\
A simple RC circuit as shown in Figure~\ref{fig:circuits:RCcircuit} contains a charged capacitor of unknown capacitance, $C$, in series with a resistor, $R=2 {\rm \Omega}$. When charged, the potential difference across the terminals of the capacitor is $9 {\rm V}$.

At time $t=0 {\rm s}$, the switch, ${\rm S}$, is closed, allowing the capacitor to discharge through the resistor. The current is then measured to be $I = 0.05 {\rm A}$ at $t = 5 {\rm s}$ after opening the switch.

\begin{itemize}
\item a. What is the capacitance of the capacitor?
\item b. What charge did the capacitor hold at $t = 2 {\rm s}$?
\end{itemize}

\begin{figure}[!htbp]
\centering
\includegraphics[width=0.3\linewidth]{files/RCcircuitCharge-10b05782ac2f5b4511023f1c0ed42294.png}
\caption[]{A simple circuit with a resistor and a capacitor.}
\label{fig:circuits:RCcircuitCharge}
\end{figure}
\end{framed}

\begin{framed}
\textbf{Problem 19.2}\\
A voltmeter with a resistance of $R_V = 20 {\rm k\Omega}$ is attached to a circuit with a battery of unknown voltage and two resistors with a resistance of $R = 2.5 {\rm k\Omega}$ as shown in Figure~\ref{fig:circuits:question2circuit}. The voltmeter reads that the voltage drop over one of the resistors is $\Delta V_{vm} = 5.647 {\rm V}$. What is the voltage drop, $V_R$, over each resistor when the voltmeter is removed from the circuit?

\begin{figure}[!htbp]
\centering
\includegraphics[width=0.45\linewidth]{files/question2circuit-ed8c1860995b2657c9e658d308b1eb62.png}
\caption[]{A circuit with a battery of unknown voltage.}
\label{fig:circuits:question2circuit}
\end{figure}
\end{framed}

\paragraph{Solutions}

\begin{framed}
\textbf{Solution 19.1}\\
\begin{itemize}
\item a. In this case, the capacitor is discharging as a function of time. At time $t=0$, the voltage across the capacitor is $\Delta V=9 {\rm V}$. We can model this discharging circuit in a similar way as we modelled the charging circuit.
\end{itemize}

We start with Kirchhoff's loop rule, which leads to a differential equation for the charge stored on the capacitor as function of time, $Q(t)$:
\begin{equation}
\Delta V - IR &=0\\
\frac{Q}{C} - IR &=0\\
\frac{Q}{C} - \frac{dQ}{dt}R &=0\\
\therefore \frac{dQ}{dt} = -\frac{1}{RC}Q
\end{equation}
This differential equation is straightforward to solve, since it says that the derivative of $Q(t)$ is equal to a constant multiplied by $Q(t)$. Thus, $Q(t)$ must be an exponential function:
\begin{equation}
Q(t) = Q_0 e^{-\frac{t}{RC}}
\end{equation}
where $Q_0$ is the (unknown) charge on the capacitor at $t=0$. You can easily verify that taking the derivative of this equation will result in the differential equation being satisfied.

Now that we have an equation for the charge as a function of time, it is straightforward to find the current, since it is just the time derivative of the charge. The current as a function of time, $I(t)$, is given by:
\begin{equation}
I &=\frac{dQ}{dt}=-\frac{1}{RC}Q=\frac{Q_0}{RC} e^{-\frac{t}{RC}}=I_0e^{-\frac{t}{RC}}
\end{equation}
where $I_0=\frac{Q_0}{RC}$ is the current at $t=0$.{\textbackslash}

We also know that the current through the resistor at $t=0$ is given by Ohm's Law, since, at that time, the voltage, $\frac{Q_0}{C}=9 {\rm V}$:
\begin{equation}
I_0=\frac{Q_0}{RC}=\frac{(9 {\rm V})}{(2 {\rm \Omega})}=4.5 {\rm A}
\end{equation}

We then know that the current, at time $t=5 {\rm s}$, is equal to $I(5)=0.05 {\rm A}$, allowing us to determine the capacitance:
\begin{equation}
I(5)&=I_0e^{-\frac{t}{RC}}\\
\ln\left( \frac{I(5)}{I_0} \right)&=-\frac{t}{RC}\\
\therefore C&=\frac{t}{R \ln\left( \frac{I_0}{I(5)} \right)}=\frac{(5 {\rm s})}{(2 {\rm \Omega})\ln\left( \frac{(4.5 {\rm A})}{(0.05 {\rm A})} \right)}=0.56 {\rm F}
\end{equation}

\begin{itemize}
\item b. To find the charge stored in the capacitor at $t = 2 {\rm s}$, we can use the function $Q(t)$ that we determined before:
\end{itemize}
\begin{equation}
Q(t=2 {\rm s})=Q_0 e^{-\frac{t}{RC}}
\end{equation}
where we can determine $Q_0$, now that we know the capacitance. $Q_0$ is the charge on the capacitor at time $t=0$, when the voltage across the capacitor is $9 {\rm V}$:
\begin{equation}
Q_0=C\Delta V = (0.56 {\rm F})(9 {\rm V})=5.0 {\rm C}
\end{equation}
At $t = 2 {\rm s}$, the charge on the capacitor is thus:
\begin{equation}
Q(t = 2 {\rm s})=(5.0 {\rm C})e^{-\frac{(2 {\rm s})}{(2 {\rm \Omega})(0.56 {\rm F})}}=0.84 {\rm C}
\end{equation}
\end{framed}

\begin{framed}
\textbf{Solution 19.2}\\
In order to know the voltage across one of the resistors, we need to determine the voltage that is across the battery. Once we have determined the voltage across the battery, the voltage across one of the resistors will just be half of that across the battery, since the two resistors have the same resistance.

We can model the circuit with the voltmeter in place, since we know the voltage across the parallel combination of the voltmeter and resistor (that voltage which is readout by the voltmeter). We can combine the voltmeter and one of the resistors into a an equivalent resistor, $R_{eff}$:
\begin{equation}
R_{eff} &= \frac{1}{R_V^{-1}+R^{-1}}\\
R_{eff} &= \frac{1}{(20 {\rm k\Omega})^{-1}+(2.5 {\rm k\Omega})^{-1}}\\
R_{eff} &= 2.22 {\rm k\Omega}\\
\end{equation}
Now that we have the effective resistance as well as the voltage drop across that effective resistor, we can solve for current through the circuit:
\begin{equation}
I &= \frac{\Delta V_{vm}}{R_{eff}}\\
I &= \frac{5.647 {\rm V}}{2.22 {\rm k\Omega}}\\
I &= 2.541 {\rm mA}\\
\end{equation}

Now that we have the current, we can combine the known resistances and determine the voltage drop across the battery.
\begin{equation}
\Delta V_{battery} &= I(R_{eff}+R)\\
\Delta V_{battery} &= (2.541 {\rm mA})(2.222 {\rm k\Omega}+2.5 {\rm k\Omega})\\
\Delta V_{battery} &= 12 {\rm V}\\
\end{equation}
Thus, with no voltmeter present, the voltage across each resistor is $6 {\rm V}$.
\end{framed}

\subsection{Chapter 20 - Magnetic force}

\subsubsection{Overview}\label{chapter:magneticforce}

This chapter introduces the tools to model the magnetic force, which is something that we have all experienced with magnets. As we will see, the magnetic force acts on moving (electric) charges, and is thus fundamentally different from the electric force which acts on stationary and moving charges. In later chapters, we will develop the tools that allow us to make connections between the electric and magnetic fields.

\begin{framed}
\textbf{Learning Objectives}\\
\begin{itemize}
\item Understand the key characteristics of a magnetic field and what makes it different from an electric field.
\item Understand how to model the magnetic force on a moving charge.
\item Understand how to model the magnetic force on a wire carrying current.
\item Understand how to model the torque exerted on a current-carrying loop by a magnetic field.
\item Understand how to model the Hall Effect.
\item Understand simple applications of the magnetic force.
\end{itemize}
\end{framed}

\begin{framed}
\textbf{Think About It}\\
When you go through airport security, they sometimes sample your luggage with sticky tape and place that tape into a machine to detect trace amounts of explosives. How does that machine work?

\begin{enumerate}
\item The machine detects trace amounts by ``sniffing'' the sample using similar chemical reactions as those in our olfactory system.
\item The machine vaporizes the sample and accelerates the resulting charged vapour around a circle to determine its constituents.
\end{enumerate}

\begin{framed}
\textbf{Answer}\\
\begin{enumerate}[resume]
\item
\end{enumerate}
\end{framed}
\end{framed}

\subsubsection{Magnetic fields}

Just as we can model the electric force on a charge by using the electric field (e.g. from another charge), we can model the force on a magnet by using a magnetic field (e.g. from another magnet). In your experience, every magnet that you have seen always has a ``North'' pole and a ``South'' pole. Most likely, you have noticed that the North pole of a magnet is attracted to the South pole of another magnet, and that the two North (or South) poles of different magnets repel each other. Thus, the magnetic force is attractive between two opposite poles, and repulsive otherwise.

The Earth itself can be modelled as a giant bar magnet, with North and South magnetic poles. The poles on a magnet are labelled North and South according to which geographic pole of the Earth they are attracted to (a magnetic compass needle has a magnetic North pole on the side that point to the Earth's North geographic pole).

\begin{framed}
\textbf{Checkpoint}\\
Is the magnetic North pole of the Earth located closer to the Earth's geographic North pole or closer to its geographic South Pole?

\begin{enumerate}
\item Earth does not have a magnetic field.
\item Earth's magnetic North pole is at Earth's geographic North pole.
\item Earth's magnetic North pole is at Earth's geographic South pole.
\item Earth's magnetic North pole depends on the charge of the observer.
\end{enumerate}

\begin{framed}
\textbf{Answer}\\
\begin{enumerate}[resume]
\item
\end{enumerate}
\end{framed}
\end{framed}

It may seem that the magnetic force can be described in the same way as the electric force, having two opposite sign ``charges'' (or poles for magnets), although this is not the case. As far as we can tell, there are no magnets that have only a North or a South pole. Every magnet must have a North \textit{and} a South pole. This is fundamentally different from the electric force, where an object can have a net positive or negative charge. In the context of magnetism, we say that ``monopoles do not exist'' (an object that has only a North or a South pole would be called a monopole). This is illustrated in Figure~\ref{fig:magneticforce:magnetcut}, which shows what happens as one cuts a bar magnet into two pieces; rather than ending up with a North and a South piece (monopoles), we end up with two smaller bar magnets, each with their own North and South pole.

\begin{figure}[!htbp]
\centering
\includegraphics[width=0.4\linewidth]{files/magnetcut-3c8cc38e7c5df139fbe1d25b9b3412a6.png}
\caption[]{When a bar magnet is cut through the middle, one obtains two magnets, each with a North and South pole, rather than a North and a South magnet.}
\label{fig:magneticforce:magnetcut}
\end{figure}

We model the magnetic force using a magnetic field vector, usually labelled $\vec B$. The magnitude of the magnetic field has the S.I. units of Teslas ({\textbackslash}si\{T\}). We draw magnetic field lines in much the same way that we draw electric field lines. The magnetic field lines are such that the magnetic field vector, $\vec B$, at some point in space is tangent to the field line at that point. The strength of the magnetic field is determined by the density of field lines at that position in space. The direction of the magnetic field, $\vec B$, indicates the direction of the force that is exerted on the North pole of a magnet. Magnetic field lines thus flow away from North poles and towards South poles.

The magnetic field description is similar to that of the electric field, with North magnetic poles being similar to positive electric charges, and vice versa.  However, because magnetic monopoles do not exist, magnetic field lines do not end (or start) on the pole of a magnet. Rather, magnetic field lines must always form \textbf{closed loops}. Figure~\ref{fig:magneticforce:barfield} shows the magnetic field lines for a bar magnet, highlighting that the field lines do not end at the poles, but rather continue through the magnet (and some of the lines only ``close'' outside of the figure). The magnetic field from a bar magnet is very similar to the electric field created by an electric dipole. For that reason, an object (such as a bar magnet) that forms magnetic field lines of this shape are often described as magnetic dipoles.

\begin{figure}[!htbp]
\centering
\includegraphics[width=0.55\linewidth]{files/barfield-6229f75cb4409fdf87afdb0d239dbb7e.png}
\caption[]{The magnetic field lines for a bar magnet always form closed loops as they do not end at the North or South pole of the magnet.}
\label{fig:magneticforce:barfield}
\end{figure}

We will discuss how to model magnetic fields in the next chapter, but it is important to understand that magnetic fields are created by moving electric charges. The electrons in the material that forms a bar magnet are the moving charges that create the magnetic field. As we will see, the magnetic field from a charge moving around in a circle (or a circular loop of current), has exactly the same shape as that of a bar magnet, as illustrated in Figure~\ref{fig:magneticforce:loopfield}. We can thus think of charge moving in a circle as a small bar magnet, or more precisely, as a magnetic dipole.

\begin{figure}[!htbp]
\centering
\includegraphics[width=0.45\linewidth]{files/loopfield-c0b99f3b0aaa1c8e1f9f3ca5dc35a9a2.png}
\caption[]{The magnetic field lines produced by a circular loop of current, $I$, are the same as those produced by a bar magnet.}
\label{fig:magneticforce:loopfield}
\end{figure}

In a magnet, the electrons in the material are moving in such a way that the magnetic fields that they generate are all in the same direction. In other words, each atom is like a small magnetic dipole, and all of these are aligned. This allows us to understand why cutting a magnet does not result in two monopoles (Figure~\ref{fig:magneticforce:magnetcut}): when we cut the bar magnet, we end up with less material, but each piece of material still contains magnetic dipoles that are aligned with each other, each having a North and South pole. Note that it is not the motion of electrons around their nuclei that results in the magnetic field, and that one  requires quantum mechanics and the notion of ``spin'' to describe this all in detail.

Most materials will respond to magnetic fields, but the behaviour is most evident in ``ferromagnetic'' materials, such as iron (Fe). Ferromagnetic materials can be magnetized by an external magnetic field, effectively transforming them into magnets. One can think of a material as containing many little magnetic dipoles (from the motion of the electrons), which themselves are like bar magnets. If that material is ferromagnetic,  an external magnetic field can act on the little ``bar magnets'', orienting them all in the same way, so that the material as a whole becomes magnetic. For some ferromagnetic materials, that common orientation will remain when the external magnetic field is removed, creating a ``permanent magnet''. For other ferromagnetic materials, the common orientation disappears when the external field is removed; those materials are thus attracted to a magnet, but they cannot form a magnet.

\subsubsection{The magnetic force on a moving charge}

\begin{framed}
\textbf{Review}\\
Before proceeding, you may wish to review:

\begin{itemize}
\item Section~\ref{sec:Vectors:vectorproduct} on the vector product.
\end{itemize}
\end{framed}

When an electric charge, $q$, has a velocity, $\vec v$, relative to a magnetic field, $\vec B$, a magnetic force is exerted on the particle:
\begin{equation}
\boxed{\vec F_B = q \vec v \times \vec B}
\end{equation}
We can make a few remarks about the magnetic force:

\begin{itemize}
\item The magnetic force is always perpendicular to the velocity and to the magnetic field (since it is given by their cross-product).
\item The direction of the magnetic force depends on the sign of the charge.
\item The magnetic force can do no work, since it is always perpendicular to the velocity (and thus to displacement).
\item There is no force if the particle's velocity is in the same direction as the magnetic field vector.
\item The force increases with charge, speed, and strength of the magnetic field.
\end{itemize}

\begin{framed}
\textbf{Checkpoint}\\
A proton moves East in Earth's magnetic field, which way is it deflected?

\begin{enumerate}
\item Away from the Earth.
\item Towards the Earth.
\item North.
\item South.
\end{enumerate}

\begin{framed}
\textbf{Answer}\\
\begin{enumerate}
\item
\end{enumerate}
\end{framed}
\end{framed}

\begin{framed}
\textbf{Checkpoint}\\
An electron moves West in Earth's magnetic field, which way is it deflected?

\begin{enumerate}
\item Away from the Earth.
\item Towards the Earth.
\item North.
\item South.
\end{enumerate}

\begin{framed}
\textbf{Answer}\\
\begin{enumerate}
\item
\end{enumerate}
\end{framed}
\end{framed}

\begin{framed}
\textbf{Josh's Thoughts}\\
\{\}
It is very important to remember what each part of the right-hand rule for cross-products represents. To help remember what each finger represents, I say ``velocity'' as I extend my thumb, ``field'' as I extend my index finger, and ``force'' as I extend my middle finger. When using the right hand rule, it is also important to remember the $q$ in the equation $\vec F_B = q\vec v \times \vec B$. This $q$ could be negative, which would mean that the force acts in the opposite direction.

\begin{figure}[!htbp]
\centering
\includegraphics[width=0.4\linewidth]{files/rhr-0a145067c9b8e339abb2d3735e257d33.png}
\caption[]{The way that Josh remembers the right hand rule for magnetism.}
\label{fig:magneticforce:rhr}
\end{figure}

If you find yourself forgetting the right-hand rule on a test or exam, just remember that you can still find the correct answer by setting up a three-dimensional coordinate system and evaluating the cross product.
\end{framed}

You should be somewhat bothered by the fact that the force depends on the velocity of the charge, since velocity depends on the frame of reference from which it is measured. The above equation has a strange implication: if we observe an electron moving in a magnetic field, we will see its motion be deflected by the magnetic field. If we move along with the electron, so that it has a velocity of zero in our frame of reference, we should not see the electron being deflected, since the magnetic force would be zero. Clearly, the motion of the electron cannot depend on the frame of reference from which we observe it. Thus, the only way that this equation can make sense is if the magnetic field also depends on our frame of reference. We will revisit this in a subsequent chapter, but for now, remember that this equation only makes sense if the velocity is measured in the same reference frame as that in which the magnetic field is defined.

Another bothersome issue with the magnetic force is that it appears to depend on the fact that most humans are right-handed. Indeed, the direction of the force requires one to use the right-hand rule, which appears arbitrary. This is a common occurrence in physics, as many quantities are defined using a cross-product. However, no physical quantity can ever depend on our choice of right or left hand for determining cross-products. It turns out that any physical quantity (such as the force on a particle, which will deflect the particle in a clearly identifiable direction that does not depends on a human's choice of right and left), always depends on two successive applications of the right-hand rule. In this case, the direction of the magnetic field is also given by a right-hand rule applied to the moving charges that create the field (as we will see in the next chapter). The successive uses of the right hand twice ``cancel''; one finds that a charge is deflected in the same direction if one had used the left hand to define the magnetic field, and then again the left-hand for the cross-product! We will revisit this issue in the next chapter.

Consider the motion of a charged particle in a region where the magnetic field is uniform (constant in magnitude and direction). If the velocity vector of the particle is perpendicular to the magnetic field, the particle will undergo uniform circular motion, as illustrated in Figure~\ref{fig:magneticforce:cyclotron}.

\begin{figure}[!htbp]
\centering
\includegraphics[width=0.35\linewidth]{files/cyclotron-e4bb14fd18f077d23ca915fb07671ad1.png}
\caption[]{The motion of a charged particle in a uniform magnetic field (out of the page) is uniform circular motion.}
\label{fig:magneticforce:cyclotron}
\end{figure}

Indeed, the force is always perpendicular to the velocity, and the force is constant in magnitude since both the speed and magnetic field remain constant. These are the only conditions required for uniform circular motion. We can easily determine the radius, $R$, of the circle, since the magnetic force is responsible for the centripetal acceleration:
\begin{equation}
F_B &= m\frac{v^2}{R}\\
qvB &= m\frac{v^2}{R}\\
\therefore R &= \frac{mv}{qB}
\end{equation}
The radius is called the ``cyclotron radius''.

\begin{framed}
\textbf{Checkpoint}\\
Is the particle illustrated in Figure~\ref{fig:magneticforce:cyclotron} positively or negatively charged?

\begin{enumerate}
\item The particle is positively charged.
\item The particle is negatively charged.
\item Not enough information to tell.
\item The particle has no charge.
\end{enumerate}

\begin{framed}
\textbf{Answer}\\
\begin{enumerate}
\item
\end{enumerate}
\end{framed}
\end{framed}

Referring to Figure~\ref{fig:magneticforce:cyclotron}, if the velocity of the particle is in the plane of the page (perpendicular to the magnetic field), as illustrated, the particle will undergo uniform circular motion. If the velocity of the particle has a component that is parallel to the magnetic field (for example a component coming out of the page, towards you), the particle will undergo ``helical motion'' (a spiral). The radius of the helix is determined by the component of the velocity, $\vec v_{\perp}$, that is perpendicular the magnetic field:
\begin{equation}
\therefore R &= \frac{mv_{\perp}}{qB}
\end{equation}
The charged particle would also have a component of velocity towards you that is constant, resulting in the spiral motion illustrated in Figure~\ref{fig:magneticforce:helix}. Note that the distance between two spirals (labelled $h$ in the figure) is called the ``pitch'', and is determined by the component of velocity that is parallel to the magnetic field, $\vec v_\parallel$, since that component is not affected by the magnetic force.

\begin{figure}[!htbp]
\centering
\includegraphics[width=0.4\linewidth]{files/helix-31124e86652b21b8c075857e84aca2d0.png}
\caption[]{The helical motion of a charged particle with a component of velocity parallel to the magnetic field. The distance, $h$, between spirals is called the ``pitch''.}
\label{fig:magneticforce:helix}
\end{figure}

\begin{framed}
\textbf{Example 20.1}\\
A particle of unknown charge and unknown mass is observed to undergo uniform circular motion with a period, $T$, when travelling perpendicular to a uniform magnetic field, $B$. What is the ratio of the particle's charge to its mass, $q/m$?

\begin{framed}
\textbf{Solution}\\
We can use the period of the motion to determine the speed of the particle in terms of the radius of the circular path:
\begin{equation}
v = \frac{2\pi R}{T}
\end{equation}
and then use the equation for the cyclotron radius to relate this to the charge-to-mass ratio of the particle:
\begin{equation}
R &= \frac{mv}{qB}\\
  &= \frac{2\pi R m}{qBT}\\
\therefore \frac{q}{m} &= \frac{2\pi}{BT}
\end{equation}
\textbf{Discussion:} When a charged particle undergoes uniform circular motion in a magnetic field, the radius of the motion depends on the particle's charge-to-mass ratio. This can often be used to measure the mass of, say, an ion, if the charge of the ion is known (usually one or two units of the electron charge). A mass spectrometer makes use of this principle in order to determine the composition of a sample. The sample is vaporized and ionized, the ions are then accelerated using an electric potential difference, before they undergo uniform circular motion. Ions of different masses (and same charge) will then undergo circular motion with different radii, which allows their masses to be determined, and thus the composition of the sample to be known.
\end{framed}
\end{framed}

\subsubsection{The magnetic force on a current-carrying wire}

\begin{framed}
\textbf{Review}\\
Before proceeding, you may wish to review:

\begin{itemize}
\item Section~\ref{sec:current:micromodel} on the microscopic model of current.
\end{itemize}
\end{framed}

In this section, we examine the force that is exerted by a magnetic field on a wire that carries electric current. Since a current is formed by moving charges, it is natural to expect that a wire that carries current will experience a force if immersed in a magnetic field.

Consider a vertical wire with cross-sectional area, $A$, carrying current, $I$, upwards that is immersed in a uniform magnetic field, $\vec B$, into the page, as illustrated in Figure~\ref{fig:magneticforce:microforce}. Inside the wire, on average, electrons have a drift velocity, $\vec v_d$, in the downwards direction (since they move in the direction opposite to that of conventional current).

\begin{figure}[!htbp]
\centering
\includegraphics[width=0.3\linewidth]{files/microforce-e8a90f317eda4fec4aff9ff43bc985d4.png}
\caption[]{A section of wire carries conventional current, $I$, upwards while being immersed in a uniform magnetic field, $\vec B$, into the page. We introduce a vector, $\vec l$, to represent a section of wire of length $l$ carrying current in the direction of $\vec l$.}
\label{fig:magneticforce:microforce}
\end{figure}

A single electron (with charge $q= -e$) will experience a magnetic force, $\vec F_e$, given by:
\begin{equation}
\vec F_e = -e \vec v_d \times \vec B
\end{equation}
as illustrated in Figure~\ref{fig:magneticforce:microforce}. A section of wire of length, $l$, will contain $N=nAl$ drifting electrons, where $n$ is the density of free electrons for the wire (the number of electrons per unit volume that are available to produce a current). Thus, the magnetic force on that section of wire will be $N$ times the force on a single electron:
\begin{equation}
\vec F = N\vec F_e = nAl (-e \vec v_d \times \vec B)=-nAle \vec v_d \times \vec B
\end{equation}
Recall the microscopic model of current to relate the drift velocity to the conventional current in the wire:
\begin{equation}
I &= -nAev_d
\end{equation}
where the minus sign indicates that negative electrons flow in the opposite direction from the conventional current. We also introduce a vector, $\vec l$, with a magnitude equal to the length of the section of wire, and a direction that is parallel to the conventional current (thus anti-parallel to the electron drift velocity). The force on the section of the length, $l$, of the wire is thus given by:
\begin{equation}
\vec F = -nAle \vec v_d \times \vec B
\end{equation}
\begin{equation}
\boxed{ \vec F= I \vec l \times \vec B}
\end{equation}

\begin{framed}
\textbf{Olivia's Thoughts}\\
If you forget this formula, you can obtain it rather easily from $\vec F=q\vec v\times \vec B$. For simplicity, we assume velocity is constant, so that velocity is the displacement, $\vec l$, per unit time. We can then rewrite the formula for the force as:
\begin{equation}
\vec F&=q\vec v\times \vec B\\
&=q\frac{\vec l}{t}\times \vec B\\
&=\frac{q}{t}\vec l\times \vec B.
\end{equation}
In the last line, I have rewritten the equation to highlight the charge per unit time $q/t$, which is just the (constant) current, $I$. Using the current, the equation becomes,
\begin{equation}
\vec F=I\vec l\times \vec B,
\end{equation}
as we found previously. This is a less rigorous derivation than what we did above, but it is helpful if you just need to remind yourself of the formula.
\end{framed}

\begin{framed}
\textbf{Checkpoint}\\
\begin{figure}[!htbp]
\centering
\includegraphics[width=0.35\linewidth]{files/horseshoemagnet-90d031fef6cc196c5bd44a9829c50a69.png}
\caption[]{A current carrying wire moving through a magnetic field.}
\label{fig:magneticforce:horseshoemagnet}
\end{figure}

In which direction does the magnetic force point on the current-carrying wire that is placed in the magnetic field between the poles of the horseshoe magnet shown in Figure~\ref{fig:magneticforce:horseshoemagnet}?

\begin{enumerate}
\item Up.
\item Down.
\item Into the page.
\item Out of the page.
\end{enumerate}

\begin{framed}
\textbf{Answer}\\
\begin{enumerate}
\item
\end{enumerate}
\end{framed}
\end{framed}

Note that if the wire is not straight, then we can model the wire as being made of many infinitesimally short sections (Figure~\ref{fig:magneticforce:bentwire}), of length $dl$, and sum the forces on those sections to get the total force on a section of length $L$:
\begin{equation}
\vec F = \int_0^L I d\vec l \times \vec B
\end{equation}
\begin{figure}[!htbp]
\centering
\includegraphics[width=0.3\linewidth]{files/bentwire-55e9968fb877a8a926c6dd1b897ed8e1.png}
\caption[]{The magnetic force on a curved current-carrying wire is obtained by modelling the forces exerted on infinitesimal sections of wire, each with length $d\vec l$, and summing together those forces to get the total force on the wire.}
\label{fig:magneticforce:bentwire}
\end{figure}

\begin{framed}
\textbf{Example 20.2}\\
A wire carrying current $I$ is bent so as to have a semi-circular section with radius $R$, as shown in Figure~\ref{fig:magneticforce:semicircle}. The wire is immersed in a uniform magnetic field, $\vec B$, that is perpendicular to the plane of the wire, as shown. Using the given coordinate system, what is the net force on the wire?

\begin{figure}[!htbp]
\centering
\includegraphics[width=0.4\linewidth]{files/semicircle-f5436b96732b568551593ddb5463e158.png}
\caption[]{A current-carrying wire with a semi-circular section is immersed in a uniform magnetic field.}
\label{fig:magneticforce:semicircle}
\end{figure}

\begin{framed}
\textbf{Solution}\\
We can model the wire as being made of three sections: a straight section carrying current in the positive $y$ direction, a curved section, and another straight section carrying current in the negative $y$ direction.

Consider the first straight section, carrying current in the positive $y$ direction. The force on that section of wire, by the right hand rule, will be towards the left (negative $x$ direction):
\begin{equation}
F_S &= I \vec l \times \vec B\\
&= I (l\hat y) \times (-B\hat z)\\
&= -IlB (\hat y \times \hat z)\\
&=-IlB\hat x
\end{equation}
where $l$ is the (unknown) length of that section of wire. The force exerted on the other straight section of wire will have the same magnitude, but the opposite direction (since the current, and thus the vector $\vec l$, is in the opposite direction). Thus, the forces from the two straight sections of the wire cancel, as illustrated in Figure~\ref{fig:magneticforce:semicircle_sol}.

\begin{figure}[!htbp]
\centering
\includegraphics[width=0.4\linewidth]{files/semicircle_sol-3a4aea39b14c75d335f7088ec6b14073.png}
\caption[]{The magnetic force on different sections of wire.}
\label{fig:magneticforce:semicircle_sol}
\end{figure}

In order to calculate the force exerted on the semi-circular section, we need to add together the forces exerted on the infinitesimal sections of the wire that make up that section. Consider the magnetic force on the two infinitesimal sections illustrated in Figure~\ref{fig:magneticforce:semicircle_sol}. The $x$ components of the forces will cancel, whereas the $y$ components will add. Thus, by symmetry, we anticipate that the net force on the semi-circular section will be in the positive $y$ direction.

Consider the small force on the section of wire located at an angle, $\theta$, as illustrated in Figure~\ref{fig:magneticforce:semicircle_sol}. We can write the vector $d\vec l$ as:
\begin{equation}
d\vec l = dl(\sin\theta\hat x + \cos\theta \hat y)
\end{equation}
Thus, the infinitesimal force on that section of wire is given by:
\begin{equation}
d\vec F &= I d\vec l \times \vec B \\
&= I dl(\sin\theta\hat x + \cos\theta \hat y)\times (-B\hat z)\\
&=-IBdl (\sin\theta\hat x \times \hat z + \cos\theta \hat y \times \hat z)\\
&=-IBdl (-\sin\theta \hat y + \cos\theta\hat x) \\
&= IBdl\sin\theta \hat y - IBdL\cos\theta \hat x\\
& = dF_y\hat y + dF_x \hat x
\end{equation}
where, in the last line, we explicitly wrote out the $x$ and $y$ components of the infinitesimal force vector. In order to sum together these infinitesimal forces, it is most convenient to use the angle $\theta$ to identify each segment. $d\theta$ is related to $dl$, since $dl$ is the length of the circle subtended by the infinitesimal angle $d\theta$:
\begin{equation}
dl = Rd\theta
\end{equation}
Summing together all of the $y$ components of the infinitesimal forces:
\begin{equation}
F_y = \int dF_y = \int_0^\pi IBR\sin\theta d\theta=IBR \int_0^\pi\sin\theta d\theta=2IBR
\end{equation}
Note that the $x$ components sum to zero, as we predicted from symmetry:
\begin{equation}
F_x = \int dF_x = -\int_0^\pi IBR\cos\theta d\theta=-IBR \int_0^\pi\cos\theta d\theta=0
\end{equation}
The net force on the wire is thus given by:
\begin{equation}
\vec F = 2IBR\hat y
\end{equation}
\textbf{Discussion:} In this example we found the magnetic force on a curved section of current-carrying wire. The calculation was simplified by symmetry arguments, as we could use the right hand rule to anticipate that the force would have no component in the $x$ direction. This is because there is as much current flowing in the positive $y$ direction as there is in the negative $y$ direction, so that the corresponding forces cancel. There is however a net flow of charges in the positive $x$ direction, leading to a net force in the positive $y$ direction. As a corollary, the net magnetic force on any closed loop of current must be zero.
\end{framed}
\end{framed}

\subsubsection{The torque on a current-carrying loop}

\begin{framed}
\textbf{Review}\\
Before proceeding, you may wish to review:

\begin{itemize}
\item Section~\ref{sec:rotationaldynamics:torque} on torque.
\item Section~\ref{sec:chargesfields:electricdipole} on electric dipoles.
\end{itemize}
\end{framed}

As noted in Example~20.2, the net magnetic force on any closed loop immersed in a uniform magnetic field is zero. Consider, for example, the current-carrying rectangular loop of height $h$ and width $w$, immersed in a uniform magnetic field, $\vec B$, as illustrated in Figure~\ref{fig:magneticforce:rectangleloop} (note that the field is not perpendicular to the plane of the loop, as it was in Example~20.2).

\begin{figure}[!htbp]
\centering
\includegraphics[width=0.4\linewidth]{files/rectangleloop-8b237152477798e6ba19d2377c451e8d.png}
\caption[]{A rectangular loop carrying counter-clockwise current in a uniform magnetic field.}
\label{fig:magneticforce:rectangleloop}
\end{figure}

The magnetic force on the two horizontal sections of the wire are zero, since the current is co-linear with the magnetic field along those sections. In the left vertical section (with current flowing downwards), the magnetic force is out of the page (positive $z$ direction), and is given by:
\begin{equation}
\vec F = IhB\hat z
\end{equation}
Similarly, the force on the right vertical section (with current flowing upwards) will have the same magnitude but the opposite direction. The net force on the loop is thus zero.

However, the net torque on the loop about its vertical axis of symmetry (shown by the vertical dashed line in the figure) is not zero. The total torque is found by summing the torques from the forces exerted on the two vertical sections of wire:
\begin{equation}
\vec \tau &= \vec r\times \vec F + (-\vec r \times - \vec F)\\
&= 2 \vec r \times F = 2 \left(-\frac{w}{2}\hat x\right) \times IhB\hat z = IBwh (-\hat x\times \hat z)\\
\therefore \vec \tau&=IBwh (\hat y)
\end{equation}
where $\vec r$ is the vector from the axis of rotation to the location where the force is exerted.

\paragraph{Magnetic dipole moment}\label{sec:MagneticForce:dipolemoment}

Describing the torque on a loop can be difficult in three dimensions, so we introduce the ``magnetic dipole moment'' to simplify the description.

If a closed loop carries a current, $I$, the magnetic dipole moment vector, $\vec \mu$, is defined such that it has a magnitude:
\begin{equation}
\mu = IA
\end{equation}
where $A$ is the area enclosed by the loop. The direction of the magnetic dipole moment vector is such that it is perpendicular to the surface defined by the loop. This leaves two possible directions, and the correct option is given by the right-hand rule for axial vectors; by curling the fingers in the direction of the current, the thumb will point in the direction of the magnetic dipole moment. This is illustrated in Figure~\ref{fig:magneticforce:momenthand}.

\begin{figure}[!htbp]
\centering
\includegraphics[width=0.4\linewidth]{files/momenthand-86ce0b6ea5974b01e75dd2e12f255f73.png}
\caption[]{The right hand rule for axial vectors is used to determine the direction of the magnetic dipole moment vector for a loop carrying current, $I$.}
\label{fig:magneticforce:momenthand}
\end{figure}

In terms of the magnetic dipole moment, the torque on a loop with magnetic dipole moment, $\vec \mu$, immersed in a magnetic field, $\vec B$, is given by:
\begin{equation}
\boxed{\vec \tau = \vec \mu \times \vec B}
\end{equation}
The magnitude of the torque is given by:
\begin{equation}
\tau =\mu B \sin\theta
\end{equation}
where $\theta$ is the angle between the magnetic dipole moment and the magnetic field vectors.

We can verify that this formula gives the correct torque for the rectangular loop in Figure~\ref{fig:magneticforce:rectangleloop} that we calculated above. The magnetic dipole moment of that loop is given by:
\begin{equation}
\vec \mu = IA \hat z = Iwh\hat z
\end{equation}
where the direction of the vector is given by the right hand rule for axial vectors (out of the page, since the current is in the counter-clockwise direction in Figure~\ref{fig:magneticforce:rectangleloop}). The torque on the loop is thus:
\begin{equation}
\vec \tau = \vec \mu \times \vec B = (Iwh\hat z) \times (B\hat x) = IBwh (\hat y)
\end{equation}
as we found previously.

The magnetic dipole moment can be used to describe a current-carrying loop in a magnetic field. That is, instead of drawing a loop carrying current, we can equivalently simply draw the associated magnetic dipole moment vector. This is useful because the magnetic dipole moment vector behaves in the same way as a bar magnet (with the tip of the arrow acting like a North pole). Indeed, a magnetic field will always create a torque that will try to align the magnetic dipole moment with the magnetic field, just as the needle of a compass experiences a torque if it is not aligned with the magnetic field of the Earth. The torque from the magnetic field is then zero when the magnetic dipole moment is parallel to the magnetic field (as the cross-product between co-linear vectors is zero).

Figure~\ref{fig:magneticforce:looptorque} shows a way to visualize a current-carrying loop in a magnetic field using its magnetic dipole moment vector, $\vec \mu$.

\begin{figure}[!htbp]
\centering
\includegraphics[width=0.4\linewidth]{files/looptorque-b927579e38704625c1187c37c60b60b6.png}
\caption[]{Three loops of current with different orientations relative to a uniform magnetic field. The loops are seen from above, and the current is shown coming in and out of the page on each loop, along with the corresponding magnetic dipole moment vector.}
\label{fig:magneticforce:looptorque}
\end{figure}

Three loops are shown (as lines), seen from above, and the direction of the current in each loop is shown as going in or out of the page. Equivalently, one can simply draw the magnetic dipole moment vector for each loop (perpendicular to the plane of the loop). For the top loop, the magnetic dipole moment is parallel to the magnetic field, so the magnetic field exerts no torque. For the middle loop, the magnetic dipole moment makes an angle $\theta$ with the magnetic field vector, so that the torque on that loop has a magnitude given by $\tau=\mu B \sin\theta$, and points into the page (clockwise rotation). The bottom loop makes an angle of $-\pi/2$ with the magnetic field, which results in a torque in the counter-clockwise direction. In all cases, the torque is such that it always tries to align the magnetic dipole moment vector with the magnetic field, just as if the magnetic dipole moment were the needle of a compass.

\begin{framed}
\textbf{Example 20.3}\\
Determine the magnetic dipole moment of the electron orbiting a hydrogen atom, if you assume that the electron is in a circular orbit with a radius of $R=0.5 \overset{\circ}{\rm A}$.

\begin{framed}
\textbf{Solution}\\
As the electron orbits around the circle, it results in a circular loop of current, $I$. The current is the rate at which charge passes through a point per unit time. If the electron orbit has a period, $T$, then the corresponding current, $I$, is given by:
\begin{equation}
I=\frac{\Delta Q}{\Delta t} = \frac{e}{T}
\end{equation}
The centripetal force on the electron is provided by the Coulomb force, $F_C$, exerted by the proton, which allows us to obtain the orbital speed, and thus the period of the orbit:
\begin{equation}
F_C &= m\frac{v^2}{R}\\
k\frac{e^2}{R^2}&= m\frac{v^2}{R}\\
\therefore v &=\sqrt{\frac{ke^2}{mR}}\\
\therefore T &= \frac{2\pi R}{v}
\end{equation}
The magnetic dipole moment is then given by:
\begin{equation}
\mu &= IA\\
&= \frac{e}{T} \pi R^2\\
& = \frac{ev}{2\pi R} \pi R^2\\
&=\frac{1}{2} evR\\
&=\frac{1}{2} \sqrt{\frac{ke^4R}{m}}\\
&=\frac{1}{2} \sqrt{\frac{(9e9 {\rm N/C^{2}\cdot m^2})(1.6e-19 {\rm C})^4(0.5 \overset{\circ}{\rm A})}{(9.1e-31 {\rm kg})}}\\
&=9\times 10^{24} {\rm A\cdot m^2}
\end{equation}
\textbf{Discussion:} In this example we calculated the orbital magnetic dipole moment of the electron in a hydrogen atom. This was a very simple model, since in reality, electrons do not orbit atoms in circular orbits, and one must use quantum mechanics to describe the motion precisely.
\end{framed}
\end{framed}

\paragraph{Potential energy for a magnetic moment in a magnetic field}

A magnetic dipole moment in a magnetic field behaves in the same way as an electric dipole in an electric field. By analogy, we can define a potential energy, $U$, for a magnetic dipole moment, $\vec \mu$, in a magnetic field, $\vec B$:
\begin{equation}
\boxed{U =-\vec \mu \cdot \vec B =- \mu B \cos\theta}
\end{equation}
where $\theta$ is the angle between the magnetic moment and the magnetic field. If a magnetic dipole is not aligned with a magnetic field and it is released, it will start to rotate (gain rotational kinetic energy) until it reaches a minimum in potential energy ($\theta = 0$). The magnetic moment would oscillate back and forth about $\theta =0$ if there are no losses. Note that the point where $\theta = \pi$, is an unstable equilibrium.

\begin{framed}
\textbf{Checkpoint}\\
When a magnetic dipole moment is parallel with a magnetic field and points in the same direction as the magnetic field, it will have...

\begin{enumerate}
\item ... its maximum torque and maximum potential energy.
\item ... its maximum torque and minimum potential energy.
\item ... its minimum torque and maximum potential energy.
\item ... its minimum torque and minimum potential energy.
\end{enumerate}

\begin{framed}
\textbf{Answer}\\
\begin{enumerate}[resume]
\item
\end{enumerate}
\end{framed}
\end{framed}

\begin{framed}
\textbf{Checkpoint}\\
When a magnetic dipole moment is placed such that the torque from the magnetic field is maximized, it will have...

\begin{enumerate}
\item ... zero potential energy.
\item ... its minimum potential energy.
\end{enumerate}

\begin{framed}
\textbf{Answer}\\
\begin{enumerate}
\item
\end{enumerate}
\end{framed}
\end{framed}

\subsubsection{The Hall Effect}

Figure~\ref{fig:magneticforce:hallV} shows a simple circuit to illustrate the Hall effect. A flat slab of metal, with width $w$, is connected to a battery, so that current flows through the slab. The slab is immersed in a uniform magnetic field, $\vec B$, that is perpendicular to the plane of the slab.

\begin{figure}[!htbp]
\centering
\includegraphics[width=0.4\linewidth]{files/hallV-6f069422571cd8cd2b314a481b286e52.png}
\caption[]{Illustration of the Hall effect, as electrons flow through a slab that is immersed in a magnetic field, the magnetic force pushes them to one side, creating an electric potential difference, $\Delta V_{Hall}$, transverse to the motion of the current through the slab.}
\label{fig:magneticforce:hallV}
\end{figure}

As the electrons enter the right-hand side of the slab (Figure~\ref{fig:magneticforce:hallV}) and drift towards the left, they will experience an upwards force from the magnetic field. As they move to the left through the slab, they also move upwards and ``pile up'' on that side of the slab. There will thus be an excess of negative charge on the top side of the slab, leading to an electric potential difference between the top and the bottom of the slab. This potential difference is called the ``Hall potential'', $\Delta V_{Hall}$. An equilibrium between the magnetic force and the electric force associated with the Hall potential is quickly reached, so that the Hall potential remains constant.

If we model the slab as two parallel plates, with a potential difference, $\Delta V_{Hall}$, between them, the electric field in the slab is constant and given by:
\begin{equation}
E= \frac{\Delta V_{Hall}}{w}
\end{equation}
The equilibrium condition (that the electric force on an electron is equal to the magnetic force) is given by:
\begin{equation}
F_E &= F_B\\
eE &= ev_dB\\
\frac{\Delta V_{Hall}}{w} &= v_d B\\
\therefore \Delta V_{Hall}&= v_d wB
\end{equation}
If the drift velocity of electrons is known, then the Hall effect can be used to measure the strength of the magnetic field by simply measuring the Hall voltage. This is the most common way to measure the strength of a magnetic field (and the device to do so is called a Hall probe). Conversely, if the magnetic field is known, the Hall effect can be used to characterize the drift velocity of electrons and other microscopic quantities for the material from which the Hall probe is made.

The Hall effect allows us to determine that it is negative charges that flow, and not positive charges. Indeed, consider Figure~\ref{fig:magneticforce:hallV}, but replace the electrons with positive charges flowing to the right, which is equivalent as far as analysing the circuit goes. In this case, those positive charges will be deflected upwards. Thus, if positive charges flow, the top side of the Hall probe becomes positive, whereas it becomes negative if it is negative charges that flow. By measuring the sign of the Hall potential, one can show that it is electrons that flow in an electric current.

\subsubsection{Applications}

In this section, we briefly outline a few applications of the magnetic force.

\paragraph{Velocity selector and mass spectrometer}

In Example~20.1, we described how charged particles with different charge-to-mass ratios will undergo uniform circular motion with different radii, if they all have the same speed. This principle is used in mass spectrometers, which are devices that are able to detect trace amounts of matter in a sample. For example, when your bag gets swiped with a sticky tape at a security check at the airport, that piece of sticky tape is then analysed by a mass spectrometer.

The tape is vaporized in a way to ionize the atoms on the tape. The ions are then accelerated through an electric potential difference and then pass through a region with a magnetic field. The ions typically execute half of a circular orbit before being detected, as illustrated in Figure~\ref{fig:magneticforce:massspec}. The charge-to-mass ratio of the ions is determined from the radius of their orbit. Usually, their charge is either one or two times the electron charge, allowing their mass to be determined.

\begin{figure}[!htbp]
\centering
\includegraphics[width=0.4\linewidth]{files/massspec-154fd37f36c600e9f118cf7854e2640c.png}
\caption[]{Illustration of how a mass spectrometer can separate ions based on their charge-to-mass ratio. A detector is placed to measure the number of ions that appear at each radius, allowing the composition of a sample to be determined.}
\label{fig:magneticforce:massspec}
\end{figure}

In order for the mass spectrometer to function as designed, it is important that all of the charged particles enter the region of magnetic field with the same speed. A velocity selector is a device that combines perpendicular electric and magnetic fields in order to select only particles of a certain speed, regardless of their mass. The velocity selector is illustrated in Figure~\ref{fig:magneticforce:vselector}

\begin{figure}[!htbp]
\centering
\includegraphics[width=0.4\linewidth]{files/vselector-f385c931befd350dfceb4d97c71b0e72.png}
\caption[]{Illustration of a velocity selector. Only charged particles with a specific speed can make it through without colliding with one of the plates.}
\label{fig:magneticforce:vselector}
\end{figure}

In a velocity selector, both an electric and a magnetic force are exerted. Figure~\ref{fig:magneticforce:vselector} shows a positive particle moving toward the right with speed $v$. The particle will experience an upwards electric force and a downwards magnetic force. If those two forces are equal, then the particle will move in a straight line. If, instead, one of the forces is larger than the other, the particle will be deflected and hit one of the charged plates. The condition for the two forces to be equal is given by:
\begin{equation}
F_B &= F_E\\
qvB &= qE\\
\therefore v=\frac{E}{B}
\end{equation}
Thus, the electric and magnetic fields can be tuned so that their ratio gives the desired speed. Note that the speed selector works regardless of the sign of the charge or its mass, which makes it ideal to filter the particles entering a mass spectrometer.

\paragraph{Galvanometer}

The galvanometer makes use of the magnetic force in order to measure electric current. In a galvanometer, a coil (composed of many loops) is placed in a known magnetic field. As current passes through the coil, the magnetic dipole moment of the coil increases, and the magnetic field exerts a torque on the coil. The torque from the magnetic force is balanced by the restoring torque of a torsional spring (a coil spring). A needle is attached to the coil, and the deflection of the needle, proportional to the current in the coil, is then a measure of the current through the coil. A galvanometer is illustrated in Figure~\ref{fig:magneticforce:galvanometer}.

\begin{figure}[!htbp]
\centering
\includegraphics[width=0.7\linewidth]{files/galvanometer-b584663c2138305af4139116ae45af13.png}
\caption[]{Illustration of a galvanometer. Current passes through the coil, and the coil rotates due to the torque from a magnetic field created by a permanent magnet. The torque from the magnetic force is balanced by a torsional spring.}
\label{fig:magneticforce:galvanometer}
\end{figure}

\paragraph{Electric motor}

In an electric motor, a current-carrying coil is immersed in a fixed and uniform magnetic field. As current passes through the coil, the coil experiences a torque and rotates. Once the coil has reached a position where its magnetic dipole moment vector is parallel to the magnetic field, the direction of the current is reversed, so that the coil continues to feel a torque for another half turn, until the direction of the current is reversed again. This is illustrated in Figure~\ref{fig:magneticforce:dcmotor}.

\begin{figure}[!htbp]
\centering
\includegraphics[width=0.65\linewidth]{files/dcmotor-719f79ce1ca8e1c208426aca4192c7e9.png}
\caption[]{Illustration of a DC electric motor. Current circulates in the coil resulting in a torque from the magnetic field. Once the coil is aligned with the magnetic field, the direction of the current in the coil is inverted, so that the coil continues to feel a torque. The current is inverted using mechanical brushes that reverse the leads on the coil every half turn.}
\label{fig:magneticforce:dcmotor}
\end{figure}

\paragraph{Cathode ray tube}

The cathode ray tube is the main component of old televisions and monitors. In those devices, a beam of electrons is accelerated by an electric potential difference. The electrons then hit a phosphorescent screen that emits light when the electrons collide with the screen. A magnetic field is used to deflect the electron beam to different parts of the screen and create the desired image, in a rapid sweeping motion, fast enough that the human eye cannot detect the sweeping motion. An example of a cathode ray tube is shown in Figure~\ref{fig:magneticforce:crt}.

\begin{figure}[!htbp]
\centering
\includegraphics[width=0.6\linewidth]{files/crt-97511c9399fef59cfc4ace3b6c9ba420.png}
\caption[]{Illustration of a cathode ray tube from a side view (top) and a top view (bottom). A magnetic field is used to deflect a beam of electrons onto a screen. The perpendicular magnetic fields are used to sweep the beam rapidly across the whole screen to create an image.}
\label{fig:magneticforce:crt}
\end{figure}

\paragraph{Loudspeaker}

In a loudspeaker, a coil is immersed in a non-uniform magnetic field. The coil is attached to a membrane so that the membrane moves with the coil when a magnetic force is exerted on the coil. AC current circulates in the coil, with the same frequencies as the desired sound. The coil then moves at those frequencies and the membrane then displaces the air, creating the desired sound waves.

\begin{figure}[!htbp]
\centering
\includegraphics[width=0.6\linewidth]{files/speaker-2732a20fcc5a641b167b5f7a3c7be97e.png}
\caption[]{Illustration of a loud speaker. As current moves through the coil, the coil is pushed back and forth by the magnetic force exerted by a permanent magnet. The motion is transferred to a membrane that move the air and creates the sound wave.}
\label{fig:magneticforce:speaker}
\end{figure}

\subsubsection{Summary}

In order to describe the magnetic force, we introduced the magnetic field, $\vec B$. While there are some similarities with the electric field, the key difference in magnetism is that there are no ``magnetic charges'' (so-called monopoles), and magnets thus always have a North \textit{and} a South pole. As a result, magnetic field lines never end and must always form closed loops. The magnetic field points in the direction of the force that would be exerted on the North pole of a magnet placed at that position.

Electric charges can feel a force from a magnetic field only if they are moving relative to the frame of reference in which the magnetic field is described. If a charge, $q$, has velocity, $\vec v$, in a magnetic field, $\vec B$, it will feel a magnetic force given by:
\begin{equation}
\vec F_B =q \vec v \times \vec B
\end{equation}
The magnetic force can do no work, since it always acts in a direction perpendicular to the velocity (and thus to the displacement). The magnetic field acts in opposite directions for charges of opposite signs.

In a uniform magnetic field, a charged particle with charge, $q$, mass $m$, and velocity vector, $\vec v$, perpendicular to a magnetic field, $\vec B$, will undergo uniform circular motion, with a cyclotron radius, $R$, given by:
\begin{equation}
R &= \frac{mv}{qB}
\end{equation}

A straight wire of length, $l$, carrying current, $I$, will experience a magnetic force in a magnetic field, $\vec B$:
\begin{equation}
\vec F_B = I \vec l \times \vec B
\end{equation}
where the vector $\vec l$ points in the same direction as the current.

If the wire is curved (or the magnetic field changes direction along the wire), then we can integrate the force, $d\vec F$, exerted on each infinitesimal section of wire with length, $d\vec l$. Again, the direction of $d\vec l$ is in the same direction as the current in the wire. The infinitesimal force on an infinitesimal section of wire, is given by:
\begin{equation}
d\vec F = I d\vec l \times \vec B
\end{equation}

A closed loop of wire carrying current will experience no net force in a uniform magnetic field. However, it will experience a torque, if the loop is not ``aligned'' with the magnetic field (the torque is zero if the magnetic field is perpendicular to the plane of the loop). We define the magnetic dipole moment, $\vec \mu$ of a loop of wire carrying current, $I$, to be a vector with magnitude:
\begin{equation}
\mu = IA
\end{equation}
where $A$ is the area enclosed by the loop. The magnetic dipole moment vector is perpendicular to the plane of the loop, and points in the direction given by the right-hand rule for axial vectors applied to the current (think of the current as rotating in the loop).

The torque from a magnetic field, $\vec B$, exerted on a loop with a magnetic dipole moment, $\vec \mu$, is given by:
\begin{equation}
\vec \tau = \vec \mu \times \vec B
\end{equation}
The torque is zero when the magnetic dipole moment vector is parallel to the magnetic field vector (corresponding to the loop being ``aligned'' with the magnetic field). One can think of the magnetic dipole moment as a small bar magnet, or the needle of a compass, that always experiences a torque to align it with a magnetic field.

We can define the potential energy of a magnetic dipole moment in a magnetic field as:
\begin{equation}
U= -\vec \mu \cdot \vec B = \mu B \cos\theta
\end{equation}

The Hall effect can be observed when current flows through a slab that is immersed in a magnetic field that is perpendicular to the slab. As the electrons move longitudinally through the slab, they will also be pushed to one side by the magnetic force, resulting in an excess of negative charge on that side. An electric potential difference (the ``Hall potential'') is then established between the two sides of the slab (in the direction perpendicular to the motion of the electrons). The Hall potential is given by:
\begin{equation}
\Delta V_{Hall}&= v_d wB
\end{equation}
where $w$ is the width of the slab in the perpendicular direction, $B$ is the strength of the magnetic field, and $v_d$ is the drift velocity of electrons. The most common use of the Hall effect is to build a Hall probe to measure magnetic fields. However, Hall probes can also measure the drift velocity of electrons and other microscopic properties. The sign of the Hall potential also indicates the sign of the charges moving in the slab.

There are many applications of the magnetic force in our daily lives, including electric motors, loudspeakers, cathode ray tubes, mass spectrometers, and galvanometers.

\begin{framed}
\textbf{Important Equations}\\
\textbf{Magnetic force on a moving charge:}
\begin{equation}
\vec F_B &= q\vec v\times \vec B
\end{equation}

\textbf{Magnetic force on a current-carrying wire:}
\begin{equation}
\vec F_B = I \vec l \times \vec B
\end{equation}

\textbf{Cyclotron radius:}
\begin{equation}
R &= \frac{mv}{qB}
\end{equation}

\textbf{Magnetic dipole moment:}
\begin{equation}
\mu = IA
\end{equation}

\textbf{Torque on a magnetic dipole:}
\begin{equation}
\vec \tau &= \vec \mu \times \vec B
\end{equation}
\end{framed}

\begin{framed}
\textbf{Important Definitions}\\
\begin{itemize}
\item \textbf{Magnetic field:} A field used to model the magnetic force. SI units: ${\rm \left[{T}\right]}$. Common variable(s): $\vec B$.
\item \textbf{Magnetic dipole moment:} A property of an object which describes the torque it will experience in a magnetic field. SI units: ${\rm \left[{C\cdot m^2\cdot s^{ -1}}\right]}$. Common variable(s): $\vec \mu$.
\end{itemize}
\end{framed}

\subsubsection{Thinking about the material}

\begin{framed}
\textbf{Reflect and research}\\
\begin{itemize}
\item When was magnetism first discovered?
\item What is the origin of the word ``magnetism''?
\item What experiments support that magnetic monopoles do not exist?
\item What did J.J. Thomson measure, and how?
\item How do debit and credit cards use magnetism?
\end{itemize}
\end{framed}

\begin{framed}
\textbf{To try at home}\\
\begin{itemize}
\item Attempt to construct a compass using household materials.
\end{itemize}
\end{framed}

\begin{framed}
\textbf{To try in the lab}\\
\begin{itemize}
\item Propose an experiment to measure the magnitude of Earth's magnetic field.
\item Propose an experiment to construct a galvanometer and test its accuracy.
\end{itemize}
\end{framed}

\subsubsection{Sample problems and solutions}

\paragraph{Problems}

\begin{framed}
\textbf{Problem 20.1}\\
A cathode ray tube in a television accelerates an electron from rest using a potential difference of $\Delta V=500 {\rm V}$. Once it exits the tube, the electron must be deflected upwards by a distance $h=3 {\rm cm}$ using a uniform magnetic field, $\vec B$, before striking the phosphorescent screen, which is a distance $d= 5 {\rm cm}$ away. What direction and magnitude must the magnetic field have in order to steer the electron towards its destination?
\end{framed}

\begin{framed}
\textbf{Problem20.2}\\
A galvanometer has a square coil with a side length of $a=2.5 {\rm cm}$ and $N=70 {\rm }$ loops between two magnets which generate a radial magnetic field of $B=8 {\rm mT}$. When a current runs through the coil, it generates a torque which is opposed by a spring with a torsional spring constant of $\kappa = 1.5\times 10^{ -8} {\rm Nm~rad^{ -1}}$. If the deflection of the galvanometer's needle is $0.7 {\rm rad}$, what is the current running through the coil?
\end{framed}

\begin{framed}
\textbf{Problem 20.3}\\
Integrate the torque over a circular path, using the equation $d\vec F = Id\vec l \times \vec B$, to show that the torque exerted on a circular loop of radius, $R$, carrying current, $I$, immersed in a uniform magnetic field, $\vec B$, has a magnitude given by $\tau=\mu B$, where $\vec \mu$ is the magnetic dipole moment of the loop. You may simplify the problem by modelling the loop when its magnetic moment is perpendicular to the magnetic field.
\end{framed}

\paragraph{Solutions}

\begin{framed}
\textbf{Solution 20.1}\\
First, we determine the velocity of the electron that were accelerated over a potential difference of $\Delta V=500 {\rm V}$. Their kinetic energy is given by their charge times the potential difference::
\begin{equation}
K &= e\Delta V \\
\frac{1}{2} mv^2 &= e\Delta V\\
\therefore v &= \sqrt{\frac{2e\Delta V}{m}}= \sqrt{\frac{2(1.602e-19 {\rm C})(500 {\rm V})}{(9.109e-31 {\rm kg})}}\\
 &= 1.326e7 {\rm ms^{-1}}
\end{equation}
Now that we have the velocity, we must determine the direction of the magnetic field. We know that the electron is moving directly towards the phosphorescent screen (which we will define as $\vec x$) and the electron must be deflected directly upwards (which we will define as $\vec z$). Knowing this, we can use the right hand rule to quickly determine that the magnetic force will be acting in the $-\vec y$ direction.

In the region with a magnetic field, the electron will undergo uniform circular motion with a radius give by the cyclotron radius, $R$:
\begin{equation}
R=\frac{mv}{qB}
\end{equation}

We thus need to determine the radius of that circle for the electron to arrive that desired location on the screen. A section of the circle about which the electron moves is illustrated in Figure~\ref{fig:magneticforce:deflection}.

\begin{figure}[!htbp]
\centering
\includegraphics[width=0.2\linewidth]{files/deflection-a76b4733fa897d4bb031ad379de8324a.png}
\caption[]{Deflection of an electron moving in a uniform magnetic field.}
\label{fig:magneticforce:deflection}
\end{figure}

From geometry and Pythagoras' Theorem, we have:
\begin{equation}
R^2 &= (R-h)^2+d^2\\
R^2 &= R^2-2Rh+h^2+d^2\\
\therefore R &= \frac{h^2+d^2}{2h}=\frac{(3 {\rm cm})^2+(5 {\rm cm})^2}{2(3 {\rm cm})}=5.67 {\rm cm}
\end{equation}
The strength of the magnetic field is then given by:
\begin{equation}
B&=\frac{mv}{qR}=\frac{(9.11e-31 {\rm kg})(1.326e7 {\rm ms^{-1}})}{(1.6e-19 {\rm C})(0.0567 {\rm m})}=0.00133 {\rm T}
\end{equation}
\end{framed}

\begin{framed}
\textbf{Solution 20.2}\\
First, we will determine the magnetic dipole moment of the square coil:
\begin{equation}
\mu &= NIA\\
\mu &=NIa^2
\end{equation}
Now that we have the magnetic dipole moment, we can calculate the torque on the square coil that is produced by the magnetic field. Note that, in a galvanometer, the magnetic field is configured such that it is radial and always perpendicular to the magnetic dipole moment of the coil:
\begin{equation}
\tau_B &= \mu B sin(90 {\rm \degree})= NIa^2B\\
\end{equation}
The deflection, $\theta$, for a given current will occur when the torque produced by the wire is equal to the torque produced by the spring. The torque produced by the spring is given by:
\begin{equation}
\tau_s =\kappa \theta
\end{equation}
where $\theta$ is measured in radians. The above equation is the rotational equivalent of Hooke's Law. Equating the torque from the spring and from the magnetic field, we can determine the current:
\begin{equation}
\tau_B&=\tau_S\\
NIa^2B &= \kappa \theta\\
I &= \frac{\kappa \theta}{Na^2B} = \frac{(1.5e-8 {\rm Nm(rad)^{-1}}) (0.7 {\rm rad})}{70(0.025 {\rm m})^2(8e-3 {\rm T})}\\
&= 30 {\rm \mu A}
\end{equation}
\end{framed}

\begin{framed}
\textbf{Solution 20.3}\\
Figure~\ref{fig:magneticforce:proof1} illustrates a loop of radius, $R$, carrying current, $I$. The loop is in the $x -z$ plane, and there is a magnetic field, $\vec B$, in the negative $x$ direction. By setting the loop up this way, it is easier to visualize some of the three-dimensional aspects.

\begin{figure}[!htbp]
\centering
\includegraphics[width=0.5\linewidth]{files/proof1-8111f1e51bf6dd8310876d5f97bc88a5.png}
\caption[]{A current-carrying loop in a magnetic field.}
\label{fig:magneticforce:proof1}
\end{figure}

Consider an infinitesimal section of the loop, with length, $dl$, located on the loop at a position labelled by the angle, $\theta$, as illustrated. The vector, $d\vec l$, is given by:
\begin{equation}
d\vec l  = dl (-\sin\theta \hat x + \cos\theta \hat z)
\end{equation}
The magnetic force on that element of the loop is given by:
\begin{equation}
d\vec F &=Id\vec l \times \vec B\\
&=Idl(-\sin\theta \hat x + \cos\theta \hat z) \times (-B\hat x)\\
&=-IBdl\cos\theta (\hat z \times \hat x)\\
&=-IBdl\cos\theta\hat y
\end{equation}
and the force on that element of wire is out of the page (negative $y$ direction), as illustrated. That infinitesimal force will create an infinitesimal torque:
\begin{equation}
d\vec \tau = \vec r \times d\vec F
\end{equation}
where $\vec r$ is the vector from the axis of rotation (through the centre of the loop, parallel to the $z$ axis) to the point where the force is exerted. The length of the vector, $\vec r$, is simply $r=R\cos\theta$, and the force is perpendicular to the vector $\vec r$. Thus, the torque on the infinitesimal element is given by:
\begin{equation}
d\vec \tau &= \vec r \times d\vec F\\
&= (R\cos\theta \hat x)\times (-IBdl\cos\theta\hat y)\\
&=-IBR\cos^2\theta dl (\hat x \times \hat y)\\
&=-IBR\cos^2\theta dl \hat z
\end{equation}
and the torque on that infinitesimal element is in the negative $z$ direction, as anticipated from the direction of the force. Note that had we considered the loop to be oriented such that the magnetic field is not in the plane of the loop, the vector $\vec r$ in the torque would have a component in the $y$ direction.

We can sum the torques on each element of the loop, from $\theta = 0$ to $\theta=2\pi$. We can express the length, $dl$, using the infinitesimal angle, $d\theta$, that subtends the arc of length, $dl$, on the circle of radius, $R$:
\begin{equation}
dl = Rd\theta
\end{equation}

The net torque is then given by:
\begin{equation}
\vec \tau &= \int d\vec \tau\\
&=\int -IBR\cos^2\theta dl \hat z\\
&= (-IBR^2\hat z)\int_0^{2\pi} \cos^2\theta d\theta\\
&=(-IBR^2\hat z)\pi
\end{equation}
The magnetic moment of the loop is:
\begin{equation}
\mu = IA = I\pi R^2
\end{equation}
so that the torque is indeed given by $\tau = \mu B$. If we had rotated the loop so that the vector, $\vec r$, had a $y$ component, then we would have found the general formula with a cross-product.
\end{framed}

\subsection{Chapter 21 - Sources of magnetic field}

\subsubsection{Overview}\label{chapter:magneticsource}

\begin{verbatim}
print("Here is a Python cell")
\end{verbatim}

\begin{verbatim}
print("Another code cell with a second optional metadata syntax")
\end{verbatim}

In this chapter, we develop the tools to model the magnetic field that is produced by an electric current. We will introduce the Biot-Savart Law, which is analoguous to Coulomb's Law in that it can be used to calculate the magnetic field produced by any current. We will also introduce Ampère's Law, which can be thought of as the analogue to Gauss' Law, allowing us to easily determine the magnetic field when there is a high degree of symmetry.

\begin{framed}
\textbf{Learning Objectives}\\
\begin{itemize}
\item Understand how to apply the Biot-Savart Law to determine the magnetic field from an electric current.
\item Understand how to apply Ampère's Law.
\item Understand how to model the forces that are exerted on each other by two wires carrying current.
\item Understand how to model a solenoid and a toroid.
\end{itemize}
\end{framed}

\begin{framed}
\textbf{Think About It}\\
How does an electromagnet work?

\begin{enumerate}
\item Current is passed through a magnet, increasing its strength.
\item Current is passed through a circular coil, creating a magnetic field.
\end{enumerate}

\begin{framed}
\textbf{Answer}\\
\begin{enumerate}[resume]
\item
\end{enumerate}
\end{framed}
\end{framed}

\subsubsection{The Biot-Savart Law}

The Biot-Savart law allows us to determine the magnetic field at some position in space that is due to an electric current. More precisely, the Biot-Savart law allows us to calculate the infinitesimal magnetic field, $d\vec B$, that is produced by a small section of wire, $d\vec l$, carrying a current, $I$, such that $d\vec l$ is co-linear with the wire and points in the direction of the electric current:
\begin{equation}
\boxed{d\vec B = \frac{\mu_0 I}{4\pi}\frac{d\vec l\times \hat r}{r^2}}
\end{equation}
where $\vec r$ is the vector from the element of wire, $d\vec l$, to the point where we would like to determine the magnetic field, as illustrated in Figure~\ref{fig:magneticsource:biotsavart}. $\mu_0$ is a constant of proportionality called the ``permeability of free space'', and has the value $\mu_0=4\pi \times 10^{ -7} {\rm T\cdot m/A}$.

\begin{figure}[!htbp]
\centering
\includegraphics[width=0.5\linewidth]{files/biotsavart-37ffa7dec2527c7a20273b520610411d.png}
\caption[]{The infinitesimal magnetic field, $d\vec B$, that is created by an infinitesimal section of wire, $d\vec l$, carrying current $I$. Note that the vector, $\vec r$, goes from $d\vec l$ to the point where we wish to calculate the field.}
\label{fig:magneticsource:biotsavart}
\end{figure}

The Biot-Savart Law has some similarities with the Coulomb Law to calculate the electric field, as the magnitude of the magnetic field decreases as the inverse of the square distance between the source and the field. However, it can only be expressed in differential form (i.e. as an infinitesimal) and it requires working in three dimensions because of the cross product. It is usually more convenient to use the Biot-Savart Law in the form:
\begin{equation}
d\vec B = \frac{\mu_0 I}{4\pi}\frac{d\vec l\times \vec r}{r^3}
\end{equation}
where the unit vector $\hat r$ was replaced by $\vec r/r$.

The procedure for applying the Biot-Savart Law is as follows:

\begin{enumerate}
\item Make a really good diagram, as you will have to include some 3D aspects.
\item Choose an infinitesimal section of wire, $d\vec l$.
\item Determine the vector $\vec r$.
\item Determine the cross-product, $d\vec l \times \vec r$, which will point in the direction of the magnetic field from that infinitesimal section of wire.
\item Write out the infinitesimal vector $d\vec B$, and determine its components.
\item Think about symmetry! As you sum the $d\vec B$, will some components cancel? If yes, you do not need to do those integrals.
\item Determine the total magnetic field, component by component, by summing (integrating) the components of $d\vec B$ over the wire.
\end{enumerate}

\paragraph{Magnetic field from a straight current-carrying wire}

In this section, we use the Biot-Savart Law to determine the magnetic field a distance, $h$, from the centre of a finite straight wire of length $L$, carrying current $I$, as illustrated in Figure~\ref{fig:magneticsource:bswire}.

\begin{figure}[!htbp]
\centering
\includegraphics[width=0.7\linewidth]{files/bswire-dc40e864732d43d30aeb03f08e4a9a7e.png}
\caption[]{Setting up the model to use the Biot-Savart Law to calculate the magnetic field a distance $h$ from the centre of a current-carrying wire of length $L$.}
\label{fig:magneticsource:bswire}
\end{figure}

We start by choosing an infinitesimal element of wire, $d\vec l$, a distance $y$ above the centre of the wire, as shown (we choose the origin to be located at the centre of the wire). The vector $d\vec l$ is thus given by:
\begin{equation}
d\vec l = dl\hat y
\end{equation}
The vector, $\vec r$, from $d\vec l$ to the point at which we would like to know the magnetic field is given by:
\begin{equation}
\vec r &= r\cos\theta\hat x -r\sin\theta\hat y\\
r &=\sqrt{h^2+y^2} =\frac{h}{\cos\theta}
\end{equation}
The cross-product between $d\vec l$ and $\vec r$ is easily found with the right-hand rule to point into the page (corresponding to the negative $z$ direction). The magnitude of the cross-product is given by:
\begin{equation}
||d\vec l \times \vec r||=dl r \sin\phi
\end{equation}
where $\phi=\pi/2+\theta$ is the angle between $d\vec l$ and $\vec r$, so that $\sin\phi=\cos\theta$. The cross-product can thus be written in terms of $\theta$ as:
\begin{equation}
d\vec l \times \vec r=-dl r \cos\theta \hat z
\end{equation}
Note that we can also determine the cross-product algebraically instead of using the right-hand rule and the magnitude:
\begin{equation}
d\vec l \times \vec r &= (dl\hat y) \times (r\cos\theta\hat x -r\sin\theta\hat y)\\
&=dlr\cos\theta (\hat y \times\hat x) - rdl\sin\theta(\hat y \times \hat y)\\
&=-dlr\cos\theta \hat z
\end{equation}
The infinitesimal magnetic field element, $d\vec B$, is given by:
\begin{equation}
d\vec B = \frac{\mu_0 I}{4\pi}\frac{d\vec l\times \vec r}{r^3}=-\frac{\mu_0 I}{4\pi}\frac{dl\cos\theta}{r^2}\hat z
\end{equation}
Any segment along the wire will result in a magnetic field that is into the page (negative $z$ direction), thus there will be no cancellations due to any symmetries. We can now proceed to perform the integral.

We can use either $\theta$ or $y$ to label the wire elements and carry out the integration. We will choose to integrate over $\theta$, requiring us to express $dl$ and $r$ in terms of $\theta$ (and constants), as those are the only quantities in $d\vec B$ that depend on the position of $d\vec l$. In order to express $dl$ in terms of $d\theta$, we first relate $\theta$ to $y$, the position of the wire element:
\begin{equation}
y = h\tan\theta\quad \to \quad
dl = dy = \frac{dy}{d\theta}d\theta = \frac{h}{\cos^2\theta}d\theta
\end{equation}
and $r$ is given by:
\begin{equation}
r=\frac{h}{\cos\theta}\quad \to \quad
\frac{1}{r^2}&=\frac{\cos^2\theta}{h^2}
\end{equation}
Putting this altogether into $d\vec B$:
\begin{equation}
d\vec B &=-\frac{\mu_0 I}{4\pi}\frac{dl\cos\theta}{r^2}\hat z\\
&= -\frac{\mu_0 I}{4\pi}\left(\frac{h}{\cos^2\theta}d\theta\right) \left( \frac{\cos^2\theta}{h^2} \right)\cos\theta\hat z\\
&=-\frac{\mu_0 I}{4\pi h}\cos\theta d\theta \hat z=dB_z\hat z
\end{equation}
We define the angle, $\theta_0$, to be the maximum amplitude of the angle $\theta$ when integrating over the wire (see Figure~\ref{fig:magneticsource:bswire}), so that we integrate $\theta$ from $-\theta_0$ to $+\theta_0$:
\begin{equation}
B_z&=\int_{-\theta_0}^{+\theta_0}dB_z\\
&= -\frac{\mu_0 I}{4\pi h} \int_{-\theta_0}^{+\theta_0}\cos\theta d\theta\\
&=-\frac{\mu_0 I}{4\pi h}(2\sin\theta_0)\\
&=-\frac{\mu_0 I}{2\pi h}\sin\theta_0
\end{equation}
Using the given dimensions:
\begin{equation}
\sin\theta_0&=\frac{L/2}{\sqrt{h^2+\frac{L^2}{4}}}\\
\end{equation}
Thus, the magnetic field, $\vec B$, a distance, $h$, from the centre of a wire of length, $L$, carrying current, $I$, in the positive $y$ direction is given by:
\begin{equation}
\boxed{\vec B = -\frac{\mu_0 I}{2\pi h}\frac{L/2}{\sqrt{h^2+\frac{L^2}{4}}}\hat z}\quad\text{(finite wire)}
\end{equation}
The magnetic field must be rotationally symmetric; that is, if the wire is vertical, the magnetic field at a distance $h$ must look the same regardless of the angle from which we view the vertical wire (we should always see the magnetic field going into the page at the point that we use in Figure~\ref{fig:magneticsource:bswire}). Thus, the magnetic field lines must form circles around the wire, as illustrated in Figure~\ref{fig:magneticsource:wirefield}. Note that the direction of the magnetic field is given by the right-hand rule for axial vectors; when you align your thumb with the current, your fingers curl in the direction of the magnetic field.

\begin{figure}[!htbp]
\centering
\includegraphics[width=0.4\linewidth]{files/wirefield-f0efa665a11f2b4ea619f8be5412e4f8.png}
\caption[]{The magnetic field from a current-carrying wire forms concentric circles centred on the wire.}
\label{fig:magneticsource:wirefield}
\end{figure}

It is of particular interest to investigate the limiting case of an infinitely long wire, in the limit of $L\to\infty$, or equivalently, $\theta_0\to\frac{\pi}{2}$. The latter is easiest to evaluate, since $\sin\theta_0\to 1$. The magnitude of the magnetic field, $\vec B$, a distance, $h$, from an infinite wire carrying current, $I$, is given by:
\begin{equation}
\boxed{ B = \frac{\mu_0 I}{2\pi h}}\quad\text{(infinite wire)}
\end{equation}
One can often make the approximation that the wire is infinite in length, when the distance, $h$, is small compared to the length, $L$, of the wire.

\paragraph{Magnetic field from a circular current-carrying wire}

In this section, we examine the magnetic field that is created by a circular current-carrying loop of wire. We can determine the shape of the magnetic field, by considering small sections as straight wires, with circular magnetic field lines around them. As we move closer to the centre of the ring, those fields sum together, as illustrated in Figure~\ref{fig:magneticsource:ringfield}. Note that the magnetic field from a loop of current is identical to that from a bar magnet (as a bar magnet is, of course, a collection of current loops).

\begin{figure}[!htbp]
\centering
\includegraphics[width=0.4\linewidth]{files/ringfield-f13bccba775229de52496b9398a5ad25.png}
\caption[]{The magnetic field from a current-carrying loop of wire can be thought of as the sum of the fields from small straight sections of wire.}
\label{fig:magneticsource:ringfield}
\end{figure}

Below, we use the Biot-Savart Law to derive an expression for the magnitude of the magnetic field at a distance, $h$, from the centre of a ring of radius, $R$, along its axis of symmetry, when there is a current, $I$, in the ring. While the mathematics are much easier than the case for the straight wire, the challenge in this case is to visualize the calculation in three dimensions! Figure~\ref{fig:magneticsource:bsring} shows the loop of current, as well as our choice of coordinate system (with the origin at the centre of the ring). In particular, we wish to calculate the magnetic field at a distance, $h$, along the $z$ axis. The $x$ axis goes into the page.

\begin{figure}[!htbp]
\centering
\includegraphics[width=0.4\linewidth]{files/bsring-740662e1ad7007d0abe905f8a7c59235.png}
\caption[]{Diagram to apply the Biot-Savart Law in order to determine the magnetic field along the symmetry axis of a ring carrying current, $I$. The $x$ axis goes into the page.}
\label{fig:magneticsource:bsring}
\end{figure}

In order to apply the Biot-Savart Law, we choose an element, $d\vec l$, of wire at the top of the ring, as illustrated. At this position, the element, $d\vec l$, points in the positive $x$ direction (into the page):
\begin{equation}
d\vec l = dl \hat x
\end{equation}
The vector, $\vec r$, from the wire element to the point where we wish to determine the magnetic field is given by:
\begin{equation}
\vec r =  - r\sin\theta \hat y+r\cos\theta \hat z
\end{equation}
and the angle $\theta$ will be the same for all wire elements along the ring. The cross-product, $d\vec l \times \vec r$, can be evaluated algebraically:
\begin{equation}
d\vec l \times \vec r &= (dl \hat x) \times ( - r\sin\theta \hat y+r\cos\theta \hat z)\\
&=-rdl\sin\theta (\hat x \times \hat y) + rdl\cos\theta (\hat x \times \hat z)\\
&=-rdl\sin\theta\hat z + rdl\cos\theta (-\hat y)\\
&=-rdl\sin\theta\hat z - rdl\cos\theta \hat y
\end{equation}
so that the element of magnetic field, $d\vec B$, corresponding to that choice of $d\vec l$, will lie in the $y -z$ plane, as illustrated in Figure~\ref{fig:magneticsource:bsring}. Note that the vector $d\vec B$ is perpendicular to the vector $\vec r$ (since it is the cross-product of $\vec r$ with another vector). The magnetic field element, $d\vec B$, is given by:
\begin{equation}
d\vec B &= \frac{\mu_0 I}{4\pi}\frac{d\vec l\times \vec r}{r^3}\\
&= \frac{\mu_0 I}{4\pi r^3}(-rdl\sin\theta\hat z - rdl\cos\theta \hat y )\\
&=\frac{\mu_0 I}{4\pi r^2}(-dl\sin\theta \hat z - dl\cos\theta \hat y)\\
&=dB_z\hat z + dB_y \hat y
\end{equation}
As the wire element, $d\vec l$, moves around the circle, the tip of the resulting magnetic field vector element traces a circle centred on the $z$ axis, as illustrated in Figure~\ref{fig:magneticsource:bsring_all}. Note that, in general, $d\vec B$ will also have an $x$ component (the $x$ component was only 0 before because we chose $d\vec l$ to be at the top of the ring). When we sum together the magnetic field elements, the $x$ and $y$ components will cancel, so that we are left with the $z$ component.

\begin{figure}[!htbp]
\centering
\includegraphics[width=0.9\linewidth]{files/bsring_all2-afdddcb12c4885a8783bea9825fd9214.png}
\caption[]{As the wire element, $d\vec l$, moves along the ring, the tip of corresponding magnetic field element vector, $d\vec B$, describes a circle centred on the $z$ axis. Thus, only the (negative) $z$ component of $d\vec B$ will survive when these are all added together.}
\label{fig:magneticsource:bsring_all}
\end{figure}

The total magnetic field will be in the negative $z$ direction, as anticipated from Figure~\ref{fig:magneticsource:ringfield}. Summing together the $z$ components of the infinitesimal magnetic fields:
\begin{equation}
dB_z &= -\frac{\mu_0 I}{4\pi r^2}dl\sin\theta\\
B_z &= \int dB_z\\
&=-\int \frac{\mu_0 I}{4\pi r^2}dl\sin\theta
\end{equation}
Note that in this case, both $r$ and $\theta$ are constant for all of the $d\vec l$, allowing us to take them out of the integral. The integral is then just a sum of the $dl$ elements, which must add up to the circumference of the ring:
\begin{equation}
B_z &= \int dB_z\\
&= -\frac{\mu_0 I}{4\pi r^2}\sin\theta \int_0^{2\pi R} dl\\
&=-\frac{\mu_0 I}{4\pi r^2}\sin\theta (2\pi R)\\
&=-\frac{\mu_0 I}{2r^2}R\sin\theta
\end{equation}
In terms of the variables that we are given:
\begin{equation}
r &= \sqrt{R^2+h^2}\\
\sin\theta &=\frac{R}{r}=\frac{R}{\sqrt{R^2+h^2}}\\
\end{equation}
\begin{equation}
\therefore\;\; \boxed{\vec B = -\frac{\mu_0 I}{2} \frac{R^2}{(R^2+h^2)^\frac{3}{2}} \hat z}\quad\text{(field from a loop of current)}
\end{equation}
In this case, the math was relatively straightforward (no substitutions to evaluate the integral), however it is challenging to visualize the problem in three dimensions.

\begin{framed}
\textbf{Checkpoint}\\
A coil is made of $N$ loops of current-carrying wire packed closely together. What is the magnetic field at the centre of the coil?

\begin{enumerate}
\item $\frac{\mu_0I}{2R}$
\item $\frac{N\mu_0I}{2R}$
\item $\frac{N\mu_0I}{2R^2}$
\item $\frac{\mu_0I}{R}$
\end{enumerate}

\begin{framed}
\textbf{Answer}\\
\begin{enumerate}[resume]
\item
\end{enumerate}
\end{framed}
\end{framed}

\subsubsection{Force between two current-carrying wires}

Consider two infinite parallel straight wires, a distance $h$ apart, carrying upwards currents, $I_1$ and $I_2$, respectively, as illustrated in Figure~\ref{fig:magneticsource:twowires}.

\begin{figure}[!htbp]
\centering
\includegraphics[width=0.4\linewidth]{files/twowires-c728efaf080baf12aff491b194a58f14.png}
\caption[]{Two parallel current-carrying wires will exert an attractive force on each other, if their currents are in the same direction.}
\label{fig:magneticsource:twowires}
\end{figure}

The first wire will create a magnetic field, $\vec B_1$, in the shape of circles concentric with the wire. At the position of the second wire, the magnetic field $B_1$ is into the page, and has a magnitude:
\begin{equation}
B_1 = \frac{\mu_0 I_1}{2\pi h}
\end{equation}
Since the second wire carries a current, $I_2$, upwards, it will experience a magnetic force, $\vec F_2$, from the magnetic field, $B_1$, that is towards the left (as illustrated in Figure~\ref{fig:magneticsource:twowires} and determined from the right-hand rule). The magnetic force, $\vec F_2$, exerted on a section of length, $l$, on the second wire has a magnitude given by:
\begin{equation}
F_2 = I_2 ||\vec l \times \vec B_1||=I_2 l B_1 =\frac{\mu_0 I_2 I_1 l}{2\pi h}
\end{equation}
where we used the fact that the angle between $\vec l$ and $\vec B$ is $90 {\rm \degree}$. We expect, from Newton's Third Law, that an equal and opposite force should be exerted on the first wire. Indeed, the second wire will create a magnetic field, $\vec B_2$, that is out of the page at the location of the first wire, with magnitude:
\begin{equation}
B_2 = \frac{\mu_0 I_2}{2\pi h}
\end{equation}
This leads to a magnetic force, $\vec F_1$, exerted on the first wire, that points to the right (from the right-hand rule). On a section of length, $l$, of the first wire, the magnetic force from the magnetic field, $\vec B_2$, has magnitude:
\begin{equation}
F_1 = I_1 ||\vec l \times \vec B_2||=I_1 l B_2 =\frac{\mu_0 I_1 I_2}{2\pi h}
\end{equation}
which does indeed have the same magnitude as the force exerted on the second wire. Thus, when two parallel wires carry current in the same direction, they exert equal and opposite attractive forces on each other.

\begin{framed}
\textbf{Checkpoint}\\
\begin{figure}[!htbp]
\centering
\includegraphics[width=0.2\linewidth]{files/oppositewires-db434c55df786bbeaad218aa318228be.png}
\caption[]{Two wires that carry current in opposite directions.}
\label{fig:magneticsource:oppositewires}
\end{figure}

Two parallel wires carry current in opposite directions, as shown in Figure~\ref{fig:magneticsource:oppositewires}. What force do they exert on each other?

\begin{enumerate}
\item There will be no force, since the currents cancel.
\item There will be an attractive force between the wires.
\item There will be a repulsive force between the wires.
\end{enumerate}

\begin{framed}
\textbf{Answer}\\
\begin{enumerate}[resume]
\item
\end{enumerate}
\end{framed}
\end{framed}

The attractive force between two wires used to be the basis for defining the Ampère, the S.I. (base) unit for electric current. Before 2019, the Ampère was defined to be ``that constant current which, if maintained in two straight parallel conductors of infinite length, of negligible circular cross-section, and placed one metre apart in vacuum, would produce between these conductors a force equal to $2\times 10^{ -17} {\rm N}$ per metre of length''. Recently, the definition was updated to be based on defining the Coulomb in such a way that the elementary charge has a numerical value of $e=1.602176634\times 10^{ -19} {\rm C}$, and the Ampère corresponds to one Coulomb per second.

The force between two wires is a good system to understand how any physical quantity cannot depend on our choice of the right-hand to define cross-products. As mentioned in the previous chapter, any physical quantity, such as the direction of the force exerted on a wire, will always depend on two successive uses of the right hand. In this system, we first used the right-hand rule for axial vectors to determine the direction of the magnetic field from one of the wires. We then used the right-hand rule to determine the direction of the cross-product to determine the direction of the force on the other wire. You can verify that you get the same answer if you, instead, use your left-hand to define the direction of the magnetic field (which will be in the opposite direction), and then again for the cross-product. This also highlights that the magnetic field (and the electric field) is just a mathematical tool that we use to, ultimately, describe the motion of charges or compass needles.

\begin{framed}
\textbf{Olivia's Thoughts}\\
For this problem, we can determine whether the force is attractive or repulsive by finding the directions of  the forces on wire 1 ($\vec{F_1}$) and wire 2 ($\vec{F_2}$). Figure~\ref{fig:magneticsource:2wires_steps} shows how to find the directions of $\vec{F_1}$ and $\vec{F_2}$ step by step using the right hand rules. The process is as follows:

\begin{enumerate}
\item Use the axial right hand rule to find the magnetic field at wire 2 due to the current in wire 1 ($\vec{B_1}$).
\item Use the right hand rule to find the direction of the force on wire 2 due to its current and the magnetic field $\vec{B_1}$.
\item Use the axial right hand rule to find the magnetic field at wire 1 due to the current in wire 2 ($\vec{B_2}$).
\item Use the right hand rule to find the direction of the force on wire 1 due to its current and the magnetic field $\vec{B_2}$.
\end{enumerate}

\begin{figure}[!htbp]
\centering
\includegraphics[width=0.8\linewidth]{files/2wire_rhr-c435f687470e9436439dd1962c8d6354.png}
\caption[]{(a) Axial right hand rule to find $\vec{B_1}$. (b) Right hand rule to find $\vec{F_2}$. (c) Axial right hand rule to find $\vec{B_2}$. (d) Right hand rule to find $\vec{F_1}$.}
\label{fig:magneticsource:2wires_steps}
\end{figure}

Both $\vec{F_1}$ and $\vec{F_2}$ point towards the centre, so the force is attractive. See what happens when you use your left hand!
\end{framed}

\begin{framed}
\textbf{Checkpoint}\\
When current is flowing in a straight cable, how to you expect the charges to be distributed radially through the cross-section of the cable?

\begin{enumerate}
\item Uniformly in radius (current density does not depend on $r$).
\item There will be an excess of positive charges on the outside of the cable.
\item There will be an excess of negative charges on the outside of the cable.
\end{enumerate}

\begin{framed}
\textbf{Answer}\\
\begin{enumerate}[resume]
\item
\end{enumerate}
\end{framed}
\end{framed}

\subsubsection{Ampère's Law}

Ampère's Law is similar to Gauss' Law, as it allows us to (analytically) determine the magnetic field that is produced by an electric current in configurations that have a high degree of symmetry. Ampère's Law states:
\begin{equation}
\boxed{\oint \vec B \cdot d\vec l =\mu_0 I^{enc}}
\end{equation}
where the integral on the left is a ``path integral'', similar to how we calculate the work done by a force over a particular path. The circle sign on the integral means that this is an integral over a ``closed'' path; a path where the starting and ending points are the same. $I^{enc}$ is the net current that crosses the surface that is defined by the closed path, often called the ``current enclosed'' by the path. This is different from Gauss' Law, where the integral is over a closed surface (not a closed path, as it is here). In the context of Gauss' Law, we refer to ``calculating the \textbf{flux} of the electric field \textbf{through} a closed surface''; in the context of Ampère's Law, we refer to ``calculating the \textbf{circulation} of the magnetic field \textbf{along} a closed path''.

We apply Ampère's Law in much the same way as we apply Gauss' Law:

\begin{enumerate}
\item Make a good diagram, identify symmetries.
\item Choose a closed path over which to calculate the circulation of the magnetic field (see below for how to choose the path). The path is often called an ``Amperian loop'' (think ``Gaussian surface'').
\item Evaluate the circulation integral.
\item Determine how much current is ``enclosed'' by the Amperian loop.
\item Apply Ampère's Law.
\end{enumerate}

Similarly to Gauss' Law, we need to \textbf{choose the path} (instead of the surface) over which we will evaluate the integral. The integral will be easy to evaluate if:

\begin{enumerate}
\item \textbf{The angle between $\vec B$ and $d\vec l$ is constant along the path}, so that:
\end{enumerate}
\begin{equation}
\oint  \vec B \cdot d\vec l = \oint B dl \cos\theta = \cos\theta \oint B dl
\end{equation}
where $\theta$ is the angle between $\vec B$ and $d\vec l$.

\begin{enumerate}[resume]
\item \textbf{The magnitude of $\vec B$ is constant along the path}, so that:
\end{enumerate}
\begin{equation}
\cos\theta \oint B dl = B\cos\theta \oint dl
\end{equation}
Choosing a path that meets these two conditions is only possible if there is a high degree of symmetry.

Consider an infinitely long straight wire, carrying current, $I$, out of the page, as illustrated in Figure~\ref{fig:magneticsource:amperewire}. The magnetic field from the wire must look the same regardless of the angle from which we view the wire (``azimuthal symmetry''). Thus, the magnetic field must either form concentric circles around the wire (which we know is the case from the Biot-Savart Law) or it must be in the radial direction (pointing towards or away from the wire). These two possibilities are illustrated in Figure~\ref{fig:magneticsource:amperewire}, and we will pretend, for now, that we do not know which is correct.

\begin{figure}[!htbp]
\centering
\includegraphics[width=0.6\linewidth]{files/amperewire-472189ba3763707d7ea318766a8855fb.png}
\caption[]{By symmetry, the magnetic field from a current-carrying infinite wire (illustrated with current coming out of the page), must either form concentric circles (left panel), or be in the radial direction (right panel). We know that the former (circles, left panel) is the correct choice. The dotted lines show ``Ampèrian loops'' that one can use to calculate the integral in Ampère's Law.}
\label{fig:magneticsource:amperewire}
\end{figure}

In order to apply Ampère's Law, we choose an Amperian loop (instead of a ``Gaussian surface''). In the case of an infinite current-carrying wire, a circle that is concentric with the wire will meet the properties above, regardless of the two possible configurations of the magnetic field: with a circular Amperian loop, the angle between the magnetic field and the element $d\vec l$ is constant along the entire loop, and the magnitude of the magnetic field is constant along the loop. Our choice of loop is illustrated in Figure~\ref{fig:magneticsource:amperewireloop}, where we have illustrated the magnetic field for the case where it forms concentric circles.

\begin{figure}[!htbp]
\centering
\includegraphics[width=0.5\linewidth]{files/amperewireloop-b47f54367ef17bf0efcd9e449152a81b.png}
\caption[]{An Amperian loop that is a circle of radius, $h$, will allow us to determine the magnetic field at a distance, $h$, from an infinitely-long current-carrying wire.}
\label{fig:magneticsource:amperewireloop}
\end{figure}

The circulation of the magnetic field along a circular path of radius, $h$, is given by:
\begin{equation}
\oint  \vec B \cdot d\vec l = \oint B dl \cos\theta = \cos\theta \oint B dl=B\cos\theta \oint dl=B\cos\theta (2\pi h)
\end{equation}
where $\cos\theta$ is 1 if the field forms circles (correct) or 0 if the field is radial (incorrect). We can now evaluate the current that is enclosed by the Amperian loop. The current that is enclosed is given by the net current that traverses the surface defined by the Amperian loop (in this case, a circle of radius $h$). Since the loop encloses the entire wire, the enclosed current is simply $I$. Applying Ampère's Law:
\begin{equation}
\oint \vec B \cdot d\vec l &=\mu_0 I^{enc}\\
B\cos\theta (2\pi h) &= \mu_0 I
\end{equation}
At this point, it is clear that $\cos\theta$ cannot be zero, since the right-hand side of the equation is not zero. We can thus conclude that the magnetic field must indeed make concentric circles, as we had previously determined. The magnitude of the magnetic field is given by:
\begin{equation}
B = \frac{\mu_0 I}{2\pi h}
\end{equation}
as we found previously with the Biot-Savart Law. Again, in analogy with Gauss' Law, one needs to apply some knowledge of symmetry and argue in which direction the magnetic field should point, in order to use Ampère's Law effectively.

\begin{framed}
\textbf{Checkpoint}\\
Ampère's law proves that the magnetic field at the centre of a current-carrying loop is zero because there is no enclosed current:

\begin{enumerate}
\item True.
\item False.
\end{enumerate}

\begin{framed}
\textbf{Answer}\\
\begin{enumerate}[resume]
\item
\end{enumerate}
\end{framed}
\end{framed}

\begin{framed}
\textbf{Olivia's Thoughts}\\
Let's compare Ampere's law to Gauss's law. Recall that Gauss's law is given by:
\begin{equation}
\oint \vec{E}\cdot d\vec{A}=\frac{Q^{enc}}{\epsilon_0}.
\end{equation}
We want to construct a Gaussian surface such that we can write this as
\begin{equation}
EA=\frac{Q^{enc}}{\epsilon_0},
\end{equation}
where $A$ is the surface area that is perpendicular to the field lines and $E$ is the (constant) magnitude of the electric field over the surface. The flux is given by $Q^{enc}/\epsilon_0$. To find the electric field at any point on the surface, we use
\begin{equation}
E=\frac{Q^{enc}/\epsilon_0}{A}.
\end{equation}
In other words, the electric field is the flux per unit area. Ampère's law is very similar, except that instead of flux, we have the new concept of ``circulation'' and instead of a surface integral, we have a line integral. When we have a high degree of symmetry in Ampère's law, we are often able to write:
\begin{equation}
\oint \vec{B}\cdot d\vec{l}=\mu_0 I^{enc}\rightarrow Bl=\mu_0 I^{enc}
\end{equation}
where $l$ is the length of the Amperian loop that is parallel to the field, $B$ is the magnitude of the magnetic field, and $\mu_0 I^{enc}$ gives us the circulation. To find the magnetic field, we use
\begin{equation}
B=\frac{\mu_0 I^{enc}}{l},
\end{equation}
so the magnetic field is like the circulation per unit length.

In the left panel of Figure~\ref{fig:magneticsource:amperecirculation}, I have shown two Amperian loops with different radii, for the example of the straight wire with a current coming out of the page. Ampère's law tells us that the circulation will be the same along each path. However, because the inner loop is smaller, the circulation per unit length (the magnetic field) will be larger at each point. This is represented in the right panel, where the length of the vectors is proportional to the magnetic field strength.

\begin{figure}[!htbp]
\centering
\includegraphics[width=0.6\linewidth]{files/Ampere_circulation-e38a2746744c28a733119cb646d86f03.png}
\caption[]{Left: 2 Amperian loops around a current carrying wire. Right: Magnetic field vectors around each loop.}
\label{fig:magneticsource:amperecirculation}
\end{figure}
\end{framed}

\begin{framed}
\textbf{Example 21.1}\\
A long solid uniform cable of radius, $R$, carries current, $I$, with a current density that is uniform through the cross-section of the cable. Determine the strength of the magnetic field as a function of $r$, the distance from the centre of the cable, inside \textit{and} outside of the cable.

\begin{framed}
\textbf{Solution}\\
In this case, we need to determine the magnetic field both inside and outside of the cable. Figure~\ref{fig:magneticsource:amperecable} shows two circular Amperian loops that we can use to apply Ampère's Law to determine the magnetic field inside and outside of the cable.

\begin{figure}[!htbp]
\centering
\includegraphics[width=0.4\linewidth]{files/amperecable-c842bce99807fda38aa74536705a03e7.png}
\caption[]{Two circular Amperian loops to determine the magnitude of the magnetic field inside and outside of a current-carrying cable of radius, $R$ (with uniform current coming out of the page).}
\label{fig:magneticsource:amperecable}
\end{figure}

By symmetry, and following the discussion in this chapter, we know that the magnetic field must form concentric circles, both inside and outside of the cable. Outside the cable, we proceed in the same fashion as above, choosing an Amperian loop of radius, $r>R$, such that the circulation is given by:
\begin{equation}
\oint \vec B \cdot d\vec l= B 2\pi r
\end{equation}
The entire cable is enclosed by the loop, so that the enclosed current is, $I$. Thus, Ampère's Law gives:
\begin{equation}
\oint \vec B \cdot d\vec l &=\mu_0 I^{enc}\\
B (2\pi r) &= \mu_0 I\\
\therefore B &= \frac{\mu_0 I}{2\pi r}\quad(r\geq R)
\end{equation}
Inside of the cable, the circulation integral around a circular path of radius, $r<R$, is the same:
\begin{equation}
\oint \vec B \cdot d\vec l= B 2\pi r
\end{equation}
However, in this case, the smaller Amperian loop does not enclose all of the current flowing through the cable. We are told that the current density, $j$, is uniform in the cable. We can thus determine the current per unit area (i.e. the current density) that flows through the whole cable, and use that to determine how much current flows through the surface with area $\pi r^2$ that is defined by the Amperian loop:
\begin{equation}
j &= \frac{I}{A}=\frac{I}{\pi R^2}\\
\therefore I^{enc} &= j(\pi r^2) = \frac{I}{\pi R^2}(\pi r^2)=I\frac{r^2}{R^2}
\end{equation}
Finally, we can apply Ampère's Law to determine the magnitude of the magnetic field inside the cable:
\begin{equation}
\oint \vec B \cdot d\vec l &=\mu_0 I^{enc}\\
B (2\pi r) &= \mu_0 I\frac{r^2}{R^2}\\
\therefore B &= \frac{\mu_0 I}{2\pi R^2}r
\end{equation}
and we find that the magnetic field is zero at the centre of the cable ($r=0$), and increases linearly up to the edge of the cable ($r=R$).

\textbf{Discussion:} In this example, we used Ampère's Law to model the strength of the magnetic field inside and outside of a current-carrying cable. In order to apply Ampère's Law inside the cable, we took into account that only a fraction of the current is enclosed by the Amperian loop. This problem is analogous to applying Gauss' Law to determine the electric field inside and outside of a uniformly charged sphere.
\end{framed}
\end{framed}

\paragraph{Interpretation of Ampère's Law and vector calculus}\label{sec:magneticsource:interpretation}

In this section, we discuss Ampère's Law in the context of vector calculus and provide a different perspective, mostly for informational purposes. The integral that appears in Ampère's Law is called the ``circulation'' of the vector field, $\vec B$:
\begin{equation}
\oint \vec B \cdot d\vec l
\end{equation}
The circulation, as its name implies, is a measure of ``how much rotation there is in the field''. To visualize this, imagine that the vector field is a velocity field for points in a fluid. Regions of the fluid where there are little whirlpools (so called ``eddies''), correspond to regions of the field with non-zero circulation (the sign of the integral tells us the direction of rotation, using the right-hand rule for axial vectors). Examples of field with and without circulation are shown in Figure~\ref{fig:magneticsource:circulation}. You will recognize that static electric charges create electric fields with no circulation (right panel), whereas static currents create magnetic fields with circulation.

\begin{figure}[!htbp]
\centering
\includegraphics[width=0.5\linewidth]{files/circulation-e4d7d465e28a0d993fee414e2645c2bf.png}
\caption[]{Examples of field with (left panel) and without (right panel) circulation, as evaluated along the closed loop shown with the dashed line.}
\label{fig:magneticsource:circulation}
\end{figure}

Ampère's Law is thus a statement that an electric current will result in a field with a magnitude proportional to the current, that has some degree of rotation to it. The direction of rotation of that field corresponds to the right-hand rule for axial vectors as applied to the current (your thumb points in the direction of the current so that your fingers curl in the direction of the rotation of the associated field).

Circulation, as defined by the integral over a closed loop, is not a local property of the field, since it depends on what the field is doing as a whole over the path of the loop. Just as one can obtain a ``local'' version of  Gauss' Law, one can also obtain a local version of Ampère's Law using techniques from advanced vector calculus (that are beyond the scope of this textbook).

Stokes' theorem allows one to convert the circulation integral (a path integral on a closed loop) into a integral over the (open) surface that is defined by the loop:
\begin{equation}
\oint_C \vec B \cdot d\vec l = \int_S (\nabla \times \vec B) \cdot d\vec A
\end{equation}
where the subscript $C$ indicates that the integral is over a one-dimensional path, whereas the subscript $S$ indicates that the integral is over a two-dimensional surface. The term, $\nabla \times \vec B$, is called the ``curl'' of the magnetic field and is a local measure of the amount of rotation in the field. Applying Stokes' theorem to Ampère's Law yield:
\begin{equation}
\oint \vec B \cdot d\vec l &= \mu_0 I^{enc}\\
\int_S (\nabla \times \vec B) \cdot d\vec A &= \mu_0 I^{enc}\\
\end{equation}
Note that we can also write the current, $I^{enc}$, that is enclosed by the loop as the integral of the current density, $\vec j$, over the surface defined by the loop:
\begin{equation}
I^{enc}=\int _S \vec j \cdot d\vec A
\end{equation}
Thus, we can write Ampère's Law with integrals over the same surface on either side of the equation, implying that the integrands must be the same:
\begin{equation}
\int_S (\nabla \times \vec B) \cdot d\vec A = \mu_0 \int _S \vec j \cdot d\vec A\\
\end{equation}
\begin{equation}
\therefore\;\;\boxed{ \nabla \times \vec B = \mu_0 \vec j}
\end{equation}
This last equation now relates a local property (current density) to the magnetic field at that point, and is the usual form in which Ampère's Law is presented (the so-called ``differential form'', rather than the ``integral form'').

The curl of the magnetic field, $\nabla \times \vec B$, is a vector that is given by the following:
\begin{equation}
\nabla \times \vec B = \left(\frac{\partial B_z}{\partial y}-\frac{\partial B_y}{\partial z}\right)\hat x + \left(\frac{\partial B_x}{\partial z}-\frac{\partial B_z}{\partial x}\right)\hat y+\left(\frac{\partial B_y}{\partial x}-\frac{\partial B_x}{\partial y}\right)\hat z
\end{equation}
and the name ``curl'' is chosen because this is a measure of the amount of rotation (curl) in the field. In differential form, Ampère's Law can read as: ``a current density will create a (magnetic) field that has non-zero curl''.

Since Ampère's Law in differential form is a vector equation (both sides are vectors), it really corresponds to three equations in Cartesian coordinates, one per component. For example, the $x$ component of the equation is a ``partial differential equation'' for the $y$ and $z$ components of the magnetic field:
\begin{equation}
\left(\frac{\partial B_z}{\partial y}-\frac{\partial B_y}{\partial z}\right) &=\mu_0j_x
\end{equation}
that is in general difficult to solve without a computer (and all three equations are required, as these are ``coupled'', since a given component of the magnetic field appears in two of three equations).

\subsubsection{Solenoids and toroids}

In order to create strong magnetic fields, the most practical method is to combine many loops of current together into a ``solenoid'' (a coil). Electromagnets function on this principle and are ubiquitous in our lives. Figure~\ref{fig:magneticsource:solenoid1} shows the magnetic field from a single loop of current.

\begin{figure}[!htbp]
\centering
\includegraphics[width=0.3\linewidth]{files/solenoid1-ed94cf3c9ed3767449e82b299f74ebb5.png}
\caption[]{The magnetic field from a single loop of current.}
\label{fig:magneticsource:solenoid1}
\end{figure}

When several loops of current are brought close together, as in Figure~\ref{fig:magneticsource:solenoid2}, the magnetic field inside the solenoid becomes uniform, and the magnetic field just outside the solenoid approaches zero.

\begin{figure}[!htbp]
\centering
\includegraphics[width=0.8\linewidth]{files/solenoid2-611d14b5c1f0d1be219876425e557c5c.png}
\caption[]{As multiple loops of current are brought together to form a solenoid, the magnetic field inside the solenoid becomes uniform and the field outside the solenoid approaches zero.}
\label{fig:magneticsource:solenoid2}
\end{figure}

We can use Ampère's Law to determine the strength of the magnetic field inside of a solenoid, under the assumption that the magnetic field is uniform in the volume of the solenoid and zero just outside. Consider a solenoid with current, $I$, going through it, that contains $n$ loops \textit{per unit length}. In order to determine the magnetic field, $B$, inside of the solenoid, consider the rectangular Amperian loop, $abcd$, of length, $l$, illustrated in Figure~\ref{fig:magneticsource:amperesolenoid}.

\begin{figure}[!htbp]
\centering
\includegraphics[width=0.4\linewidth]{files/amperesolenoid-59b241c750f8105b78b9fb29bbfc8a6b.png}
\caption[]{We use Ampère's Law with a rectangular loop to evaluate the strength of the magnetic field inside a solenoid.}
\label{fig:magneticsource:amperesolenoid}
\end{figure}

In order to evaluate the circulation of the magnetic field around the loop, $abcd$, we divide the loop up into segments, and evaluate the path integral ($\int \vec B \cdot d\vec l$) over each segment, then add those together to obtain the integral over the closed path:
\begin{equation}
\oint_{abcd} \vec B \cdot d\vec l = \int_a^b \vec B \cdot d\vec l + \int_b^c \vec B \cdot d\vec l + \int_c^d \vec B \cdot d\vec l + \int_d^a \vec B \cdot d\vec l
\end{equation}
Over each segment, the vector $d\vec l$ will be parallel to that segment. Only the last term is non-zero. The integrals over the segments $ab$ and $cd$ are zero because the magnetic field is perpendicular to $d\vec l$ over those segments (so the scalar product is zero). The integral over the segment $bc$ is zero because the magnetic field is zero just outside the solenoid. The integral over the last segment, where $d\vec l$ and $\vec B$ are parallel, is simply given by:
\begin{equation}
\oint_{abcd} \vec B \cdot d\vec l &= \int_d^a \vec B \cdot d\vec l\\
&= B \int_d^a dl\\
&= Bl
\end{equation}
since the length of the segment is $l$, and the magnetic field is constant in magnitude.

In order to apply Ampère's Law, we must determine the current that is enclosed by our Amperian loop. Since the rectangular loop has a length, $l$, it will enclose $N=nl$ loops of current, $I$, since there are $n$ loops per unit length. Thus the enclosed current is $I^{enc}=nlI$. Applying Ampère's Law, we find the magnetic field inside a solenoid:
\begin{equation}
\oint \vec B \cdot d\vec l &=\mu_0 I^{enc}\\
Bl &= \mu_0 nIl\\
\end{equation}
\begin{equation}
\therefore\;\; \boxed{B = \mu_0 nI}\quad \text{(Field inside a solenoid)}
\end{equation}
which does not depend on our (arbitrary) choice of making an Amperian loop with an arbitrary length of $l$. In practice, when solenoids are used as electromagnets, they are typically filled with a ferromagnetic material, which will magnetise when there is a current, resulting in a stronger magnetic field. This is usually done by winding a wire around an iron rod.

Note that if we extend the Amperian loop so that the bottom segment is also outside the solenoid, as in Figure~\ref{fig:magneticsource:amperesolenoid2}, it is easy to show that the magnetic field immediately outside of the solenoid must be zero. Indeed, in this case, there are an equal number of currents coming out of the page as there are going into the page, so that the net current that is enclosed by the Amperian loop (the net current that crosses the plane of the loop) is identically zero, so that the circulation must be zero, implying that the magnetic field is zero just outside the solenoid.

\begin{figure}[!htbp]
\centering
\includegraphics[width=0.4\linewidth]{files/amperesolenoid2-e48761aed6a4094ac357cc8c4af2916b.png}
\caption[]{By extending the Amperian loop to both sides of the solenoid, we conclude that the magnetic field just outside the solenoid must be zero, because the net current enclosed is zero.}
\label{fig:magneticsource:amperesolenoid2}
\end{figure}

A toroid can be thought of as a solenoid that has been bent into the shape of a circle (or rather, a torus), as illustrated in Figure~\ref{fig:magneticsource:amperetoroid}. Inside the toroid, the magnetic field forms concentric circles (not shown).

\begin{figure}[!htbp]
\centering
\includegraphics[width=0.6\linewidth]{files/amperetoroid-cb02fa5ffa0c5002d59a27beff662361.png}
\caption[]{An Amperian loop of radius $r$ to determine the magnetic field inside of a toroid. Note that the magnetic field everywhere outside the toroid must be zero (think of the current enclosed by Amperian loops).}
\label{fig:magneticsource:amperetoroid}
\end{figure}

Again, we can use Ampère's Law to determine the strength of the magnetic field inside the toroid. Consider the circular Amperian loop of radius $r$ that is illustrated in Figure~\ref{fig:magneticsource:amperetoroid}. Since the magnetic field is parallel to the Amperian loop everywhere along the loop, and the magnetic field does not change magnitude (by symmetry), the circulation is given by:
\begin{equation}
\oint \vec B \cdot d\vec l &= B (2\pi r)
\end{equation}
If the toroid contains $N$ loops of current, then the enclosed current is given by $I^{enc}=NI$, since the Amperian loop include $N$ times the current $I$ coming out of the page. Ampère's Law thus gives the magnitude of the magnetic field as:
\begin{equation}
\oint \vec B \cdot d\vec l &=\mu_0 I^{enc}\\
B (2\pi r) &= \mu_0 NI\\
\therefore B&=\frac{\mu_0 NI}{2\pi r}
\end{equation}
which decreases in magnitude with increasing radius, as long as we are inside the toroid. It is easy to show, by using Amperian loops that are either smaller or bigger than the toroid, that the magnetic field everywhere outside of the toroid is exactly zero (as those Amperian loops will enclose no net current). In a toroid, the magnetic field lines form closed circles. For a solenoid, there must exist a magnetic field somewhere outside the solenoid, in order for the field lines inside the solenoid to close. We can usually ignore these if the solenoid is long, as the field outside will be very weak, and very close to zero very close the solenoid (as we showed with Ampère's Law above).

\begin{framed}
\textbf{Checkpoint}\\
In Figure~\ref{fig:magneticsource:amperetoroid}, the magnetic field makes concentric circles. What direction do the field lines point?:

\begin{enumerate}
\item Clockwise.
\item Counter clockwise.
\item Upwards.
\item Not enough information to tell.
\end{enumerate}

\begin{framed}
\textbf{Answer}\\
\begin{enumerate}[resume]
\item
\end{enumerate}
\end{framed}
\end{framed}

% %%:::{tip} Checkpoint
% %%What range or position of $r$ encloses the most current in [](#fig:magneticsource:amperetoroid)?:
% %%1.  All $r$ between the $\odot$ and the $\otimes$ symbols.
% %%2.  The point directly between the $\odot$ and the $\otimes$ symbols.
% %%3.  All $r$ from the centre of the toroid to the $\odot$ symbols.
% %%4.  All $r$ beyond the $\otimes$ symbols.
% %%%	:::{tip} Answer
% %%:class: dropdown
% %%1.
% %%:::

\subsubsection{Summary}

Magnetic fields are created by moving charges. The Biot-Savart Law allows us to determine the infinitesimal magnetic field, $d\vec B$, that is produced by the current, $I$, flowing in an infinitesimal section of wire, $d\vec l$:
\begin{equation}
d\vec B = \frac{\mu_0 I}{4\pi}\frac{d\vec l\times \hat r}{r^2}
\end{equation}
where $\mu_0$ is a constant called the permeability of free space. The vector $\vec r$ points from the wire element, $d\vec l$, to the point at which we want to determine the magnetic field. In order to determine the magnetic field from a finite wire, one must sum (integrate) the contributions that come from each section of wire. It is often easier to work with the Biot-Savart law written without the unit vector, $\hat r$:
\begin{equation}
d\vec B = \frac{\mu_0 I}{4\pi}\frac{d\vec l\times \vec r}{r^3}
\end{equation}
The magnetic field at a distance, $h$, from an infinitely long wire carrying current, $I$, is given by:
\begin{equation}
B&=\frac{\mu_0 I}{2\pi h}
\end{equation}
The magnetic field from a straight current-carrying wire forms concentric circles centred around the wire. The direction of the magnetic field is given by the right-hand rule for axial vectors; with the thumb pointing in the direction of current, the fingers curl in the direction of the magnetic field.

The magnitude of the magnetic field, a distance, $h$, from the centre of a circular loop of wire with radius, $R$, carrying current, $I$, along the axis of symmetry of the loop is given by:
\begin{equation}
B=\frac{\mu_0 I}{2} \frac{R^2}{(R^2+h^2)^\frac{3}{2}}
\end{equation}
The direction of the magnetic field can also be found using the right-hand rule for axial currents. In this case, if your fingers curl in the direction of the current loop, your thumb points in the same direction as the magnetic field at the centre of the loop.

Two parallel wires carrying currents, $I_1$ and $I_2$, separated by a distance, $h$, will exert equal and opposite forces on each other with a magnitude:
\begin{equation}
F = \frac{\mu_0I_1I_2}{2\pi h}
\end{equation}
The force is attractive if the two currents flow in the same direction and repulsive otherwise.

Ampère's Law is the magnetism analogue to Gauss' Law. Just like Gauss' Law, it requires a high degree of symmetry to be applied analytically, although it is always valid. Ampère's Law relates the circulation of the magnetic field around a closed path to the current enclosed by that path:
\begin{equation}
\oint \vec B \cdot d\vec l =\mu_0 I^{enc}
\end{equation}
In order to apply Ampère's Law, we must first choose an Amperian loop over which to compute the closed path integral (instead of choosing a Gaussian surface to calculate the flux of the electric field on a closed surface). The circulation integral will be straightforward to evaluate if:

\begin{enumerate}
\item \textbf{The angle between $\vec B$ and $d\vec l$ is constant along the path}, so that:
\end{enumerate}
\begin{equation}
\oint  \vec B \cdot d\vec l &= \oint B dl \cos\theta\\
&= \cos\theta \oint B dl
\end{equation}
where $\theta$ is the angle between $\vec B$ and $d\vec l$.

\begin{enumerate}[resume]
\item \textbf{The magnitude of $\vec B$ is constant along the path}, so that:
\end{enumerate}
\begin{equation}
\cos\theta \oint B dl = B\cos\theta \oint dl
\end{equation}
The current enclosed, $I^{enc}$, corresponds to the net current that crosses the surface that is defined by the Amperian loop (a closed path always defines a surface).

Ampère's Law is straightforward to use in situations with a high degree of symmetry, such as infinitely long wires carrying current.

Solenoids are formed by combining many loops of current together, in order to form a strong and uniform magnetic field. The magnetic field inside of a solenoid has a magnitude of:
\begin{equation}
B=\mu_0nI
\end{equation}
where, $I$, is the current in the solenoid, and $n$, is the number of loops per unit length in the solenoid.  The magnetic field just outside of a solenoid is zero, and generally, the magnetic field is negligible outside of a solenoid.

A toroid is formed by bending a solenoid into a circle to form a torus. The magnetic field lines inside of a toroid form concentric circles. The magnetic field decreases with radius inside of a toroid and is identically zero everywhere outside a toroid.

\begin{framed}
\textbf{Important Equations}\\
\textbf{Biot-Savart law:}
\begin{equation}
d\vec B = \frac{\mu_0 I}{4\pi}\frac{d\vec l\times \hat r}{r^2}=\frac{\mu_0 I}{4\pi}\frac{d\vec l\times \vec r}{r^3}
\end{equation}

\textbf{Magnetic field from a finite wire:}
\begin{equation}
B = \frac{\mu_0 I}{2\pi h}
\end{equation}

\textbf{Magnetic field from an infinitely long wire:}
\begin{equation}
B = \frac{\mu_0 I}{2\pi h}
\end{equation}

\textbf{Magnetic field from a circular loop of current:}
\begin{equation}
B=\frac{\mu_0 I}{2} \frac{R^2}{(R^2+h^2)^\frac{3}{2}}
\end{equation}

\textbf{Force between two wires:}
\begin{equation}
F = \frac{\mu_0I_1I_2}{2\pi h}
\end{equation}

\textbf{Ampère's law:}
\begin{equation}
\oint \vec B \cdot d\vec l =\mu_0 I^{enc}
\end{equation}
\end{framed}

\subsubsection{Thinking about the material}

\begin{framed}
\textbf{Reflect and research}\\
\begin{itemize}
\item Who discovered Ampère's Law? What did Ampère discover?
\item What are three common uses of electromagnets.
\item How does a coil gun work?
\end{itemize}
\end{framed}

\begin{framed}
\textbf{To try at home}\\
\begin{itemize}
\item Use a battery and some wire to build an electromagnet. Is it stronger if you use a ferromagnetic core?
\item Research the current that your household electronics have. Are design choices affected by the magnetic fields generated by the current in these electronics?
\end{itemize}
\end{framed}

\begin{framed}
\textbf{To try in the lab}\\
\begin{itemize}
\item (Simulation) Calculate the magnetic field from a loop of current at all positions in space (not only on the axis of symmetry).
\item Propose an experiment to characterize the magnetic field produced by Helmholtz coils.
\item Propose an experiment to build and test a coil gun.
\end{itemize}
\end{framed}

\subsubsection{Sample problems and solutions}

\paragraph{Problems}

\begin{framed}
\textbf{Problem 21.1}\\
A square loop of wire with side length, $L$, carries current, $I$, as shown in Figure~\ref{fig:magneticsource:squareloop}. What is the magnetic field at the centre of the loop?

\begin{figure}[!htbp]
\centering
\includegraphics[width=0.3\linewidth]{files/squareloop-b9f08214664e39de556569ccfa71efa9.png}
\caption[]{A square loop of current.}
\label{fig:magneticsource:squareloop}
\end{figure}
\end{framed}

\begin{framed}
\textbf{Problem 21.2}\\
Helmholtz coils are an arrangement of two parallel loops of current that produce a nearly uniform magnetic field. Helmholtz coils are formed by two identical circular loops of radius $R$, carrying the same current, $I$, where the centres of the coils are separated by a distance, $R$, as illustrated in Figure~\ref{fig:magneticsource:helmholtzdiagram}. Determine the magnetic field as a function of $z$, along the axis of symmetry of the coils, where the origin is located half way between the two coils. Make a plot of the magnetic field as a function of $z$ from each coil, as well as the total magnetic field to show that it is close to uniform between the coils.

\begin{figure}[!htbp]
\centering
\includegraphics[width=0.4\linewidth]{files/helmholtzdiagram-5e71f8519586a40e19b5c4bd4c183262.png}
\caption[]{A Helmholtz coil arrangement.}
\label{fig:magneticsource:helmholtzdiagram}
\end{figure}
\end{framed}

\paragraph{Solutions}

\begin{framed}
\textbf{Solution 21.1}\\
The square loop is simply made of four straight sections of wire of length, $L$. The magnetic field from each section of wire is into the page, which you can easily verify with your right-hand (with your thumb in the direction of current, your fingers curl in the direction of the resulting magnetic field).

The magnetic field at the centre is just four times the magnetic field produced by a single segment, which we determined in this chapter. The magnetic field at the centre of the loop is thus four times the magnetic field at a distance, $h=\frac{L}{2}$, from a wire of length, $L$:
\begin{equation}
B &= 4\times\frac{\mu_0 I}{2\pi \frac{L}{2}}\frac{L/2}{\sqrt{\frac{L^2}{4}+\frac{L^2}{4}}}=2\sqrt 2\frac{\mu_0 I}{\pi L}
\end{equation}
\end{framed}

\begin{framed}
\textbf{Solution 21.2}\\
We know that the magnetic field at a distance, $h$, from the centre of a loop of current, along its axis of symmetry is given by:
\begin{equation}
 B(h) = \frac{\mu_0 I}{2}\frac{R^2}{(R^2+h^2)^{\frac{3}{2}}}
\end{equation}
For the two coils in the Helmholtz configuration, the magnetic field from each coil will be in the same direction. The centre of the two coils are located at $z=\pm\frac{R}{2}$. Thus, if we are located at position, $z$, along the $z$ axis, one coil will be at a distance of $z+\frac{R}{2}$, and the other at a distance $z -\frac{R}{2}$. The total magnetic field as a function of $z$ is then given by:
\begin{equation}
B^{tot}(z) &= B\left(z+\frac{R}{2}\right)+B\left(z-\frac{R}{2}\right)\\
&=\frac{\mu_0 I}{2}\frac{R^2}{(R^2+\left(z+\frac{R}{2}\right)^2)^{\frac{3}{2}}}+\frac{\mu_0 I}{2}\frac{R^2}{(R^2+\left(z-\frac{R}{2}\right)^2)^{\frac{3}{2}}}
\end{equation}
We can plot this function, as well as the two individual terms using python. For information, we show the code below. In order to make the plot, we need to choose some reasonable values for the radius of the coils and the current through the coils, for example:

\begin{itemize}
\item $R=0.3 {\rm m}$
\item $I=0.5 {\rm A}$
\end{itemize}

\begin{verbatim}
#Import the modules that we need:
import numpy as np
import matplotlib.pyplot as plt

#Define some constants:
mu0 = 4*np.pi*1e -7 #4 pi 
I = 0.5
R = 0.3

#Define the values on the z axis, from -2R to +2R, in 100 increments
z = np.linspace( -2*R,2*R,100)

#Determine the magnetic field from the coils at those values of z
#The coil at z = - R/2:
B1 = (mu0*I)/2 * R**2/((R**2+(z+R/2)**2)**(3/2))
#The coil at z = + R/2:
B2 = (mu0*I)/2 * R**2/((R**2+(z -R/2)**2)**(3/2))
#The sum:
B = B1 + B2

#Make the plot
plt.figure(figsize=(10,6))
plt.plot(z,B1,label='Coil at z= -R/2')
plt.plot(z,B2,label='Coil at z=+R/2')
plt.plot(z,B,label='Total')
plt.legend()
plt.xlabel('z position [m]')
plt.ylabel('Magnetic field [T]')
plt.show()
\end{verbatim}

\begin{figure}[!htbp]
\centering
\begin{quote}
\begin{figure}[!htbp]
\centering
\includegraphics[width=0.8\linewidth]{files/helmholtzcoil-179de3570ec926b3e29eaadde6cb1a4b.png}
\caption[]{Magnetic field from each coil, as well as their sum, for two coils in the Helmholtz configuration}
\label{fig:magneticsource:helmholtzcoil}
\end{figure}

As advertised, we see a region between the Helmholtz coils where the magnetic field is nearly uniform.
\end{quote}
\end{figure}
\end{framed}

\subsection{Chapter 22 - Electromagnetic induction}

\subsubsection{Overview}\label{chapter:induction}

In this chapter, we introduce the tools to model the connection between the magnetic and the electric field. In particular, we will see how a changing magnetic field can be used to induce an electric current, which is the basic principle behind the electric generators that power our life. We will also briefly discuss how electromagnetic waves are formed.

\begin{framed}
\textbf{Learning Objectives}\\
\begin{itemize}
\item Understand how to apply Faraday's law to determine an induced voltage.
\item Understand how to model the induced voltage in a moving conductor.
\item Understand how to model an electric generator.
\item Understand how electromagnetic induction affects electric motors.
\item Understand how to model electric transformers.
\item Understand how electromagnetic waves are formed.
\end{itemize}
\end{framed}

\begin{framed}
\textbf{Think About It}\\
How does one make electricity with a hydroelectric dam?

\begin{enumerate}
\item By running water through a coil to induce a current.
\item By using water to rotate a coil inside of a fixed magnetic field.
\item By using water to charge a metallic surface by friction, and then maintaining that potential difference.
\end{enumerate}

\begin{framed}
\textbf{Answer}\\
\begin{enumerate}[resume]
\item
\end{enumerate}
\end{framed}
\end{framed}

\subsubsection{Faraday's Law}

In the previous chapter, we described how an electric current produces a magnetic field. In this chapter, we describe how an electric current can be produced (or rather, ``induced'') by a magnetic field. The most important aspect of electromagnetic induction is that it always involves quantities that change with time. In past chapters, we have only dealt with static electric and magnetic fields, static charges (for electric fields), and static currents (for magnetic fields).

Faraday's law connects the flux of a \textbf{time-varying} magnetic field to an induced voltage (rather than a current). For historical reasons, the induced voltage is also called an induced ``electromotive force'' (emf), even if it is a voltage and not a force. Faraday's law is as follows:
\begin{equation}
\boxed{\Delta V = -\frac{d\Phi_B}{dt}}
\end{equation}
where $\Delta V$ is the induced voltage, and  $\Phi_B$ is the flux of the magnetic field through an open surface, defined in the same way as the flux of the electric field (Section~\ref{sec:gauss:flux}):
\begin{equation}
\Phi_B = \int_S \vec B\cdot d\vec A
\end{equation}
If the magnetic field has a constant magnitude over the surface, $S$, and always makes the same angle with the surface, then the flux can be written as:
\begin{equation}
\Phi_B =  \vec B\cdot\vec A
\end{equation}
where the magnitude of the vector $\vec A$ is equal to the area of the surface, and the vector $\vec A$ is normal to the surface.

The surface, $S$, is defined by a closed path. The induced voltage can be thought of as an ideal battery placed in the closed path that defines the surface (right-hand panel of Figure~\ref{fig:induction:faraday}). The minus sign in Faraday's Law indicates the direction of the current associated with the induced voltage. It is important to note that an induced voltage only exists if the flux of the magnetic field changes (since the induced voltage is given by the time-derivative of the flux). Remember, induction is all about time-varying fields! This is better illustrated with an example.

Consider a loop of wire that is immersed in a uniform magnetic field, $\vec B$, that is perpendicular to the plane of the loop, as illustrated in Figure~\ref{fig:induction:faraday}. As time goes by, the magnetic field increases in strength, as shown in going from the left panel to the right panel. The flux of the magnetic field through the loop increases in magnitude, and a voltage is thus induced across the wire (illustrated by the ideal battery on the loop in the right panel), leading to an induced current, $I$.

\begin{figure}[!htbp]
\centering
\includegraphics[width=0.6\linewidth]{files/faraday-e5e8ce9b9d57169da249773be45edc7c.png}
\caption[]{As the magnetic field increases, so does the flux through the loop that is shown. The changing flux results in an induced voltage, which produces an induced current that has a magnetic moment, $\vec \mu_I$. The induced current produces a magnetic field in a direction to oppose the changing flux.}
\label{fig:induction:faraday}
\end{figure}

When calculating the flux of the magnetic field, we have to choose the surface element vector, $d\vec A$, to be perpendicular to the surface over which we calculate the flux. There are two choices\footnote{Recall that this ambiguity is resolved when using Gauss' Law by always choosing $d\vec A$ to point ``outwards'', which only makes sense when the surface is closed. With an open surface, there is no inside or outside, and we are left with the ambiguity.} (upwards or downwards, referring to Figure~\ref{fig:induction:faraday}); we \textbf{chose} to define $d\vec A$ to point upwards. Thus, the magnetic flux is positive in both panels, and increases with time. The derivative, $d\vec B/dt$, is positive and the right-hand side of Faraday's equation is negative because of the negative sign in front. Had we chosen to define $d\vec A$ to point downwards, the right-hand side of Faraday's law would be negative.

We can describe the direction of the induced current, $I$, in terms of its magnetic dipole moment (Section~\ref{sec:MagneticForce:dipolemoment}), $\vec\mu_I$, also shown in Figure~\ref{fig:induction:faraday}. The overall sign on the right-hand side of Faraday's law is determined by our (arbitrary) choice of the direction $d\vec A$. With this choice, we found that the right-hand side of Faraday's law is negative:
\begin{equation}
\Delta V = -\frac{d\Phi_B}{dt}=\text{a negative number}
\end{equation}
\textbf{The overall sign of $\Delta V$ indicates whether the magnetic moment of the induced current is parallel ($\Delta V$ positive) or anti-parallel ($\Delta V$ negative) to $d\vec A$}. This allows us to determine the direction of the induced current, and thus the direction of the ideal battery that represents the induced voltage. In general, when possible, it is common to choose the direction of $d\vec A$ to be parallel to the magnetic field vector, so that the flux is positive (although this does not guarantee that its derivative is positive).

\paragraph{Lenz's law}

The minus sign in Faraday's law is sometimes called ``Lenz's law'', and ultimately comes from the conservation of energy. In Figure~\ref{fig:induction:faraday} above, we found that as the magnetic flux increases through the loop, a current is induced. That \textbf{induced current will also produce a magnetic field} (in the direction of its magnetic dipole moment vector, $\vec \mu_I$).

Lenz's law states that the ``induced current will always be such that the magnetic field that it produces counteracts the changing magnetic field that induced the current''. In Figure~\ref{fig:induction:faraday}, the magnetic field points in the upwards direction, and increases in magnitude with time. The induced current produces a magnetic field that points downwards to counteract the changing magnetic field, and preserve a constant flux through the loop. If this were not the case, the induced current would be in the opposite direction, contributing to the increasing magnetic flux through the loop, inducing more current, producing more flux, inducing more current, etc. Clearly, this would lead to an infinite current and solve the world's energy crisis. Unfortunately, conservation of energy (expressed here as Lenz's law) prevents this from happening.

You can use Lenz's law to determine the direction of induced currents. In general:

\begin{itemize}
\item If the magnitude of the magnetic \textbf{flux is increasing} in the loop, then the induced current produces a magnetic field that is in the \textbf{opposite direction} from the original magnetic field.
\item If the magnitude of the magnetic \textbf{flux is decreasing} in the loop, then the induced current produces a magnetic field that is in the \textbf{same direction} as the original magnetic field.
\end{itemize}

The negative sign in Faraday's law is not arbitrary (as we saw above, it gives the correct direction for the magnetic moment of the induced current, given our arbitrary choice of direction for $d\vec A$). In practice, one can often use Lenz's law to determine the direction of the induced current (so that it counteracts the changing flux), and Faraday's law to determine the magnitude of the induced voltage.

\begin{framed}
\textbf{Checkpoint}\\
A loop of wire is immersed in a constant and uniform magnetic field out of the page, perpendicular to the plane of the loop, as shown in Figure~\ref{fig:induction:areaup}. If the radius of the loop increases with time, in which direction will be the current induced in the loop?

\begin{figure}[!htbp]
\centering
\includegraphics[width=0.2\linewidth]{files/areaup-dd85169b19d9857148685da749c5d617.png}
\caption[]{A loop whose radius increases with time.}
\label{fig:induction:areaup}
\end{figure}

\begin{enumerate}
\item Since the magnetic field is constant, there is no induced current.
\item Clockwise.
\item Counter-clockwise.
\end{enumerate}

\begin{framed}
\textbf{Answer}\\
\begin{enumerate}[resume]
\item
\end{enumerate}
\end{framed}
\end{framed}

\begin{framed}
\textbf{Checkpoint}\\
A loop of wire is immersed in a constant and uniform magnetic field out of the page, perpendicular to the plane of the loop, as shown in Figure~\ref{fig:induction:loopout}. If the loop is pulled out of the region of magnetic field, as shown, in which direction is the induced current in the loop?

\begin{figure}[!htbp]
\centering
\includegraphics[width=0.2\linewidth]{files/loopout-c0263da4236539f3c5aaa3e78bcf7ece.png}
\caption[]{A loop being pulled out of a region with uniform magnetic field.}
\label{fig:induction:loopout}
\end{figure}

\begin{enumerate}
\item Since the magnetic field is constant, there is no induced current.
\item Clockwise.
\item Counter-clockwise.
\end{enumerate}

\begin{framed}
\textbf{Answer}\\
\begin{enumerate}[resume]
\item
\end{enumerate}
\end{framed}
\end{framed}

\begin{framed}
\textbf{Example 22.1}\\
A uniform time-varying magnetic field is given by:
\begin{equation}
\vec B(t) = B_0(1+at)\hat z
\end{equation}
where $B_0$ and $a$ are positive constants. A coil, made of $N$ circular loops of radius, $r$, lies in the $x -y$ plane. If the coil has a total resistance, $R$, what is the magnitude and direction of the current induced in the coil?

\begin{framed}
\textbf{Solution}\\
The coil is made of $N$ loops of wire. Each loop of wire can be treated independently, and each will have its own induced voltage across it. Since each loop is the same, they will all have the same induced voltage, and the total voltage induced across the coil, $\Delta V$, will be given by:
\begin{equation}
\Delta V = -N \frac{d\Phi_B}{dt}
\end{equation}
where $\Phi_B$ is the flux through any one of the loops. That is, each loop is similar to an ideal battery, and the coil is similar to placing all of these batteries in series, so that the voltages from each battery sum together.

The coil lies the $x -y$ plane, perpendicular to the increasing magnetic field, similar to the situation depicted in Figure~\ref{fig:induction:faraday}. Since the magnetic field is uniform over the surface of the coil, we do not need an integral to determine the flux. We define the area vector, $\vec A$, to be in the positive $z$ direction (parallel to the magnetic field):
\begin{equation}
\vec A = A \hat z = \pi r^2 \hat z
\end{equation}
The flux through one circular loop of radius, $r$, is given by:
\begin{equation}
\Phi_B (t) &= \vec B \cdot \vec A = ( B_0(1+at)\hat z) \cdot (\pi r^2 \hat z) =B_0(1+at) (\pi r^2)
\end{equation}
We can apply Faraday's law to determine the induced voltage:
\begin{equation}
\Delta V &= -N \frac{d\Phi_B}{dt} = -N \frac{d}{dt} B_0(1+at) (\pi r^2)\\
&=-NB_0a\pi r^2
\end{equation}
Since the induced voltage is negative, the magnetic moment of the induced current points in the negative $z$ direction (opposite to our choice of direction for $\vec A$). This is consistent with Lenz's law, since the magnetic field increases in the positive $z$ direction, the induced current will produce a magnetic field in the negative $z$ direction to counteract the changing flux. The magnitude of the induced current is given by Ohm's Law:
\begin{equation}
I = \frac{\Delta V}{R}=\frac{NB_0a\pi r^2}{R}
\end{equation}

\textbf{Discussion:} In this example, we determined the induced voltage and current in a coil made of $N$ identical loops. We argued that one can sum the induced voltages from the $N$ loops, as these can be thought of as ideal batteries in series. We found that the direction of the induced current as obtained from Faraday's law was consistent with the expectation from Lenz's law.
\end{framed}
\end{framed}

\begin{framed}
\textbf{Olivia's Thoughts}\\
Here are some steps you can follow to find the direction of the current using Lenz's law.

\begin{enumerate}
\item Draw a diagram showing the loop/coil and the magnetic field lines.
\item I like to indicate whether the flux is increasing or decreasing by drawing a ``flux arrow'' (a term that I made up, so please don't use it around physicists because they won't know what you're talking about). If the flux is increasing, this will point in the same direction as the field lines. If it is decreasing, the flux arrow will point opposite to the field lines.
\item Assume that the induced current is in the clockwise direction and use the axial right hand rule to determine the direction of the induced magnetic field.
\item Repeat Step 2 assuming the counter-clockwise direction.
\item Decide which direction for the induced current will give you the desired field. We want the induced field to point opposite to the flux arrow.
\end{enumerate}

This is how you would apply this method to Example~22.1 (note that I am using the example of a loop instead of a coil but the idea is the same):

\begin{enumerate}
\item We draw the diagram, as in Figure~\ref{fig:induction:lenzs_example}a.
\item To show that the flux is increasing, I have drawn a ``flux arrow'' (again, made up) in the direction of the field lines (the $+z$ direction).
\item If the current is clockwise, the induced field points in the $-z$ direction inside the loop (Figure~\ref{fig:induction:lenzs_example}b).
\item If the current is counter-clockwise, the induced field points in the $+z$ direction inside the loop (Figure~\ref{fig:induction:lenzs_example}c).
\item Since the flux is increasing, we want an induced current that will decrease the flux. We choose the clockwise current because the induced field points opposite to the flux arrow.
\end{enumerate}

\begin{figure}[!htbp]
\centering
\includegraphics[width=0.9\linewidth]{files/lenzs_example-d7f248103d55bc911f37a92a8fc83217.png}
\caption[]{(a) An increasing magnetic field through a loop. (b) A clockwise current induces a magnetic field in the $-z$ direction. (c) A counter-clockwise current induces a magnetic field in the $+z$ direction.}
\label{fig:induction:lenzs_example}
\end{figure}
\end{framed}

\subsubsection{Induction in a moving conductor}

If we define a loop of wire, there are two ways in which the magnetic flux through that loop can change:

\begin{enumerate}
\item The magnetic field can change magnitude or direction, as we saw in Example~22.1.
\item The loop can change size or orientation relative to the magnetic field.
\end{enumerate}

In this section, we examine the latter case, sometimes called ``motional emf'', as the induced voltage is the result of motion from the loop in which the voltage is induced.

\paragraph{Motion of a bar on two parallel rails}

Consider a U-shaped rail in a uniform magnetic field on top of which a bar can slide with no friction, as illustrated in Figure~\ref{fig:induction:rail}. The bar of length $L$ moves to the right with a constant speed, $v$.

\begin{figure}[!htbp]
\centering
\includegraphics[width=0.6\linewidth]{files/rail-8a3f3d543097c528778d803819acd89a.png}
\caption[]{A U-shaped rail on top of which a bar of length, $L$, can slide. The system is immersed in a magnetic field that points out of the page. The bar moves to the right with a constant speed, $v$.}
\label{fig:induction:rail}
\end{figure}

The bar and the rails form a closed loop of area:
\begin{equation}
A(t)=Lw(t)=Lvt
\end{equation}
that increases with time. The magnitude of the flux through the loop will increase with time, resulting in an induced current (clockwise, according to Lenz's law). At some time $t$, the flux through the loop is given by:
\begin{equation}
\Phi_B (t) &=  \vec B \cdot \vec A =BA=BLvt
\end{equation}
where we chose $\vec A$ to be parallel to the magnetic field vector.

Since we already used Lenz's law to argue that the current must be in the clockwise direction, we can use Faraday's law to determine the magnitude of the induced voltage and ignore the negative sign:
\begin{equation}
\Delta V = \frac{d \Phi_B}{dt}=\frac{d}{dt}BLvt = BLv
\end{equation}

Suppose that the rails are superconducting (have no resistance), and that the bar has a resistance, $R$. The current through the loop is then given by Ohm's Law:
\begin{equation}
I=\frac{\Delta V}{R}=\frac{BLv}{R}
\end{equation}
As the current moves through the bar, it will heat up the bar by dissipating energy at a rate of:
\begin{equation}
P=I^2 R = \frac{B^2L^2v^2}{R}
\end{equation}
Thus, the bar cannot possibly move at a constant speed on its own, or energy would be produced out of nothing. There must be a force exerted on the bar to keep it moving at constant speed.

Recall that a current-carrying wire in a magnetic field will experience a force from the magnetic field. In this case, the bar of length $L$ carries a current, $I$, in a magnetic field, $\vec B$ (perpendicular to the current), so that the force exerted on the bar is given by:
\begin{equation}
\vec F_B = I \vec L \times \vec B
\end{equation}
and points to the left (right-hand rule). The magnitude of the force is given by:
\begin{equation}
F_B = ILB = \frac{B^2L^2v}{R}
\end{equation}
Thus, in order for the bar to move at constant velocity towards the right, a force with the same magnitude must be exerted towards the right. In other words, work must be done to pull the bar to the right, by exerting a force with the magnitude, $F_B$. The rate at which that work must be done is given by:
\begin{equation}
P &= \frac{d}{dt}W\\
&=\frac{d}{dt}\vec F \cdot dx\\
&=\vec F\cdot \frac{dx}{dt}\\
&=\vec F\cdot \vec v = Fv\\
&=\frac{B^2L^2v^2}{R}
\end{equation}
where we assumed that the bar moves in the positive $x$ direction. This is exactly the rate at which electric energy is dissipated in the bar! In other words, by doing mechanical work on the bar, we can create an induced current that will dissipate that energy at the same rate at which we do work. We can convert mechanical work into electrical energy!

\begin{framed}
\textbf{Olivia's Thoughts}\\
I'll quickly sum up what is happening in this example:

\begin{enumerate}
\item We pull the bar to the right.
\item The flux in the loop increases, which produces a current in the clockwise direction according to Lenz's law. We can calculate the potential difference to produce this current using Faraday's law.
\item The downwards current in the bar produces a force to the left according to the Lorentz force, $F=I\vec{l}\times \vec{B}$. To move the bar at a constant velocity, the force we apply to pull the bar should be equal to the Lorentz force.
\item Assuming the bar has a resistance, power is dissipated in the resistor. Power is dissipated at the same rate as the work done by us to pull the bar.
\end{enumerate}
\end{framed}

Finally, also note that this situation is closely related to the Hall effect, which is simply a different way to think about this problem. Consider the electrons that are in the bar, as the bar moves at constant speed to the right through the magnetic field (ignore the existence of the U-shaped rail). The electrons will experience a magnetic force that is upwards (consistent with the direction of the induced current discussed above). Eventually, electrons accumulate at the top of the bar, and start preventing more electrons from accumulating there, by producing an electric field, $\vec E$, in the bar. The equilibrium condition is that the magnetic force and the electric force have the same magnitude (and opposite directions):
\begin{equation}
qvB &= qE\\
E &= vB
\end{equation}
The (Hall) potential difference, across the bar of length, $L$, with an electric field, $E$, is given by:
\begin{equation}
\Delta V_{Hall} = EL = vBL
\end{equation}
where we assumed that the electric field is uniform in the bar. This potential difference is identical to the one that we calculated from Faraday's law. Viewing this example as a different manifestation of the Hall effect provides some insight into what is actually happening at the microscopic level when a current is induced.

\paragraph{The generator}

An electrical generator is used to create an alternating induced voltage/current by rotating a coil inside of a constant and uniform magnetic field. In this case, the current is induced because the angle between the magnetic field and the surface element vector $d\vec A$ changes with time.

Consider a single loop of wire with area $A$ that can rotate in a uniform and constant magnetic field, $\vec B$, as illustrated in Figure~\ref{fig:induction:generator}.

\begin{figure}[!htbp]
\centering
\includegraphics[width=0.6\linewidth]{files/generator-d90ce872c1eb265a4b57a78c94dce269.png}
\caption[]{A loop of wire rotates in a constant and uniform magnetic field. At time $t=0$ (left panel), the loop lies in the $yz$ plane. The loop rotates about the $y$ axis, with a constant angular velocity, $\vec \omega$. At some time $t$ later, the loop has rotated through an angle $\theta = \omega t$ (right panel, as seen from above, looking down on the $xz$ plane).}
\label{fig:induction:generator}
\end{figure}

Referring to the coordinate system that is illustrated in Figure~\ref{fig:induction:generator}, the loop has a constant angular velocity, $\vec\omega$, in the positive $y$ direction and rotates about the $y$ axis (with the origin at the centre of the coil). At time $t=0$ (left panel), the loop lies in the $yz$ plane, and we choose the vector $\vec A$ (used to calculate the flux) to be in the positive $x$ direction at time $t=0$. As the coil rotates, so will the vector $\vec A$, which is easier to visualize than the coil. At some time $t$, the vector $\vec A$ will make an angle $\theta=\omega t$ with the $x$ axis (right panel). The magnetic field is constant and in the positive $x$ direction, $\vec B = B\hat x$. That is, the angle between the vector $\vec A$ and the magnetic field, $\vec B$, will be given by $\theta = \omega t$.

At some time, $t$, the vector $\vec A$ is given by:
\begin{equation}
\vec A(t) = A(\cos\theta \hat x -\sin\theta \hat z) = A(\cos(\omega t) \hat x -\sin(\omega t)\hat z)
\end{equation}

We can calculate the flux of the magnetic field through the loop at some time $t$:
\begin{equation}
\Phi_B(t) =  \vec B \cdot \vec A = (B\hat x) \cdot (A\cos(\omega t) \hat x -A\sin(\omega t)\hat z)=AB\cos(\omega t)
\end{equation}
where we did not use the integral for the flux, since the magnetic field is constant over the area of the loop. The induced voltage is given by Faraday's law:
\begin{equation}
\Delta V = - \frac{d\Phi_B}{dt}  =  - \frac{d}{dt}AB\cos(\omega t) =  AB\omega\sin(\omega t)
\end{equation}
If the generator includes $N$ loops in a coil, then the induced voltage is given by:
\begin{equation}
\Delta V = NAB\omega\sin(\omega t)
\end{equation}
As you can see, the voltage oscillates with time, between $\pm NAB\omega$, corresponding to alternating voltage. Furthermore, since the sign of $\Delta V$ changes with time (due to the sine function), the relative orientation between $\vec A$ and the magnetic dipole moment of the induced current, also changes with time, indicating that the induced current in the coil changes direction every half-turn (alternating current).

The generators that produce the alternating voltages that we find in our outlets work on the same principle. For example, in a hydro-electric dam, the water pressure from the height of the dam is used to force water through a turbine (essentially a propeller) that rotates a set of coils inside of a strong permanent magnet. Various controls allow the rotational frequency of the turbine to be adjusted in order to produce alternating current of the desired frequency ($50 {\rm Hz}$ in most of the world, $60 {\rm Hz}$ in North America and a few other countries).

Since the generator produces current that can dissipate electrical energy, one has to do work in order to keep the coil in the generator rotating. As the coil rotates, a current is induced in the coil. A current in a circular loop that is immersed in a magnetic field will experience a torque, $\vec \tau$, given by:
\begin{equation}
\vec \tau = \vec \mu \times \vec B
\end{equation}
where $\vec \mu$ is the magnetic dipole moment of the coil with induced current, $I$. If the current from the coil dissipates its energy in a system with resistance, $R$, then the current in the coil is given by Ohm's Law:
\begin{equation}
I = \frac{\Delta V}{R}=\frac{NAB\omega\sin(\omega t)}{R}
\end{equation}
The magnetic moment, $\vec \mu$, for the current in the coil is given by:
\begin{equation}
\vec \mu &= I\vec A = \frac{NAB\omega\sin(\omega t)}{R} (A(\cos(\omega t) \hat x -\sin(\omega t)\hat z))\\
&=\frac{NA^2B\omega\sin(\omega t)}{R} (\cos(\omega t) \hat x -\sin(\omega t)\hat z)
\end{equation}
The torque exerted by the magnetic field on the coil with the induced current is thus given by:
\begin{equation}
\vec \tau &= \vec \mu \times \vec B = \left(\frac{NA^2B\omega\sin(\omega t)}{R} (\cos(\omega t) \hat x -\sin(\omega t)\hat z)\right) \times (B\hat x)\\
&=\frac{NA^2B^2\omega\sin(\omega t)}{R}(\cos\omega(t)(\hat x \times \hat x)-\sin(\omega t)(\hat z \times \hat x))\\
&=-\frac{NA^2B^2\omega\sin^2(\omega t)}{R}\hat y
\end{equation}
Note that the torque exerted on the loop is always in the negative $y$ direction, as every term in the torque is either strictly positive ($N,R$) or squared ($\sin^2(\omega t)$). The torque exerted by the magnetic field on the coil is thus always in the opposite direction of rotation (recall that the coil has an angular velocity in the positive $y$ direction). This is sometimes called ``counter torque''. If we want the coil to maintain a constant angular velocity, then we must exert a torque in the positive $y$ direction to counter the torque from the magnetic field. Note that the torque that we must exert to keep the coil rotating with constant angular velocity is not constant in time (but always in the same direction).

You can easily verify that the work that you must do by exerting the torque is the same as the electrical power dissipated by the current in the resistor, $R$. The generator is thus a device to convert mechanical work into electrical energy (with AC current, in particular).

\subsubsection{Back EMF in an electric motor}

There are many similarities between electric motors and generators, and in fact, they can be thought of as the same device. In an electric motor, current is passed through a coil in a magnetic field, so that a torque is exerted on the coil, and it starts to rotate. In a generator, one exerts a torque to rotate the coil, thus inducing a current.

Consider an electric motor. As we supply current to the motor, the coil starts to rotate. But, a rotating coil in a magnetic field results in an induced current. By Lenz's law, the induced current in the coil of a motor has to be in the direction opposite to the current that we put in, since otherwise, the motor would start to spin infinitely fast. We call this effect ``back emf'', as the motor effectively acts like a battery that opposes current, as illustrated in Figure~\ref{fig:induction:backemf}

\begin{figure}[!htbp]
\centering
\includegraphics[width=0.3\linewidth]{files/backemf-884fc6af7b406bccea4cff3c4f67981b.png}
\caption[]{A simple circuit illustrating how a motor, with resistance, $R_{motor}$, will generate a ``back emf'', equivalent to a battery that produces a voltage in the direction to oppose the current from the actual battery that is powering the motor, $\Delta V$.}
\label{fig:induction:backemf}
\end{figure}

If you connect an electric motor to a voltage source, initially, the motor is at rest, so there will be no back emf and the current through the circuit will be very large (motors have a small resistance, so that the electrical energy is converted into work rather than heating up the motor). As the motor starts to spin faster, the back emf from the motor grows, reducing the current in the circuit. If there is no load on the motor (i.e. the motor can rotate freely with no friction), then the rotational speed of the motor will increase until the back emf exactly matches the voltage supplied to the motor. The motor will then rotate at constant speed, with (almost) no current in the circuit (if the motor slows down, the emf will decrease, and the current will increase to speed up the motor). If there is a load on the motor (because it's making something turn), then the motor will rotate at a speed that is lower than that which would result in zero current, since some of that current is now used by the motor to exert a torque.

You may notice that the lights in your house dim briefly as your refrigerator turns on. This is because your refrigerator uses an electric motor that initially draws a large current when it turns on, large enough to produce a voltage drop in the circuit of your house to observe a dimming of your lights. You may also notice that if you plug the inlet or outlet of a hair dryer, the hair dryer turns off quickly. In this case, by blocking the flow of air, you prevent the motor in the hair dryer from rotating; this results in a large current through its coil, since there is no back emf. Most hair dryers have a circuit breaker that will detect this large current and open the circuit to prevent the coil in the motor from over heating and melting. In general, one should not prevent an electric motor from rotating, as this will result in a large current through the motor that could melt its internal components.

\subsubsection{The induced electric field and eddy currents}

So far, we have described electromagnetic induction in terms of the voltage that is induced by a changing magnetic field. This voltage is related to an electric field, which we discuss in this section. In Faraday's  Law, the voltage is induced across a closed loop (and can be thought of as an ideal battery placed in the loop). This is illustrated in Figure~\ref{fig:induction:inducedE} which shows a loop in the plane of the page, and a magnetic field out of the plane of the page.

\begin{figure}[!htbp]
\centering
\includegraphics[width=0.4\linewidth]{files/inducedE-7bdece303343897b313b1edc5e6f97b8.png}
\caption[]{A varying magnetic field will induce a circular electric field.}
\label{fig:induction:inducedE}
\end{figure}

\begin{framed}
\textbf{Checkpoint}\\
In Figure~\ref{fig:induction:inducedE}, in order to produce the induced voltage as shown, is the magnetic field increasing or decreasing?

\begin{enumerate}
\item The magnetic field is increasing.
\item The magnetic field is decreasing.
\end{enumerate}

\begin{framed}
\textbf{Answer}\\
\begin{enumerate}[resume]
\item
\end{enumerate}
\end{framed}
\end{framed}

As you recall, the electric potential difference between two points, $A$ and $B$, is obtained from the electric field:
\begin{equation}
\Delta V = \int_A^B \vec E\cdot d\vec l
\end{equation}
In the case of an induced voltage across a loop, the points $A$ and $B$ are the same. The integral is thus over a closed path:
\begin{equation}
\Delta V = \oint \vec E\cdot d\vec l
\end{equation}
We can include this into Faraday's law by using the electric field instead of the potential difference:
\begin{equation}
\Delta V =-\frac{d\Phi_B}{dt}\\
\end{equation}
\begin{equation}
\therefore \;\;\boxed{\oint \vec E\cdot d\vec l = -\frac{d\Phi_B}{dt}}
\end{equation}
where the last line is a more general form of Faraday's law. Note that in the case of electrostatics, where the electric field is produced by a distribution of charges, the integral $\oint \vec E\cdot d\vec l$ must be zero, since the electric force is conservative; the work done by the electric field on a charge $q$ over a closed path, which is just a charge $q$ multiplied by that integral, must be zero. The force from an electric field that is induced by a time-varying magnetic field is not conservative!

Faraday's law as expressed with the electric field is much more general, and implies that a time-varying magnetic field will induce an electric field. This is true, independently of there existing a physical wire to carry the induced current.

\begin{framed}
\textbf{Example 22.2}\\
A circular region with radius $R$ contains a magnetic field that is uniform, and decreasing in magnitude with time:
\begin{equation}
\vec B(t) = B_0(1-at)\hat z
\end{equation}
where $a$ and $B_0$ are positive constants. Determine the electric field at a distance, $r$, from the centre of the region, inside and outside of the region with the magnetic field.

\begin{framed}
\textbf{Solution}\\
Figure~\ref{fig:induction:inducedE2} shows the circular region of magnetic field, as well as a circular path of radius $r$ that defines the region over which we calculate the flux of the magnetic field.

\begin{figure}[!htbp]
\centering
\includegraphics[width=0.4\linewidth]{files/inducedE2-d4dcf4f817ea58569abf12bbaa5f7727.png}
\caption[]{The induced electric field lines form closed circles when the magnetic field changes.}
\label{fig:induction:inducedE2}
\end{figure}

First, we consider the induced electric field in the region with a magnetic field, where $r<R$. We choose a circle of radius $r$ to calculate the flux of the magnetic field. Since the magnetic field is uniform within that region, the flux is given by:
\begin{equation}
\Phi_B = \vec B \cdot \vec A = BA = B_0(1-at) \pi r^2
\end{equation}
The circulation of the electric field is easily found, since the electric field forms concentric circles (by symmetry):
\begin{equation}
\oint \vec E \cdot d\vec l = \oint Edl = E \oint dl = E(2\pi r)
\end{equation}
Applying Faraday's law, the electric field is found to be:
\begin{equation}
\oint \vec E\cdot d\vec l &= -\frac{d\Phi_B}{dt}\\
E(2\pi r) &= -\frac{d}{dt} B_0(1-at) \pi r^2\\
2E &=  B_0ar\\
\therefore E&=\frac{B_0a}{2}r\quad\text{(inside the region of magnetic field)}
\end{equation}
and we see that, inside the region with the magnetic field, the strength of the induced electric field is proportional to the distance from the centre of the region (i.e. it increases linearly with $r$).

For the region where the magnetic field is zero, we again calculate the circulation of the electric field around a circular loop of radius $r>R$:
\begin{equation}
\oint \vec E \cdot d\vec l = \oint Edl = E \oint dl = E(2\pi r)
\end{equation}
The flux of the magnetic field through that loop is however related to the area of the region with the magnetic field (of radius, $R$):
\begin{equation}
\Phi_B = \vec B \cdot \vec A = BA = B_0(1-at) \pi R^2
\end{equation}
Again, applying Faraday's law:
\begin{equation}
\oint \vec E\cdot d\vec l &= -\frac{d\Phi_B}{dt}\\
E(2\pi r) &= -\frac{d}{dt} B_0(1-at) \pi R^2\\
2Er&=  B_0aR^2\\
\therefore E&=\frac{B_0aR^2}{2r}\quad\text{(outside the region of magnetic field)}
\end{equation}
Outside the region with a magnetic field, the magnitude of the electric field decreases with the distance from the centre of the region.

\textbf{Discussion:} In this example, we determined the electric field that is induced by a varying magnetic field. In this case, the electric field lines form closed circles and result in a non-conservative force. When the electric field is formed by a distribution of electric charges, the field lines begin and end on charges, which is not the case for an induced electric field.
\end{framed}
\end{framed}

\begin{framed}
\textbf{Olivia's Thoughts}\\
You'll notice that this version of Faraday's law has a very similar structure to Ampère 's law. Recall that Ampere's law states:
\begin{equation}
\oint \vec B \cdot d\vec l = \mu_0I^{enc}
\end{equation}
The difference is that now we have the circulation of the electric field rather than the magnetic field and instead of the enclosed current we are dealing with the rate of change of the enclosed flux. This is why, in the last example, we used an almost identical process to how we use Ampère 's law.
\end{framed}

\paragraph{Magnetic braking}

When a conducting material moves into a region of magnetic field, an electric field forming closed loops is induced in the material, thus inducing small current loops, called ``eddy currents''. The magnetic field can then exert a force on those currents, effectively resulting in a force on the material. This is the principle behind magnetic braking, which is used in some trains and in other applications.

Figure~\ref{fig:induction:magneticbrake} illustrates how a magnetic brake can be used to slow a rotating wheel made of a conducting material (the material must conduct or the induced electric field will not produce any current). A magnetic field is produced (e.g. by a fixed permanent magnet) in a direction perpendicular to the wheel, over a small area (shown at the bottom of the wheel in Figure~\ref{fig:induction:magneticbrake}).

\begin{figure}[!htbp]
\centering
\includegraphics[width=0.6\linewidth]{files/magneticbrake-912a80c2e0d665014177fc2883ff92a1.png}
\caption[]{A rotating wheel made of a conducting material has a small region with a magnetic field. The eddy currents in the region of changing flux result in a net downwards current at the centre of the region. The magnetic force that is exerted on that current slows down the wheel.}
\label{fig:induction:magneticbrake}
\end{figure}

For material located at the bottom left of the wheel, the magnetic flux is increasing, since the material is moving from a region with no magnetic field into a region with a magnetic field. In that part of the region, clockwise eddy currents will form, as those result in a magnetic field into the page, to counter the increasing magnetic flux (Lenz's law). The bottom right side of the wheel is leaving the magnetic field, and will thus have eddy currents in the opposite direction. The currents from both sides add up in the centre, resulting in a net downwards current. The magnetic force on that downwards current is to the left, resulting in a torque that slows the wheel. This is magnetic braking.

Again, this is no more than conservation of energy at play. Since we induce currents by making the wheel move into/out of a region of magnetic field, the electrical energy in those currents must come from somewhere (either we do work to keep the wheel rotating, or the wheel loses kinetic energy). Any time that we try to move a conductor through a magnetic field, in a way that current is induced, we will have to exert a force and do work. In the case of magnetic braking, the wheel will convert its rotational kinetic energy into heat (the eddy currents will heat up the wheel). The main issue with magnetic braking is that one needs to be able to dissipate the heat. The main advantage is that there are no parts that wear out, as opposed to braking with friction. In addition, magnetic braking is very smooth, and only acts when there is motion. As soon as the wheel stops rotating, the magnetic flux is constant everywhere and the eddy currents disappear.

\begin{framed}
\textbf{Checkpoint}\\
Suppose that the magnetic field in Figure~\ref{fig:induction:magneticbrake} pointed into the page. Would the magnetic break still work?

\begin{enumerate}
\item Yes.
\item No.
\end{enumerate}

\begin{framed}
\textbf{Answer}\\
\begin{enumerate}
\item
\end{enumerate}
\end{framed}
\end{framed}

\subsubsection{Transformers}

The electric power generated in power stations is transmitted using high-voltage transmission lines, typically with voltages above $300000 {\rm V}$ for long distances. However, that voltage is not usable in our households, as our appliances expect a voltage around $120 {\rm V}$ (or $220 {\rm V}$ in Europe). Transformers use electromagnetic induction to transform one \textbf{alternating voltage} into another. Figure~\ref{fig:induction:transformer} illustrates a transformer.

\begin{figure}[!htbp]
\centering
\includegraphics[width=0.5\linewidth]{files/transformer-bcf11362c9aec157a40b39b5e34827ca.png}
\caption[]{A transformer converts a primary alternating voltage, $\Delta V_p$, to a secondary alternating voltage, $\Delta V_s$. The magnetic flux produced in one coil is transmitted by an iron core to the secondary coil, where a different voltage is induced, depending on the ratio of the number of windings in each coil.}
\label{fig:induction:transformer}
\end{figure}

The transformer has two coils, the ``primary'' and the ``secondary'', with different numbers of loops, $N_p$ and $N_s$, respectively. The coils are wrapped around an iron core, which can transmit the magnetic flux generated in the primary coil to the secondary coil. In the transformer, an alternating voltage, $\Delta V_p$, is applied to the primary coil, and transformed into the desired voltage, $\Delta V_s$, in the secondary coil.

The current in the primary coil creates a magnetic field. Those field lines are transmitted by the iron core into the second coil. A voltage is only induced in the secondary coil if the magnetic flux through the secondary coil changes with time. Thus, transformers only work with alternating voltages, so that the magnetic field created by the primary coil changes continuously. Both coils will have the same magnetic flux, $\Phi_B$, through them, since they have the same area. The voltage in the primary coil is given by Faraday's law:
\begin{equation}
\Delta V_p = N_p \frac{d\Phi_B}{dt}
\end{equation}
as is the voltage in the secondary coil:
\begin{equation}
\Delta V_s = N_s \frac{d\Phi_B}{dt}
\end{equation}
Since the flux (and thus its time-derivative) are the same in both coils, we can isolate the time-derivative in each equation to obtain the relationship between the voltages in the two coils:
\begin{equation}
\frac{\Delta V_p}{N_p}&=\frac{\Delta V_s}{N_s}\\
\therefore \Delta V_s &= \frac{N_s}{N_p}\Delta V_p
\end{equation}
Thus, with a transformer, one simply needs to set the ratio of the number of loops in each coil in order to transform one voltage into another.

\begin{framed}
\textbf{Checkpoint}\\
Which coil in Figure~\ref{fig:induction:transformer} has the highest voltage?

\begin{enumerate}
\item The one with the most loops.
\item The one with the least loops.
\end{enumerate}

\begin{framed}
\textbf{Answer}\\
\begin{enumerate}
\item
\end{enumerate}
\end{framed}
\end{framed}

\begin{framed}
\textbf{Checkpoint}\\
Which coil in Figure~\ref{fig:induction:transformer} will have the highest current?

\begin{enumerate}
\item The one with the most loops.
\item The one with the least loops.
\item Not enough information to tell.
\end{enumerate}

\begin{framed}
\textbf{Answer}\\
\begin{enumerate}[resume]
\item
\end{enumerate}
\end{framed}
\end{framed}

\begin{framed}
\textbf{Example 22.3}\\
A power plant produces energy at rate of $P=150 {\rm kW}$, and wishes to transmit this power as efficiently as possible to a town. The power lines between the power plant and the town have a resistance of $R=0.5 {\rm \Omega}$. Compare the amount of power dissipated in the transmission lines depending on whether the power is transmitted through a voltage of $300000 {\rm V}$ or $300 {\rm V}$.

\begin{framed}
\textbf{Solution}\\
We model the transmission of power from the power plant to the town using the circuit shown in Figure~\ref{fig:induction:powerplant}.

\begin{figure}[!htbp]
\centering
\includegraphics[width=0.25\linewidth]{files/powerplant-b4ec4cc447657225fc944a1bd4e3ab72.png}
\caption[]{Circuit for a power plant transmitting power to a town.}
\label{fig:induction:powerplant}
\end{figure}

We do not know the resistance of the town, but we can still calculate the power that is dissipated in the transmission lines that have a total resistance of $R=0.5 {\rm \Omega}$. The power plant produces power, $P$, and transmits it through the lines at a potential difference, $\Delta V$, resulting in a current, $I$:
\begin{equation}
P &= I\Delta V\\
\therefore I &= \frac{P}{\Delta V}
\end{equation}
The current, $I$, will dissipate power in the lines at a rate of:
\begin{equation}
P_{line} = I^2 R = \frac{P^2}{\Delta V^2}R
\end{equation}
With the two different voltages, this corresponds to:
\begin{equation}
P_{line}&=\frac{P^2}{\Delta V^2}R=\frac{(150e3 {\rm W})^2}{(300000 {\rm V})^2}(0.5 {\rm \Omega})=0.1 {\rm W}\\
P_{line}&=\frac{P^2}{\Delta V^2}R=\frac{(150e3 {\rm W})^2}{(300 {\rm V})^2}(0.5 {\rm \Omega})=125000 {\rm W}\\
\end{equation}
Thus, when the power is transmitted at low voltage, more than 80\% is dissipated in the transmission lines, whereas an insignificant fraction is dissipated when the power is transmitted at high voltage. This is why we need transformers.
\end{framed}
\end{framed}

\subsubsection{Maxwell's equations and electromagnetic waves}

This section is meant to be informative, as the material is beyond the scope of this textbook. Nonetheless, it is worth summarizing what we have learned about electricity and magnetism, as Maxwell did. We can summarize the main laws from electromagnetism as follows:
\begin{equation}
\oint \vec E\cdot d\vec A &= \frac{Q}{\epsilon_0}&\text{(Gauss' Law)}\\
\oint \vec B\cdot d\vec A &= 0 &\text{(No magnetic monopoles)}\\
\oint \vec B\cdot d\vec l &= \mu_0 I^{enc} &\text{(Ampère's Law)}\\
\oint \vec E\cdot d\vec l &= -\frac{d}{dt}\int \vec B\cdot d\vec A  &\text{(Faraday's law)}\\
\end{equation}
where we wrote the magnetic flux in Faraday's law using the integral explicitly. As you recall, Gauss' Law is equivalent to Coulomb's Law, relating the electric field to electric charges that produce the electric field. Although we did not explicitly use the second equation, it is the equivalent to Gauss' Law for the magnetic field. The flux of the magnetic field out of a closed surface must always be zero, since there are no magnetic monopoles, so that magnetic field lines never end.

When we covered Ampère's Law, we only considered a static current as the source of the magnetic field. However, if there is an electric field present that is created by charges that are moving, then those can also contribute a current to Ampère's Law:
\begin{equation}
\oint \vec E\cdot d\vec A &= \frac{Q}{\epsilon_0}\quad \text{(Gauss' Law)}\\
\therefore Q &= \epsilon_0 \oint \vec E\cdot d\vec A\\
\therefore I &= \frac{dQ}{dt} = \epsilon_0\frac{d}{dt} \oint \vec E\cdot d\vec A\\
\end{equation}
so that Ampère's Law, in its most general form, is written:
\begin{equation}
\oint \vec B\cdot d\vec l &= \mu_0 \left(I^{enc}+\epsilon_0\frac{d}{dt} \oint \vec E\cdot d\vec A\right)\quad \text{(Ampère's Law)}
\end{equation}
Writing out the four equations again:
\begin{equation}
\oint \vec E\cdot d\vec A &= \frac{Q}{\epsilon_0} &\text{(Gauss' Law)}\\
\oint \vec B\cdot d\vec A &= 0 &\text{(No magnetic monopoles)}\\
\oint \vec B\cdot d\vec l &= \mu_0 \left(I^{enc}+\epsilon_0\frac{d}{dt} \oint \vec E\cdot d\vec A\right) &\text{(Ampère's Law)}\\
\oint \vec E\cdot d\vec l &= -\frac{d}{dt}\int \vec B\cdot d\vec A  &\text{(Faraday's law)}\\
\end{equation}
These four equations are known as Maxwell's equations, and form our most complete theory of classical electromagnetism. It is quite interesting to note the similarities and relations between the electric and magnetic field. Maxwell's equations contain equations for the circulation and the total flux out of a closed surface for both fields. Ampère's Law implies that a changing electric field will produce a magnetic field. Faraday's law implies that a changing magnetic field produces an electric field. If a point charge oscillates up and down, it will produce a changing electric field, which will produce a changing magnetic field, which will induce a changing magnetic field, etc. This is precisely what an electromagnetic wave is! The light that we see, the Wi-Fi signals for our phones, and the highly penetrating radiation from nuclear reactors are all examples of electromagnetic waves (of different wavelengths).

In fact, as Maxwell did, we can obtain the wave equation (Section~\ref{sec:waves:waveequation}) from Maxwell's equations. We sketch out the derivation here, but it is definitely beyond the scope of this textbook. However, you're so close to seeing one of the most exciting revelations of physics that it would be a shame to skip it!

We first write out Maxwell's equations in differential form, as we have already shown for Gauss' Law ((Section~\ref{sec:gauss:interpretation}) and Ampère's Law ((Section~\ref{sec:magneticsource:interpretation})
\begin{equation}
 \nabla \cdot \vec E &= \frac{\rho}{\epsilon_0} &\text{(Gauss' Law)}\\
 \nabla \cdot \vec B&= 0 &\text{(No magnetic monopoles)}\\
 \nabla \times \vec B &= \mu_0 \left(\vec j + \epsilon_0\frac{\partial\vec E}{\partial t}\right) &\text{(Ampère's Law)}\\
 \nabla \times \vec E &= -\frac{\partial\vec B}{\partial t} &\text{(Faraday's law)}\\
\end{equation}
If we consider a vacuum region in space, with no charges and no currents, these equations reduce to:
\begin{equation}
\nabla \cdot \vec E &= 0 ~~~~&\nabla \cdot \vec B&= 0\\
\nabla \times \vec B &= \mu_0 \epsilon_0\frac{\partial\vec E}{\partial t} ~~~~& \nabla \times \vec E &= -\frac{\partial\vec B}{\partial t}
\end{equation}
We will make use of the following identity from vector calculus:
\begin{equation}
\nabla \times (\nabla \times \vec E)=\nabla(\nabla\cdot \vec E)-\nabla^2\vec E
\end{equation}
where:
\begin{equation}
\nabla^2\vec E &= \frac{\partial^2 \vec E}{\partial x^2}+\frac{\partial^2 \vec E}{\partial y^2} + \frac{\partial^2 \vec E}{\partial z^2}\\
&=\left(\frac{\partial^2 E_x}{\partial x^2}+\frac{\partial^2  E_x}{\partial y^2} + \frac{\partial^2 E_x}{\partial z^2} \right) \hat x+ \left( \frac{\partial^2 E_y}{\partial x^2}+\frac{\partial^2  E_y}{\partial y^2} + \frac{\partial^2 E_y}{\partial z^2} \right) \hat y \\
&+ \left(\frac{\partial^2 E_z}{\partial x^2}+\frac{\partial^2  E_z}{\partial y^2} + \frac{\partial^2 E_z}{\partial z^2}  \right) \hat z
\end{equation}
is called the ``vector Laplacian''.

Consider taking the curl ($\nabla \times$) of the equation that has the curl of the electric field (Faraday's law):
\begin{equation}
\nabla \times \bigg(\nabla \times \vec E &= -\frac{\partial\vec B}{\partial t}\bigg)\\
\nabla(\nabla\cdot \vec E)-\nabla^2\vec E &= -\nabla \times \frac{\partial\vec B}{\partial t}\\
-\nabla^2\vec E &= - \frac{\partial}{\partial t} \nabla \times \vec B\\
-\nabla^2\vec E &= - \frac{\partial}{\partial t} \mu_0 \epsilon_0\frac{\partial E}{\partial t}\\
-\nabla^2\vec E &= - \mu_0 \epsilon_0\frac{\partial^2\vec E}{\partial t^2}
\end{equation}
where, in the third line, we made use of Gauss' Law ($\nabla \cdot \vec E=0$), and, in the fourth line, Ampère's Law ($\nabla \times \vec B = \mu_0 \epsilon_0\frac{\partial E}{\partial t}$). The last equation that we obtained is a vector equation (the vector Laplacian has three components, as does the time-derivative of $\vec E$ on the right-hand side). Consider the $x$ component of this equation:
\begin{equation}
\frac{\partial^2 E_x}{\partial x^2}+\frac{\partial^2  E_x}{\partial y^2} + \frac{\partial^2 E_x}{\partial z^2}  &= \mu_0 \epsilon_0\frac{\partial^2 E_x}{\partial t^2}
\end{equation}
If we define the quantity:
\begin{equation}
c = \frac{1}{\sqrt{\epsilon_0\mu_0}}
\end{equation}
then, the $x$ component of the equation can be written as:
\begin{equation}
\frac{\partial^2 E_x}{\partial x^2}+\frac{\partial^2  E_x}{\partial y^2} + \frac{\partial^2 E_x}{\partial z^2} &= \frac{1}{c^2}\frac{\partial^2 E_x}{\partial t^2}
\end{equation}
which is exactly the wave equation for the component, $E_x$, of the electric field, propagating with speed $c$, the speed of light! Thus, the speed of light is directly related to the constants $\epsilon_0$ and $\mu_0$. You can write out similar equations for the $y$ and $z$ components of the electric field, and find the similar equations for the magnetic field if you start by taking the curl of Ampère's Law instead of Faraday's law.

We have just shown that electric and magnetic fields can behave as waves, which we now understand to be the waves that are responsible for light, radio waves, gamma rays, infra-red radiation, etc. All of these are types of electromagnetic waves with different frequencies. Although we did not demonstrate this, the electromagnetic waves that propagate are such that the magnetic and electric field vectors are always perpendicular to each other. Electromagnetic waves also carry energy. Thus, a charge that is oscillating (say on a spring) and creating an electromagnetic wave must necessarily be losing energy (or work must be done to keep the charge oscillating with the same amplitude). Finally, it is worth noting that, according to Quantum Mechanics, light (and the other frequencies of radiation), are really carried by particles called ``photons''. Those particles are strange, since their propagation is described by a wave equation.

\subsubsection{Summary}

Faraday's law connects a \textbf{changing} magnetic flux to an induced voltage:
\begin{equation}
\Delta V = -\frac{d\Phi_B}{dt}
\end{equation}
The magnetic flux, $\Phi_B$, is calculated as the flux of the magnetic field through an open surface, $S$:
\begin{equation}
\Phi_B = \int_S \vec B\cdot d\vec A
\end{equation}
The induced voltage, $\Delta V$, is the potential difference that is induced along the closed path (a ``loop'') that bounds the surface, $S$. If a charge, $q$, were to move around that closed path, it would gain (or lose) energy, $q\Delta V$. Note that the potential difference that is induced corresponds to a non-conservative electric force, as a charge can gain/lose energy by moving along a closed path. The induced voltage is often called an induced electromotive force (emf), even if it is a voltage.

The minus sign in Faraday's law is sometime referred to as ``Lenz's law'', since it indicates in which direction the induced voltage will be. It is easiest to think of the closed path as a physical wire (e.g. a loop of wire) through which a current will be induced as a result of the induced voltage. The minus sign is easiest to interpret in terms of the relative direction between the area vector used to define the flux, and the magnetic dipole moment vector, $\vec \mu$, associated with the induced current (which points in the same direction as the magnetic field that is produced by the induced current).

When calculating the flux of the magnetic field, the surface element vector $d\vec A$, must be perpendicular to the surface through which the flux is calculated, which leads to two possible choices. Once a choice is made, and Faraday's law has been applied, the sign of $\Delta V$ will indicate if the magnetic dipole moment of the induced current points in the same direction as $d\vec A$ (positive $\Delta V$) or in the opposite direction (negative $\Delta V$).

If $N$ loops of wire are combined together into a coil, the voltages across each loop sum together, so that the voltage induced across the coil is given by:
\begin{equation}
\Delta V = -N\frac{d\Phi_B}{dt}
\end{equation}

Lenz's law is a statement about conservation of energy. Indeed, the induced current must create a magnetic field that \textbf{opposes} the change in flux, otherwise, the induced current would grow indefinitely. Lenz's law can be summarized as follows:

\begin{itemize}
\item If the magnitude of the magnetic \textbf{flux is increasing} in the loop, then the induced current produces a magnetic field that is in the \textbf{opposite direction} from the original magnetic field.
\item If the magnitude of the magnetic \textbf{flux is decreasing} in the loop, then the induced current produces a magnetic field that is in the \textbf{same direction} as the original magnetic field.
\end{itemize}

A voltage is induced along a closed path any time that the flux of the magnetic field through the corresponding surface changes. The flux can change either because the magnetic field is changing, or because the loop is changing (in size or orientation relative to the magnetic field). In the latter case (changing loop), one speaks of a ``motional emf''. A generator creates a motional emf by rotating a coil (with $N$ loops, each with area, $A$), inside a fixed uniform magnetic field, $\vec B$. The voltage produced by a generator is given by:
\begin{equation}
\Delta V = NAB\omega\sin(\omega t)
\end{equation}
where $\omega$ is the angular speed of the coil. A generator thus produces alternating voltage/current. The current that is induced in the coil of the generator will dissipate energy as it flows through a resistance, $R$. Thus, one must do work in order to keep the generator spinning. The current induced in the coil of the generator will also result in a magnetic moment, and a ``counter torque'' will be exerted on the coil. One must thus exert a torque in order to keep the generator spinning (and the work done by exerting that torque is converted into the electrical energy dissipated in the resistor). The counter torque on the generator is always in the same direction, and has a magnitude:
\begin{equation}
\tau = \frac{NA^2B^2\omega\sin^2(\omega t)}{R}
\end{equation}

When an electric motor is used, a ``back emf'' is induced in the coil of the motor. The back emf is such that it resists the direction of current (Lenz's law), or else the motor would spin infinitely fast. As the motor spins faster, the back emf grows, until it reaches an equilibrium. Motors thus draw a large current when they first start up, since at low speed, they have no back emf.

Since a changing magnetic flux induces a voltage, an electric field is also induced. We can replace the voltage in Faraday's law with the circulation of the electric field to write a more general version of Faraday's law:
\begin{equation}
\oint \vec E\cdot d\vec l &= -\frac{d\Phi_B}{dt}
\end{equation}
The induced electric field forms closed field lines, and is different than the electric field that is produced by static charges, since the latter will have field lines that start and end on charges. The force associated with the induced electric field is not conservative.

When a metallic object passes through a region of magnetic field, the induced electric field will induce current loops in the material called eddy currents. The magnetic field will also exert a force on these eddy currents to oppose the motion that is creating the currents (Lenz's law); as the eddy currents dissipate electrical energy in the material, the metallic object must lose kinetic energy unless a force is acting on it. Magnetic brakes make use of this principle.

Transformers are used to convert an alternating voltage, $\Delta V_p$, into a different alternating voltage, $\Delta V_s$. A ``primary'' coil, with $N_p$ windings, creates a changing magnetic flux that is guided (e.g. by an iron core) to a ``secondary'' coil, with $N_s$ windings. The voltage induced in the secondary coil is given by:
\begin{equation}
\Delta V_s &= \frac{N_p}{N_s}\Delta V_p
\end{equation}

Maxwell's four equations form our best classical theory of electromagnetism. Those equations imply that a changing magnetic field produces an electric field (Faraday's law), while a changing electric field can produce a magnetic field (Ampère's Law). By combining Maxwell's equation (with some heavy vector calculus), one can show that this leads to the formation of electromagnetic waves, that propagate with a speed, $c$, given by:
\begin{equation}
c = \frac{1}{\sqrt{\epsilon_0\mu_0}}
\end{equation}

\begin{framed}
\textbf{Important Equations}\\
\textbf{Magnetic flux:}
\begin{equation}
\Phi_B = \int_S \vec B\cdot d\vec A
\end{equation}

\textbf{Faraday's law:}
\begin{equation}
\Delta V = -N\frac{d\Phi_B}{dt}
\end{equation}

\textbf{Faraday's law:}
\begin{equation}
\oint \vec E\cdot d\vec l &= -\frac{d\Phi_B}{dt}
\end{equation}

\textbf{Voltage produced by a generator:}
\begin{equation}
\Delta V = NAB\omega\sin(\omega t)
\end{equation}

\textbf{Counter torque on a generator:}
\begin{equation}
\tau = \frac{NA^2B^2\omega\sin^2(\omega t)}{R}
\end{equation}

\textbf{Secondary voltage in a transformer:}
\begin{equation}
\Delta V_s &= \frac{N_p}{N_s}\Delta V_p
\end{equation}
\end{framed}

\subsubsection{Thinking about the material}

\begin{framed}
\textbf{Reflect and research}\\
\begin{itemize}
\item Who first discovered induction? Why is it called Faraday's law?
\item Give a few examples of applications of magnetic braking.
\item How does a microphone make use of electromagnetic induction?
\item What is magnetic damping?
\item How does an induction stove work?
\item How does a credit card swipe reader make use of induction?
\item What is the origin of Maxwell's equations? When did he publish them?
\item Who was the first to detect electromagnetic waves? How were they produced and detected?
\end{itemize}
\end{framed}

\begin{framed}
\textbf{To try at home}\\
\begin{itemize}
\item Demonstrate magnetic braking by moving a conducting piece of material through a magnetic field.
\end{itemize}
\end{framed}

\begin{framed}
\textbf{To try in the lab}\\
\begin{itemize}
\item Construct an AC generator.
\item Propose an experiment to measure Earth's magnetic field using induction.
\item Propose an experiment to measure a bar magnet's strength using induction.
\end{itemize}
\end{framed}

\subsubsection{Sample problems and solutions}

\paragraph{Problems}

\begin{framed}
\textbf{Problem 22.1}\\
In the 1950s, the Royal Canadian Air Force developed a jet airplane called the Avro Arrow. This jet reached a speed of Mach 1.9 ($652 {\rm ms^{ -1}}$), and was considered one of the most advanced airplanes that existed at the time. Suppose that the Avro Arrow is travelling at a velocity of $v = 652 {\rm ms^{ -1}}$ above the South Pole through Earth's vertical magnetic field, $B = 5.2e -5 {\rm T}$, as shown in Figure~\ref{fig:induction:avro}. If the Avro Arrow had a wingspan of $l = 15 {\rm m}$, determine the induced voltage across its wings.

\begin{figure}[!htbp]
\centering
\includegraphics[width=0.25\linewidth]{files/avro-2bee885122f39a9c5de989faa5ab7a4b.png}
\caption[]{The Avro Arrow moving through a magnetic field.}
\label{fig:induction:avro}
\end{figure}
\end{framed}

\begin{framed}
\textbf{Problem 22.2}\\
A generator is made of $N$ circular loops of radius $R=0.3 {\rm m}$, rotating at a frequency of $f=60 {\rm Hz}$ in a uniform magnetic field, $B=0.1 {\rm T}$. How many coils must the generator have in order for it to produce an alternating voltage with a maximum amplitude of $\Delta V =110 {\rm V}$.
\end{framed}

\paragraph{Solutions}

\begin{framed}
\textbf{Solution 22.1}\\
This is identical to the motional emf that is generated by a bar moving in a magnetic field. As the airplane moves as illustrated (towards the left, in an upwards magnetic field), the electrons in the wing of the airplane will be pushed into the page. Eventually, the electric field from the electrons will prevent further electrons from accumulating at that side of the wing, and there will be a constant (Hall) voltage, $\Delta V$, across the wing tips. This will happen when the magnetic and electric force are equal and opposite:
\begin{equation}
qvB &= qE = q\frac{\Delta V}{L}
\end{equation}
where $L$ is the wingspan of the airplane. The induced potential is thus given by:
\begin{equation}
\Delta V = BLv = (5.2e-5 {\rm T})(15 {\rm m})(652 {\rm ms^{-1}})=0.51 {\rm V}
\end{equation}
\end{framed}

\begin{framed}
\textbf{Solution 22.2}\\
The voltage produced by a generator is given by:
\begin{equation}
\Delta V&=NAB\omega\sin(\omega t)
\end{equation}
and the angular frequency is given by $\omega = 2\pi f$. The number of required coils is thus:
\begin{equation}
N=\frac{\Delta V}{AB\omega}=\frac{\Delta V}{\pi R^2B2\pi f}=\frac{(110 {\rm V})}{2\pi^2(0.3 {\rm m})^2(0.1 {\rm T})(60 {\rm Hz})}=10.3
\end{equation}
Thus, one requires 10 loops in the coil to generate the desired voltage.
\end{framed}

\include{ModelingWithPhysics-specialrelativity}

\section{Appendices}

\include{ModelingWithPhysics-calculus}

\subsection{Appendix B - An Introduction to Visual Python}

\subsubsection{Overview}\label{app:visualpython}

This appendix gives a very brief introduction to calculus with a focus on the tools needed in physics.

\begin{framed}
\textbf{Learning Objectives}\\
\begin{itemize}
\item Understand the \href{http://trinket.io}{trinket.io} platform
\item Understand how to create and position objects in trinket.
\item Understand how to create motion with objects in trinket.
\end{itemize}
\end{framed}

\subsubsection{Code in this textbook}

When referring to general ideas or something very brief, Visual Python code may be written in-line. For example, \texttt{this is a general code statement}. When giving explicit code to be implemented, Visual Python code will be written in ``display mode''. For example,

\begin{verbatim}
print("This code is meant to be implemented into a trinket.")
\end{verbatim}

The distinction should become clear as you work through this appendix. We hope to provide in this textbook all of the instruction necessary to create physics simulations using Visual Python in the trinket platform. In the event that more information is desired, a good reference is \href{https://www.glowscript.org/docs/VPythonDocs/index.html}{https://www.glowscript.org/docs/VPythonDocs/index.html}.

\subsubsection{The \href{http://trinket.io}{trinket.io} platform}

\href{http://Trinket.io}{Trinket.io} is web-based programming platform with a few different programming languages available,e.g., Python, Visual Python, R, Java, and HTML5. In this textbook, we will use primarily Visual Python, which is a language built on Python that allows Python programming to manipulate simulated objects in three-dimensional space. The trinket Figure~\ref{app:visualpython:earthmoonorbit} is a Visual Python program that simulates a moon-like mass orbiting an Earth-like mass. Click the play button to see the simulation.

\begin{figure}[!htbp]
\centering
\caption[]{The moon orbiting the Earth.}
\label{app:visualpython:earthmoonorbit}
\end{figure}

A few notes about the embedded trinkets in this book:

\begin{itemize}
\item The menu in the upper left corner allows full-screen viewing.
\item The center bar between the code and the simulation can be adjusted to make the either consume more of the screen for easier viewing.
\item The right mouse button allows rotation of the view.
\item The mouse wheel allows zooming in and out of the view.
\item If signed into trinket, readers should be able to click ``Remix'' in the upper right corner to save a trinket to their own account.
\end{itemize}

\paragraph{Trinket accounts}

The website \href{http://trinket.io}{trinket.io} offers a variety of accounts. You can learn more about these at \href{https://trinket.io/schools}{https://trinket.io/schools}. The three main account modes trinket might be implemented in your course are

\begin{itemize}
\item with free accounts for everyone where the instructor shares pre-written shells for students to complete
\item with free accounts for students and a pay account for the instructor. This allows the instructor to build trinket courses with assignments.
\item with pay accounts for everyone that offers the most options.
\end{itemize}

\paragraph{What is a trinket?}

A trinket is a Visual Python program written in the \href{http://trinket.io}{trinket.io} web platform. When starting a new program, a trinket will look like Figure~\ref{app:visualpython:blanktrinket}. The \texttt{Web Vpython} line indicates the language. This line should always be in a trinket that uses the Visual Python language. The version number may change as updates are made to the Visual Python language.

\begin{figure}[!htbp]
\centering
\caption[]{A blank trinket ready to begin programming.}
\label{app:visualpython:blanktrinket}
\end{figure}

Trinkets in this book are interactive. Readers can enter code and run the code while reading the textbook. Give it a try. Enter the following code and click the play button.

\begin{verbatim}
print("Hello World! This is my first trinket.")
\end{verbatim}

If you wish, you can Remix the trinket to your own account. It will appear in your account as ``BlankTrinket''. You may rename your trinkets from the \href{http://trinket.io}{trinket.io} website.

\subsubsection{Programming in Visual Python}

\paragraph{Create objects in a trinket}

The choice of Visual Python is so that readers can visualize the physics being discussed in the book. Therefore, almost every program will be focused on manipulating Visual Python objects. It is important for readers to become comfortable with creating these objects. In the following examples, we will demonstrate spheres and cylinders. All Visual Python objects have ``attributes'' such as position, color, and size. \href{\#tab:visualpython:objattrs}{} lists common objects and their attributes. Notice the attributes vary in definition such as vector, scalar, or some text-based variable. It is also possible to create new, user-defined attributes, which will be discussed below.

\begin{table}
\centering
\caption[]{A quick reference for a few commonly used objects and their attributes}
\label{app:visualpython:objsattrs}
\begin{tabular}{p{\dimexpr 0.500\linewidth-2\tabcolsep}p{\dimexpr 0.500\linewidth-2\tabcolsep}}
\toprule
Object & Attributes \\
\hline
sphere & pos (vector) -- Position of center. Default (0,0,0). \\
 & radius (scalar) -- Default 1. \\
 & color (vector) -- Default color.white \\
 & size (vector) -- Dimensions of a box surrounding the sphere. Default (2,2,2). \\
 & axis (vector) -- Default (1,0,0) \\
box & pos (vector) -- Position of center. Default (0,0,0). \\
 & axis (vector) -- Extends from left to right end. Default \textless 1,0,0\textgreater . \\
 & color (vector) -- Default color.white. \\
 & length (scalar) -- Length of box. Default 1. \\
 & height (scalar) -- Height of box. Default 1. \\
 & width (scalar) -- Width of box. Default 1. \\
 & size (vector) -- Length, width, and height in one vector. An alternative to length, height, width. \\
arrow & pos (vector) -- Position of tail. Default (0,0,0). \\
 & axis (vector) -- Extends from tail to tip. Default (1,0,0). \\
 & color (vector) -- Default is color.white \\
 & round (boolean) -- Makes shaft and head round instead of square. Default is False. \\
 & shaftwidth (scalar) -- Width of tail. Default 0.1*(length of arrow) \\
 & headwidth (scalar) -- Default 2*shaftwidth \\
 & headlength (scalar) -- Default 3*shaftwidth \\
cylinder & pos (vector) -- Position of left end. Default (0,0,0). \\
 & axis (vector) -- Extends from pos to end. Default (1,0,0). \\
 & color (vector) -- Default color.white. \\
 & radius -- Radius of the cylinder. Default is 1. \\
 & length (scalar) -- Length of axis. Setting length sets magnitude of axis. Default is 1. \\
 & size (vector) -- Length, height, width of a box surrounding the cylinder. Default is (1,1,1). \\
\bottomrule
\end{tabular}
\end{table}

\subparagraph{Spheres}

To create a sphere at the origin of a cartesian coordinate system, enter the following code.

\begin{verbatim}
ball = sphere(pos=vec(0,0,0), color=color.green, radius=0.1)
\end{verbatim}

One can think of the spatial coordinate system being in meters so that all lengths are denoted in meters. Of course, this is arbitrary, and the scale could be considered any length desired, and the programming would need to match the chosen scale. Notice the color is defined with \texttt{color=color.green}. It is important to define colors this way, i.e., with the ``color-dot'' format. Readily available colors are red, green, blue, cyan, magenta, yellow, orange, purple, and white. The \texttt{radius} variable is often chosen based on the scale of motion that will be simultated. For example, if the sphere is going to be moved centimeters, the radius should be smaller than 1 cm, and if the sphere is going to move meters, the radius would be smaller than 1 meter.

\begin{framed}
\textbf{Example}\\
Make a collection of spheres each a different color arranged on the corners of a 1x1x1 cube with the origin as the center of the cube.

\begin{framed}
\textbf{Solution}\\
Notice in the code, we define the half-length of the cube edges \texttt{a} and the sphere radii \texttt{r}. Then, we displace each sphere the half-length along the various axes. This makes the program easily edited to change the size of the cube and size of the spheres.

\begin{figure}[!htbp]
\centering
\caption[]{Spheres on the corners of a 1x1x1 cube.}
\label{ex:visualpython:simplecube}
\end{figure}
\end{framed}
\end{framed}

\subparagraph{Cylinders}

To create a magenta cylinder 5 meters long, radius 0.1 meter, named \texttt{cyl}, with one end at the origin and along the $x$-axis, enter the following code.

\begin{verbatim}
cyl = cylinder(pos=vec(0,0,0), axis=vec(1,0,0), radius=0.1, length=5, size=2, color=color.magenta)
\end{verbatim}

The axis can easily be changed to orient a cylinder along the cartesian axes. If it desired for the cylinder to be oriented at an angle, one would use the Pythagorean Theorem or trigonometry to determine the unit vector pointed in the desired direction. Then, this would become the axis vector. For example, a cylinder oriented at $45\degree$ in the $xy$-plane would be found by
\begin{equation}
x &= r\cos 45\degree\\
y &= r\sin 45\degree\\
z &= 0
\end{equation}
Since we want a unit vector ($r=1$), and $x=y=\sqrt{1/2}$. In Python power are expressed using \texttt{**} rather than \texttt{\^}. The command to orient a cylinder at $45\degree$ is

\begin{verbatim}
cyl = cylinder(pos=vec(0,0,0), axis=vec(0.5**0.5,0.5**0.5,0), radius=0.1, length=5, size=2, color=color.magenta)
\end{verbatim}

However, we may also use trigonometric functions.

\begin{verbatim}
cyl = cylinder(pos=vec(0,0,0), axis=vec(cos(pi/4),sin(pi/4),0), radius=0.1, length=5, size=2, color=color.magenta)
\end{verbatim}

\begin{framed}
\textbf{Example}\\
Make a collection of twelve cylinders arranged to evenly divide a circle into sections like the hours on the face of a clock. Adjust the length and radius to suit your desired look.

\begin{framed}
\textbf{Solution}\\
Notice in the code, we define the radius and length, \texttt{R} and \texttt{L}. This makes the program easily edited to change the size of the model.

\begin{figure}[!htbp]
\centering
\caption[]{Spheres on the corners of a 1x1x1 cube.}
\label{ex:visualpython:cylinderclock}
\end{figure}
\end{framed}
\end{framed}

\subparagraph{\texttt{for} loops}

It is possible to simplify the code using a \texttt{for} loop. This loop will increment through the desired angles to create the clock face with a single cylinder object command. The advantage is streamlined code. The disadvantage is that the cylinder objects cannot be manipulated unless we complicate the program by tracking the object names. This kind of programming is beyond the scope of this textbook. Figure~\ref{fig:visualpython:forloopcylinder} shows the code using a \texttt{for} loop. In this program, we define the number of cylinders we want \texttt{numcyl}. This number is used to divide $2\pi$ into that many angle segments. The \texttt{for} loop uses the Python \texttt{range()} function, which is an ``inclusive-exclusive'' list. That is, \texttt{for i in range(N):} would increment \texttt{i} from 0 to N-1. Notice the tabbing after the line beginning the loop indicates which commands are inside the loop.

\begin{figure}[!htbp]
\centering
\caption[]{For loop to place cylinders every $30\degree$.}
\label{fig:visualpython:forloopcylinder}
\end{figure}

\subparagraph{User-defined attributes}

Object attributes are accessed using the ``dot'' notation. For example, the position of a sphere named \texttt{ball} can be accessed with \texttt{ball.pos}. It is possible to add user-defined attributes to an object. Assigning attributes is a simple way to keep track of variables and constants associated with objects. These attributes do not affect the visualization of the object. Some examples of assigning attributes to a sphere named \texttt{ball} are shown in Table~\ref{tab:visualpython:userattributes}. Users are free to choose the names of the attributes they assign. Rembember that computer code is unit agnostic. The units are whatever the programmer decides, and it is up to the programmer to keep track of units. In this book, we will primarily use SI (or mks) units.

\begin{table}
\centering
\caption[]{Some examples of user-defined object attributes.}
\label{tab:visualpython:userattributes}
\begin{tabular}{p{\dimexpr 0.500\linewidth-2\tabcolsep}p{\dimexpr 0.500\linewidth-2\tabcolsep}}
\toprule
\textbf{Attribute} & \textbf{Visual Python} \\
\hline
mass & \texttt{ball.m = 5} \\
velocity & \texttt{ball.vel = vec(1,1,1)} \\
acceleration & \texttt{ball.acc = vec(0,0,-9.8)} \\
momentum & \texttt{ball.p = ball.m * ball.vel} \\
\bottomrule
\end{tabular}
\end{table}

\subsubsection{Move objects using code and physics}

Once we have objects, we can apply physical principles to make them move. One way to make the ball move is to repeatedly update the position using kinematic descriptions of motion. If we define an object's position as $\vec r$, the object will change position over time as
\begin{equation}
\vec r(t + \Delta t) = \vec r(t) + \vec v\Delta t + \frac{1}{2}\vec a\Delta t^2
\end{equation}
Suppose we want to model a situation where there is not acceleration. We could write lines of code that take the current position $\vec r(t)$ and update it with some amount $\vec v\Delta t$. Since we are in a 3D world, we must do this with vectors. For simplicity, let's assume the velocity is only the $x$-direction that is a speed of $v = 0.5 {\rm m/s}$, i.e., \texttt{ball.vel=vec(0.5,0,0)}. Suppose we want to update every second, i.e., \texttt{t = 0, 1, 2, 3,...} or $\Delta t = 1$. To update the position of our object we could write code to position the ball each second that passes.

\begin{verbatim}
ball = sphere(pos=vec(0,0,0), radius=0.1, color=color.red)
ball.vel=vec(0.5, 0, 0)
dt = 1
ball.pos = ball.pos + ball.vel * dt #move ball after 1s
ball.pos = ball.pos + ball.vel * dt #move ball after 2s
ball.pos = ball.pos + ball.vel * dt #move ball after 3s
ball.pos = ball.pos + ball.vel * dt # move ball after 4s
ball.pos = ball.pos + ball.vel * dt # move ball after 5s
\end{verbatim}

The statements at the end of the lines after \texttt{\#} are comments that are ignored by the computer. Copy the code above run these lines in the trinket below.

\begin{figure}[!htbp]
\centering
\caption[]{A blank trinket ready to begin programming.}
\label{app:visualpython:blanktrinket}
\end{figure}

You will see that your program runs all of the lines instantly, and you do not see the motion. Slow the computation down by adding the command \texttt{rate(1)} between each line updating the ball's position. This will delay each line by 1 second. For example,

\begin{verbatim}
ball.pos = ball.pos + ball.vel * dt
rate(1)
ball.pos = ball.pos + ball.vel * dt
rate(1)
\end{verbatim}

It might also be helpful to keep track of the time and position. To do this, create a variable for time, \texttt{t}, and add print statements between each move. Notice below that we can access the $x$ position with \texttt{ball.pos.x}.

\begin{verbatim}
ball = sphere(pos=vec(0,0,0), radius=0.1, color=color.red)
ball.vel=vec(0.5, 0, 0)
t = 0
dt = 1
print(t, ball.pos.x)
rate(1)
ball.pos = ball.pos + ball.vel * dt #move ball after 1s
t = t + dt #increase the total time
print(t, ball.pos.x)
rate(1)
ball.pos = ball.pos + ball.vel * dt #move ball after 2s
t = t + dt #increase the total time
print(t, ball.pos.x)
rate(1)
ball.pos = ball.pos + ball.vel * dt #move ball after 3s
t = t + dt #increase the total time
print(t, ball.pos.x)
rate(1)
ball.pos = ball.pos + ball.vel * dt # move ball after 4s
t = t + dt #increase the total time
print(t, ball.pos.x)
rate(1)
ball.pos = ball.pos + ball.vel * dt # move ball after 5s
t = t + dt #increase the total time
print(t, ball.pos.x)
\end{verbatim}

Finally, let's make the ball's motion obvious by adding a trail to it. In the definition of the ball add \texttt{make\_trail = True}.

\begin{verbatim}
ball = sphere(pos=vec(0,0,0), radius=0.1, color=color.red, make_trail=True)
\end{verbatim}

\paragraph{\texttt{while} loops}

Hopefully it is clear to you this is an inefficient way to write a program, especially if we want to run many iterations of changing the ball's position. To make the process more streamlined, we use loops. In this case, we will use a \texttt{while} loop. The while loop runs ``while'' a condition is met. For example, we can run the ball simulation while the time is less than 5 seconds. Because the loop streamlines the computation, we can get higher time resolution and update the position every 0.1 second. A \texttt{while} loop looks similar to a \texttt{for} loop.

\begin{verbatim}
while t < 5:
\end{verbatim}

Since we are using the total time as the condition for the \texttt{while} loop, it is important to keep track of the total time as we did in the previous section. To implement a \texttt{while} loop, we first initialize variables and objects. Then, every command inside the \texttt{while} loop will repeat until the condition is met.

\begin{verbatim}
#Initialization of objects and variables
ball = sphere(pos=vec(0,0,0), radius=0.1, color=color.red, make_trail=True)
ball.vel=vec(0.5, 0, 0)
t = 0
dt = 0.1

while t < 5:
	rate(1)
	ball.pos = ball.pos + ball.vel * dt #move ball after dt
	t = t + dt #increase the total time
	print(t, ball.pos.x) #print time and position
\end{verbatim}

Copy and paste this code to replace the code from the previous section. It will appear to run very slowly. Increase the rate until the motion looks continuous.

\begin{framed}
\textbf{Example}\\
Change the code for the ball example above so that the ball is thrown upward along the $y$-axis with an initial speed of $10 {\rm m/s}$, i.e., \texttt{ball.vel=vec(0, 10, 0)}. Add the acceleration due to gravity using the kinematic equation HINT: Acceleration will change both the position and the velocity over time.
\begin{equation}
\vec r(t + \Delta t) = \vec r(t) + \vec v\Delta t + \frac{1}{2}\vec a\Delta t^2
\end{equation}
\begin{framed}
\textbf{Solution}\\
The velocity will now initially be \texttt{ball.vel = vec(0,10,0)}. There is an acceleration \texttt{ball.acc = vec(0,-9.8,0)}. The position and velocity update will follow the kinematic equations
\begin{equation}
\vec r(t + \Delta t) &= \vec r(t) + \vec v\Delta t + \frac{1}{2}\vec a\Delta t^2\\
\vec v(t + \Delta t) &= \vec v(t) + \vec a\Delta t
\end{equation}
and in Visual Python, the code is

\begin{verbatim}
ball.pos = ball.pos + ball.vel * dt + 0.5 * ball.acc * dt**2 #move ball after dt
ball.vel = ball.vel + ball.acc * dt #update the ball's velocity after dt
\end{verbatim}

The entire program will be

\begin{verbatim}
#Initialization of objects and variables
ball = sphere(pos=vec(0,0,0), radius=0.1, color=color.red, make_trail=True)
ball.vel=vec(0, 10, 0)
ball.acc=vec(0, -9.8,0)
t = 0
dt = 0.1

while t < 5:
	rate(10)
	ball.pos = ball.pos + ball.vel * dt + 0.5 * ball.acc * dt**2 #move ball after dt
	ball.vel = ball.vel + ball.acc * dt #update the ball's velocity after dt
	t = t + dt #increase the total time
	print(t, ball.pos.y) #print time and position
\end{verbatim}
\end{framed}
\end{framed}

\include{ModelingWithPhysics-labs}



\printbibliography

% DON'T EDIT. If "endfloat" option is enabled all floats appear before appendices
\if@endfloat\clearpage\processdelayedfloats\clearpage\fi


%%%%%%%%%%%%%%%%%%%%%%%%%%%%%%%%%%%%%%%%%%%%%%%%%%%%%%%%%%%%
%%% SUPPLEMENTARY MATERIAL / APPENDICES
%%%%%%%%%%%%%%%%%%%%%%%%%%%%%%%%%%%%%%%%%%%%%%%%%%%%%%%%%%%%
%% Sadly, we can't use floats in the appendix boxes. So they don't "float", but use \captionof{figure}{...} and \captionof{table}{...} to get them properly caption.

%%%%%%%%%%%%%%%%%%%%%%%%%%%%%%%%%%%%%%%%%%%%%%%%%%%%%%%%%%%%
%%% ARTICLE END
%%%%%%%%%%%%%%%%%%%%%%%%%%%%%%%%%%%%%%%%%%%%%%%%%%%%%%%%%%%%

\end{document}
